\documentclass{article}
\usepackage{graphicx}
\usepackage{amsthm}
\usepackage{amsmath}
\usepackage{amssymb}
\usepackage{geometry}
\usepackage{tikz}
\usepackage[hidelinks]{hyperref}
\usetikzlibrary{arrows}

\geometry{a4paper, total={170mm,257mm}, left=20mm, top=20mm}
\AtBeginEnvironment{align}{\setcounter{equation}{0}} 
\AtBeginEnvironment{eqnarray}{\setcounter{equation}{0}} 

\newcommand\blfootnote[1]{
    \begingroup
    \renewcommand\thefootnote{}\footnote{#1}
    \addtocounter{footnote}{-1}
    \endgroup
}

\title{Module 6 Notes (MATH-211)}
\author{Lillie Donato}
\date{15 July 2024}

\begin{document}

\maketitle

\section*{General Notes (and Definitions)}
\begin{itemize}
    \item L'Hôpital's Rule \\
    \textbf{Indeterminate Form}: An expression involving two components where the limit cannot be determined by evaluating the limits of the individual components. \\
    \textbf{L'Hôpital's Rule}: Suppose $f$ and $g$ are differentiable functions on an open interval $I$ containing the point $x = a$, with $g'(x) \neq 0$ on $I$ when $x \neq a$. \\
    If $\lim\limits_{x \to a}{\frac{f(x)}{g(x)}}$ has any of the indeterminate forms: $\frac{0}{0}$, $\frac{\infty}{\infty}$, $-\frac{\infty}{\infty}$, then
    $$\lim_{x \to a}{\frac{f(x)}{g(x)}} = \lim_{x \to a}{\frac{f'(x)}{g'(x)}}$$
    provided that one of the following is the case:
    $$\lim_{x \to a}{\frac{f'(x)}{g'(x)}} \in \mathbb{R}$$
    $$\lim_{x \to a}{\frac{f'(x)}{g'(x)}} = \infty$$
    $$\lim_{x \to a}{\frac{f'(x)}{g'(x)}} = -\infty$$
    L'Hôpital's Rule is still valid if $x \to a$ is replaced by any of $x \to a^+$, $x \to a^-$, $x \to \infty$, or $x \to -\infty$. In the last two of these cases, there must be a greatest $x$-value beyond which both $f$ and $g$ are differentiable at every point. \\
    \textbf{Exponential Indeterminate forms}: $1^{\infty}$, $0^0$, $\infty^0$ \\
    \textbf{Method for evaluating limits of indeterminate forms $1^{\infty}$, $0^0$, $\infty^0$}: \\
    Assume that $L = \lim\limits_{x \to a}{f(x)^{g(x)}}$ has one of these indeterminate forms.
    \begin{enumerate}
        \item Use the fact that the natural logarithm and natural exponential functions are inverses to write
        $$L = \lim_{x \to a}{e^{\ln{\left(f(x)^{g(x)}\right)}}}$$
        \item Use the power property of logarithm arguments to write
        $$L = \lim_{x \to a}{e^{g(x)\ln{\left(f(x)\right)}}}$$
        \item Use continuity of the exponential function to write
        $$L = e^{\lim\limits_{x \to a}{g(x)\ln{\left(f(x)\right)}}}$$
        \item Rewrite multiplication as division by the reciprocal:
        $$L = e^{\lim\limits_{x \to a}{\left(\frac{\ln{\left(f(x)\right)}}{\frac{1}{g(x)}}\right)}}$$
        \item Use L'Hôpital's Rule to evaluate this limit expression
    \end{enumerate}
    \textbf{Growth Rates}: Suppose $f$ and $g$ are functions with $\lim\limits_{x \to \infty}{f(x)} = \infty$ and $\lim\limits_{x \to \infty}{g(x)} = \infty$ \\
    \begin{enumerate}
        \item If one of the following are true, \textbf{$f$ grows faster than $g$}, and we use the notation $f \gg g$
        \begin{eqnarray}
            \lim_{x \to \infty}{\frac{g(x)}{f(x)}} &=& 0 \\
            \lim_{x \to \infty}{\frac{f(x)}{g(x)}} &=& \infty
        \end{eqnarray}
        \item \textbf{$f$ and $g$ have comparable growth rates}, if there is some non-zero finite number $M$ such that
        $$\lim_{x \to \infty}{\frac{f(x)}{g(x)}} = M$$
    \end{enumerate}
    \textbf{Ranked Growth Rates as $x \to \infty$} \\
    For any base $b > 1$, and for any positive numbers $p$, $q$, $r$, and $s$
    $$\ln^q{x} \ll x^p \ll x^p \ln^r{x} \ll x^{p + s} \ll b^x \ll x^x$$
\end{itemize}

\section*{Examples}
\begin{enumerate}
    \item Use L'Hôpital's Rule to evaluate a limit with indeterminate form $\frac{0}{0}$
        \begin{eqnarray}
            \lim_{x \to 0}{\frac{e^x - x - 1}{5x^2}} &=& \lim_{x \to 0}{\frac{e^x - 1}{10x}} \\
                                                     &=& \lim_{x \to 0}{\frac{e^x}{10}} \\
                                                     &=& \frac{e^0}{10} \\
                                                     &=& \frac{1}{10}
        \end{eqnarray}
    \item Use L'Hôpital's Rule to evaluate a limit with indeterminate form $\frac{\infty}{\infty}$
        \begin{eqnarray}
            \lim_{x \to 0^+}{\frac{1 - \ln{x}}{1 + \ln{x}}} &=& \lim_{x \to 0^+}{\frac{-\frac{1}{x}}{\frac{1}{x}}} \\
                                                            &=& \lim_{x \to 0^+}{\frac{-\frac{1}{x}}{\frac{1}{x}}} \\
                                                            &=& \frac{-1}{1} \\
                                                            &=& -1
        \end{eqnarray}
    \item Use L'Hôpital's Rule to evaluate a limit with indeterminate form $0 \cdot \infty$
        \begin{eqnarray}
            \lim_{x \to 1^-}{\left(1 - x\right)\tan{\left(\frac{\pi x}{2}\right)}} &=& \lim_{x \to 1^-}{\frac{\left(1 - x\right)}{\cot{\left(\frac{\pi x}{2}\right)}}} \\
                                                                                   &=& \lim_{x \to 1^-}{\frac{-1}{-\frac{\pi}{2}\csc^2{\left(\frac{\pi x}{2}\right)}}} \\
                                                                                   &=& \lim_{x \to 1^-}{\frac{2}{\pi}\sin^2{\left(\frac{\pi x}{2}\right)}} \\
                                                                                   &=& \frac{2}{\pi}
        \end{eqnarray}
    \item Use L'Hôpital's Rule to evaluate a limit with exponential indeterminate form
        \begin{eqnarray}
            \lim_{x \to 0^+}{x^{\tan{x}}} &=& e^{\lim\limits_{x \to 0^+}{\frac{\ln{x}}{\frac{1}{\tan{x}}}}} \\
                                          &=& e^{\lim\limits_{x \to 0^+}{\frac{\ln{x}}{\cot{x}}}} \\
                                          &=& e^{\lim\limits_{x \to 0^+}{\frac{1}{-x\csc^2{x}}}} \\
                                          &=& e^{\lim\limits_{x \to 0^+}{\frac{-\sin^2{x}}{x}}} \\
                                          &=& e^{\lim\limits_{x \to 0^+}{\frac{-2\sin{x}\cos{x}}{1}}} \\
                                          &=& e^{\lim\limits_{x \to 0^+}{-2\sin{x}\cos{x}}} \\
                                          &=& e^0 \\
                                          &=& 1
        \end{eqnarray}
    \item Compare the growth rates of functions
        $$f(x) = x^2\ln{x}$$
        $$g(x) = x\ln^2{x}$$
        \begin{eqnarray}
            \lim_{x \to \infty}{\frac{f(x)}{g(x)}} &=& \lim_{x \to \infty}{\frac{x^2\ln{x}}{x\ln^2{x}}} \\
                                                   &=& \lim_{x \to \infty}{\frac{x}{\ln{x}}} \\
                                                   &=& \lim_{x \to \infty}{\frac{1}{\frac{1}{x}}} \\
                                                   &=& \lim_{x \to \infty}{x} \\
                                                   &=& \infty
        \end{eqnarray}
        Since $\lim\limits_{x \to \infty}{\frac{f(x)}{g(x)}} = \infty$, $f \gg g$
\end{enumerate}

\section*{Related Exercises}
\begin{enumerate}
    \item Example
\end{enumerate}

\blfootnote{A copy of my notes (in \LaTeX) are available on my \href{https://github.com/onlinechronically/MATH-211}{GitHub}}
\end{document}
