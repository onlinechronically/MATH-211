\documentclass{article}
\usepackage{graphicx}
\usepackage{amsthm}
\usepackage{amsmath}
\usepackage{amssymb}
\usepackage{geometry}
\usepackage{tikz}
\usepackage[hidelinks]{hyperref}
\usetikzlibrary{arrows}

\geometry{a4paper, total={170mm,257mm}, left=20mm, top=20mm}
\AtBeginEnvironment{align}{\setcounter{equation}{0}} 
\AtBeginEnvironment{eqnarray}{\setcounter{equation}{0}} 

\newcommand\blfootnote[1]{
    \begingroup
    \renewcommand\thefootnote{}\footnote{#1}
    \addtocounter{footnote}{-1}
    \endgroup
}

\title{Module 6 Notes (MATH-211)}
\author{Lillie Donato}
\date{15 July 2024}

\begin{document}

\maketitle

\section*{General Notes (and Definitions)}
\begin{itemize}
    \item L'Hôpital's Rule \\
    \textbf{Indeterminate Form}: An expression involving two components where the limit cannot be determined by evaluating the limits of the individual components. \\
    \textbf{L'Hôpital's Rule}: Suppose $f$ and $g$ are differentiable functions on an open interval $I$ containing the point $x = a$, with $g'(x) \neq 0$ on $I$ when $x \neq a$. \\
    If $\lim\limits_{x \to a}{\frac{f(x)}{g(x)}}$ has any of the indeterminate forms: $\frac{0}{0}$, $\frac{\infty}{\infty}$, $-\frac{\infty}{\infty}$, then
    $$\lim_{x \to a}{\frac{f(x)}{g(x)}} = \lim_{x \to a}{\frac{f'(x)}{g'(x)}}$$
    provided that one of the following is the case:
    $$\lim_{x \to a}{\frac{f'(x)}{g'(x)}} \in \mathbb{R}$$
    $$\lim_{x \to a}{\frac{f'(x)}{g'(x)}} = \infty$$
    $$\lim_{x \to a}{\frac{f'(x)}{g'(x)}} = -\infty$$
    L'Hôpital's Rule is still valid if $x \to a$ is replaced by any of $x \to a^+$, $x \to a^-$, $x \to \infty$, or $x \to -\infty$. In the last two of these cases, there must be a greatest $x$-value beyond which both $f$ and $g$ are differentiable at every point. \\
    \textbf{Exponential Indeterminate forms}: $1^{\infty}$, $0^0$, $\infty^0$ \\
    \textbf{Method for evaluating limits of indeterminate forms $1^{\infty}$, $0^0$, $\infty^0$}: \\
    Assume that $L = \lim\limits_{x \to a}{f(x)^{g(x)}}$ has one of these indeterminate forms.
    \begin{enumerate}
        \item Use the fact that the natural logarithm and natural exponential functions are inverses to write
        $$L = \lim_{x \to a}{e^{\ln{\left(f(x)^{g(x)}\right)}}}$$
        \item Use the power property of logarithm arguments to write
        $$L = \lim_{x \to a}{e^{g(x)\ln{\left(f(x)\right)}}}$$
        \item Use continuity of the exponential function to write
        $$L = e^{\lim\limits_{x \to a}{g(x)\ln{\left(f(x)\right)}}}$$
        \item Rewrite multiplication as division by the reciprocal:
        $$L = e^{\lim\limits_{x \to a}{\left(\frac{\ln{\left(f(x)\right)}}{\frac{1}{g(x)}}\right)}}$$
        \item Use L'Hôpital's Rule to evaluate this limit expression
    \end{enumerate}
    \textbf{Growth Rates}: Suppose $f$ and $g$ are functions with $\lim\limits_{x \to \infty}{f(x)} = \infty$ and $\lim\limits_{x \to \infty}{g(x)} = \infty$ \\
    \begin{enumerate}
        \item If one of the following are true, \textbf{$f$ grows faster than $g$}, and we use the notation $f \gg g$
        \begin{eqnarray}
            \lim_{x \to \infty}{\frac{g(x)}{f(x)}} &=& 0 \\
            \lim_{x \to \infty}{\frac{f(x)}{g(x)}} &=& \infty
        \end{eqnarray}
        \item \textbf{$f$ and $g$ have comparable growth rates}, if there is some non-zero finite number $M$ such that
        $$\lim_{x \to \infty}{\frac{f(x)}{g(x)}} = M$$
    \end{enumerate}
    \textbf{Ranked Growth Rates as $x \to \infty$} \\
    For any base $b > 1$, and for any positive numbers $p$, $q$, $r$, and $s$
    $$\ln^q{x} \ll x^p \ll x^p \ln^r{x} \ll x^{p + s} \ll b^x \ll x^x$$
    \item Antiderivatives \\
    \textbf{Antiderivative}: A function $F$ is an antiderivative of another function $f$ on an interval $I$ if for all $x$ in $I$:
    $$F'(x) = f(x)$$
    \textbf{Family of Antiderivatives}: Let $F(x)$ be any antiderivative of $f(x)$ on an interval $I$. Then all antiderivatives of $f$ on $I$ have the form $F(x) + C$, where $C$ is an arbitrary constant. \\
    \textbf{Differential Equations}: Any equation involving an unknown function and its derivatives
    \begin{itemize}
        \item Infinite family of solutions
        \item No two solutions from the family pass through the same point
        \item Given an initial condition $f(a) = b$, we can identify the particular family member that solves the given problem by solving for $C$
    \end{itemize}
\end{itemize}

\section*{Antiderivative Rules}
\begin{itemize}
    \item Power Rule \\
        If $p \neq -1$ and $C$ is an arbitrary constant:
        $$\int{x^p} dx = \frac{x^{p + 1}}{p + 1} + C$$
    \item Integral of $x^{-1}$
        $$\int{x^{-1}}dx = \int{\frac{1}{x}}dx = \ln{|x|} + C$$
    \item Constant Multiple and Sum Rules \\
        If $c \in \mathbb{R}$:
        $$\int{cf(x)}dx = c\int{f(x)}dx$$
        $$\int{\left(f(x) + g(x)\right)}dx = \int{f(x)}dx + \int{g(x)}dx$$
    \item Integral of $e^x$
        $$\int{e^x}dx = e^x + C$$
\end{itemize}

\section*{Trigonometric (and inverse) Integrals}
\begin{eqnarray}
    \int{\cos{\left(x\right)}}dx &=& \sin{x} + C \\
    \int{\sin{\left(x\right)}}dx &=& -\cos{x} + C \\
    \int{\sec^2{\left(x\right)}}dx &=& \tan{x} + C \\
    \int{\csc^2{\left(x\right)}}dx &=& -\cot{x} + C \\
    \int{\sec{\left(x\right)}\tan{\left(x\right)}}dx &=& \sec{x} + C \\
    \int{\csc{\left(x\right)}\cot{\left(x\right)}}dx &=& -\csc{x} + C \\
    \int{\frac{1}{\sqrt{1 - x^2}}}dx &=& \sin^{-1}{x} + C \\
    \int{\frac{1}{1 + x^2}}dx &=& \tan^{-1}{x} + C \\
    \int{\frac{1}{x\sqrt{x^2 - 1}}}dx &=& \sec^{-1}{|x|} + C
\end{eqnarray}

\section*{Examples}
\begin{enumerate}
    \item Use L'Hôpital's Rule to evaluate a limit with indeterminate form $\frac{0}{0}$
        \begin{eqnarray}
            \lim_{x \to 0}{\frac{e^x - x - 1}{5x^2}} &=& \lim_{x \to 0}{\frac{e^x - 1}{10x}} \\
                                                     &=& \lim_{x \to 0}{\frac{e^x}{10}} \\
                                                     &=& \frac{e^0}{10} \\
                                                     &=& \frac{1}{10}
        \end{eqnarray}
    \item Use L'Hôpital's Rule to evaluate a limit with indeterminate form $\frac{\infty}{\infty}$
        \begin{eqnarray}
            \lim_{x \to 0^+}{\frac{1 - \ln{x}}{1 + \ln{x}}} &=& \lim_{x \to 0^+}{\frac{-\frac{1}{x}}{\frac{1}{x}}} \\
                                                            &=& \lim_{x \to 0^+}{\frac{-\frac{1}{x}}{\frac{1}{x}}} \\
                                                            &=& \frac{-1}{1} \\
                                                            &=& -1
        \end{eqnarray}
    \item Use L'Hôpital's Rule to evaluate a limit with indeterminate form $0 \cdot \infty$
        \begin{eqnarray}
            \lim_{x \to 1^-}{\left(1 - x\right)\tan{\left(\frac{\pi x}{2}\right)}} &=& \lim_{x \to 1^-}{\frac{\left(1 - x\right)}{\cot{\left(\frac{\pi x}{2}\right)}}} \\
                                                                                   &=& \lim_{x \to 1^-}{\frac{-1}{-\frac{\pi}{2}\csc^2{\left(\frac{\pi x}{2}\right)}}} \\
                                                                                   &=& \lim_{x \to 1^-}{\frac{2}{\pi}\sin^2{\left(\frac{\pi x}{2}\right)}} \\
                                                                                   &=& \frac{2}{\pi}
        \end{eqnarray}
    \item Use L'Hôpital's Rule to evaluate a limit with exponential indeterminate form
        \begin{eqnarray}
            \lim_{x \to 0^+}{x^{\tan{x}}} &=& e^{\lim\limits_{x \to 0^+}{\frac{\ln{x}}{\frac{1}{\tan{x}}}}} \\
                                          &=& e^{\lim\limits_{x \to 0^+}{\frac{\ln{x}}{\cot{x}}}} \\
                                          &=& e^{\lim\limits_{x \to 0^+}{\frac{1}{-x\csc^2{x}}}} \\
                                          &=& e^{\lim\limits_{x \to 0^+}{\frac{-\sin^2{x}}{x}}} \\
                                          &=& e^{\lim\limits_{x \to 0^+}{\frac{-2\sin{x}\cos{x}}{1}}} \\
                                          &=& e^{\lim\limits_{x \to 0^+}{-2\sin{x}\cos{x}}} \\
                                          &=& e^0 \\
                                          &=& 1
        \end{eqnarray}
    \item Compare the growth rates of functions
        $$f(x) = x^2\ln{x}$$
        $$g(x) = x\ln^2{x}$$
        \begin{eqnarray}
            \lim_{x \to \infty}{\frac{f(x)}{g(x)}} &=& \lim_{x \to \infty}{\frac{x^2\ln{x}}{x\ln^2{x}}} \\
                                                   &=& \lim_{x \to \infty}{\frac{x}{\ln{x}}} \\
                                                   &=& \lim_{x \to \infty}{\frac{1}{\frac{1}{x}}} \\
                                                   &=& \lim_{x \to \infty}{x} \\
                                                   &=& \infty
        \end{eqnarray}
        Since $\lim\limits_{x \to \infty}{\frac{f(x)}{g(x)}} = \infty$, $f \gg g$
        \item Use knowledge of derivatives to find antiderivatives
            $$f(x) = -4\cos{x} - x$$
            $$F(x) = -4\sin{x} - \frac{1}{2}x^2$$
            $$F'(x) = -4\cos{x} - x$$
            $$\int{\left(-4\cos{x} - x\right)} dx = -4\sin{x} - \frac{1}{2}x^2 + C$$
        \item Determine indefinite integrals using antiderivative rules
            $$\int{\frac{3}{x^4} + 2 - 3x^2}dx = \int{3x^{-4} + 2 - 3x^2}dx = \frac{-1}{x^3} + 2x - x^3 + C$$
        \item Rewrite an indefinite integral to find an antiderivative
            \begin{eqnarray}
                \int{\frac{2 + 3\cos{y}}{\sin^2{y}}}dy &=& \int{2\csc^2{y} + 3\cot{y}\csc{y}}dy \\
                                                       &=& 2\int{\csc^2{y}}dy + 3\int{\cot{y}\csc{y}}dy \\
                                                       &=& -2\cot{y} - 3\csc{y} + C
            \end{eqnarray}
        \item Solve an initial value problem
            $$f'(u) = 4\cos{u} - 4\sin{u}$$
            $$f(\pi) = 0$$
            $$f(u) = 4\int{\cos{u}}\,du - 4\int{\sin{u}}\,du = 4\sin{u} + 4\cos{u} + C$$
            \begin{eqnarray}
                4\sin{\pi} + 4\cos{\pi} + C &=& 0 \\
                0 - 4 + C &=& 0 \\
                C &=& 4
            \end{eqnarray}
            $$f(u) = 4\sin{u} + 4\cos{u} + 4$$
        \item Application of differential equations to linear motion
            $$a(t) = 2 + 3\sin{t}$$
            $$v(0) = 1$$
            $$s(0) = 10$$
            \begin{eqnarray}
                v(t) &=& \int{2 + 3\sin{t}}\,dt \\
                     &=& 2\int{t^0}\,dt + 3\int{\sin{t}}\,dt \\
                     &=& 2t - 3\cos{t} + C \\
                2(0) - 3\cos{0} + C &=& 1 \\
                - 3 + C &=& 1 \\
                C &=& 4 \\
                v(t) &=& 2t - 3\cos{t} + 4 \\
                s(t) &=& \int{2t - 3\cos{t} + 4}\,dt \\
                     &=& 2\int{t}\,dt - 3\int{\cos{t}}\,dt + 4\int{t^0}\,dt \\
                     &=& 2\frac{t^2}{2} - 3\sin{t} + 4t + C \\
                     &=& t^2 - 3\sin{t} + 4t + C \\
                0^2 - 3\sin{0} + 4(0) + C &=& 10 \\
                C &=& 10 \\
                s(t) &=& t^2 - 3\sin{t} + 4t + 10
            \end{eqnarray}
\end{enumerate}

\section*{Related Exercises}
\begin{enumerate}
    \item (Section 4.7, Exercise 17)
        \begin{eqnarray}
            \lim_{x \to 2}{\frac{x^2 - 2x}{x^2 - 6x + 8}} &=& \lim_{x \to 2}{\frac{2x - 2}{2x - 6}} \\
                                                          &=& \frac{2(2) - 2}{2(2) - 6} \\
                                                          &=& \frac{4 - 2}{4 - 6} \\
                                                          &=& \frac{2}{-2} \\
                                                          &=& -1
        \end{eqnarray}
    \item (Section 4.7, Exercise 18)
        \begin{eqnarray}
            \lim_{x \to -1}{\frac{x^4 + x^3 + 2x + 2}{x + 1}} &=& \lim_{x \to -1}{\frac{4x^3 + 3x^2 + 2}{1}} \\
                                                              &=& \lim_{x \to -1}{4x^3 + 3x^2 + 2} \\
                                                              &=& 4(-1)^3 + 3(-1)^2 + 2 \\
                                                              &=& -4 + 3 + 2 \\
                                                              &=& 1
        \end{eqnarray}
    \item (Section 4.7, Exercise 36)
        \begin{eqnarray}
            \lim_{x \to 0}{\frac{e^x - x - 1}{5x^2}} &=& \lim_{x \to 0}{\frac{e^x - 1}{10x}} \\
                                                     &=& \lim_{x \to 0}{\frac{e^x}{10}} \\
                                                     &=& \frac{e^0}{10} \\
                                                     &=& \frac{1}{10}
        \end{eqnarray}
    \item (Section 4.7, Exercise 39)
        \begin{eqnarray}
            \lim_{x \to 0}{\frac{e^x - \sin{x} - 1}{x^4 + 8x^3 + 12x^2}} &=& \lim_{x \to 0}{\frac{e^x - \cos{x}}{4x^3 + 24x^2 + 24x}} \\
                                                                         &=& \lim_{x \to 0}{\frac{e^x + \sin{x}}{12x^2 + 48x + 24}} \\
                                                                         &=& \frac{e^0 + \sin{0}}{12(0)^2 + 48(0) + 24} \\
                                                                         &=& \frac{1 + 0}{24} \\
                                                                         &=& \frac{1}{24}
        \end{eqnarray}
    \item (Section 4.7, Exercise 38)
        \begin{eqnarray}
            \lim_{x \to \infty}{\frac{e^{3x}}{3e^{3x} + 5}} &=& \lim_{x \to \infty}{\frac{3e^{3x}}{9e^{3x}}} \\
                                                            &=& \lim_{x \to \infty}{\frac{1}{3}\cdot\frac{e^{3x}}{e^{3x}}} \\
                                                            &=& \lim_{x \to \infty}{\frac{1}{3}} \\
                                                            &=& \frac{1}{3}
        \end{eqnarray}
    \item (Section 4.7, Exercise 51)
        \begin{eqnarray}
            \lim_{x \to \infty}{\frac{x^2 - \ln{\frac{2}{x}}}{3x^2 + 2x}} &=& \lim_{x \to \infty}{\frac{2x + \frac{1}{x}}{6x + 2}} \\
                                                                          &=& \lim_{x \to \infty}{\frac{2 - \frac{1}{x^2}}{6}} \\
                                                                          &=& \frac{2 - 0}{6} \\
                                                                          &=& \frac{2}{6} \\
                                                                          &=& \frac{1}{3}
        \end{eqnarray}
    \item (Section 4.7, Exercise 53)
        \begin{eqnarray}
            \lim_{x \to 0}{x\csc{x}} &=& \lim_{x \to 0}{\frac{x}{\sin{x}}} \\
                                     &=& \lim_{x \to 0}{\frac{1}{\cos{x}}} \\
                                     &=& \frac{1}{\cos{0}} \\
                                     &=& \frac{1}{1} \\
                                     &=& 1
        \end{eqnarray}
    \item (Section 4.7, Exercise 63)
        \begin{eqnarray}
            \lim_{x \to \infty}{\left(x^2 - \sqrt{x^4 + 16x^2}\right)} &=& \lim_{x \to \infty}{\left(x^2 - \sqrt{x^4\left(1 + \frac{16}{x^2}\right)}\right)} \\
                                                                       &=& \lim_{x \to \infty}{\left(x^2 - x^2\sqrt{1 + \frac{16}{x^2}}\right)} \\
                                                                       &=& \lim_{x \to \infty}{x^2\left(1 - \sqrt{1 + \frac{16}{x^2}}\right)} \\
                                                                       &=& \lim_{x \to \infty}{\frac{1 - \sqrt{1 + \frac{16}{x^2}}}{\frac{1}{x^2}}} \\
                                                                       &=& \lim_{x \to \infty}{\frac{\frac{16}{x^3}}{\frac{-2}{x^3}\sqrt{1 + \frac{16}{x^2}}}} \\
                                                                       &=& \lim_{x \to \infty}{\frac{\frac{16}{x^3}\cdot\frac{x^3}{-2}}{\sqrt{1 + \frac{16}{x^2}}}} \\
                                                                       &=& \lim_{x \to \infty}{\frac{\frac{16}{-2}\cdot\frac{x^3}{x^3}}{\sqrt{1 + \frac{16}{x^2}}}} \\
                                                                       &=& \lim_{x \to \infty}{\frac{-8}{\sqrt{1 + \frac{16}{x^2}}}} \\
                                                                       &=& \frac{-8}{\sqrt{1 + 0}} \\
                                                                       &=& \frac{-8}{1} \\
                                                                       &=& -8
        \end{eqnarray}
    \item (Section 4.7, Exercise 64)
        \begin{eqnarray}
            \lim_{x \to \infty}{\left(x - \sqrt{x^2 + 4x}\right)} &=& \lim_{x \to \infty}{\left(x - \sqrt{x^2\left(1 + \frac{4}{x}\right)}\right)} \\
                                                                  &=& \lim_{x \to \infty}{\left(x - x\sqrt{1 + \frac{4}{x}}\right)} \\
                                                                  &=& \lim_{x \to \infty}{x\left(1 - \sqrt{1 + \frac{4}{x}}\right)} \\
                                                                  &=& \lim_{x \to \infty}{\frac{1 - \sqrt{1 + \frac{4}{x}}}{\frac{1}{x}}} \\
                                                                  &=& \lim_{x \to \infty}{\frac{\frac{2}{x^2}}{\frac{-1}{x^2}\sqrt{1 + \frac{4}{x}}}} \\
                                                                  &=& \lim_{x \to \infty}{\frac{\frac{2}{x^2}\cdot\frac{x^2}{-1}}{\sqrt{1 + \frac{4}{x}}}} \\
                                                                  &=& \lim_{x \to \infty}{\frac{\frac{2}{-1}\cdot\frac{x^2}{x^2}}{\sqrt{1 + \frac{4}{x}}}} \\
                                                                  &=& \lim_{x \to \infty}{\frac{-2}{\sqrt{1 + \frac{4}{x}}}} \\
                                                                  &=& \frac{-2}{\sqrt{1 + 0}} \\
                                                                  &=& -2
        \end{eqnarray}
    \item (Section 4.7, Exercise 75)
        \begin{eqnarray}
            \lim_{x \to 0^+}{x^{2x}} &=& e^{\lim\limits_{x \to 0^+}{\frac{\ln{x}}{\frac{1}{2x}}}} \\
                                     &=& e^{\lim\limits_{x \to 0^+}{\frac{\frac{1}{x}}{\frac{-1}{2x^2}}}} \\
                                     &=& e^{\lim\limits_{x \to 0^+}{\frac{1}{x}\cdot\frac{2x^2}{-1}}} \\
                                     &=& e^{\lim\limits_{x \to 0^+}{-\frac{2x^2}{x}}} \\
                                     &=& e^{\lim\limits_{x \to 0^+}{-2x}} \\
                                     &=& e^{-2(0)} \\
                                     &=& e^{0} \\
                                     &=& 1
        \end{eqnarray}
    \item (Section 4.7, Exercise 76)
        \begin{eqnarray}
            \lim_{x \to 0}{\left(1 + 4x\right)^{\frac{3}{x}}} &=& e^{\lim\limits_{x \to 0^+}{\frac{\ln{\left(1 + 4x\right)}}{\frac{1}{\frac{3}{x}}}}} \\
                                                              &=& e^{\lim\limits_{x \to 0^+}{\frac{\ln{\left(1 + 4x\right)}}{\frac{x}{3}}}} \\
                                                              &=& e^{\lim\limits_{x \to 0^+}{\frac{\frac{4}{\left(1 + 4x\right)}}{\frac{1}{3}}}} \\
                                                              &=& e^{\lim\limits_{x \to 0^+}{\frac{4}{\left(1 + 4x\right)}\cdot\frac{3}{1}}} \\
                                                              &=& e^{\lim\limits_{x \to 0^+}{\frac{12}{\left(1 + 4x\right)}}} \\
                                                              &=& e^{\frac{12}{1}} \\
                                                              &=& e^{12}
        \end{eqnarray}
    \item (Section 4.7, Exercise 96)
        \begin{eqnarray}
            f(x) &=& x^2\ln{x} \\
            g(x) &=& \ln^2{x}
        \end{eqnarray}
        \begin{eqnarray}
            \lim_{x \to \infty}{\frac{f(x)}{g(x)}} &=& \lim_{x \to \infty}{\frac{x^2\ln{x}}{\ln^2{x}}} \\
                                                   &=& \lim_{x \to \infty}{\frac{x^2}{\ln{x}}} \\
                                                   &=& \lim_{x \to \infty}{\frac{2x}{\frac{1}{x}}} \\
                                                   &=& \lim_{x \to \infty}{2x^2} \\
                                                   &=& \infty
        \end{eqnarray}
        $$f \gg g$$
    \item (Section 4.7, Exercise 100)
        \begin{eqnarray}
            f(x) &=& x^2\ln{x} \\
            g(x) &=& x^3
        \end{eqnarray}
        \begin{eqnarray}
            \lim_{x \to \infty}{\frac{f(x)}{g(x)}} &=& \lim_{x \to \infty}{\frac{x^2\ln{x}}{x^3}} \\
                                                   &=& \lim_{x \to \infty}{\frac{\ln{x}}{x}} \\
                                                   &=& \lim_{x \to \infty}{\frac{\frac{1}{x}}{1}} \\
                                                   &=& \lim_{x \to \infty}{\frac{1}{x}} \\
                                                   &=& \frac{1}{\infty} \neq \infty
        \end{eqnarray}
        $$g \gg f$$
    \item (Section 4.7, Exercise 95)
        \begin{eqnarray}
            f(x) &=& x^{10} \\
            g(x) &=& e^{0.01x}
        \end{eqnarray}
        \begin{eqnarray}
            \lim_{x \to \infty}{\frac{f(x)}{g(x)}} &=& \lim_{x \to \infty}{\frac{x^{10}}{e^{0.01x}}} \\
                                                   &=& \lim_{x \to \infty}{\frac{10x^9}{0.01e^{0.01x}}} \\
                                                   &=& \lim_{x \to \infty}{\frac{90x^8}{0.01^2e^{0.01x}}} \\
                                                   &=& \lim_{x \to \infty}{\frac{7200x^7}{0.01^3e^{0.01x}}} \\
                                                   &=& \lim_{x \to \infty}{\frac{50400x^6}{0.01^4e^{0.01x}}} \\
                                                   &=& \lim_{x \to \infty}{\frac{302400x^5}{0.01^5e^{0.01x}}} \\
                                                   &=& \lim_{x \to \infty}{\frac{1512000x^4}{0.01^6e^{0.01x}}} \\
                                                   &=& \lim_{x \to \infty}{\frac{6048000x^3}{0.01^7e^{0.01x}}} \\
                                                   &=& \lim_{x \to \infty}{\frac{18144000x^2}{0.01^8e^{0.01x}}} \\
                                                   &=& \lim_{x \to \infty}{\frac{36288000x}{0.01^9e^{0.01x}}} \\
                                                   &=& \lim_{x \to \infty}{\frac{36288000}{0.01^{10}e^{0.01x}}} \\
                                                   &=& \frac{36288000}{\infty} \neq \infty
        \end{eqnarray}
        $$g \gg f$$
    \item (Section 4.7, Exercise 101)
        \begin{eqnarray}
            f(x) &=& x^{20} \\
            g(x) &=& 1.00001^x
        \end{eqnarray}
        \begin{eqnarray}
            \lim_{x \to \infty}{\frac{f(x)}{g(x)}} &=& \lim_{x \to \infty}{\frac{x^{20}}{1.00001^x}} \\
                                                   &=& \frac{2432902008176640000}{\infty} \neq \infty
        \end{eqnarray}
        $$g \gg f$$
\end{enumerate}

\blfootnote{A copy of my notes (in \LaTeX) are available on my \href{https://github.com/onlinechronically/MATH-211}{GitHub}}
\end{document}
