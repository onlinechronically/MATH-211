\documentclass{article}
\usepackage{graphicx}
\usepackage{amsthm}
\usepackage{amsmath}
\usepackage{amssymb}
\usepackage{geometry}
\usepackage{tikz}
\usepackage[hidelinks]{hyperref}
\usetikzlibrary{arrows}

\geometry{a4paper, total={170mm,257mm}, left=20mm, top=20mm}
\AtBeginEnvironment{align}{\setcounter{equation}{0}} 
\AtBeginEnvironment{eqnarray}{\setcounter{equation}{0}} 

\newcommand\blfootnote[1]{
    \begingroup
    \renewcommand\thefootnote{}\footnote{#1}
    \addtocounter{footnote}{-1}
    \endgroup
}

\title{Module 6 Notes (MATH-211)}
\author{Lillie Donato}
\date{15 July 2024}

\begin{document}

\maketitle

\section*{General Notes (and Definitions)}
\begin{itemize}
    \item L'Hôpital's Rule \\
    \textbf{Indeterminate Form}: An expression involving two components where the limit cannot be determined by evaluating the limits of the individual components. \\
    \textbf{L'Hôpital's Rule}: Suppose $f$ and $g$ are differentiable functions on an open interval $I$ containing the point $x = a$, with $g'(x) \neq 0$ on $I$ when $x \neq a$. \\
    If $\lim\limits_{x \to a}{\frac{f(x)}{g(x)}}$ has any of the indeterminate forms: $\frac{0}{0}$, $\frac{\infty}{\infty}$, $-\frac{\infty}{\infty}$, then
    $$\lim_{x \to a}{\frac{f(x)}{g(x)}} = \lim_{x \to a}{\frac{f'(x)}{g'(x)}}$$
    provided that one of the following is the case:
    $$\lim_{x \to a}{\frac{f'(x)}{g'(x)}} \in \mathbb{R}$$
    $$\lim_{x \to a}{\frac{f'(x)}{g'(x)}} = \infty$$
    $$\lim_{x \to a}{\frac{f'(x)}{g'(x)}} = -\infty$$
    L'Hôpital's Rule is still valid if $x \to a$ is replaced by any of $x \to a^+$, $x \to a^-$, $x \to \infty$, or $x \to -\infty$. In the last two of these cases, there must be a greatest $x$-value beyond which both $f$ and $g$ are differentiable at every point. \\
    \textbf{Exponential Indeterminate forms}: $1^{\infty}$, $0^0$, $\infty^0$ \\
    \textbf{Method for evaluating limits of indeterminate forms $1^{\infty}$, $0^0$, $\infty^0$}: \\
    Assume that $L = \lim\limits_{x \to a}{f(x)^{g(x)}}$ has one of these indeterminate forms.
    \begin{enumerate}
        \item Use the fact that the natural logarithm and natural exponential functions are inverses to write
        $$L = \lim_{x \to a}{e^{\ln{\left(f(x)^{g(x)}\right)}}}$$
        \item Use the power property of logarithm arguments to write
        $$L = \lim_{x \to a}{e^{g(x)\ln{\left(f(x)\right)}}}$$
        \item Use continuity of the exponential function to write
        $$L = e^{\lim\limits_{x \to a}{g(x)\ln{\left(f(x)\right)}}}$$
        \item Rewrite multiplication as division by the reciprocal:
        $$L = e^{\lim\limits_{x \to a}{\left(\frac{\ln{\left(f(x)\right)}}{\frac{1}{g(x)}}\right)}}$$
        \item Use L'Hôpital's Rule to evaluate this limit expression
    \end{enumerate}
    \textbf{Growth Rates}: Suppose $f$ and $g$ are functions with $\lim\limits_{x \to \infty}{f(x)} = \infty$ and $\lim\limits_{x \to \infty}{g(x)} = \infty$ \\
    \begin{enumerate}
        \item If one of the following are true, \textbf{$f$ grows faster than $g$}, and we use the notation $f \gg g$
        \begin{eqnarray}
            \lim_{x \to \infty}{\frac{g(x)}{f(x)}} &=& 0 \\
            \lim_{x \to \infty}{\frac{f(x)}{g(x)}} &=& \infty
        \end{eqnarray}
        \item \textbf{$f$ and $g$ have comparable growth rates}, if there is some non-zero finite number $M$ such that
        $$\lim_{x \to \infty}{\frac{f(x)}{g(x)}} = M$$
    \end{enumerate}
    \textbf{Ranked Growth Rates as $x \to \infty$} \\
    For any base $b > 1$, and for any positive numbers $p$, $q$, $r$, and $s$
    $$\ln^q{x} \ll x^p \ll x^p \ln^r{x} \ll x^{p + s} \ll b^x \ll x^x$$
    \item Antiderivatives \\
    \textbf{Antiderivative}: A function $F$ is an antiderivative of another function $f$ on an interval $I$ if for all $x$ in $I$:
    $$F'(x) = f(x)$$
    \textbf{Family of Antiderivatives}: Let $F(x)$ be any antiderivative of $f(x)$ on an interval $I$. Then all antiderivatives of $f$ on $I$ have the form $F(x) + C$, where $C$ is an arbitrary constant. \\
    \textbf{Differential Equations}: Any equation involving an unknown function and its derivatives
    \begin{itemize}
        \item Infinite family of solutions
        \item No two solutions from the family pass through the same point
        \item Given an initial condition $f(a) = b$, we can identify the particular family member that solves the given problem by solving for $C$
    \end{itemize}
    \item Approximating Areas Under Curves
    \begin{itemize}
        \item If we know the velocity function of a moving object, what can we learn about its position function?
        \item Given an object with velocity function $v(t)$, the displacement of the moving object over the interval $[a,b]$ is the area between the velocity curve and the $t$-axis from $t = a$ to $t = b$.
        \item Because objects do not necessarily move at a constant velocity, we can extend this idea to positive velocities that change over an interval of time.
        \item The strategy is to divide the time interval into many subintervals, approximate the velocity on each subinterval with a constant velocity, calculate the individual displacements and sum the results.
    \end{itemize}
    \textbf{Riemann Sums}
    \begin{itemize}
        \item Suppose $f(x)$ is continuous and non-negative on $[a,b]$.
        \item Goal is to approximate the area of the region $R$ bounded by the graph of $f(x)$ and the $x$-axis from $x = a$ to $x = b$.
        \item Divide $[a,b]$ into $n$ subintervals $[x_0,x_1],[x_1,x_2],...,[x_{n-1},x_n]$ where $a = x_0, b = x_n$.
        \item The length of each subinterval is $\Delta x = \frac{b - a}{n}$
        \item \textbf{Regular Partition}: Suppose $[a,b]$ is a closed interval containing $n$ subintervals
            $$[x_0,x_1],[x_1,x_2],...,[x_{n-1},x_n]$$
        of equal length $\Delta x = \frac{b - a}{n}$, with $a = x_0$ and $b = x_n$. The endpoints $x_0, x_1, x_2,...,x_{n - 1},x_n$ of the subintervals are called \textbf{grid points}, and they create a \textbf{regular partition} of the interval $[a,b]$. In general the $k$th grid point is
            $$x_k = a + k\Delta x\text{, for } k = 0,1,2,...,n$$
        \item In the $k$th subinterval $[x_{k - 1}, x_k]$, choose any point $x_k^*$ and build a rectangle whose height is $f(x_k^*)$.
        \item The area of the rectangle of the $k$th subinterval is
            $$\text{height} \cdot \text{base} = f(x_k^*)\Delta x\text{, where } k = 1,2,...,n$$
        \item Summing the areas of these rectangles, we obtain an approximation to the area of $R$, which is called a \textbf{Riemann sum}:
            $$f(x_1^*)\Delta x + f(x_2^*)\Delta x + ... + f(x_n^*)\Delta x$$
        \item Three notable Riemann sums are the left, right, and midpoint Riemann sums.
    \end{itemize}
    \textbf{Riemann Sum}: Suppose $f$ is defined on a closed interval $[a,b]$, which is divided into $n$ subintervals of equal length $\Delta x$. If $x_k^*$ is any point in the $k$th subinterval $[x_{k - 1}, x_k]$, for $k = 1,2,...,n$, then
        $$f(x_1^*)\Delta x + f(x_2^*)\Delta x + ... + f(x_n^*)\Delta x$$
    is called a \textbf{Riemann sum} for $f$ on $[a,b]$. This sum is called
        \begin{itemize}
            \item a \textbf{left Riemann sum} if $x_k^*$ is the left endpoint of $[x_{k - 1}, x_k]$
            \item a \textbf{right Riemann sum} if $x_k^*$ is the right endpoint of $[x_{k - 1}, x_k]$
            \item a \textbf{midpoint Riemann sum} if $x_k^*$ is the midpoint of $[x_{k - 1}, x_k]$
        \end{itemize}
    \textbf{Summation notation ($\Sigma$)}:
        \begin{itemize}
            \item Working with Riemann sums is cumbersome when $n$ is large
            \item We introduce sigma (summation) notation as a shorthand:
                $$1 + 2 + ... + 49 + 50 = \sum_{k = 1}^{50}{k}$$
            \item The symbol $\Sigma$ (sigma) stands for sum
            \item $k$ is the index, and takes on all integer values from $k = 1$ to $k = 50$
            \item The expression immediately following $\Sigma$, the summand, is evaluated for each $k$, and the resulting values are summed
            \item The index is a dummy variable, and it does not matter which symbol is chosen for the index:
                $$\sum_{k = 1}^{99}{k} = \sum_{n = 1}^{99}{n} = \sum_{p - 1}^{99}{p}$$
            \item Two Properties of Sums and Sigma Notation
                \begin{enumerate}
                    \item Constant Multiple Rule:
                        $$\sum_{k = 1}^{n}{ca_k} = c\sum_{k = 1}^{n}{a_k}$$
                    \item Addition Rule:
                        $$\sum_{k = 1}^{n}{\left(a_k + b_k\right)} = \sum_{k = 1}^{n}{a_k} + \sum_{k = 1}^{n}{b_k}$$
                \end{enumerate}
            \item \textbf{Theorem}: Sums of Power of Integers \\
                Let $n \in \mathbb{Z}$ such that $n > 0$ and $c \in \mathbb{R}$
                \begin{eqnarray}
                    \sum_{k = 1}^{n}{c} &=& cn \\
                    \sum_{k = 1}^{n}{k} &=& \frac{n\left(n + 1\right)}{2} \\
                    \sum_{k = 1}^{n}{k^2} &=& \frac{n\left(n + 1\right)\left(2n + 1\right)}{6} \\
                    \sum_{k = 1}^{n}{k^3} &=& \frac{n^2\left(n + 1\right)^2}{4}
                \end{eqnarray}
        \end{itemize}
    \textbf{Left, Right, and Midpoint Riemann Sums in Sigma Notation}: \\
        Suppose $f$ is defined on a closed interval $[a,b]$, which is divided into subintervals of equal length $\Delta{x}$. If $x_k^*$ is a point in the $k$th subinterval $[x_{k - 1}, x_k]$, for $k = 1,2,...,n$, then the \textbf{Riemann sum} for $f$ on $[a,b]$ is $$\sum_{k = 1}^{n}{f(x_k^*)\Delta{x}}$$
        Three cases arise in practice
    \begin{itemize}
        \item $\sum\limits_{k = 1}^{n}{f(x_k^*)\Delta{x}}$ is a \textbf{left Riemann sum} if $x_k^* = a + (k - 1)\Delta{x}$
        \item $\sum\limits_{k = 1}^{n}{f(x_k^*)\Delta{x}}$ is a \textbf{right Riemann sum} if $x_k^* = a + k\Delta{x}$
        \item $\sum\limits_{k = 1}^{n}{f(x_k^*)\Delta{x}}$ is a \textbf{midpoint Riemann sum} if $x_k^* = a + (k - \frac{1}{2})\Delta{x}$
    \end{itemize}
\end{itemize}

\section*{Antiderivative Rules}
\begin{itemize}
    \item Power Rule \\
        If $p \neq -1$ and $C$ is an arbitrary constant:
        $$\int{x^p} dx = \frac{x^{p + 1}}{p + 1} + C$$
    \item Integral of $x^{-1}$
        $$\int{x^{-1}}dx = \int{\frac{1}{x}}dx = \ln{|x|} + C$$
    \item Constant Multiple and Sum Rules \\
        If $c \in \mathbb{R}$:
        $$\int{cf(x)}dx = c\int{f(x)}dx$$
        $$\int{\left(f(x) + g(x)\right)}dx = \int{f(x)}dx + \int{g(x)}dx$$
    \item Integral of $e^x$
        $$\int{e^x}dx = e^x + C$$
    \item Integral of $\frac{1}{x}$
        $$\int{\frac{1}{x}}\,dx = \ln{|x|} + C$$
\end{itemize}

\section*{Trigonometric (and inverse) Integrals}
\begin{eqnarray}
    \int{\cos{\left(x\right)}}dx &=& \sin{x} + C \\
    \int{\sin{\left(x\right)}}dx &=& -\cos{x} + C \\
    \int{\sec^2{\left(x\right)}}dx &=& \tan{x} + C \\
    \int{\csc^2{\left(x\right)}}dx &=& -\cot{x} + C \\
    \int{\sec{\left(x\right)}\tan{\left(x\right)}}dx &=& \sec{x} + C \\
    \int{\csc{\left(x\right)}\cot{\left(x\right)}}dx &=& -\csc{x} + C \\
    \int{\frac{1}{\sqrt{1 - x^2}}}dx &=& \sin^{-1}{x} + C \\
    \int{\frac{1}{1 + x^2}}dx &=& \tan^{-1}{x} + C \\
    \int{\frac{1}{x\sqrt{x^2 - 1}}}dx &=& \sec^{-1}{|x|} + C
\end{eqnarray}

\section*{Examples}
\begin{enumerate}
    \item Use L'Hôpital's Rule to evaluate a limit with indeterminate form $\frac{0}{0}$
        \begin{eqnarray}
            \lim_{x \to 0}{\frac{e^x - x - 1}{5x^2}} &=& \lim_{x \to 0}{\frac{e^x - 1}{10x}} \\
                                                     &=& \lim_{x \to 0}{\frac{e^x}{10}} \\
                                                     &=& \frac{e^0}{10} \\
                                                     &=& \frac{1}{10}
        \end{eqnarray}
    \item Use L'Hôpital's Rule to evaluate a limit with indeterminate form $\frac{\infty}{\infty}$
        \begin{eqnarray}
            \lim_{x \to 0^+}{\frac{1 - \ln{x}}{1 + \ln{x}}} &=& \lim_{x \to 0^+}{\frac{-\frac{1}{x}}{\frac{1}{x}}} \\
                                                            &=& \lim_{x \to 0^+}{\frac{-\frac{1}{x}}{\frac{1}{x}}} \\
                                                            &=& \frac{-1}{1} \\
                                                            &=& -1
        \end{eqnarray}
    \item Use L'Hôpital's Rule to evaluate a limit with indeterminate form $0 \cdot \infty$
        \begin{eqnarray}
            \lim_{x \to 1^-}{\left(1 - x\right)\tan{\left(\frac{\pi x}{2}\right)}} &=& \lim_{x \to 1^-}{\frac{\left(1 - x\right)}{\cot{\left(\frac{\pi x}{2}\right)}}} \\
                                                                                   &=& \lim_{x \to 1^-}{\frac{-1}{-\frac{\pi}{2}\csc^2{\left(\frac{\pi x}{2}\right)}}} \\
                                                                                   &=& \lim_{x \to 1^-}{\frac{2}{\pi}\sin^2{\left(\frac{\pi x}{2}\right)}} \\
                                                                                   &=& \frac{2}{\pi}
        \end{eqnarray}
    \item Use L'Hôpital's Rule to evaluate a limit with exponential indeterminate form
        \begin{eqnarray}
            \lim_{x \to 0^+}{x^{\tan{x}}} &=& e^{\lim\limits_{x \to 0^+}{\frac{\ln{x}}{\frac{1}{\tan{x}}}}} \\
                                          &=& e^{\lim\limits_{x \to 0^+}{\frac{\ln{x}}{\cot{x}}}} \\
                                          &=& e^{\lim\limits_{x \to 0^+}{\frac{1}{-x\csc^2{x}}}} \\
                                          &=& e^{\lim\limits_{x \to 0^+}{\frac{-\sin^2{x}}{x}}} \\
                                          &=& e^{\lim\limits_{x \to 0^+}{\frac{-2\sin{x}\cos{x}}{1}}} \\
                                          &=& e^{\lim\limits_{x \to 0^+}{-2\sin{x}\cos{x}}} \\
                                          &=& e^0 \\
                                          &=& 1
        \end{eqnarray}
    \item Compare the growth rates of functions
        $$f(x) = x^2\ln{x}$$
        $$g(x) = x\ln^2{x}$$
        \begin{eqnarray}
            \lim_{x \to \infty}{\frac{f(x)}{g(x)}} &=& \lim_{x \to \infty}{\frac{x^2\ln{x}}{x\ln^2{x}}} \\
                                                   &=& \lim_{x \to \infty}{\frac{x}{\ln{x}}} \\
                                                   &=& \lim_{x \to \infty}{\frac{1}{\frac{1}{x}}} \\
                                                   &=& \lim_{x \to \infty}{x} \\
                                                   &=& \infty
        \end{eqnarray}
        Since $\lim\limits_{x \to \infty}{\frac{f(x)}{g(x)}} = \infty$, $f \gg g$
        \item Use knowledge of derivatives to find antiderivatives
            $$f(x) = -4\cos{x} - x$$
            $$F(x) = -4\sin{x} - \frac{1}{2}x^2$$
            $$F'(x) = -4\cos{x} - x$$
            $$\int{\left(-4\cos{x} - x\right)} dx = -4\sin{x} - \frac{1}{2}x^2 + C$$
        \item Determine indefinite integrals using antiderivative rules
            $$\int{\frac{3}{x^4} + 2 - 3x^2}dx = \int{3x^{-4} + 2 - 3x^2}dx = \frac{-1}{x^3} + 2x - x^3 + C$$
        \item Rewrite an indefinite integral to find an antiderivative
            \begin{eqnarray}
                \int{\frac{2 + 3\cos{y}}{\sin^2{y}}}dy &=& \int{2\csc^2{y} + 3\cot{y}\csc{y}}dy \\
                                                       &=& 2\int{\csc^2{y}}dy + 3\int{\cot{y}\csc{y}}dy \\
                                                       &=& -2\cot{y} - 3\csc{y} + C
            \end{eqnarray}
        \item Solve an initial value problem
            $$f'(u) = 4\cos{u} - 4\sin{u}$$
            $$f(\pi) = 0$$
            $$f(u) = 4\int{\cos{u}}\,du - 4\int{\sin{u}}\,du = 4\sin{u} + 4\cos{u} + C$$
            \begin{eqnarray}
                4\sin{\pi} + 4\cos{\pi} + C &=& 0 \\
                0 - 4 + C &=& 0 \\
                C &=& 4
            \end{eqnarray}
            $$f(u) = 4\sin{u} + 4\cos{u} + 4$$
        \item Application of differential equations to linear motion
            $$a(t) = 2 + 3\sin{t}$$
            $$v(0) = 1$$
            $$s(0) = 10$$
            \begin{eqnarray}
                v(t) &=& \int{2 + 3\sin{t}}\,dt \\
                     &=& 2\int{t^0}\,dt + 3\int{\sin{t}}\,dt \\
                     &=& 2t - 3\cos{t} + C \\
                2(0) - 3\cos{0} + C &=& 1 \\
                - 3 + C &=& 1 \\
                C &=& 4 \\
                v(t) &=& 2t - 3\cos{t} + 4 \\
                s(t) &=& \int{2t - 3\cos{t} + 4}\,dt \\
                     &=& 2\int{t}\,dt - 3\int{\cos{t}}\,dt + 4\int{t^0}\,dt \\
                     &=& 2\frac{t^2}{2} - 3\sin{t} + 4t + C \\
                     &=& t^2 - 3\sin{t} + 4t + C \\
                0^2 - 3\sin{0} + 4(0) + C &=& 10 \\
                C &=& 10 \\
                s(t) &=& t^2 - 3\sin{t} + 4t + 10
            \end{eqnarray}
        \item Approximating displacement
            $$v = \sqrt{10t}$$
            $$1 \leq t \leq 7$$
            \begin{enumerate}
                \item
                $$n = 3$$
                $$[1,3],[3,5],[5,7]$$
                $$2,4,6$$
                $$d \approx 2v(2) + 2v(4) + 2v(6) = 2\sqrt{20} + 2\sqrt{40} + 2\sqrt{60} \approx 37.085$$
                \item
                $$n = 6$$
                $$1.5,2.5,3.5,4.5,5.5,6.5$$
                \begin{eqnarray}
                    d &\approx& v(1.5) + v(2.5) + v(3.5) + v(4.5) + v(5.5) + v(6.5) \\
                      &=& \sqrt{15} + \sqrt{25} + \sqrt{35} + \sqrt{45} + \sqrt{55} + \sqrt{65} \\
                      &\approx& 36.976
                \end{eqnarray}
            \end{enumerate}
        \item Left and right Riemann sums (1)
            $$f(x) = x + 1$$
            $$[1,6]$$
            $$n = 5$$
            \begin{enumerate}
                \item
                \begin{eqnarray}
                    \left(1 + 1\right) + \left(2 + 1\right) + \left(3 + 1\right) + \left(4 + 1\right) + \left(5 + 1\right) &=& 2 + 3 + 4 + 5 + 6 \\
                                                                                                                           &=& 20
                \end{eqnarray}
                \item
                \begin{eqnarray}
                    \left(2 + 1\right) + \left(3 + 1\right) + \left(4 + 1\right) + \left(5 + 1\right) + \left(6 + 1\right) &=& 3 + 4 + 5 + 6 + 7 \\
                                                                                                                           &=& 25
                \end{eqnarray}
            \end{enumerate}
        \item Left and right Riemann sums (2)
            $$f(x) = 9 - x$$
            $$[3,8]$$
            $$n = 5$$
            $$\Delta{x} = \frac{8 - 3}{5} = \frac{5}{5} = 1$$
            \begin{enumerate}
                \item Grid Points: \\
                    \begin{eqnarray}
                        x_0 &=& 3 \\
                        x_1 &=& 4 \\
                        x_2 &=& 5 \\
                        x_3 &=& 6 \\
                        x_4 &=& 7 \\
                        x_5 &=& 8
                    \end{eqnarray}
                \item Riemann Sums
                    \begin{eqnarray}
                        \left(9 - 3\right) + \left(9 - 4\right) + \left(9 - 5\right) + \left(9 - 6\right) + \left(9 - 7\right) &=& 6 + 5 + 4 + 3 + 2 \\
                                                                                                                               &=& 20 \text{ (Overestimation)}
                    \end{eqnarray}
                    \begin{eqnarray}
                        \left(9 - 4\right) + \left(9 - 5\right) + \left(9 - 6\right) + \left(9 - 7\right) + \left(9 - 8\right) &=& 5 + 4 + 3 + 2 + 1 \\
                                                                                                                               &=& 15 \text{ (Underestimation)}
                    \end{eqnarray}
            \end{enumerate}
        \item Midpoint Riemann sum
            $$f(x) = 100 - x^2$$
            $$[0, 10]$$
            $$n = 5$$
            $$\Delta{x} = 2$$
            \begin{eqnarray}
                2f(1) + 2f(3) + 2f(5) + 2f(7) + 2f(9) &=& 2(99) + 2(91) + 2(75) + 2(51) + 2(19) \\
                                                      &=& 198 + 182 + 150 + 102 + 38 \\
                                                      &=& 670
            \end{eqnarray}
        \item Riemann sums from tables
            $$[0,2]$$
            $$n = 4$$
            $$\Delta{x} = \frac{1}{2}$$
            \begin{eqnarray}
                \frac{5}{2} + \frac{3}{2} + \frac{2}{2} + \frac{1}{2} &=& 5.5 \\
                \frac{3}{2} + \frac{2}{2} + \frac{1}{2} + \frac{1}{2} &=& 3.5
            \end{eqnarray}
        \item Computing Riemann sums for large values of $n$
            $$f(x) = x^2 + 1$$
            $$[-1, 1]$$
            $$n = 50$$
            \begin{eqnarray}
                \sum_{k = 1}^{50}{0.04f\left(-1 + 0.04\left(k - 1\right)\right)} &=& \sum_{k = 1}^{50}{0.04\left(\left(-1 + 0.04\left(k - 1\right)\right)^2 + 1\right)} \\
                                                                                 &\approx& 2.6672 \\
                                                                                 \nonumber \\
                \sum_{k = 1}^{50}{0.04f\left(-1 + 0.04k\right)} &=& \sum_{k = 1}^{50}{0.04\left(\left(-1 + 0.04k\right)^2 + 1\right)} \\
                                                                &\approx& 2.6672 \\
                                                                \nonumber \\
                \sum_{k - 1}^{50}{0.04f\left(-1 + 0.04\left(k - \frac{1}{2}\right)\right)} &=& \sum_{k - 1}^{50}{0.04\left(\left(-1 + 0.04\left(k - \frac{1}{2}\right)\right)^2 + 1\right)} \\
                                                                                           &\approx& 2.6664
            \end{eqnarray}
\end{enumerate}

\section*{Related Exercises}
\begin{enumerate}
    \item (Section 4.7, Exercise 17)
        \begin{eqnarray}
            \lim_{x \to 2}{\frac{x^2 - 2x}{x^2 - 6x + 8}} &=& \lim_{x \to 2}{\frac{2x - 2}{2x - 6}} \\
                                                          &=& \frac{2(2) - 2}{2(2) - 6} \\
                                                          &=& \frac{4 - 2}{4 - 6} \\
                                                          &=& \frac{2}{-2} \\
                                                          &=& -1
        \end{eqnarray}
    \item (Section 4.7, Exercise 18)
        \begin{eqnarray}
            \lim_{x \to -1}{\frac{x^4 + x^3 + 2x + 2}{x + 1}} &=& \lim_{x \to -1}{\frac{4x^3 + 3x^2 + 2}{1}} \\
                                                              &=& \lim_{x \to -1}{4x^3 + 3x^2 + 2} \\
                                                              &=& 4(-1)^3 + 3(-1)^2 + 2 \\
                                                              &=& -4 + 3 + 2 \\
                                                              &=& 1
        \end{eqnarray}
    \item (Section 4.7, Exercise 36)
        \begin{eqnarray}
            \lim_{x \to 0}{\frac{e^x - x - 1}{5x^2}} &=& \lim_{x \to 0}{\frac{e^x - 1}{10x}} \\
                                                     &=& \lim_{x \to 0}{\frac{e^x}{10}} \\
                                                     &=& \frac{e^0}{10} \\
                                                     &=& \frac{1}{10}
        \end{eqnarray}
    \item (Section 4.7, Exercise 39)
        \begin{eqnarray}
            \lim_{x \to 0}{\frac{e^x - \sin{x} - 1}{x^4 + 8x^3 + 12x^2}} &=& \lim_{x \to 0}{\frac{e^x - \cos{x}}{4x^3 + 24x^2 + 24x}} \\
                                                                         &=& \lim_{x \to 0}{\frac{e^x + \sin{x}}{12x^2 + 48x + 24}} \\
                                                                         &=& \frac{e^0 + \sin{0}}{12(0)^2 + 48(0) + 24} \\
                                                                         &=& \frac{1 + 0}{24} \\
                                                                         &=& \frac{1}{24}
        \end{eqnarray}
    \item (Section 4.7, Exercise 38)
        \begin{eqnarray}
            \lim_{x \to \infty}{\frac{e^{3x}}{3e^{3x} + 5}} &=& \lim_{x \to \infty}{\frac{3e^{3x}}{9e^{3x}}} \\
                                                            &=& \lim_{x \to \infty}{\frac{1}{3}\cdot\frac{e^{3x}}{e^{3x}}} \\
                                                            &=& \lim_{x \to \infty}{\frac{1}{3}} \\
                                                            &=& \frac{1}{3}
        \end{eqnarray}
    \item (Section 4.7, Exercise 51)
        \begin{eqnarray}
            \lim_{x \to \infty}{\frac{x^2 - \ln{\frac{2}{x}}}{3x^2 + 2x}} &=& \lim_{x \to \infty}{\frac{2x + \frac{1}{x}}{6x + 2}} \\
                                                                          &=& \lim_{x \to \infty}{\frac{2 - \frac{1}{x^2}}{6}} \\
                                                                          &=& \frac{2 - 0}{6} \\
                                                                          &=& \frac{2}{6} \\
                                                                          &=& \frac{1}{3}
        \end{eqnarray}
    \item (Section 4.7, Exercise 53)
        \begin{eqnarray}
            \lim_{x \to 0}{x\csc{x}} &=& \lim_{x \to 0}{\frac{x}{\sin{x}}} \\
                                     &=& \lim_{x \to 0}{\frac{1}{\cos{x}}} \\
                                     &=& \frac{1}{\cos{0}} \\
                                     &=& \frac{1}{1} \\
                                     &=& 1
        \end{eqnarray}
    \item (Section 4.7, Exercise 63)
        \begin{eqnarray}
            \lim_{x \to \infty}{\left(x^2 - \sqrt{x^4 + 16x^2}\right)} &=& \lim_{x \to \infty}{\left(x^2 - \sqrt{x^4\left(1 + \frac{16}{x^2}\right)}\right)} \\
                                                                       &=& \lim_{x \to \infty}{\left(x^2 - x^2\sqrt{1 + \frac{16}{x^2}}\right)} \\
                                                                       &=& \lim_{x \to \infty}{x^2\left(1 - \sqrt{1 + \frac{16}{x^2}}\right)} \\
                                                                       &=& \lim_{x \to \infty}{\frac{1 - \sqrt{1 + \frac{16}{x^2}}}{\frac{1}{x^2}}} \\
                                                                       &=& \lim_{x \to \infty}{\frac{\frac{16}{x^3}}{\frac{-2}{x^3}\sqrt{1 + \frac{16}{x^2}}}} \\
                                                                       &=& \lim_{x \to \infty}{\frac{\frac{16}{x^3}\cdot\frac{x^3}{-2}}{\sqrt{1 + \frac{16}{x^2}}}} \\
                                                                       &=& \lim_{x \to \infty}{\frac{\frac{16}{-2}\cdot\frac{x^3}{x^3}}{\sqrt{1 + \frac{16}{x^2}}}} \\
                                                                       &=& \lim_{x \to \infty}{\frac{-8}{\sqrt{1 + \frac{16}{x^2}}}} \\
                                                                       &=& \frac{-8}{\sqrt{1 + 0}} \\
                                                                       &=& \frac{-8}{1} \\
                                                                       &=& -8
        \end{eqnarray}
    \item (Section 4.7, Exercise 64)
        \begin{eqnarray}
            \lim_{x \to \infty}{\left(x - \sqrt{x^2 + 4x}\right)} &=& \lim_{x \to \infty}{\left(x - \sqrt{x^2\left(1 + \frac{4}{x}\right)}\right)} \\
                                                                  &=& \lim_{x \to \infty}{\left(x - x\sqrt{1 + \frac{4}{x}}\right)} \\
                                                                  &=& \lim_{x \to \infty}{x\left(1 - \sqrt{1 + \frac{4}{x}}\right)} \\
                                                                  &=& \lim_{x \to \infty}{\frac{1 - \sqrt{1 + \frac{4}{x}}}{\frac{1}{x}}} \\
                                                                  &=& \lim_{x \to \infty}{\frac{\frac{2}{x^2}}{\frac{-1}{x^2}\sqrt{1 + \frac{4}{x}}}} \\
                                                                  &=& \lim_{x \to \infty}{\frac{\frac{2}{x^2}\cdot\frac{x^2}{-1}}{\sqrt{1 + \frac{4}{x}}}} \\
                                                                  &=& \lim_{x \to \infty}{\frac{\frac{2}{-1}\cdot\frac{x^2}{x^2}}{\sqrt{1 + \frac{4}{x}}}} \\
                                                                  &=& \lim_{x \to \infty}{\frac{-2}{\sqrt{1 + \frac{4}{x}}}} \\
                                                                  &=& \frac{-2}{\sqrt{1 + 0}} \\
                                                                  &=& -2
        \end{eqnarray}
    \item (Section 4.7, Exercise 75)
        \begin{eqnarray}
            \lim_{x \to 0^+}{x^{2x}} &=& e^{\lim\limits_{x \to 0^+}{\frac{\ln{x}}{\frac{1}{2x}}}} \\
                                     &=& e^{\lim\limits_{x \to 0^+}{\frac{\frac{1}{x}}{\frac{-1}{2x^2}}}} \\
                                     &=& e^{\lim\limits_{x \to 0^+}{\frac{1}{x}\cdot\frac{2x^2}{-1}}} \\
                                     &=& e^{\lim\limits_{x \to 0^+}{-\frac{2x^2}{x}}} \\
                                     &=& e^{\lim\limits_{x \to 0^+}{-2x}} \\
                                     &=& e^{-2(0)} \\
                                     &=& e^{0} \\
                                     &=& 1
        \end{eqnarray}
    \item (Section 4.7, Exercise 76)
        \begin{eqnarray}
            \lim_{x \to 0}{\left(1 + 4x\right)^{\frac{3}{x}}} &=& e^{\lim\limits_{x \to 0^+}{\frac{\ln{\left(1 + 4x\right)}}{\frac{1}{\frac{3}{x}}}}} \\
                                                              &=& e^{\lim\limits_{x \to 0^+}{\frac{\ln{\left(1 + 4x\right)}}{\frac{x}{3}}}} \\
                                                              &=& e^{\lim\limits_{x \to 0^+}{\frac{\frac{4}{\left(1 + 4x\right)}}{\frac{1}{3}}}} \\
                                                              &=& e^{\lim\limits_{x \to 0^+}{\frac{4}{\left(1 + 4x\right)}\cdot\frac{3}{1}}} \\
                                                              &=& e^{\lim\limits_{x \to 0^+}{\frac{12}{\left(1 + 4x\right)}}} \\
                                                              &=& e^{\frac{12}{1}} \\
                                                              &=& e^{12}
        \end{eqnarray}
    \item (Section 4.7, Exercise 96)
        \begin{eqnarray}
            f(x) &=& x^2\ln{x} \\
            g(x) &=& \ln^2{x}
        \end{eqnarray}
        \begin{eqnarray}
            \lim_{x \to \infty}{\frac{f(x)}{g(x)}} &=& \lim_{x \to \infty}{\frac{x^2\ln{x}}{\ln^2{x}}} \\
                                                   &=& \lim_{x \to \infty}{\frac{x^2}{\ln{x}}} \\
                                                   &=& \lim_{x \to \infty}{\frac{2x}{\frac{1}{x}}} \\
                                                   &=& \lim_{x \to \infty}{2x^2} \\
                                                   &=& \infty
        \end{eqnarray}
        $$f \gg g$$
    \item (Section 4.7, Exercise 100)
        \begin{eqnarray}
            f(x) &=& x^2\ln{x} \\
            g(x) &=& x^3
        \end{eqnarray}
        \begin{eqnarray}
            \lim_{x \to \infty}{\frac{f(x)}{g(x)}} &=& \lim_{x \to \infty}{\frac{x^2\ln{x}}{x^3}} \\
                                                   &=& \lim_{x \to \infty}{\frac{\ln{x}}{x}} \\
                                                   &=& \lim_{x \to \infty}{\frac{\frac{1}{x}}{1}} \\
                                                   &=& \lim_{x \to \infty}{\frac{1}{x}} \\
                                                   &=& \frac{1}{\infty} \neq \infty
        \end{eqnarray}
        $$g \gg f$$
    \item (Section 4.7, Exercise 95)
        \begin{eqnarray}
            f(x) &=& x^{10} \\
            g(x) &=& e^{0.01x}
        \end{eqnarray}
        \begin{eqnarray}
            \lim_{x \to \infty}{\frac{f(x)}{g(x)}} &=& \lim_{x \to \infty}{\frac{x^{10}}{e^{0.01x}}} \\
                                                   &=& \lim_{x \to \infty}{\frac{10x^9}{0.01e^{0.01x}}} \\
                                                   &=& \lim_{x \to \infty}{\frac{90x^8}{0.01^2e^{0.01x}}} \\
                                                   &=& \lim_{x \to \infty}{\frac{7200x^7}{0.01^3e^{0.01x}}} \\
                                                   &=& \lim_{x \to \infty}{\frac{50400x^6}{0.01^4e^{0.01x}}} \\
                                                   &=& \lim_{x \to \infty}{\frac{302400x^5}{0.01^5e^{0.01x}}} \\
                                                   &=& \lim_{x \to \infty}{\frac{1512000x^4}{0.01^6e^{0.01x}}} \\
                                                   &=& \lim_{x \to \infty}{\frac{6048000x^3}{0.01^7e^{0.01x}}} \\
                                                   &=& \lim_{x \to \infty}{\frac{18144000x^2}{0.01^8e^{0.01x}}} \\
                                                   &=& \lim_{x \to \infty}{\frac{36288000x}{0.01^9e^{0.01x}}} \\
                                                   &=& \lim_{x \to \infty}{\frac{36288000}{0.01^{10}e^{0.01x}}} \\
                                                   &=& \frac{36288000}{\infty} \neq \infty
        \end{eqnarray}
        $$g \gg f$$
    \item (Section 4.7, Exercise 101)
        \begin{eqnarray}
            f(x) &=& x^{20} \\
            g(x) &=& 1.00001^x
        \end{eqnarray}
        \begin{eqnarray}
            \lim_{x \to \infty}{\frac{f(x)}{g(x)}} &=& \lim_{x \to \infty}{\frac{x^{20}}{1.00001^x}} \\
                                                   &=& \frac{2432902008176640000}{\infty} \neq \infty
        \end{eqnarray}
        $$g \gg f$$
    \item (Section 4.9, Exercise 12)
        $$f(x) = 11x^{10}$$
        \begin{eqnarray}
            \int{11x^{10}}\,dx &=& 11\int{x^{10}}\,dx \\
                               &=& 11\frac{x^{11}}{11} + C \\
                               &=& x^{11} + C
        \end{eqnarray}
    \item (Section 4.9, Exercise 13)
        $$f(x) = 2\sin{x} + 1$$
        \begin{eqnarray}
            \int{2\sin{x} + 1}\,dx &=& 2\int{\sin{x}} + \int{x^0} \\
                                   &=& -2\cos{x} + x + C
        \end{eqnarray}
    \item (Section 4.9, Exercise 24)
        \begin{eqnarray}
            \int{3u^{-2} - 4u^2 + 1}\,du &=& 3\int{u^{-2}}\,du - 4\int{u^2}\,du + \int{u^0}\,du \\
                                         &=& 3\frac{u^{-1}}{-1} - 4\frac{u^3}{3} + u + C \\
                                         &=& \frac{-3}{u} - \frac{4u^3}{3} + u + C
        \end{eqnarray}
    \item (Section 4.9, Exercise 25)
        \begin{eqnarray}
            \int{4\sqrt{x} - \frac{4}{\sqrt{x}}}\,dx &=& 4\int{x^{\frac{1}{2}}}\,dx - 4\int{x^{-\frac{1}{2}}}\,dx \\
                                                     &=& 4\frac{x^{\frac{3}{2}}}{\frac{3}{2}} - 4\frac{x^{\frac{1}{2}}}{\frac{1}{2}} + C \\
                                                     &=& \frac{8x^{\frac{3}{2}}}{3} - 8x^{\frac{1}{2}} + C \\
                                                     &=& \frac{8x\sqrt{x}}{3} - 8\sqrt{x} + C
        \end{eqnarray}
    \item (Section 4.9, Exercise 31)
        \begin{eqnarray}
            \int{\left(3x + 1\right)\left(4 - x\right)}\,dx &=& \int{-3x^2 + 11x + 4}\,dx \\
                                                            &=& -3\int{x^2}\,dx + 11\int{x}\,dx + 4\int{x^0}\,dx \\
                                                            &=& -3\frac{x^3}{3} + 11\frac{x^2}{2} + 4x + C \\
                                                            &=& -x^3 + \frac{11x^2}{2} + 4x + C
        \end{eqnarray}
    \item (Section 4.9, Exercise 35)
        \begin{eqnarray}
            \int{\frac{4x^4 - 6x^2}{x}}\,dx &=& \int{4x^3 - 6x}\,dx \\
                                            &=& 4\int{x^3}\,dx - 6\int{x}\,dx \\
                                            &=& 4\frac{x^4}{4} - 6\frac{x^2}{2} + C \\
                                            &=& x^4 - 3x^2 + C
        \end{eqnarray}
    \item (Section 4.9, Exercise 40)
        \begin{eqnarray}
            \int{\left(\csc^2{\theta} + 1\right)}\,d\theta &=& \int{\csc^2{\theta}}\,d\theta + \int{\theta^0}\,d\theta \\
                                                           &=& -\cot{\theta} + \theta + C
        \end{eqnarray}
    \item (Section 4.9, Exercise 41)
        \begin{eqnarray}
            \int{\frac{2 + 3\cos{y}}{\sin^2{y}}}\,dy &=& \int{\left(\frac{2}{\sin^2{y}} + \frac{3\cos{y}}{\sin^2{y}}\right)}\,dy \\
                                                     &=& \int{\left(2\csc^2{y} + 3\cot{y}\csc{y}\right)}\,dy \\
                                                     &=& 2\int{\csc^2{y}}\,dy + 3\int{\cot{y}\csc{y}}\,dy \\
                                                     &=& -2\cot{y} - 3\csc{y} + C
        \end{eqnarray}
    \item (Section 4.9, Exercise 47)
        \begin{eqnarray}
            \int{\left(3t^2 + 2\csc^2{t}\right)}\,dt &=& \int{3t^2}\,dt + \int{2\csc^2{t}}\,dt \\
                                                     &=& 3\int{t^2}\,dt + 2\int{\csc^2{t}}\,dt \\
                                                     &=& 3\frac{t^3}{3} - 2\cot{t} + C \\
                                                     &=& t^3 - 2\cot{t} + C
        \end{eqnarray}
    \item (Section 4.9, Exercise 45)
        \begin{eqnarray}
            \int{\left(sec^2{\theta} + \sec{\theta}\tan{\theta}\right)}\,d\theta &=& \int{sec^2{\theta}}\,d\theta + \int{\sec{\theta}\tan{\theta}}\,d\theta \\
                                                                                 &=& \tan{\theta} + \sec{\theta} + C
        \end{eqnarray}
    \item (Section 4.9, Exercise 50)
        \begin{eqnarray}
            \int{\frac{\csc^3{x} + 1}{\csc{x}}}\,dx &=& \int{\frac{\csc^3{x}}{\csc{x}} + \frac{1}{\csc{x}}}\,dx \\
                                                    &=& \int{\csc^2{x} + \sin{x}}\,dx \\
                                                    &=& \int{\csc^2{x}}\,dx + \int{\sin{x}}\,dx \\
                                                    &=& -\cot{x} + -\cos{x} + C
        \end{eqnarray}
    \item (Section 4.9, Exercise 51)
        \begin{eqnarray}
            \int{\frac{1}{2y}}\,dy &=& \frac{1}{2}\int{\frac{1}{y}}\,dy \\
                                   &=& \frac{1}{2}\ln{|y|} + C \\
                                   &=& \frac{\ln{|y|}}{2} + C
        \end{eqnarray}
    \item (Section 4.9, Exercise 53)
        \begin{eqnarray}
            \int{\frac{6}{\sqrt{4 - 4x^2}}}\,dx &=& \int{\frac{6}{\sqrt{4\left(1 - x^2\right)}}}\,dx \\
                                                &=& \int{\frac{6}{2\sqrt{1 - x^2}}}\,dx \\
                                                &=& 3\int{\frac{1}{\sqrt{1 - x^2}}}\,dx \\
                                                &=& 3\sin^{-1}{x} + C
        \end{eqnarray}
    \item (Section 4.9, Exercise 59)
        \begin{eqnarray}
            \int{\frac{t + 1}{t}}\,dt &=& \int{\left(\frac{t}{t} + \frac{1}{t}\right)}\,dt \\
                                      &=& \int{\frac{t}{t}}\,dt + \int{\frac{1}{t}}\,dt \\
                                      &=& \int{t^0}\,dt + \ln{|t|} + C \\
                                      &=& t + \ln{|t|} + C
        \end{eqnarray}
    \item (Section 4.9, Exercise 61)
        \begin{eqnarray}
            \int{e^{x + 2}}\,dx &=& \int{e^2e^x}\,dx \\
                                &=& e^2\int{e^x}\,dx \\
                                &=& e^2e^x \\
                                &=& e^{x + 2}
        \end{eqnarray}
    \item (Section 4.9, Exercise 54)
        \begin{eqnarray}
            \int{\frac{v^3 + v + 1}{1 + v^2}}\,dv &=& \int{\frac{v^3 + v}{1 + v^2} + \frac{1}{1 + v^2}}\,dv \\
                                                  &=& \int{v + \frac{1}{1 + v^2}}\,dv \\
                                                  &=& \int{v}\,dv + \int{\frac{1}{1 + v^2}}\,dv \\
                                                  &=& \frac{v^2}{2} + \tan^{-1}{v} + C
        \end{eqnarray}
    \item (Section 4.9, Exercise 62)
        \begin{eqnarray}
            \int{\frac{10t^5 - 3}{t}}\,dt &=& \int{\left(\frac{10t^5}{t} - \frac{3}{t}\right)}\,dt \\
                                          &=& \int{10t^4}\,dt - \int{\frac{3}{t}}\,dt \\
                                          &=& 10\int{t^4}\,dt - 3\int{\frac{1}{t}}\,dt \\
                                          &=& 10\frac{t^5}{5} - 3\ln{|t|} + C \\
                                          &=& 2t^5 - 3\ln{|t|} + C
        \end{eqnarray}
    \item (Section 4.9, Exercise 78)
        \begin{eqnarray}
            g'(x) &=& 7x^6 - 4x^3 + 12 \\
            g(1) &=& 24 \\
            g(x) &=& \int{7x^6 - 4x^3 + 12}\,dx \\
                 &=& 7\int{x^6}\,dx - 4\int{x^3}\,dx + 12\int{x^0}\,dx \\
                 &=& 7\frac{x^7}{7} - 4\frac{x^4}{4} + 12x + C \\
                 &=& x^7 - x^4 + 12x + C \\
            (1)^7 - (1)^4 + 12(1) + C &=& 24 \\
            1 - 1 + 12 + C &=& 24 \\
            12 + C &=& 24 \\
            C &=& 12 \\
            g(x) &=& x^7 - x^4 + 12x + 12
        \end{eqnarray}
    \item (Section 4.9, Exercise 83)
        \begin{eqnarray}
            y'(t) &=& \frac{3}{t} + 6 \\
            y(1) &=& 8, t > 0 \\
            y(t) &=& \int{\left(\frac{3}{t} + 6\right)}\,dt \\
                 &=& 3\int{\frac{1}{t}}\,dt + 6\int{t^0}\,dt \\
                 &=& 3\ln{|t|} + 6t + C \\
            3\ln{1} + 6(1) + C &=& 8 \\
            6 + C &=& 8 \\
            C &=& 2 \\
            y(t) &=& 3\ln{|t|} + 6t + 2
        \end{eqnarray}
    \item (Section 4.9, Exercise 105)
        \begin{eqnarray}
            v(t) &=& \sin{t} \\
            s(0) &=& 0 \\
            s(t) &=& \int{\sin{t}}\,dt \\
                 &=& -\cos{t} + C \\
            -\cos{0} + C &=& 0 \\
            -1 + C &=& 0 \\
            C &=& 1 \\
            s(t) &=& -\cos{t} + 1
        \end{eqnarray}
        \begin{eqnarray}
            V(t) &=& \cos{t} \\
            S(0) &=& 0 \\
            S(t) &=& \int{\cos{t}}\,dt \\
                 &=& \sin{t} + C \\
            \sin{0} + C &=& 0 \\
            C &=& 0 \\
            S(t) &=& \sin{t}
        \end{eqnarray}
    \item (Section 4.9, Exercise 106)
        \begin{eqnarray}
            v(t) &=& e^t \\
            s(0) &=& 0 \\
            s(t) &=& \int{e^t}\,dt \\
                 &=& e^t + C \\
            e^0 + C &=& 0 \\
            1 + C &=& 0 \\
            C &=& -1 \\
            s(t) &=& e^t - 1
        \end{eqnarray}
        \begin{eqnarray}
            V(t) &=& 2 + \cos{t} \\
            S(0) &=& 3 \\
            S(t) &=& \int{2 + \cos{t}}\,dt \\
                 &=& \int{2}\,dt + \int{\cos{t}}\,dt \\
                 &=& 2\int{t^0}\,dt + \sin{t} + C \\
                 &=& 2t + \sin{t} + C \\
            2(0) + \sin{0} + C &=& 3 \\
            C &=& 3
        \end{eqnarray}
    \item (Section 5.1, Exercise 3)
        $$4(40) + 4(70) = 440$$
        $$2(30) + 2(50) + 2(80) + 2(40) = 400$$
    \item (Section 5.1, Exercise 15)
        $$v = 3t^2 + 1$$
        $$0 \leq t \leq 4$$
        \begin{eqnarray}
            f(0.5) + f(1.5) + f(2.5) + f(3.5) &=& 1.75 + 7.75 + 19.75 + 37.75 \\
                                              &=& 67 \\
            \frac{f(0.25)}{2} + \frac{f(0.75)}{2} + \frac{f(1.25)}{2} + \frac{f(1.75)}{2} + \frac{f(2.25)}{2} + \frac{f(2.75)}{2} + \frac{f(3.25)}{2} + \frac{f(3.75)}{2} &=& 67.75
        \end{eqnarray}
    \item (Section 5.1, Exercise 16)
        $$v = \sqrt{10t}$$
        $$1 \leq t \leq 7$$
        \begin{enumerate}
            \item
            $$n = 3$$
            $$[1,3],[3,5],[5,7]$$
            $$2,4,6$$
            $$d \approx 2v(2) + 2v(4) + 2v(6) = 2\sqrt{20} + 2\sqrt{40} + 2\sqrt{60} \approx 37.085$$
            \item
            $$n = 6$$
            $$1.5,2.5,3.5,4.5,5.5,6.5$$
            \begin{eqnarray}
                d &\approx& v(1.5) + v(2.5) + v(3.5) + v(4.5) + v(5.5) + v(6.5) \\
                  &=& \sqrt{15} + \sqrt{25} + \sqrt{35} + \sqrt{45} + \sqrt{55} + \sqrt{65} \\
                  &\approx& 36.976
            \end{eqnarray}
        \end{enumerate}
    \item (Section 5.1, Exercise 23)
        $$f(x) = x + 1$$
        $$[1,6]$$
        $$n = 5$$
        \begin{enumerate}
            \item
            \begin{eqnarray}
                \left(1 + 1\right) + \left(2 + 1\right) + \left(3 + 1\right) + \left(4 + 1\right) + \left(5 + 1\right) &=& 2 + 3 + 4 + 5 + 6 \\
                                                                                                                       &=& 20
            \end{eqnarray}
            \item
            \begin{eqnarray}
                \left(2 + 1\right) + \left(3 + 1\right) + \left(4 + 1\right) + \left(5 + 1\right) + \left(6 + 1\right) &=& 3 + 4 + 5 + 6 + 7 \\
                                                                                                                       &=& 25
            \end{eqnarray}
        \end{enumerate}
    \item (Section 5.1, Exercise 24)
        $$f(x) = \frac{1}{x}$$
        $$[1,5]$$
        $$n = 4$$
        $$\Delta{x} = 1$$
        \begin{eqnarray}
            \frac{1}{1} + \frac{1}{2} + \frac{1}{3} + \frac{1}{4} &=& \frac{25}{12} \\
            \frac{1}{2} + \frac{1}{3} + \frac{1}{4} + \frac{1}{5} &=& \frac{77}{60}
        \end{eqnarray}
    \item (Section 5.1, Exercise 29)
        $$f(x) = x^2 - 1$$
        $$[2,4]$$
        $$n = 4$$
        $$\Delta{x} = \frac{1}{2}$$
        $$[2,2.5],[2.5,3],[3,3.5],[3.5,4]$$
        \begin{eqnarray}
            \frac{3}{2} + \frac{5.25}{2} + \frac{8}{2} + \frac{11.25}{2} &=& 13.75 \\
            \frac{5.25}{2} + \frac{8}{2} + \frac{11.25}{2} + \frac{15}{2} &=& 19.75
        \end{eqnarray}
    \item (Section 5.1, Exercise 33)
        $$f(x) = 100 - x^2$$
        $$[0,10]$$
        $$n = 5$$
        $$\Delta{x} = 2$$
        \begin{eqnarray}
            2\left(99\right) + 2\left(91\right) + 2\left(75\right) + 2\left(51\right) + 2\left(19\right) &=& 198 + 182 + 150 + 102 + 38 \\
                                                                                                         &=& 670
        \end{eqnarray}
    \item (Section 5.1, Exercise 34)
        $$f(t) = \cos{\frac{t}{2}}$$
        $$[0, \pi]$$
        $$n = 4$$
        $$\Delta{x} = \frac{\pi}{4}$$
        \begin{eqnarray}
            \frac{\pi}{4}\left(\cos{\frac{\pi}{16}}\right) + \frac{\pi}{4}\left(\cos{\frac{3\pi}{16}}\right) + \frac{\pi}{4}\left(\cos{\frac{5\pi}{16}}\right) + \frac{\pi}{4}\left(\cos{\frac{7\pi}{16}}\right) &=& \frac{704}{105} \\
                                                                                                                                                                                                                 &\approx& 6.704761905
        \end{eqnarray}
    \item (Section 5.1, Exercise 39)
        $$f(x) = \sqrt{x}$$
        $$[1,3]$$
        $$n = 4$$
        $$\Delta{x} = \frac{1}{2}$$
        \begin{eqnarray}
            \frac{\sqrt{1.25}}{2} + \frac{\sqrt{1.75}}{2} + \frac{\sqrt{2.25}}{2} + \frac{\sqrt{2.75}}{2} &\approx& 2.853
        \end{eqnarray}
    \item (Section 5.1, Exercise 43)
        $$n = 4$$
        $$[0,2]$$
        $$\Delta{x} = \frac{1}{2}$$
        \begin{eqnarray}
            \frac{5}{2} + \frac{3}{2} + \frac{2}{2} + \frac{1}{2} &=& 5.5 \\
            \frac{3}{2} + \frac{2}{2} + \frac{1}{2} + \frac{1}{2} &=& 3.5
        \end{eqnarray}
    \item (Section 5.1, Exercise 44)
        $$n = 8$$
        $$[1,5]$$
        $$\Delta{x} = \frac{1}{2}$$
        \begin{eqnarray}
            \frac{0}{2} + \frac{2}{2} + \frac{3}{2} + \frac{2}{2} + \frac{2}{2} + \frac{1}{2} + \frac{0}{2} + \frac{2}{2} &=& 6 \\
            \frac{2}{2} + \frac{3}{2} + \frac{2}{2} + \frac{2}{2} + \frac{1}{2} + \frac{0}{2} + \frac{2}{2} + \frac{3}{2} &=& 7.5
        \end{eqnarray}
    \item (Section 5.1, Exercise 51)
        $$f(x) = 3\sqrt{x}$$
        $$[0,4]$$
        $$n = 40$$
        $$\Delta{x} = \frac{1}{10}$$
        \begin{eqnarray}
            \sum_{k = 1}^{40}{\frac{3}{10}\sqrt{\frac{k - 1}{10}}} &\approx& 15.681 \\
            \sum_{k = 1}^{40}{\frac{3}{10}\sqrt{\frac{k}{10}}} &\approx& 16.281 \\
            \sum_{k = 1}^{40}{\frac{3}{10}\sqrt{\frac{k - \frac{1}{2}}{10}}} &\approx& 16.005
        \end{eqnarray}
    \item (Section 5.1, Exercise 52)
        $$f(x) = x^2 + 1$$
        $$[-1,1]$$
        $$n = 50$$
        $$\Delta{x} = \frac{1}{25}$$
        \begin{eqnarray}
            \sum_{k = 1}^{50}{0.04\left(\left(-1 + 0.04\left(k - 1\right)\right)^2 + 1\right)} &\approx& 2.6672 \\
            \sum_{k = 1}^{50}{0.04\left(\left(-1 + 0.04k\right)^2 + 1\right)} &\approx& 2.6672 \\
            \sum_{k = 1}^{50}{0.04\left(\left(-1 + 0.04\left(k - \frac{1}{2}\right)\right)^2 + 1\right)} &\approx& 2.6664
        \end{eqnarray}
\end{enumerate}

\blfootnote{A copy of my notes (in \LaTeX) are available on my \href{https://github.com/onlinechronically/MATH-211}{GitHub}}
\end{document}
