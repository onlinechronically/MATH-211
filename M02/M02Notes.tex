\documentclass{article}
\usepackage{graphicx}
\usepackage{amsthm}
\usepackage{amsmath}
\usepackage{amssymb}
\usepackage{geometry}
\usepackage{tikz}
\usepackage[hidelinks]{hyperref}
\usetikzlibrary{arrows}

\geometry{a4paper, total={170mm,257mm}, left=20mm, top=20mm}
\AtBeginEnvironment{align}{\setcounter{equation}{0}} 
\AtBeginEnvironment{eqnarray}{\setcounter{equation}{0}} 

\newcommand\blfootnote[1]{
    \begingroup
    \renewcommand\thefootnote{}\footnote{#1}
    \addtocounter{footnote}{-1}
    \endgroup
}

\title{Module 2 Notes (MATH-211)}
\author{Lillie Donato}
\date{17 June 2024}

\begin{document}

\maketitle

\section*{General Notes (and Definitions)}
\begin{itemize}
    \item Derivatives
    \begin{itemize}
        \item A \textbf{derivative} is a new function made up of the slopes of the tangent lines as they change along a curve
        \item If a curve represents the trajectory of a moving object, the tangent line at a point indicates the direction of motion at that point
        \item As $x \to a$, the slope of the secant lines approaches the slope of the tangent line
        \item Alternative definition for Tangent Line(s): Consider the curve $y = f(x)$ and a secant line intersecting the curve at points $P(a, f(a))$ and $Q(a + h, f(a + h))$, with $m_{sec}$ and $m_{tan}$
        $$\text{Interval:} (a, a + h)$$
        $$m_{sec} = \frac{f(a + h) - f(a)}{h}$$
        $$m_{tan} = \lim_{h \to 0}{\frac{f(a + h) - f(a)}{h}}$$
        \item \textbf{Definition}: The derivative of $f$ at $a$, denoted $f'(a)$, is given by either the two following limits, provided the limits exist and $a$ is in the domain of $f$
        \begin{eqnarray}
            f'(a) &=& \lim_{x \to a}{\frac{f(x) - f(a)}{x - a}} \\
            f'(a) &=& \lim_{h \to 0}{\frac{f(a + h) - f(a)}{h}}
        \end{eqnarray}
        If $f'(a)$ exists, we say that $f$ is \textbf{differentiable} at $a$
    \end{itemize}
\end{itemize}

\section*{Examples}
\begin{enumerate}
    \item Instantaneous Velocity
    $$s(t) = -16t^2 + 128t + 192$$
    $$t = 2$$
    \begin{eqnarray}
        \lim_{t \to 2}{\frac{s(t) - s(2)}{t - 2}} &=& \lim_{t \to 2}{\frac{(-16t^2 + 128t + 192) - (-16(2^2) + 128(2) + 192)}{t - 2}} \\
        &=& \lim_{t \to 2}{\frac{(-16t^2 + 128t + 192) - (-64 + 256 + 192)}{t - 2}} \\
        &=& \lim_{t \to 2}{\frac{(-16t^2 + 128t + 192) - 384}{t - 2}} \\
        &=& \lim_{t \to 2}{\frac{-16t^2 + 128t - 192}{t - 2}} \\
        &=& \lim_{t \to 2}{\frac{(t - 2)(-16t + 96)}{t - 2}} \\
        &=& \lim_{t \to 2}{-16t + 96} \\
        &=& -32 + 96 \\
        &=& 64
    \end{eqnarray}
    \item Secant Lines
    $$y = f(x)$$
    Intersection Points: $P(a, f(a))$ and $Q(x, f(x))$ \\
    $$\text{Secant Line Slope} = \frac{f(x) - f(a)}{x - a}$$
    \item Tangent Lines
    $$f(x) = 2x^2 + 4x - 3$$
    $$(-1, 5)$$
    \begin{eqnarray}
        \lim_{x \to -1}{\frac{f(x) - f(-1)}{x - (-1)}} &=& \lim_{x \to -1}{\frac{2x^2 + 4x - 3 - (-5)}{x + 1}} \\
        &=& \lim_{x \to -1}{\frac{2x^2 + 4x + 2}{x + 1}} \\
        &=& \lim_{x \to -1}{\frac{(x + 1)(2x + 2)}{x + 1}} \\
        &=& \lim_{x \to -1}{\frac{(x + 1)(2x + 2)}{x + 1}} \\
        &=& \lim_{x \to -1}{2x + 2} \\
        &=& 2(-1) + 2 \\
        &=& -2 + 2 \\
        &=& 0
    \end{eqnarray}
    \item Alternative Tangent Lines
    $$f(x) = 5 - x^3$$
    $$(2, -3)$$
    $$a = 2$$
    $$h = -3 - 2 = -5$$
    \begin{eqnarray}
        \lim_{h \to 0}{\frac{f(2 + h) - f(2)}{h}} &=& \lim_{h \to 0}{\frac{f(2 + h) - (-3)}{h}} \\
        &=& \lim_{h \to 0}{\frac{f(2 + h) + 3}{h}} \\
        &=& \lim_{h \to 0}{\frac{5 - (2 + h)^3 + 3}{h}} \\
        &=& \lim_{h \to 0}{\frac{8 - (2 + h)^3}{h}} \\
        &=& \lim_{h \to 0}{\frac{2^3 - (2 + h)^3}{h}} \\
        &=& \lim_{h \to 0}{\frac{(2 - (2 + h))(2^2 + 2(2 + h) + (2 + h)^2)}{h}} \\
        &=& \lim_{h \to 0}{\frac{-h(4 + 4 + 2h + h^2 + 4h + 4)}{h}} \\
        &=& \lim_{h \to 0}{\frac{-h(h^2 + 6h + 12)}{h}} \\
        &=& \lim_{h \to 0}{-(h^2 + 6h + 12)} \\
        &=& -12 \\
    \end{eqnarray}
    $$y+3 = -12(x-2) = -12x + 24$$
    $$y = -12x + 21$$
    \item Derrivative Example
    $$f(x) = \sqrt{x - 1}$$
    $$x = 2$$
    $$f(x) = f(2) = \sqrt{2 - 1} = \sqrt{1} = 1$$
\end{enumerate}

\section*{Related Exercises}
\begin{enumerate}
    \item Example
\end{enumerate}

\blfootnote{A copy of my notes (in \LaTeX) are available on my \href{https://github.com/onlinechronically/MATH-211}{GitHub}}
\end{document}
