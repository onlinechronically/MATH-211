\documentclass{article}
\usepackage{graphicx}
\usepackage{amsthm}
\usepackage{amsmath}
\usepackage{amssymb}
\usepackage{geometry}
\usepackage{tikz}
\usepackage[hidelinks]{hyperref}
\usetikzlibrary{arrows}

\geometry{a4paper, total={170mm,257mm}, left=20mm, top=20mm}
\AtBeginEnvironment{align}{\setcounter{equation}{0}} 
\AtBeginEnvironment{eqnarray}{\setcounter{equation}{0}} 

\newcommand\blfootnote[1]{
    \begingroup
    \renewcommand\thefootnote{}\footnote{#1}
    \addtocounter{footnote}{-1}
    \endgroup
}

\title{Module 2 Notes (MATH-211)}
\author{Lillie Donato}
\date{17 June 2024}

\begin{document}

\maketitle

\section*{General Notes (and Definitions)}
\begin{itemize}
    \item Derivatives
    \begin{itemize}
        \item A \textbf{derivative} is a new function made up of the slopes of the tangent lines as they change along a curve
        \item If a curve represents the trajectory of a moving object, the tangent line at a point indicates the direction of motion at that point
        \item As $x \to a$, the slope of the secant lines approaches the slope of the tangent line
        \item Alternative definition for Tangent Line(s): Consider the curve $y = f(x)$ and a secant line intersecting the curve at points $P(a, f(a))$ and $Q(a + h, f(a + h))$, with $m_{sec}$ and $m_{tan}$
        $$\text{Interval:} (a, a + h)$$
        $$m_{sec} = \frac{f(a + h) - f(a)}{h}$$
        $$m_{tan} = \lim_{h \to 0}{\frac{f(a + h) - f(a)}{h}}$$
        $$y - f(a) = m_{tan}(x - a)$$
        \item \textbf{Definition}: The derivative of $f$ at $a$, denoted $f'(a)$, is given by either the two following limits, provided the limits exist and $a$ is in the domain of $f$
        \begin{eqnarray}
            f'(a) &=& \lim_{x \to a}{\frac{f(x) - f(a)}{x - a}} \\
            f'(a) &=& \lim_{h \to 0}{\frac{f(a + h) - f(a)}{h}}
        \end{eqnarray}
        If $f'(a)$ exists, we say that $f$ is \textbf{differentiable} at $a$
    \end{itemize}
\item Derivatives as Functions
    \begin{itemize}
        \item The slope of the tangent line of somefunction $f$ is a function called the derivative of $f$
            $$f'(x) = \lim_{h \to 0}{\frac{f(x + h) - f(x)}{h}}$$
        \item If $f'(x)$ exists, we say that $f$ is \textbf{differentiable} at x
        \item If $f$ is differentiable at every point in some open interval $I$, we say that $f$ is differentiable on $I$
        \item For some function $f$ we can denote the derivative of $f$ like such:
        \begin{eqnarray}
            f'(x) \\
            \frac{dy}{dx} \\
            \frac{df}{dx} \\
            \frac{d}{dx}(f(x)) \\
            D_x (f(x)) \\
            y'(x)
        \end{eqnarray}
        \item When evaluating some derivative $f$ at $a$, we can use the following:
        \begin{eqnarray}
            f'(a) \\
            y'(a) \\
            \frac{df}{dx}\Bigr|_{\substack{x=a}} \\
            \frac{dy}{dx}\Bigr|_{\substack{x=a}}
        \end{eqnarray}
        \item If $f$ is differentiable at $a$, then $f$ is continuous at $a$
        \item If $f$ is not continuous at $a$, then $f$ is not differentiable at $a$
    \end{itemize}
\end{itemize}

\section*{Rules of Differentiation}
\begin{itemize}
    \item Constant Rule
    $$\text{If } c \in \mathbb{R} \text{, then } \frac{d}{dx}\left(c\right) = 0$$
    \item Power Rule
    $$\text{If } n \in \mathbb{Z} \text{ and } n > 0 \text{, then } \frac{d}{dx}\left(x^n\right) = nx^{n - 1}$$
    \item Derivative of a Root
    $$\frac{d}{dx}\left(\sqrt{x}\right) = \frac{1}{2\sqrt{x}}$$
    \item Constant Multiple Rule
    $$\text{If } f \text{ is differentiable at } x \text{ and } c \text{ is a constant, then } \frac{d}{dx}\left(cf(x)\right) = cf'\left(x\right)$$
    \item Sum Rule
    $$\text{If } f \text{ and } g \text{ are differentiable at } x \text{, then } \frac{d}{dx}\left(f(x) + g(x)\right) = f'(x) + g'(x)$$
    \item Generalized Sum Rule
    $$\frac{d}{dx}\left(f_1(x) + f_2(x) + ... + f_x(x)\right) = f_1'(x) + f_2'(x) + ... + f_n'(x)$$
    \item Difference Rule
    $$\frac{d}{dx}\left(f(x) - g(x)\right) = f'(x) - g'(x)$$
    \item Euler's Number
        $$\text{The function } f(x) = e^x \text{ is differentiable for all } x \in \mathbb{R} \text{, and } \frac{d}{dx}\left(e^x\right) = e^x$$
    \item Higher-order Derivatives
    \\ Assuming $y = f(x)$ can be differentiated as often as necessary, the \textbf{second derivative} of $f$ is
    $$f''(x) = \frac{d}{dx}\left(f'(x)\right)$$
    For $n \in \mathbb{Z}$ where $n \geq 1$, the \textbf{nth derivative} of $f$ is
    $$f^{(n)}\left(x\right) = \frac{d}{dx}\left(f^{(n - 1)}\left(x\right)\right)$$
    \item Product Rule
    $$\text{If } f \text{ and } g \text{ are differentiable at } x \text{, then } \frac{d}{dx}\left(f(x)g(x)\right) = f'(x)g(x) + f(x)g'(x)$$
    \item Quotient Rule
        $$\text{If } f \text{ and } g \text{ are differentiable at } x \text{ and } g(x) \neq 0 \text{, then the derivative of } \frac{f}{g} \text{ at } x \text{ exists and}$$
        $$\frac{d}{dx}\left(\frac{f(x)}{g(x)}\right) = \frac{g(x)f'(x) - f(x)g'(x)}{(g(x))^2}$$
    \item Power Rule for Negative Integers
    \begin{proof}
        \begin{align*}
            &\quad \text{Choose } n \in \mathbb{Z} \\
            &\quad \text{Assume that } n < 0 \\
            &\quad \hspace{1cm} \text{Let } m = -n \\
            &\quad \hspace{1cm} \text{Since } m = -n \text{ and } n < 0 \text{, we know } m > 0
        \end{align*}
        \begin{eqnarray}
            \frac{d}{dx}\left(x^n\right) &=& \frac{d}{dx}\left(\frac{1}{x^m}\right) \\
                                         &=& \frac{x^m \cdot 0 - 1 \cdot (mx^{m - 1})}{(x^m)^2} \\
                                         &=& \frac{-mx^{m - 1}}{x^{2m}} \\
                                         &=& -mx^{(m - 1) - 2m} \\
                                         &=& -mx^{- m - 1} \\
                                         &=& nx^{n - 1} \\
        \end{eqnarray} \\
        $$\text{Under the assumption that } n \in \mathbb{Z} \text{ and } n < 0 \text{, we proved } \frac{d}{dx}\left(x^n\right) = nx^{n - 1}$$
    \end{proof}
\end{itemize}

\section*{Examples}
\begin{enumerate}
    \item Instantaneous Velocity
    $$s(t) = -16t^2 + 128t + 192$$
    $$t = 2$$
    \begin{eqnarray}
        \lim_{t \to 2}{\frac{s(t) - s(2)}{t - 2}} &=& \lim_{t \to 2}{\frac{(-16t^2 + 128t + 192) - (-16(2^2) + 128(2) + 192)}{t - 2}} \\
        &=& \lim_{t \to 2}{\frac{(-16t^2 + 128t + 192) - (-64 + 256 + 192)}{t - 2}} \\
        &=& \lim_{t \to 2}{\frac{(-16t^2 + 128t + 192) - 384}{t - 2}} \\
        &=& \lim_{t \to 2}{\frac{-16t^2 + 128t - 192}{t - 2}} \\
        &=& \lim_{t \to 2}{\frac{(t - 2)(-16t + 96)}{t - 2}} \\
        &=& \lim_{t \to 2}{-16t + 96} \\
        &=& -32 + 96 \\
        &=& 64
    \end{eqnarray}
    \item Secant Lines
    $$y = f(x)$$
    Intersection Points: $P(a, f(a))$ and $Q(x, f(x))$ \\
    $$\text{Secant Line Slope} = \frac{f(x) - f(a)}{x - a}$$
    \item Tangent Lines
    $$f(x) = 2x^2 + 4x - 3$$
    $$(-1, 5)$$
    \begin{eqnarray}
        \lim_{x \to -1}{\frac{f(x) - f(-1)}{x - (-1)}} &=& \lim_{x \to -1}{\frac{2x^2 + 4x - 3 - (-5)}{x + 1}} \\
        &=& \lim_{x \to -1}{\frac{2x^2 + 4x + 2}{x + 1}} \\
        &=& \lim_{x \to -1}{\frac{(x + 1)(2x + 2)}{x + 1}} \\
        &=& \lim_{x \to -1}{\frac{(x + 1)(2x + 2)}{x + 1}} \\
        &=& \lim_{x \to -1}{2x + 2} \\
        &=& 2(-1) + 2 \\
        &=& -2 + 2 \\
        &=& 0
    \end{eqnarray}
    \item Alternative Tangent Lines
    $$f(x) = 5 - x^3$$
    $$(2, -3)$$
    $$a = 2$$
    $$h = -3 - 2 = -5$$
    \begin{eqnarray}
        \lim_{h \to 0}{\frac{f(2 + h) - f(2)}{h}} &=& \lim_{h \to 0}{\frac{f(2 + h) - (-3)}{h}} \\
        &=& \lim_{h \to 0}{\frac{f(2 + h) + 3}{h}} \\
        &=& \lim_{h \to 0}{\frac{5 - (2 + h)^3 + 3}{h}} \\
        &=& \lim_{h \to 0}{\frac{8 - (2 + h)^3}{h}} \\
        &=& \lim_{h \to 0}{\frac{2^3 - (2 + h)^3}{h}} \\
        &=& \lim_{h \to 0}{\frac{(2 - (2 + h))(2^2 + 2(2 + h) + (2 + h)^2)}{h}} \\
        &=& \lim_{h \to 0}{\frac{-h(4 + 4 + 2h + h^2 + 4h + 4)}{h}} \\
        &=& \lim_{h \to 0}{\frac{-h(h^2 + 6h + 12)}{h}} \\
        &=& \lim_{h \to 0}{-(h^2 + 6h + 12)} \\
        &=& -12 \\
    \end{eqnarray}
    $$y+3 = -12(x-2) = -12x + 24$$
    $$y = -12x + 21$$
    \item Derivative Example
    $$f(x) = \sqrt{x - 1}$$
    $$x = 2$$
    $$f(x) = f(2) = \sqrt{2 - 1} = \sqrt{1} = 1$$
    $$(2, 1)$$
    \begin{eqnarray}
        f'(2) &=& \lim_{x \to 2}{\frac{f(x) - f(2)}{x - 2}} \\
              &=& \lim_{x \to 2}{\frac{\sqrt{x - 1} - 1}{x - 2}} \\
              &=& \lim_{x \to 2}{\frac{\sqrt{x - 1} - 1}{x - 2} \cdot \frac{\sqrt{x - 1} + 1}{\sqrt{x - 1} + 1}} \\
              &=& \lim_{x \to 2}{\frac{x - 2}{(x - 2)(\sqrt{x - 1} + 1)}} \\
              &=& \lim_{x \to 2}{\frac{1}{\sqrt{x - 1} + 1}} \\
              &=& \frac{1}{\sqrt{2 - 1} + 1} \\
              &=& \frac{1}{\sqrt{1} + 1} \\
              &=& \frac{1}{1 + 1} \\
              &=& \frac{1}{2} \\
    \end{eqnarray}
    \begin{eqnarray}
        y - 1 &=& \frac{1}{2}(x - 2) \\
        y &=& \frac{1}{2}(x - 2) + 1 \\
        &=& \frac{1}{2}x - 1 + 1 \\
        &=& \frac{1}{2}x
    \end{eqnarray}
    \item Derivative Application Example
    $$V(t) = 3t$$
    \begin{eqnarray}
        V'(12) &=& \lim_{x \to 12}{\frac{V(x) - V(12)}{x - 12}} \\
               &=& \lim_{x \to 12}{\frac{3x - 36}{x - 12}} \\
               &=& \lim_{x \to 12}{\frac{3(x - 12)}{x - 12}} \\
               &=& \lim_{x \to 12}{3} \\
               &=& 3
    \end{eqnarray}
    \begin{eqnarray}
        y - 36 &=& 3(x - 12) \\
        y &=& 3x - 36 + 36 \\
          &=& 3x \\
    \end{eqnarray}
    \item Find the Derivative
    $$f(x) = 4x^2 - 5x + 6$$
    \begin{eqnarray}
        \lim_{h \to 0}{\frac{f(x + h) - f(x)}{h}} &=& \lim_{h \to 0}{\frac{4(x + h)^2 - 5(x + h) + 6 - (4x^2 - 5x + 6)}{h}} \\
        &=& \lim_{h \to 0}{\frac{4(x + h)^2 - 5x - 5h + 6 - 4x^2 + 5x - 6}{h}} \\
        &=& \lim_{h \to 0}{\frac{4(x + h)^2 - 5h - 4x^2}{h}} \\
        &=& \lim_{h \to 0}{\frac{4h^2 + 8xh - 5h}{h}} \\
        &=& \lim_{h \to 0}{4h + 8x - 5} \\
        &=& 4(0) + 8x - 5 \\
        &=& 8x - 5
    \end{eqnarray}
    \item Calculating a Derivative
    $$f(x) = \frac{1}{x}$$
    $$(-5, - \frac{1}{5})$$
    \begin{eqnarray}
        \frac{dy}{dx} &=& \lim_{h \to 0}{\frac{f(x + h) - f(x)}{h}} \\
                      &=& \lim_{h \to 0}{\frac{\frac{1}{x + h} - \frac{1}{x}}{h}} \\
                      &=& \lim_{h \to 0}{\frac{\frac{x - (x + h)}{x(x + h)}}{h}} \\
                      &=& \lim_{h \to 0}{\frac{\frac{-h}{x(x + h)}}{h}} \\
                      &=& \lim_{h \to 0}{\frac{-h}{x(x + h)} \cdot \frac{1}{h}} \\
                      &=& \lim_{h \to 0}{\frac{-h}{h(x(x + h))}} \\
                      &=& \lim_{h \to 0}{\frac{-1}{x(x + h)}} \\
                      &=& \frac{-1}{x(x + 0)} \\
                      &=& \frac{-1}{x^2}
    \end{eqnarray}
    $$m_{\tan} = \frac{dy}{dx}\Bigr|_{\substack{x=-5}} = \frac{-1}{(-5)^2} = \frac{-1}{25}$$
    $$y - (- \frac{1}{5}) = \frac{-1}{25}(x - (-5))$$
    $$y = \frac{-1}{25}x - \frac{1}{5} - \frac{1}{5} = \frac{-1}{25}x - \frac{2}{5}$$
    \item Constant and Power Rules
    $$\frac{d}{dx}\left(x^6\right) = 6x^5$$
    $$\frac{d}{dx}\left(x\right) = 1x^0 = 1$$
    $$\frac{d}{dx}\left(\pi^2\right) = 0$$
    \item Constant Multiple Rule
    $$\frac{d}{dx}\left(-4x^9\right) = -4\left(9x^8\right) = -36x^8$$
    $$\frac{d}{dt}\left(\frac{2}{5}t^5\right) = \frac{2}{5}\left(5t^4\right) = 2t^4$$
    \item Sum and Difference Rules with a Polynomial
    $$\frac{d}{dx}\left(6x^5 - \frac{5}{2}x^2 + x + 5\right) = 30x^4 - 5x + 1$$
    \item Euler's Number with Derivatives
    $$f(x) = 5x + \frac{1}{3}e^x$$
    $$\left(0, \frac{1}{3}\right)$$
    $$\frac{d}{dx}\left(5x + \frac{1}{3}e^x\right) = 5 + \frac{1}{3}e^x$$
    $$y = \frac{16}{3}x + \frac{1}{3}$$
    \item Higher-order derivatives
    $$f(x) = 3x^4 - 2x^2 + 7x - e^x$$
    $$f'(x) = 12x^3 - 4x + 7 - e^x$$
    $$f''(x) = 36x^2 - 4 - e^x$$
    $$f'''(x) = 72x - e^x$$
    $$f^{(4)}(x) = 72 - e^x$$
    \item Product Rule
    $$\frac{d}{dt}\left((t + 1)(t^2 - t + 1)\right) = 1(t^2 - t + 1) + (2t - 1)(t + 1) = 3t^2$$
    $$\frac{d}{dx}\left(x^5e^x\right) = 5x^4(e^x) + e^x(x^5) = e^x\left(x^5 + 5x^4\right)$$
    \item Quotient Rule
    \begin{eqnarray}
        \frac{d}{dx}\left(\frac{x^4 + 5x^2 + x}{\sin{x}}\right) &=& \frac{\sin{x}(4x^3 + 10x + 1) - (x^4 + 5x^2 + x)\cos{x}}{\sin^2{x}} \\
                                                                &=& \frac{(4x^3 + 10x + 1) - (x^4 + 5x^2 + x)\cot{x}}{\sin{x}}
    \end{eqnarray}
    \begin{eqnarray}
        \frac{d}{dx}\left(\frac{2e^x - 1}{3e^x + 1}\right) &=& \frac{2e^x(2e^x + 1) - 2e^x(2e^x - 1)}{(2e^x + 1)^2} \\
        &=& \frac{2e^x(2e^x + 1 - 2e^x + 1)}{(2e^x + 1)^2} \\
        &=& \frac{2e^x(2)}{(2e^x + 1)^2} \\
        &=& \frac{4e^x}{(2e^x + 1)^2}
    \end{eqnarray}
    \item Tangent Lines with Quotient Rule
    $$f(x) = \frac{2x^2}{3x - 1}$$
    $$(1, 1)$$
    $$f(1) = 1$$
    $$f'(1) = \frac{4x(3x - 1) - 3(2x^2)}{(3x - 1)^2} = \frac{2}{4} = \frac{1}{2}$$
    $$y = \frac{1}{2}x + \frac{1}{2}$$
    \item Power Rule with Negative Powers
    \begin{eqnarray}
        \frac{d}{dx}\left(\frac{-5}{x^6}\right) &=& \frac{d}{dx}\left(-5x^{-6}\right) \\
        &=& -5\left(-6x^{-7}\right) \\
        &=& 30x^{-7} \\
        &=& \frac{1}{30x^{7}}
    \end{eqnarray}
    \begin{eqnarray}
    \frac{d}{dp}\left(\frac{2p^8 - 7}{p^3}\right) &=& \frac{d}{dp}2p^5 - 7p^{-3} \\
    &=& 10p^4 + 21p^{-4} \\
    &=& \frac{10p^4}{21p^4}
    \end{eqnarray}
    \item Combining Rules to Evaluate Derivatives
        $$f(x) = \left(\frac{x^2 + 1}{x}\right)e^x = \left(x + x^{-1}\right)e^x$$
        $$f'(x) = \left(1 + -x^{-2}\right)e^x + \left(x + x^{-1}\right)e^x$$
\end{enumerate}

\section*{Related Exercises}
\begin{enumerate}
    \item (Section 3.1, Related Exercise 13)
    $$s(t) = -16t^2 + 100t$$
    $$a = 1$$
    \begin{eqnarray}
        \lim_{h \to 0}{\frac{s(a + h) - s(a)}{h}} &=& \lim_{h \to 0}{\frac{s(1 + h) - 84}{h}} \\
        &=& \lim_{h \to 0}{\frac{-16(1 + h)^2 + 100(1 + h) - 84}{h}} \\
        &=& \lim_{h \to 0}{\frac{-16(h^2 + 2h + 1) + 100 + 100h - 84}{h}} \\
        &=& \lim_{h \to 0}{\frac{-16h^2 - 32h - 16 + 100 + 100h - 84}{h}} \\
        &=& \lim_{h \to 0}{\frac{-16h^2 + 68h}{h}} \\
        &=& \lim_{h \to 0}{-16h + 68} \\
        &=& -16(0) + 68 \\
        &=& 68
    \end{eqnarray}
    \item (Section 3.1, Related Exercise 14)
    $$s(t) = -16t^2 + 128t + 192$$
    $$a = 2$$
    \begin{eqnarray}
        \lim_{h \to 0}{\frac{s(a + h) - s(a)}{h}} &=& \lim_{h \to 0}{\frac{s(2 + h) - 384}{h}} \\
        &=& \lim_{h \to 0}{\frac{-16(2 + h)^2 + 128(2 + h) + 192 - 384}{h}} \\
        &=& \lim_{h \to 0}{\frac{-16(h^2 + 4h + 4) + 128(2 + h) + 192 - 384}{h}} \\
        &=& \lim_{h \to 0}{\frac{-16h^2 - 64h - 64 + 256 + 128h + 192 - 384}{h}} \\
        &=& \lim_{h \to 0}{\frac{-16h^2 + 64h}{h}} \\
        &=& \lim_{h \to 0}{-16h + 64} \\
        &=& -16(0) + 64 \\
        &=& 64
    \end{eqnarray}
    \item (Section 3.1, Related Exercise 17)
    $$f(x) = \frac{1}{x}$$
    $$P(-1, -1)$$
    \begin{eqnarray}
        \lim_{x \to -1}{\frac{f(x) - f(-1)}{x - (-1)}} &=& \lim_{x \to -1}{\frac{f(x) - (-1)}{x + 1}} \\
        &=& \lim_{x \to -1}{\frac{f(x) + 1}{x + 1}} \\
        &=& \lim_{x \to -1}{\frac{\frac{1}{x} + 1}{x + 1}} \\
        &=& \lim_{x \to -1}{\frac{\frac{1+x}{x}}{x + 1}} \\
        &=& \lim_{x \to -1}{\frac{\frac{1+x}{x}}{x + 1} \cdot \frac{x}{x}} \\
        &=& \lim_{x \to -1}{\frac{1+x}{(x + 1)x}} \\
        &=& \lim_{x \to -1}{\frac{1}{x}} \\
        &=& \frac{1}{-1} \\
        &=& -1
    \end{eqnarray}
    $$y - (-1) = -1(x - (-1))$$
    $$y = -1(x + 1) - 1 = -x - 1 - 1 = -x - 2$$
    \item (Section 3.1, Related Exercise 18)
    $$f(x) = \frac{4}{x^2}$$
    $$(-1, 4)$$
    \begin{eqnarray}
        \lim_{x \to -1}{\frac{f(x) - 4}{x - (-1)}} &=& \lim_{x \to -1}{\frac{f(x) - 4}{x + 1}} \\
                                                 &=& \lim_{x \to -1}{\frac{\frac{4}{x^2} - 4}{x + 1}} \\
                                                 &=& \lim_{x \to -1}{\frac{\frac{4 - 4x^2}{x^2}}{x + 1}} \\
                                                 &=& \lim_{x \to -1}{\frac{\frac{4 - 4x^2}{x^2}}{x + 1} \cdot \frac{x^2}{x^2}} \\
                                                 &=& \lim_{x \to -1}{\frac{4 - 4x^2}{x^2(x + 1)}} \\
                                                 &=& \lim_{x \to -1}{\frac{4(1 - x^2)}{x^2(x + 1)}} \\
                                                 &=& \lim_{x \to -1}{\frac{4(1 - x)(1 + x)}{x^2(x + 1)}} \\
                                                 &=& \lim_{x \to -1}{\frac{4(1 - x)}{x^2}} \\
                                                 &=& \frac{4(1 - (-1)^2)}{(-1)^2} \\
                                                 &=& \frac{4(1 - 1)}{1} \\
                                                 &=& 0
    \end{eqnarray}
    $$y - 4 = 0$$
    $$y = 4$$
    \item (Section 3.1, Related Exercise 23)
    $$f(x) = 3x^2 - 4x$$
    $$(1, -1)$$
    \begin{eqnarray}
        \lim_{h \to 0}{\frac{f(a + h) - f(a)}{h}} &=& \lim_{h \to 0}{\frac{f(1 + h) - (-1)}{h}} \\
                                                &=& \lim_{h \to 0}{\frac{3(1 + h)^2 - 4(1 + h) +1}{h}} \\
                                                &=& \lim_{h \to 0}{\frac{3h^2 + 6h + 3 - 4 - 4h +1}{h}} \\
                                                &=& \lim_{h \to 0}{\frac{3h^2 + 2h + 3 - 4 + 1}{h}} \\
                                                &=& \lim_{h \to 0}{\frac{3h^2 + 2h}{h}} \\
                                                &=& \lim_{h \to 0}{3h + 2} \\
                                                &=& 3(0) + 2 \\
                                                &=& 2
    \end{eqnarray}
    $$y - (-1) = 2(x - 1)$$
    $$y = 2(x - 1) - 1 = 2x - 2 - 1 = 2x - 3$$
    \item (Section 3.1, Related Exercise 27)
    $$f(x) = x^3$$
    $$(1, 1)$$
    \begin{eqnarray}
        \lim_{h \to 0}{\frac{f(a + h) - f(a)}{h}} &=& \lim_{h \to 0}{\frac{f(1 + h) - 1}{h}} \\
        &=& \lim_{h \to 0}{\frac{(1 + h)^3 - 1}{h}} \\
        &=& \lim_{h \to 0}{\frac{h^3 + h^2 + 2h + 2h^2 + h + 1 - 1}{h}} \\
        &=& \lim_{h \to 0}{\frac{h^3 + 2h^2 + 3h}{h}} \\
        &=& \lim_{h \to 0}{h^2 + 2h + 3} \\
        &=& 0^2 + 2(0) + 3 \\
        &=& 3
    \end{eqnarray}
    $$y - 1 = 3(x - 1)$$
    $$y = 3x - 3 + 1 = 3x - 2$$
    \item (Section 3.1, Related Exercise 39)
    $$f(x) = \sqrt{2x+1}$$
    $$a = 4$$
    \begin{eqnarray}
        \lim_{x \to 4}{\frac{f(x) - f(a)}{x - a}} &=& \lim_{x \to 4}{\frac{f(x) - 3}{x - 4}} \\
                                                &=& \lim_{x \to 4}{\frac{\sqrt{2x + 1} - 3}{x - 4}} \\
                                                &=& \lim_{x \to 4}{\frac{\sqrt{2x + 1} - 3}{x - 4} \cdot \frac{\sqrt{2x + 1} + 3}{\sqrt{2x + 1} + 3}} \\
                                                &=& \lim_{x \to 4}{\frac{2x + 1 - 9}{(x - 4)(\sqrt{2x + 1} + 3)}} \\
                                                &=& \lim_{x \to 4}{\frac{2(x - 4)}{(x - 4)(\sqrt{2x + 1} + 3)}} \\
                                                &=& \lim_{x \to 4}{\frac{2}{\sqrt{2x + 1} + 3}} \\
                                                &=& \frac{2}{\sqrt{9} + 3} \\
                                                &=& \frac{2}{3 + 3} \\
                                                &=& \frac{2}{6} \\
                                                &=& \frac{1}{3}
    \end{eqnarray}
    $$y - 3 = \frac{1}{3}(x - 4)$$
    $$y = \frac{1}{3}x - \frac{4}{3} + \frac{9}{3} = \frac{1}{3}x + \frac{5}{3}$$
    \item (Section 3.1, Related Exercise 40)
    $$f(x) = \sqrt{3x}$$
    $$a = 12$$
    \begin{eqnarray}
        \lim_{x \to 12}{\frac{f(x) - f(a)}{x - a}} &=& \lim_{x \to 12}{\frac{f(x) - 6}{x - 12}} \\
                                                 &=& \lim_{x \to 12}{\frac{\sqrt{3x} - 6}{x - 12}} \\
                                                 &=& \lim_{x \to 12}{\frac{\sqrt{3x} - 6}{x - 12} \cdot \frac{\sqrt{3x} + 6}{\sqrt{3x} + 6}} \\
                                                 &=& \lim_{x \to 12}{\frac{3x - 36}{(x - 12)(\sqrt{3x} + 6)}} \\
                                                 &=& \lim_{x \to 12}{\frac{3(x - 12)}{(x - 12)(\sqrt{3x} + 6)}} \\
                                                 &=& \lim_{x \to 12}{\frac{3}{\sqrt{3x} + 6}} \\
                                                 &=& \frac{3}{\sqrt{3(12)} + 6} \\
                                                 &=& \frac{3}{\sqrt{36} + 6} \\
                                                 &=& \frac{3}{6 + 6} \\
                                                 &=& \frac{3}{12} \\
                                                 &=& \frac{1}{4}
    \end{eqnarray}
    $$y - 6 = \frac{1}{4}(x - 12)$$
    $$y = \frac{1}{4}x - 3 + 6 = \frac{1}{4}x + 3$$
    \item (Section 3.1, Related Exercise 49)
    $$d(t) = 16t^2$$
    $$a = 4$$
    \begin{eqnarray}
        \lim_{x \to 4}{\frac{f(x) - f(a)}{x - a}} &=& \lim_{h \to 0}{\frac{f(x) - 256}{x - 4}} \\
        &=& \lim_{x \to 4}{\frac{16x^2 - 256}{x - 4}} \\
        &=& \lim_{x \to 4}{\frac{(16x + 64)(x - 4)}{x - 4}} \\
        &=& \lim_{x \to 4}{16x + 64} \\
        &=& 16(4) + 64 \\
        &=& 64 + 64 \\
        &=& 128
    \end{eqnarray}
    \item (Section 3.1, Related Exercise 50)
    $$F(x) = \frac{k}{x^2} \text{ where } k \text{ is some constant}$$
    $$a = 1$$
    \begin{eqnarray}
        \lim_{h \to 0}{\frac{F(a + h) - F(a)}{h}} &=& \lim_{h \to 0}{\frac{F(1 + h) - \frac{k}{1}}{h}} \\
                                                &=& \lim_{h \to 0}{\frac{F(1 + h) - \frac{k}{1}}{h}} \\
                                                &=& \lim_{h \to 0}{\frac{\frac{k}{(1 + h)^2} - \frac{k}{1}}{h}} \\
                                                &=& \lim_{h \to 0}{\frac{\frac{k}{(1 + h)^2} - \frac{k(1 + h)^2}{(1 + h)^2}}{h}} \\
                                                &=& \lim_{h \to 0}{\frac{\frac{k - k(1 + h)^2}{(1 + h)^2}}{h}} \\
                                                &=& \lim_{h \to 0}{\frac{\frac{k - k(1 + h)^2}{(1 + h)^2}}{h}} \\
                                                &=& \lim_{h \to 0}{\frac{\frac{k - (kh^2 + 2kh + k)}{(1 + h)^2}}{h}} \\
                                                &=& \lim_{h \to 0}{\frac{\frac{k - kh^2 - 2kh - k}{(1 + h)^2}}{h}} \\
                                                &=& \lim_{h \to 0}{\frac{\frac{- kh^2 - 2kh}{(1 + h)^2}}{h}} \\
                                                &=& \lim_{h \to 0}{\frac{- kh^2 - 2kh}{(1 + h)^2} \cdot \frac{1}{h}} \\
                                                &=& \lim_{h \to 0}{\frac{h(- kh - 2k)}{h(1 + h)^2}} \\
                                                &=& \lim_{h \to 0}{\frac{- kh - 2k}{(1 + h)^2}} \\
                                                &=& \frac{- kh - 2k}{(1 + h)^2} \\
                                                &=& \frac{- k(0) - 2k}{(1 + 0)^2} \\
                                                &=& \frac{- 2k}{1} \\
                                                &=& - 2k
    \end{eqnarray}
    \item (Section 3.1, Related Exercise 53)
        \\ Hint: Sketch a Secant Line
        $$L'(1.5) \approx 4$$
        $$L'(a) \approx 0 \text{ where } a \geq 4$$
    \item (Section 3.1, Related Exercise 54)
        $$D'(60) \approx 0.6$$
        $$D'(170) \approx 0$$
    \item (Section 3.2, Related Exercise 23)
    $$f(x) = 4x^2 + 1$$
    $$a = 2, 4$$
    \begin{eqnarray}
        f'(x) &=& \lim_{h \to 0}{\frac{f(x + h) - f(x)}{h}} \\
              &=& \lim_{h \to 0}{\frac{4(x + h)^2 + 1 - (4x^2 + 1)}{h}} \\
              &=& \lim_{h \to 0}{\frac{4(x^2 + 2xh + h^2) + 1 - 4x^2 - 1}{h}} \\
              &=& \lim_{h \to 0}{\frac{4x^2 + 8xh + 4h^2 - 4x^2}{h}} \\
              &=& \lim_{h \to 0}{\frac{8xh + 4h^2}{h}} \\
              &=& \lim_{h \to 0}{8x + 4h} \\
              &=& 8x + 4(0) \\
              &=& 8x
    \end{eqnarray}
    $$f'(2) = 8(2) = 16$$
    $$f'(4) = 8(4) = 32$$
    \item (Section 3.2, Related Exercise 24)
    $$f(x) = x^2 + 3x$$
    $$a = -1, 4$$
    \begin{eqnarray}
        f'(x) &=& \lim_{h \to 0}{\frac{f(x + h) - f(x)}{h}} \\
        &=& \lim_{h \to 0}{\frac{(x + h)^2 + 3(x + h) - (x^2 + 3x)}{h}} \\
        &=& \lim_{h \to 0}{\frac{x^2 + h^2 + xh + 3x + 3h - x^2 - 3x}{h}} \\
        &=& \lim_{h \to 0}{\frac{h^2 + xh + 3h}{h}} \\
        &=& \lim_{h \to 0}{h + x + 3} \\
        &=& 0 + x + 3 \\
        &=& x + 3
    \end{eqnarray}
    $$f'(-1) = -1 + 3 = 2$$
    $$f'(4) = 4 + 3 = 7$$
    \item (Section 3.2, Related Exercise 37)
    $$f(x) = \sqrt{3x + 1}$$
    $$a = 8$$
    \begin{eqnarray}
        f'(x) &=& \lim_{h \to 0}{\frac{f(x + h) - f(x)}{h}} \\
              &=& \lim_{h \to 0}{\frac{\sqrt{3(x + h) + 1} - \sqrt{3x + 1}}{h}} \\
              &=& \lim_{h \to 0}{\frac{\sqrt{3(x + h) + 1} - \sqrt{3x + 1}}{h} \cdot \frac{\sqrt{3(x + h) + 1} + \sqrt{3x + 1}}{\sqrt{3(x + h) + 1} + \sqrt{3x + 1}}} \\
              &=& \lim_{h \to 0}{\frac{3(x + h) + 1 - (3x + 1)}{h(\sqrt{3(x + h) + 1} + \sqrt{3x + 1})}} \\
              &=& \lim_{h \to 0}{\frac{3x + 3h + 1 - 3x - 1}{h(\sqrt{3(x + h) + 1} + \sqrt{3x + 1})}} \\
              &=& \lim_{h \to 0}{\frac{3h}{h(\sqrt{3(x + h) + 1} + \sqrt{3x + 1})}} \\
              &=& \lim_{h \to 0}{\frac{3}{\sqrt{3(x + h) + 1} + \sqrt{3x + 1}}} \\
              &=& \frac{3}{\sqrt{3(x + 0) + 1} + \sqrt{3x + 1}} \\
              &=& \frac{3}{\sqrt{3x + 1} + \sqrt{3x + 1}} \\
              &=& \frac{3}{2 \sqrt{3x + 1}}
    \end{eqnarray}
    $$f'(8) = \frac{3}{2\sqrt{3(8) + 1}} = \frac{3}{2\sqrt{24 + 1}} = \frac{3}{2\sqrt{25}} = \frac{3}{2(5)} = \frac{3}{10}$$
    $$y - f(8) = f'(8)(x - 8)$$
    $$y = \frac{3}{10}(x - 8) + 5 = \frac{3}{10}x - \frac{12}{5} + 5 = \frac{3}{10}x + \frac{13}{5}$$
    \item (Section 3.2, Related Exercise 38)
    $$f(x) = \sqrt{x + 2}$$
    $$a = 7$$
    \begin{eqnarray}
        f'(x) &=& \lim_{h \to 0}{\frac{f(x + h) - f(x)}{h}} \\
              &=& \lim_{h \to 0}{\frac{\sqrt{x + h + 2} - \sqrt{x + 2}}{h}} \\
              &=& \lim_{h \to 0}{\frac{\sqrt{x + h + 2} - \sqrt{x + 2}}{h} \cdot \frac{\sqrt{x + h + 2} + \sqrt{x + 2}}{\sqrt{x + h + 2} + \sqrt{x + 2}}} \\
              &=& \lim_{h \to 0}{\frac{x + h + 2 - (x + 2)}{h(\sqrt{x + h + 2} + \sqrt{x + 2})}} \\
              &=& \lim_{h \to 0}{\frac{x + h + 2 - x - 2}{h(\sqrt{x + h + 2} + \sqrt{x + 2})}} \\
              &=& \lim_{h \to 0}{\frac{h}{h(\sqrt{x + h + 2} + \sqrt{x + 2})}} \\
              &=& \lim_{h \to 0}{\frac{1}{\sqrt{x + h + 2} + \sqrt{x + 2}}} \\
              &=& \frac{1}{\sqrt{x + 0 + 2} + \sqrt{x + 2}} \\
              &=& \frac{1}{\sqrt{x + 2} + \sqrt{x + 2}} \\
              &=& \frac{1}{2\sqrt{x + 2}}
    \end{eqnarray}
    $$f'(7) = \frac{1}{2\sqrt{7 + 2}} = \frac{1}{2\sqrt{9}} = \frac{1}{2 \cdot 3} = \frac{1}{6}$$
    $$y - f(7) = f'(7)(x - 7)$$
    $$y = \frac{1}{6}(x - 7) + 3 = \frac{1}{6}x - \frac{7}{6} + 3 = \frac{1}{6}x + \frac{11}{6}$$
    \item (Section 3.2, Related Exercise 25)
        $$f(x) = \frac{1}{x + 1}$$
        $$a = - \frac{1}{2}, 5$$
        \begin{eqnarray}
            f'(x) &=& \lim_{h \to 0}{\frac{f(x + h) - f(x)}{h}} \\
                  &=& \lim_{h \to 0}{\frac{\frac{1}{x + h + 1} - \frac{1}{x+1}}{h}} \\
                  &=& \lim_{h \to 0}{\frac{\frac{1}{x + h + 1} - \frac{1}{x+1}}{h}} \\
                  &=& \lim_{h \to 0}{\frac{\frac{x + 1 - (x + h + 1)}{(x + h + 1)(x + 1)}}{h}} \\
                  &=& \lim_{h \to 0}{\frac{x + 1 - x - h - 1}{h(x + h + 1)(x + 1)}} \\
                  &=& \lim_{h \to 0}{\frac{-h}{h(x + h + 1)(x + 1)}} \\
                  &=& \lim_{h \to 0}{\frac{-1}{(x + h + 1)(x + 1)}} \\
                  &=& \frac{-1}{(x + 0 + 1)(x + 1)} \\
                  &=& \frac{-1}{(x + 1)(x + 1)} \\
                  &=& \frac{-1}{(x + 1)^2}
        \end{eqnarray}
        $$f'(- \frac{1}{2}) = \frac{-1}{(- \frac{1}{2} + 1)^2} = \frac{-1}{(\frac{1}{2})^2} = \frac{-1}{\frac{1}{4}} = -1 \cdot 4 = -4$$
        $$f'(5) = \frac{-1}{(5 + 1)^2} = \frac{-1}{(6)^2} = \frac{-1}{36}$$
    \item (Section 3.2, Related Exercise 27)
        $$f(t) = \frac{1}{\sqrt{t}}$$
        $$a = 9, \frac{1}{4}$$
        \begin{eqnarray}
            f'(t) &=& \lim_{h \to 0}{\frac{f(t + h) - f(t)}{h}} \\
                  &=& \lim_{h \to 0}{\frac{\frac{1}{\sqrt{t + h}} - \frac{1}{\sqrt{t}}}{h}} \\
                  &=& \lim_{h \to 0}{\frac{\frac{\sqrt{t} - \sqrt{t + h}}{\sqrt{t + h}\sqrt{t}}}{h}} \\
                  &=& \lim_{h \to 0}{\frac{\frac{\sqrt{t} - \sqrt{t + h}}{\sqrt{t + h}\sqrt{t}} \cdot \frac{\sqrt{t} + \sqrt{t + h}}{\sqrt{t} + \sqrt{t + h}}}{h}} \\
                  &=& \lim_{h \to 0}{\frac{t - (t + h)}{h(t\sqrt{t + h} + (t + h)\sqrt{t})}} \\
                  &=& \lim_{h \to 0}{\frac{- h}{h(t\sqrt{t + h} + (t + h)\sqrt{t})}} \\
                  &=& \lim_{h \to 0}{\frac{- 1}{t\sqrt{t + h} + (t + h)\sqrt{t}}} \\
                  &=& \frac{- 1}{t\sqrt{t + 0} + (t + 0)\sqrt{t}} \\
                  &=& \frac{- 1}{t\sqrt{t} + t\sqrt{t}} \\
                  &=& \frac{- 1}{2(t\sqrt{t})}
        \end{eqnarray}
        $$f'(9) = \frac{-1}{2 \left (9\sqrt{9} \right )} = \frac{-1}{2 \left (9 \cdot 3 \right )} = \frac{-1}{2\left(27\right)} = \frac{-1}{54}$$
        $$f'\left(\frac{1}{4}\right) = \frac{-1}{2\left(\frac{1}{4}\sqrt{\frac{1}{4}}\right)} = \frac{-1}{2\left(\frac{1}{4} \cdot \frac{1}{2}\right)} = \frac{-1}{2\left(\frac{1}{8}\right)} = \frac{-1}{\frac{2}{8}} = -1\left(4\right) = -4$$
    \item (Section 3.2, Related Exercise 53)
    \\ $f$ is not continuous at $x = 1$
    \\ $f$ is not differentiable at $x = 1, 2$
    \item (Section 3.2, Related Exercise 54)
    \\ $f$ is not continuous at $x = 1$
    \\ $f$ is not differentiable at $x = 1, 2$
    \item (Section 3.3, Related Exercise 19)
        $$\frac{d}{dx}\left(x^5\right) = 5x^4$$
    \item (Section 3.3, Related Exercise 22)
        $$\frac{d}{dx}\left(e^3\right) = 0$$
    \item (Section 3.3, Related Exercise 23)
        $$\frac{d}{dx}\left(5x^3\right) = 15x^2$$
    \item (Section 3.3, Related Exercise 24)
        $$\frac{d}{dx}\left(\frac{5}{6}w^{12}\right) = 10w^{11}$$
    \item (Section 3.3, Related Exercise 28)
        $$\frac{d}{dx}\left(6\sqrt{t}\right) = \frac{3}{\sqrt{t}}$$
    \item (Section 3.3, Related Exercise 31)
        $$\frac{d}{dx}\left(3x^4 + 7x\right) = 12x^3 + 7$$
    \item (Section 3.3, Related Exercise 33)
        $$\frac{d}{dx}\left(10x^4 - 32x + e^2\right) = 40x^3 - 32$$
    \item (Section 3.3, Related Exercise 60)
        $$f(x) = x^3 - 4x^2 + 2x - 1$$
        $$a = 2$$
        $$f'(x) = 3x^2 - 8x + 2$$
        $$y = -2x - 1$$
    \item (Section 3.3, Related Exercise 61)
        $$f(x) = e^x$$
        $$a = \ln{3}$$
        $$f'(x) = e^x$$
        $$y = 3x - 3(\ln{3}) + 3$$
    \item (Section 3.3, Related Exercise 63)
        $$f(x) = x^2 - 6x + 5$$
        $$f'(x) = 2x - 6$$
        $$y = f(x) \text{ is } 0 \text{, when } x = 3$$
        $$y = f(x) \text{ is } 2 \text{, when } x = 4$$
    \item (Section 3.3, Related Exercise 64)
        $$f(t) = t^3 - 27t + 5$$
        $$f'(t) = 3t^2 - 27$$
        $$y = f(t) \text{ is } 0 \text{, when } x = 3$$
        $$y = f(t) \text{ is } 21 \text{, when } x = 4$$
    \item (Section 3.3, Related Exercise 69)
        $$f(x) = 5x^4 + 10x^3 + 3x + 6$$
        $$f'(x) = 20x^3 + 30x^2 + 3$$
        $$f''(x) = 60x^2 + 60x$$
        $$f'''(x) = 120x + 60$$
    \item (Section 3.3, Related Exercise 70)
        $$f(x) = 3x^2 + 5e^x$$
        $$f'(x) = 6x + 5e^x$$
        $$f''(x) = 6 + 5e^x$$
        $$f'''(x) = 5e^x$$
    \item (Section 3.4, Related Exercise 19)
        $$\frac{d}{dx}\left(3x^4(2x^2 - 1)\right) = 12x^3(2x^2 - 1) + 3x^4(4x) = 36x^5 - 12x^3$$
    \item (Section 3.4, Related Exercise 20)
        $$\frac{d}{dx}\left(6x - 2xe^x\right) = 6 - 2(e^x + 2xe^x) = 6 - 2e^x - 2xe^x$$
    \item (Section 3.4, Related Exercise 22)
        \begin{eqnarray}
            \frac{d}{dx}\left(\frac{x^3-4x^2+x}{x-2}\right) &=& \frac{(3x^2 - 8x + 1)(x-2) - (x^3 - 4x^2 + x)}{(x-2)^2} \\
            &=& \frac{3x^3 - 6x^2 - 8x^2 + 16x + x - 2 - x^3 + 4x^2 - x}{(x-2)^2} \\
            &=& \frac{2x^3 + -10x^2 + 16x - 2}{(x-2)^2}
        \end{eqnarray}
    \item (Section 3.4, Related Exercise 27)
        \begin{eqnarray}
            \frac{d}{dx}\left(xe^{-x}\right) &=& \frac{d}{dx}\left(\frac{x}{e^x}\right) \\
            &=& \frac{e^x - xe^x}{\left(e^x\right)^2} \\
            &=& \frac{e^x(1 - x)}{\left(e^x\right)^2} \\
            &=& \frac{1 - x}{e^x} \\
            &=& e^{-x}\left(1 - x\right)
        \end{eqnarray}
    \item (Section 3.4, Related Exercise 61)
        $$f(x) = \frac{x + 5}{x - 1}$$
        $$a = 3$$
        \begin{eqnarray}
            \frac{d}{dx}\left(\frac{x+5}{x-1}\right) &=& \frac{(x-1) - (x+5)}{(x-1)^2} \\
            &=& \frac{x - 1 - x - 5}{(x-1)^2} \\
            &=& \frac{- 6}{(x-1)^2}
        \end{eqnarray}
        $$y = - \frac{3}{2}x + \frac{17}{2}$$
    \item (Section 3.4, Related Exercise 62)
        $$f(x) = \frac{2x^2}{3x - 1}$$
        $$a = 1$$
        \begin{eqnarray}
            \frac{d}{dx}\left(\frac{2x^2}{3x - 1}\right) &=& \frac{4x(3x - 1) - 3(2x^2)}{(3x - 1)^2} \\
            &=& \frac{12x^2 - 4x - 6x^2}{(3x - 1)^2} \\
            &=& \frac{6x^2 - 4x}{(3x - 1)^2}
        \end{eqnarray}
        $$y = \frac{1}{2}x + \frac{1}{2}$$
    \item (Section 3.4, Related Exercise 28)
        \begin{eqnarray}
            \frac{d}{dx}\left(e^x \sqrt[3]{x}\right) &=& e^x \sqrt[3]{x} + \frac{e^x}{3x^{\frac{2}{3}}}
        \end{eqnarray}
    \item (Section 3.4, Related Exercise 39)
        \begin{eqnarray}
            \frac{d}{dx}\left(3x^{-9}\right) &=& -27x^{-10} \\
                                             &=& -\frac{27}{x^{10}}
        \end{eqnarray}
    \item (Section 3.4, Related Exercise 43)
        \begin{eqnarray}
            \frac{d}{dt}\left(\frac{t^3 + 3t^2 + t}{t^3}\right) &=& \frac{t^3(3t^2 + 6t + 1) - 3t^2(t^3 + 3t^2 + t)}{(t^3)^2} \\
            &=& \frac{3t^5 + 6t^4 + t^3 - 3t^5 - 9t^4 - 3t^3}{(t^3)^2} \\
            &=& \frac{-3t^4 - 2t^3}{t^6} \\
            &=& \frac{-3t^4}{t^6} - \frac{2t^3}{t^6} \\
            &=& \frac{-3}{t^2} - \frac{2}{t^3}
        \end{eqnarray}
    \item (Section 3.4, Related Exercise 51)
        \begin{eqnarray}
            \frac{d}{dw}\left(\frac{w^{\frac{5}{3}}}{w^{\frac{5}{3}} + 1}\right) &=& ? \\
        \end{eqnarray}
    \item (Section 3.4, Related Exercise 45)
        \begin{eqnarray}
            \frac{d}{dx}\left(\frac{(x + 1)e^x}{x - 2}\right) &=& ? \\
        \end{eqnarray}
    \item (Section 3.4, Related Exercise 46)
        \begin{eqnarray}
            \frac{d}{dx}\left(\frac{(x - 1)(2x^2 - 1)}{(x^3 - 1)}\right) &=& ? \\
        \end{eqnarray}
\end{enumerate}

\blfootnote{A copy of my notes (in \LaTeX) are available on my \href{https://github.com/onlinechronically/MATH-211}{GitHub}}
\end{document}
