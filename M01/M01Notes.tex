\documentclass{article}
\usepackage{graphicx}
\usepackage{amsthm}
\usepackage{amsmath}
\usepackage{amssymb}
\usepackage{geometry}
\usepackage{tikz}
\usepackage[hidelinks]{hyperref}
\usetikzlibrary{arrows}

\geometry{a4paper, total={170mm,257mm}, left=20mm, top=20mm}
\AtBeginEnvironment{align}{\setcounter{equation}{0}} 
\AtBeginEnvironment{eqnarray}{\setcounter{equation}{0}} 

\newcommand\blfootnote[1]{
    \begingroup
    \renewcommand\thefootnote{}\footnote{#1}
    \addtocounter{footnote}{-1}
    \endgroup
}

\title{Module 1 Notes (MATH-211)}
\author{Lillie Donato}
\date{10 June 2024}

\begin{document}

\maketitle

\section*{General Notes (and Definitions)}
\begin{itemize}
    \item Limit Definition(s):
        \begin{itemize}
            \item Simple: The value that the outputs of a function approach as inputs approach a certain value
            \item Preliminary: Suppose a function $f$ is defined for all $x$ near $a$ except possibly at $a$. If $f(x)$ is arbitrarily close to $L$ all $x$ sufficiently close (but not equal) to $a$, we write the following.
        \end{itemize}
        $$\lim_{x \to a}=L$$
    \item Secant Line: a line passing through two points $(t_0, s(t_0))$ and $(t_1, s(t_1))$. The slope is given by
    $$\frac{s(t_1)-s(t_0)}{t_1-t_0}$$
    \item Tangent Line: the line passing through $(t_0, s(t_0))$ with slope $$\lim_{t \to t_0}\frac{s(t)-s(t_0)}{t-t_0}$$
    \item One Sided limits:
        \begin{itemize}
            \item Right-hand (Definition): Suppose a function $f$ is defined for all $x$ near $a$ with $x > a$. If $f(x)$ is arbitrarily close to $L$ for all $x$ sufficiently close to $a$ with $x > a$ we write
            $$\lim_{x \to a^+}{f(x) = L}$$
            \item Left-hand (Definition): Suppose a function $f$ is defined for all $x$ near $a$ with $x < a$. If $f(x)$ is arbitrarily close to $L$ for all $x$ sufficiently close to $a$ with $x < a$ we write
            $$\lim_{x \to a^-}{f(x) = L}$$
            \item In order for their to be a double sided limit, we must have:
            $$\lim_{x \to a^-}{f(x)} = \lim_{x \to a^+}{f(x)}$$
            \item If the limits from sides are not equal, then a the double sided limit, "does not exist"
        \end{itemize}
    \item Limits can be simplified/solved in an easier way (as compared to numerically/graphically) using Limit Rules/Laws
    \item Limit Example Types:
        \begin{itemize}
            \item Tangent lines
            \item Velocity
        \end{itemize}
    \item Velocity
        \begin{itemize}
            \item Average Veolcity
            \begin{itemize}
                \item The average velocity over some interval $[t_0, t_1]$ is defined as
                $$v_{av} = \frac{s(t_1) - s(t_0)}{t_1 - t_0}$$
            \end{itemize}
            \item Instantaneous Veolcity
            \begin{itemize}
                \item The average velocity over some interval $[t_0, t_1]$ is defined as
                $$v_{inst} = \lim_{t \to a}{v_{av}} = = \frac{s(t) - s(a)}{t - a}$$
            \end{itemize}
        \end{itemize}
    \item Solving Techniques
        \begin{itemize}
    	    \item Factoring and canceling out
    	    \item Using conjugates
            \begin{itemize}
    	    	\item When direct substitution is not possible, you may rationalize the numerator
            \end{itemize}
        \end{itemize}
	\item Infinite Limits: In either case, the limit does not exist (not a real number) if it is infinite
	\begin{itemize}
		\item Suppose $f$ is defined for all $x$ near $a$. If $f(x)$ gorws arbitrarily large for all $x$ sufficinetly close (but not equal) to $a$, we write
		$$\lim_{x \to a}{f(x)} = \infty$$
		\item If $f(x)$ is negative and gorws arbitrarily large in magnitude for all $x$ sufficinetly close (but not equal) to $a$, we write
		$$\lim_{x \to a}{f(x)} = - \infty$$
		\item The line $x = a$ is a vertical asymptote for $f$ if any of the following hold
		$$\lim_{x \to a}{f(x)} = \pm \infty$$
		$$\lim_{x \to a^+}{f(x)} = \pm \infty$$
		$$\lim_{x \to a^-}{f(x)} = \pm \infty$$
		\item A vertical asymptote exists at $x = a$ if any one sided limit as $x \to a$ is $\infty$ or $- \infty$
		\item If you have a limit of a rational function, where $p(a) = L \neq 0$ and $q(a) = 0$, then the one sided limits for $\frac{p(x)}{q(x)}$ approach $\pm \infty$
		$$\lim_{x \to a}{\frac{p(x)}{q(x)}} = \frac{L}{0}$$
	\end{itemize}
\end{itemize}
\section*{Limit Rules/Laws}
$$\text{Assume }\lim_{x \to a}{f(x)}\text{ and }\lim_{x \to a}{g(x)}\text{ exist.}$$
The following properties hold where $c$ is a real number, and $n > 0$ is an integer.
\begin{itemize}
\item \textbf{Sum Rule}
  $$\lim_{x \to a}{(f(x) + g(x))} = \lim_{x \to a}{f(x)} + \lim_{x \to a}{g(x)}$$
\item \textbf{Difference Rule}
  $$\lim_{x \to a}{(f(x) - g(x))} = \lim_{x \to a}{f(x)} - \lim_{x \to a}{g(x)}$$
\item \textbf{Constant Multiple Rule}
  $$\lim_{x \to a}{(cf(x))} = c \lim_{x \to a}{f(x)}$$
\item \textbf{Product Rule}
  $$\lim_{x \to a}{(f(x)g(x))} = (\lim_{x \to a}{f(x)})(\lim_{x \to a}{g(x)})$$
\item \textbf{Quotient Rule}
  $$\lim_{x \to a}{\frac{f(x)}{g(x)}} = \frac{\lim\limits_{x \to a}{f(x)}}{\lim\limits_{x \to a}{g(x)}}\text{, provided}\lim_{x \to a}{g(x) \neq 0}$$
\item \textbf{Power Rule}
$$\lim_{x \to a}{f(x)^n} = (\lim_{x \to a}{f(x)})^n$$
\item \textbf{Root Rule}
$$\lim_{x \to a}{\sqrt[n]{f(x)}} = \sqrt[n]{\lim_{x \to a}{f(x)}}\text{, provided} f(x) > 0 \text{, for } x \text{ near } a \text{, if } n \text{ is even}$$
\item \textbf{Polynomials} \\
    A \textbf{Polynomial} is defined as A function of the form $x_n x^n + a_{n-1} x^{n-1} + ... + a_1x + a_0$ where $n \geq 0$ is an integer
    If $p(x)$ is a polynomial then:
    $$\lim_{x \to a}{p(x)} = p(a)$$
    If $p(x)$ and $q(x)$ are polynomials and $q(a) \neq 0$ then (Direct Substitution):
    $$\lim_{x \to a}{\frac{p(x)}{q(x)}} = \frac{p(a)}{q(a)}$$
\item \textbf{The Squeeze Theorem} \\
    Assume for some functions $f$, $g$ and $h$ that satisfy $f(x) \leq g(x) \leq h(x)$ for $x$ near $a$ (except possibly at $x = a$). If
    $$\lim_{x \to a}{f(x)} = \lim_{x \to a}{h(x)} = L$$
    then
    $$\lim_{x \to a}{g(x)} = L$$
    As $x \to a, h(x) \to L$. Therefore, $g(x) \to L$.
    As $x$ approaches $a$, if $f$ and $h$ approach the same value, so does $g$.
\end{itemize}
\section*{Examples}
\begin{enumerate}
    \item (Describing Limits) As $x$ approaches $3$, $x^2$ approaches $9$
    $$\lim_{x \to 3}{x^2 = 9}$$
    \item (Common Use) Values that are undefined can still have limits, given a graph $G$ where $f(3) = \text{undefined}$ ($f(3)$ is a hole), the following limit is valid:
    $$\lim_{x \to 3}{f(x) = 4}$$
    \item Calculating Limits Numerically:
	$$f(x) = \frac{x^3-1}{x-1}$$
	\begin{tabular}{| c | c | c | c |}
        \hline
	    0.9 & 0.99 & 0.999 & 0.9999 \\
        \hline
	    2.71 & 2.9701 & 2.997001 & 2.99970001 \\
        \hline
    \end{tabular}
	  

	\begin{tabular}{| c | c | c | c |}
        \hline
	    1.1 & 1.01 & 1.001 & 1.0001 \\
        \hline
	    3.31 & 3.0301 & 3.003001 & 3.00030001 \\
        \hline
    \end{tabular} \\
	As $x$ approaches $1$, $f(x)$ approaches $3$: $\lim\limits_{x \to 1}{\frac{x^3-1}{x-1}} = 3$

    \item Calculating One-sided limits:
	$$g(x) = \frac{x^3 - 4x}{8|x-2|}$$
	  
	\begin{tabular}{| c | c | c | c |}
        \hline
	    1.9 & 1.09 & 1.009 & 1.0009 \\
        \hline
	    -0.92625 & -0.9925125 & -0.999250125 & -0.9999250013 \\
        \hline
    \end{tabular} \\
	  
	\begin{tabular}{| c | c | c | c |}
        \hline
	    2.1 & 2.01 & 2.001 & 2.0001 \\
        \hline
	    1.07625 & 1.0075125 & 1.000750125 & 1.000075001 \\
        \hline
    \end{tabular} \\
	  
	$$\lim_{x \to 2}{g(x)} = \text{Does not exist}$$
	$$\lim_{x \to 2^-}{g(x)} = -1$$
	$$\lim_{x \to 2^+}{g(x)} = 1$$

    \item Calculating piecewise function limits
	    $$f(x) = \begin{cases}3 - x \text{ if } x < 2 \\ x - 1 \text{ if } x > 2 \end{cases}$$
	    $$a = 2$$
	  
	\begin{tabular}{| c | c | c | c |}
        \hline
	    1.92 & 1.99 & 1.999 & 1.9999 \\
        \hline
	    1.1 & 1.01 & 1.001 & 1.0001 \\
        \hline
    \end{tabular} \\
	  
	\begin{tabular}{| c | c | c | c |}
        \hline
	    2.1 & 2.01 & 2.001 & 2.0001 \\
        \hline
	    1.1 & 1.01 & 1.001 & 1.0001 \\
        \hline
    \end{tabular} \\
	  
	\textbf{Explanation}: Since $f(2)$ is not defined within the piece wise function, a graph representing this function would have a whole where $x = a$ and have two lines with inverse slopes
	      
	$$f(a) = \text{undefined}$$
	$$\lim_{x \to a}{f(x)} = 1$$
	$$\lim_{x \to a^-}{f(x)} = 1$$
	$$\lim_{x \to a^+}{f(x)} = 1$$

    \item Limit Rules/Laws:
    \begin{enumerate}
        \item Definitions:
    		$$\lim\limits_{x \to 3}{f(x)} = 2$$
    		$$\lim\limits_{x \to 3}{g(x)} = -1$$
    		$$\lim\limits_{x \to 3}{h(x)} = 6$$
        \item Problems:
        \begin{enumerate}
    		\item Sum, Constant Multiple
    			\begin{eqnarray}
    			\lim_{x \to 3}{(f(x) + 2g(x))} &=& \lim_{x \to 3}{f(x)} + \lim_{x \to 3}{2g(x)} \\
    			&=& \lim_{x \to 3}{f(x)} + 2(\lim_{x \to 3}{g(x)}) \\
    			&=& 2 + 2(-1) \\
    			&=& 0
    			\end{eqnarray}
    		\item Quotient
    			\begin{eqnarray}
    			\lim_{x \to 3}{\frac{h(x)}{g(x)}} &=& \frac{\lim\limits_{x \to 3}{h(x)}}{\lim\limits_{x \to 3}{g(x)}} \\
    			&=& \frac{6}{-1} \\
    			&=& -6
    			\end{eqnarray}
    		\item Quotient, Root, Difference
    			\begin{eqnarray}
    			\lim_{x \to 3}{\frac{h(x)}{\sqrt{f(x) - g(x)}}} &=& \frac{\lim\limits_{x \to 3}{h(x)}}{\lim\limits_{x \to 3}{\sqrt{f(x) - g(x)}}}\\
    			&=& \frac{\lim\limits_{x \to 3}{h(x)}}{\sqrt{\lim\limits_{x \to 3}{(f(x) - g(x))}}} \\
    			&=& \frac{\lim\limits_{x \to 3}{h(x)}}{\sqrt{\lim\limits_{x \to 3}f(x) - \lim\limits_{x \to 3}g(x)}} \\
    			&=& \frac{6}{\sqrt{2 + 1}} \\
    			&=& \frac{6}{\sqrt{3}} \\
    			&=& 2 \sqrt{3}
    			\end{eqnarray}
        \end{enumerate}
    \end{enumerate}
    \item
    	\begin{eqnarray}
    	    \lim_{x \to 1}{\frac{3x^2 - 7x + 1}{x + 2}} &=& \frac{3(1)^2 - 7(1) + 1}{1 + 2} \\
    	    &=& \frac{3 - 7 + 1}{1+2} \\
    	    &=& \frac{-3}{3} \\
    	    &=& -1
    	\end{eqnarray}
    \item
    	\begin{eqnarray}
    	    \lim_{x \to 4}{\frac{\left ( \frac{1}{x} - \frac{1}{4} \right )}{x - 4}} &=& \lim_{x \to 4}{\frac{\left ( \frac{4}{4x} - \frac{x}{4x} \right )}{x - 4}} \\
    	    &=& \lim_{x \to 4}{\frac{\left ( \frac{4 - x}{4x} \right )}{x - 4}} \\
    	    &=& \lim_{x \to 4}{\frac{\left ( \frac{4 - x}{4x} \right )}{\left ( \frac{x - 4}{1} \right )}} \\
    	    &=& \lim_{x \to 4}{ \left (\frac{4 - x}{4x} \right ) \left (\frac{1}{x-4} \right )} \\
    	    &=& \lim_{x \to 4}{\frac{4 - x}{4x(x - 4)}} \\
    	    &=& \lim_{x \to 4}{\frac{-(-4 + x)}{4x(x - 4)}} \\
    	    &=& \lim_{x \to 4}{\frac{-(x - 4)}{4x(x - 4)}} \\
    	    &=& \lim_{x \to 4}{\frac{-1}{4x}} \\
    	    &=& \lim_{x \to 4}{\frac{-1}{4(4)}} \\
    	    &=& -\frac{1}{16}
    	\end{eqnarray}
    \item
    	\begin{eqnarray}
    	    \lim_{x \to 9}{\frac{x - 9}{\sqrt{x} - 3}} &=& \lim_{x \to 9}{\frac{x - 9}{\sqrt{x} - 3} \cdot \frac{\sqrt{x} + 3}{\sqrt{x} + 3}} \\
    	    &=& \lim_{x \to 9}{\frac{(x - 9)(\sqrt{x} + 3)}{(\sqrt{x} - 3)(\sqrt{x} + 3)}} \\
    	    &=& \lim_{x \to 9}{\frac{(x - 9)(\sqrt{x} + 3)}{x - 9}} \\
    	    &=& \lim_{x \to 9}{\sqrt{x} + 3} \\
    	    &=& \sqrt{9} + 3 \\
    	    &=& 3 + 3\\
    	    &=& 6
    	\end{eqnarray}
    \item
    	$$1 - \frac{x^2}{2} \leq \cos{x} \leq 1$$
    	\begin{eqnarray}
    	    \lim_{x \to 0}{\left ( 1 - \frac{x^2}{2} \right )} &=& 1 - \frac{0^2}{2} \\
    	    &=& 1 - 0 \\
    	    &=& 1 \\
    	    &=& \lim_{x \to 0}{1} \\
    	    \lim_{x \to 0}{\cos{x}} &=& 1 \hspace{2cm} \text{(By the Squeeze Theorem)}
    	\end{eqnarray}
    \item
    	\begin{eqnarray}
    	    \lim_{x \to 0}{\sin{x}} &=& 0 \hspace{1.5cm} \text{(By the Squeeze Theorem)} \\
    	    \lim_{x \to 0}{\cos{x}} &=& 1 \hspace{1.5cm} \text{(By the Squeeze Theorem)}
    	\end{eqnarray}
    
    	\begin{eqnarray}
    	    \lim_{x \to 0}{\frac{\sin{2x}}{\sin{x}}} &=& \lim_{x \to 0}{\frac{2 \sin{x} \cos{x}}{\sin{x}}} \\
    	    &=& \lim_{x \to 0}{2 \cos{x}} \\
    	    &=& 2 \lim_{x \to 0}{\cos{x}} \\
    	    &=& 2 \cdot 1 \\
    	    &=& 2
    	\end{eqnarray}
	\item Infinite Limits Numerically
		$$f(x) = \frac{x}{(x-2)^2}$$
		\begin{tabular}{| c | c | c | c |}
			\hline
			2.1 & 2.01 & 2.001 & 2.0001 \\
			\hline
			210 & 20100 & 2001000 & 200010000 \\
			\hline
		\end{tabular}

		\begin{tabular}{| c | c | c | c |}
			\hline
			1.9 & 1.99 & 1.999 & 1.9999 \\
			\hline
			190 & 19900 & 1999000 & 199990000 \\
			\hline
		\end{tabular}

		$$\lim_{x \to 2}{f(x)} = \infty$$
	\item Infinite Limits Graphically
		$$\lim_{x \to -2^-}{h(x)} = - \infty$$
		$$\lim_{x \to -2^+}{h(x)} = - \infty$$
		$$\lim_{x \to -2}{h(x)} = - \infty$$
		$$\lim_{x \to 3^-}{h(x)} = \infty$$
		$$\lim_{x \to 3^+}{h(x)} = - \infty$$
		$$\lim_{x \to 3}{h(x)} = \text{Does not exist}$$
	\item Infinite Limits Analytically
		\\ Hint: Look at the signs of the fractions
		$$\frac{x^2-5x+6}{x^4-4x^2} = \frac{(x - 3)(x - 2)}{x^2(x + 2)(x - 2)} = \frac{x-3}{x^2(x+2)}$$
		$$\lim_{x \to -2^+}{\frac{x^2 - 5x + 6}{x^4 - 4x^2}} = \lim_{x \to -2^+}{\frac{x-3}{x^2(x+2)}} = - \infty$$
		$$\lim_{x \to -2^-}{\frac{x^2 - 5x + 6}{x^4 - 4x^2}} = \lim_{x \to -2^-}{\frac{x-3}{x^2(x+2)}} = \infty$$
		$$\lim_{x \to -2}{\frac{x^2 - 5x + 6}{x^4 - 4x^2}} = \text{Does not exist}$$
	\item Infinite Limits Analytically with Square Root
		$$\lim_{x \to 1^+}{\frac{x + 3}{\sqrt{x^2 - 5x + 4}}} = \lim_{x \to 1^+}{\frac{x + 3}{\sqrt{(x-4)(x-1)}}} = \text{Does not exist}$$
		$$\lim_{x \to 1^-}{\frac{x + 3}{\sqrt{x^2 - 5x + 4}}} = \lim_{x \to 1^-}{\frac{x + 3}{\sqrt{(x-4)(x-1)}}} = \infty$$
		$$\lim_{x \to 1}{\frac{x + 3}{\sqrt{x^2 - 5x + 4}}} = \text{Does not exist}$$
	\item Infinite Limit with a Trigonometric Function
		$$\lim_{\theta \to 0^-}{\frac{\sin{\theta}}{\cos^2{\theta} - 1}} = \lim_{\theta \to 0^-}{\frac{\sin{\theta}}{- \sin^2{\theta}}} = \lim_{\theta \to 0^-}{\frac{1}{- \sin^2{\theta}}} = \infty$$
	\item Locating Veritical Asymptotes
		$$f(x) = \frac{x + 7}{x^4 - 49x^2} = \frac{x+7}{x^2(x^2-49)} = \frac{x+7}{x^2((x-7)(x+7))} = \frac{1}{x^2(x-7)}$$
		Denominator is $0$ at $x=0$, $x=-7$, $x=7$ \\
		$x=-7$ does not fit, as it is connected with $x+7$, but cancels out \\
		Vertical Asympotetes: $x = 0$, $x = 7$
    \item (Section 2.1, Related Exercise 13):
    	\\Hint: use the secant line slope formula
    	$$s(t) = -16t^2 + 128t$$
        \begin{enumerate}
    	    \item $[1, 4]$
    	        $$\frac{256 - 112}{4 - 1} = \frac{144}{3} = 48$$
    	    \item $[1, 3]$
    	        $$\frac{240 - 112}{3 - 1} = \frac{128}{2} = 64$$
    	    \item $[1, 2]$
    	        $$\frac{192 - 112}{2 - 1} = \frac{80}{1} = 84$$
    	    \item $[1, 1 + h]$, where $h > 0$ is a real number
    	        $$\frac{112 + -16h^2 + 128h - 112}{1 + h - 1} = \frac{-16h^2 + 128h}{h} = -16h + 128 = 16(-h + 6)$$
        \end{enumerate}
    \item (Section 2.1, Related Exercise 15):
    	Hint: we use the slope formula for the secant line, and the relationship is referring to the interval
    	$$s(t) = -16^t + 100t$$
    	\begin{eqnarray}
    	\frac{s(t_1) - s(t_0)}{t_1 - t_0} &=& \frac{s(2) - s(0.5)}{2 - 0.5} \\
    	&=& \frac{136 - 46}{1.5} \\
    	&=& \frac{90}{1.5} \\
    	&=& 60
    	\end{eqnarray}
    	The slope of this secant line, through the lens of average velocity could be viewed as the average velocity over the interval $[0.5, 2]$
    \item (Section 2.1, Related Exercise 17):
        $$s(t) = -16t^2 + 128t$$
    
	    \begin{tabular}{| c | c | c | c | c |}
            \hline
            $[1,2]$&$[1,1.5]$&$[1,1.1]$&$[1,1.01]$&$[1,1.001]$ \\
            \hline
    	    80 & 88 & 94.4 & 95.84 & 95.984 \\
            \hline
        \end{tabular} \\

    	$$v_{inst} = \lim_{t \to 1}{s(t)} = 96$$
    \item (Section 2.1, Related Exercise 19):
	    $$s(t) = -16t^2 + 100t$$
    
	    \begin{tabular}{| c | c | c | c | c |}
            \hline
	        $[2,3]$&$[2.9,3]$&$[2.99,3]$&$[2.999,3]$&$[2.9999,3]$ \\
            \hline
	        20 & 5.6 & 4.16 & $4.016$ & $4.002$ \\
            \hline
        \end{tabular} \\
    
	    $$v_{inst} = \lim_{t \to 3}{s(t)} = 4$$
    \item (Section 2.2, Related Exercise 3):
        \begin{itemize}
		    \item $h(2) = 5$
		    \item $\lim\limits_{x \to 2}{h(x)} = 3$
		    \item $h(4) = \text{Does not exist}$
		    \item $\lim\limits_{x \to 4}{h(x)} = 1$
		    \item $\lim\limits_{x \to 5}{h(x)} = 2$
        \end{itemize}
    \item (Section 2.2, Related Exercise 4):
        \begin{itemize}
		    \item $g(0) = 0$
		    \item $\lim\limits_{x \to 0}{g(x)} = 1$
		    \item $g(1) = 2$
		    \item $\lim\limits_{x \to 1}{g(x)} = 2$
        \end{itemize}
    \item (Section 2.2, Related Exercise 7):
	    $$f(x) = \frac{x^2 - 4}{x - 2}$$
    
	    \begin{tabular}{| c | c | c | c |}
            \hline
	        1.9 & 1.99 & 1.999 & 1.9999 \\
            \hline
	        3.9 & 3.99 & 3.999 & 3.9999 \\
            \hline
        \end{tabular} \\
    
	    \begin{tabular}{| c | c | c | c |}
            \hline
	        2.1 & 2.01 & 2.001 & 2.0001 \\
            \hline
	        4.1 & 4.01 & 4.001 & 4.0001 \\
            \hline
        \end{tabular} \\
    
	    $$\lim_{x \to 2}{f(x)} = 4$$
    \item (Section 2.2, Related Exercise 8):
	    $$f(x) = \frac{x^3 - 1}{x - 1}$$
    
	    \begin{tabular}{| c | c | c | c |}
            \hline
	        0.9 & 0.99 & 0.999 & 0.9999 \\
            \hline
	        2.71 & 2.9701 & 3.997001 & 3.99970001 \\
            \hline
        \end{tabular} \\
    
	    \begin{tabular}{| c | c | c | c |}
            \hline
	        1.1 & 1.01 & 1.001 & 1.0001 \\
            \hline
	        3.31 & 3.0301 & 3.003001 & 3.00030001 \\
            \hline
        \end{tabular} \\
    
	    $$\lim_{x \to 1}{f(x)} = 3$$
    \item (Section 2.2, Related Exercise 27):
	    $$f(x) = \frac{x-2}{\ln|x-2|}$$
	    $$\lim_{x \to 2}{f(x)} = 2$$
    \item (Section 2.2, Related Exercise 28):
	    $$f(x) = \frac{e^{2x} - 2x - 1}{x^2}$$
	    $$\lim_{x \to 0}{f(x)} = 0$$
    \item (Section 2.2, Related Exercise 19):
	    $$\begin{aligned}f(x) = \begin{cases}
	    x^2 + 1 &\text{ if } x \leq -1 \\
	    3 &\text{ if } x > -1
	    \end{cases}\end{aligned}$$
		$$\lim_{x \to -1^-}{f(x)} = 2$$
		$$\lim_{x \to -1^+}{f(x)} = 3$$
		$$\lim_{x \to -1}{f(x)} = \text{Does not exist}$$
    \item (Section 2.2, Related Exercise 20):
	    $$\begin{aligned}f(x) = \begin{cases}
	    3 - x &\text{ if } x < 2 \\
	    x - 1 &\text{ if } x >2
	    \end{cases}\end{aligned}$$
	    $$\lim_{x \to 2^-}{f(x)} = 1$$
	    $$\lim_{x \to 2^+}{f(x)} = 1$$
	    $$\lim_{x \to 2}{f(x)} = 1$$

    \item (Section 2.3, Related Exercise 19):
	    $$\lim_{x \to 4}{3x-7} = 3(4)-7 = 12 - 7 = 5$$
    \item (Section 2.3, Related Exercise 22):
	    $$\lim_{x \to 6}{4} = 4$$
    \item (Section 2.3, Related Exercise 11): Quotient, Difference
	    \begin{eqnarray}
	    \lim_{x \to 1}{\frac{f(x)}{g(x) - h(x)}} &=& \frac{\lim\limits_{x \to 1}{f(x)}}{\lim\limits_{x \to 1}{g(x) - h(x)}} \\
	    &=& \frac{\lim\limits_{x \to 1}{f(x)}}{\lim\limits_{x \to 1}{g(x) - h(x)}} \\
	    &=& \frac{\lim\limits_{x \to 1}{f(x)}}{\lim\limits_{x \to 1}{g(x)} - \lim\limits_{x \to 1}h(x)} \\
	    &=& \frac{8}{3 - 2} \\
	    &=& \frac{8}{1} \\
	    &=& 8
	    \end{eqnarray}
    \item (Section 2.3, Related Exercise 12): Root, Sum, Product
	    \begin{eqnarray}
	    \lim_{x \to 1}{\sqrt[3]{f(x)g(x) + 3}} &=& \sqrt[3]{\lim_{x \to 1}{f(x)g(x) + 3}} \\
	    &=& \sqrt[3]{\lim_{x \to 1}{f(x)g(x) + 3}} \\
	    &=& \sqrt[3]{\lim_{x \to 1}{f(x)g(x)} + \lim_{x \to 1}{3}} \\
	    &=& \sqrt[3]{\lim_{x \to 1}{f(x)}\lim_{x \to 1}{g(x)} + \lim_{x \to 1}{3}} \\
	    &=& \sqrt[3]{8 \cdot 3 + 3} \\
	    &=& \sqrt[3]{24 + 3} \\
	    &=& \sqrt[3]{27} \\
	    &=& 3
	    \end{eqnarray}
    \item (Section 2.3, Related Exercise 25):
	    \begin{eqnarray}
	    \lim_{x \to 1}{\frac{5x^2+6x+1}{8x-4}} &=& \frac{5(1^2)+6(1)+1}{8(1)-4} \\
	    &=& \frac{5+6+1}{8-4} \\
	    &=& \frac{12}{4} \\
	    &=& 3
	    \end{eqnarray}
    \item (Section 2.3, Related Exercise 26):
	    \begin{eqnarray}
	    \lim_{t \to 3}{\sqrt[3]{t^2-10}} &=& \sqrt[3]{\lim_{t \to 3}{t^2-10}} \\
	    &=& \sqrt[3]{3^2 - 10} \\
	    &=& \sqrt[3]{9 - 10} \\
	    &=& \sqrt[3]{-1} \\
	    &=& -1
	    \end{eqnarray}
    \item (Section 2.3, Related Exercise 27):
	    \begin{eqnarray}
	    \lim_{p \to 2}{\frac{3p}{\sqrt{4p+1}-1}} &=& \frac{\lim\limits_{p \to 2}{3p}}{\lim\limits_{p \to 2}{\sqrt{4p+1}-1}} \\
	    &=& \frac{3(2)}{\sqrt{\lim\limits_{p \to 2}{4p+1}}-1} \\
	    &=& \frac{6}{\sqrt{4(2)+1}-1} \\
	    &=& \frac{6}{\sqrt{8+1}-1} \\
	    &=& \frac{6}{\sqrt{9}-1} \\
	    &=& \frac{6}{3-1} \\
	    &=& \frac{6}{2} \\
	    &=& 3
	    \end{eqnarray}
    \item (Section 2.3, Related Exercise 72):
	    $$g(x) = \begin{aligned}\begin{cases}
	    5x-15 &\text{ if } x < 4 \\
	    \sqrt{6x+1} &\text{ if } x \geq 4
	    \end{cases}\end{aligned}$$
	    $$\lim_{x \to 4^-}{g(x)} = 5$$
	    $$\lim_{x \to 4^+}{g(x)} = 5$$
	    $$\lim_{x \to 4}{g(x)} = 5$$
    \item (Section 2.3, Related Exercise 73):
        $$g(x) = \begin{aligned}
            \begin{cases}
	            x^2+1 &\text{ if } x < -1 \\
	            \sqrt{x+1} &\text{ if } x \geq -1
	        \end{cases}
        \end{aligned}$$
	    $$\lim_{x \to -1^-}{g(x)} = 2$$
	    $$\lim_{x \to -1^+}{g(x)} = 0$$
	    $$\lim_{x \to -1}{g(x)} = \text{Does not exist}$$
    \item (Section 2.3, Related Exercise 34):
	    \begin{eqnarray}
	    \lim_{x \to 3}{\frac{x^2 - 2x - 3}{x - 3}} &=& \lim_{x \to 3}{\frac{(x-3)(x+1)}{x - 3}} \\
	    &=& \lim_{x \to 3}{x+1} \\
	    &=& 3+1 \\
	    &=& 4
	    \end{eqnarray}
    \item (Section 2.3, Related Exercise 41):
	    \begin{eqnarray}
	    \lim_{x \to 9}{\frac{\sqrt{x} - 3}{x - 9}} &=& \lim_{x \to 9}{\frac{\sqrt{x} - 3}{x - 9}} \cdot \frac{\sqrt{x}+3}{\sqrt{x}+3} \\
	    &=& \lim_{x \to 9}{\frac{(\sqrt{x} - 3)(\sqrt{x} + 3)}{(x - 9)(\sqrt{x} + 3)}} \\
	    &=& \lim_{x \to 9}{\frac{x-9}{(x-9)(\sqrt{x}+3)}} \\
	    &=& \lim_{x \to 9}{\frac{1}{\sqrt{x}+3}} \\
	    &=& \frac{1}{\sqrt{9}+3} \\
	    &=& \frac{1}{3+3} \\
	    &=& \frac{1}{6}
	    \end{eqnarray}
    \item (Section 2.3, Related Exercise 69):
	    $$\lim_{x \to 1^+}{\frac{x-1}{\sqrt{x^2-1}}} = \text{Does not exist}$$
    \item (Section 2.3, Related Exercise 70):
	    \begin{eqnarray}
	    \lim_{x \to 1^+}{\frac{x-1}{\sqrt{x^2-1}}} &=& \lim_{x \to 1^+}{\frac{x-1}{\sqrt{x^2-1}} \cdot \frac{x+1}{x+1}} \\
	    &=& \lim_{x \to 1^+}{\frac{x^2-1}{\sqrt{x^2-1} (x+1)}} \\
	    &=& \lim_{x \to 1^+}{\frac{x^2-1}{(x^2-1)^{\frac{1}{2}} (x+1)}} \\
	    &=& \lim_{x \to 1^+}{\frac{(x^2-1)^{\frac{1}{2}}}{x+1}} \\
	    &=& \lim_{x \to 1^+}{\frac{\sqrt{x^2-1}}{x+1}} \\
	    &=& \frac{\sqrt{1-1}}{1+1} \\
	    &=& \frac{\sqrt{0}}{2} \\
	    &=& \frac{0}{2} \\
	    &=& 0
	    \end{eqnarray}
    \item (Section 2.3, Related Exercise 95):
		    $$\frac{2^x - 2^0}{x-0} = \frac{2^x - 1}{x}$$

            \begin{tabular}{| c | c | c | c | c | c |}
                \hline
		        -1 & -0.1 & -0.01 & -0.001 & -0.0001 & -0.00001 \\
                \hline
		        0.5 & 0.6696700846 & 0.6907504563 & 0.6929070095 & 0.6931231585 & 0.6931447783 \\
                \hline
            \end{tabular}

            $$\lim_{x \to 0^1}{\frac{2^x-1}{x}} = 0.693$$
    \item (Section 2.3, Related Exercise 96):
		$$\frac{3^x - 3^0}{x-0} = \frac{3^x - 1}{x}$$

            \begin{tabular}{| c | c | c | c |}
                \hline
		        -0.1 & -0.01 & -0.001 & -0.0001 \\
                \hline
		        1.040415402 & 1.092599583 & 1.098009035 & 1.098551943 \\
                \hline
            \end{tabular}

            \begin{tabular}{| c | c | c | c |}
                \hline
		        0.0001 & 0.001 & 0.01 & 0.1 \\
                \hline
		        1.098672638 & 1.099215984 & 1.104669194 & 1.161231740 \\
                \hline
            \end{tabular}
            
		    $$\lim_{x \to 0^1}{\frac{3^x-1}{x}} = 1.0986$$
    \item (Section 2.3, Related Exercise 81):
	    \\ $-|x| < 0 < |x|$ and $\sin{\frac{1}{x}} \leq 1$, so $|x| \sin{\frac{1}{x}} \leq |x|$ and $-|x| \sin{\frac{1}{x}} \geq -|x|$
		$$\lim_{x \to 0}{-|x|} = -|0| = 0$$
		$$\lim_{x \to 0}{|x|} = |0| = 0$$
		$$\lim_{x \to 0}{x \sin{\frac{1}{x}}} = 0$$
		By the Squeeze Theorem, since $\lim\limits_{x \to 0}{-|x|} = \lim\limits_{x \to 0}{|x|}$ and the functions are chronologically greater than the last
    \item (Section 2.3, Related Exercise 82):
		$$\lim_{x \to 0}{1 - \frac{x^2}{2}} = 1 -\frac{0}{2} = 1 - 0 = 1$$
		$$\lim_{x \to 0}{1} = 1$$
		$$\lim_{x \to 0}{\cos{x}} = 1$$
		By the Squeeze Theorem, since $\lim\limits_{x \to 0}{1 - \frac{x^2}{2}} = \lim\limits_{x \to 0}{1}$ and the functions are chronologically greater than the last
    \item (Section 2.3, Related Exercise 60):
	    \begin{eqnarray}
	    \lim_{x \to 0}{\frac{\sin{2x}}{\sin{x}}} &=& \lim_{x \to 0}{\frac{2\sin{x}\cos{x}}{\sin{x}}} \\
	    &=& \lim_{x \to 0}{2 \cos{x}} \\
	    &=& 2 \cos{0} \\
	    &=& 2 \cdot 1 \\
	    &=& 2
	    \end{eqnarray}
    \item (Section 2.3, Related Exercise 61):
	    \begin{eqnarray}
	    \lim_{x \to 0}{\frac{1 - \cos{x}}{\cos^2{x} - 3 \cos{x} + 2}} &=& \lim_{x \to 0}{\frac{1}{\cos^2{x} - 2 \cos{x} + 2}} \\
	    &=& \lim_{x \to 0}{\frac{1}{\cos{x} \cos{x} - 2 \cos{x} + 2}} \\
	    &=& \frac{1}{\cos{0} \cos{0} - 2 \cos{0} + 2} \\
	    &=& \frac{1}{1 \cdot 1 - 2(1) + 2} \\
	    &=& \frac{1}{1 - 2 + 2} \\
	    &=& \frac{1}{1} \\
	    &=& 1
	    \end{eqnarray}
	\item (Section 2.4, Related Exercise 6):
		$$f(x) = \frac{x}{(x^2 - 2x - 3)^2}$$
		$$\lim_{x \to -1}{f(x)} = - \infty$$
		$$\lim_{x \to 3}{f(x)} = \infty$$
	\item (Section 2.4, Related Exercise 7):
		$$\lim_{x \to 1^-}{f(x)} = \infty$$
		$$\lim_{x \to 1^+}{f(x)} = \infty$$
		$$\lim_{x \to 1}{f(x)} = \infty$$
		$$\lim_{x \to 2^-}{f(x)} = \infty$$
		$$\lim_{x \to 2^+}{f(x)} = - \infty$$
		$$\lim_{x \to 2}{f(x)} = \text{Does not exist}$$
	\item (Section 2.4, Related Exercise 8):
		$$\lim_{x \to 2^-}{g(x)} = \infty$$
		$$\lim_{x \to 2^+}{g(x)} = - \infty$$
		$$\lim_{x \to 2}{g(x)} = \text{Does not exist}$$
		$$\lim_{x \to 4^-}{g(x)} = - \infty$$
		$$\lim_{x \to 4^+}{g(x)} = - \infty$$
		$$\lim_{x \to 4}{g(x)} = - \infty$$
	\item (Section 2.4, Related Exercise 21):
		$$\lim_{x \to 2^+}{\frac{1}{x-2}} = \infty$$
		$$\lim_{x \to 2^-}{\frac{1}{x-2}} = - \infty$$
		$$\lim_{x \to 2}{\frac{1}{x-2}} = \text{Does not exist}$$
	\item (Section 2.4, Related Exercise 22):
		$$\lim_{x \to 3^+}{\frac{2}{(x - 3)^3}} = \infty$$
		$$\lim_{x \to 3^-}{\frac{2}{(x - 3)^3}} = - \infty$$
		$$\lim_{x \to 3}{\frac{2}{(x - 3)^3}} = \text{Does not exist}$$
	\item (Section 2.4, Related Exercise 28):
		$$\lim_{t \to -2^+}{\frac{t^3 - 5t^2 + 6t}{t^4 - 4t^2}} = \lim_{t \to -2^+}{\frac{t(t-2)(t-3)}{t^2(t^2 - 4)}} = \lim_{t \to -2^+}{\frac{t(t-2)(t-3)}{t^2(t-2)(t+2)}} = \lim_{t \to -2^+}{\frac{t(t-3)}{t^2(t+2)}} = \lim_{t \to -2^+}{\frac{t^2 - 3t}{t^3 + 2t^2}} = - \infty$$
		$$\lim_{t \to -2^-}{\frac{t^3 - 5t^2 + 6t}{t^4 - 4t^2}} = \lim_{t \to -2^-}{\frac{t(t-2)(t-3)}{t^2(t^2 - 4)}} = \lim_{t \to -2^-}{\frac{t(t-2)(t-3)}{t^2(t-2)(t+2)}} = \lim_{t \to -2^-}{\frac{t(t-3)}{t^2(t+2)}} = \lim_{t \to -2^-}{\frac{t^2 - 3t}{t^3 + 2t^2}} = - \infty$$
		$$\lim_{t \to -2}{\frac{t^3 - 5t^2 + 6t}{t^4 - 4t^2}} = \lim_{t \to -2}{\frac{t(t-2)(t-3)}{t^2(t^2 - 4)}} = \lim_{t \to -2}{\frac{t(t-2)(t-3)}{t^2(t-2)(t+2)}} = \lim_{t \to -2}{\frac{t(t-3)}{t^2(t+2)}} = \lim_{t \to -2}{\frac{t^2 - 3t}{t^3 + 2t^2}} = - \infty$$
		$$\lim_{t \to 2}{\frac{t^3 - 5t^2 + 6t}{t^4 - 4t^2}} = \lim_{t \to 2}{\frac{t(t-2)(t-3)}{t^2(t^2 - 4)}} = - \frac{1}{8}$$
	\item (Section 2.4, Related Exercise 31): Remember, if you are able to solve by direct substitution after canceling terms (where the denominator does not equal zero), that's your answer
		$$\lim_{x \to 0}{\frac{x - 3}{x^4 - 9x^2}} = \lim_{x \to 0}{\frac{x - 3}{x^2(x-3)(x+3)}} = \lim_{x \to 0}{\frac{1}{x^2(x+3)}} = \lim_{x \to 0}{\frac{1}{x^3+3x^2}} = \infty$$
		$$\lim_{x \to 3}{\frac{x - 3}{x^4 - 9x^2}} = \lim_{x \to 3}{\frac{x - 3}{x^2(x-3)(x+3)}} = \lim_{x \to 3}{\frac{1}{x^2(x+3)}} = \lim_{x \to -3}{\frac{1}{x^3+3x^2}} = \frac{1}{54}$$
		$$\lim_{x \to -3}{\frac{x - 3}{x^4 - 9x^2}} = \lim_{x \to -3}{\frac{x - 3}{x^2(x-3)(x+3)}} = \lim_{x \to -3}{\frac{1}{x^2(x+3)}} = \lim_{x \to -3}{\frac{1}{x^3+3x^2}} = \text{Does not exist}$$
	\item (Section 2.4, Related Exercise 45):
		$$f(x) = \frac{x-5}{x^2 - 25} = \frac{x-5}{(x-5)(x+5)} = \frac{1}{x+5}$$
		Vertical Asymptotes: $x=-5$
		$$\lim_{x \to 5}{f(x)} = \lim_{x \to 5}{\frac{1}{x+5}} = \frac{1}{5+5} = \frac{1}{10}$$
		$$\lim_{x \to -5^-}{f(x)} = \lim_{x \to -5^-}{\frac{1}{x+5}} = - \infty$$
		$$\lim_{x \to -5^+}{f(x)} = \lim_{x \to -5^+}{\frac{1}{x+5}} = \infty$$
	\item (Section 2.4, Related Exercise 46):
		$$f(x) = \frac{x+7}{x^4 - 49x^2} = \frac{x+7}{x^2(x^2 - 49)} = \frac{x+7}{x^2(x+7)(x-7)} = \frac{1}{x^2(x-7)} = \frac{1}{x^3 - 7x^2}$$
		Vertical Asymptotes: $x = 0$, $x = 7$, $x = -7$
		$$\lim_{x \to 7^-}{f(x)} = \lim_{x \to 7^-}{\frac{1}{x^3 - 6x^2}} = - \infty$$
		$$\lim_{x \to 7^+}{f(x)} = \lim_{x \to 7^+}{\frac{1}{x^3 - 6x^2}} = \infty$$
		$$\lim_{x \to -7}{f(x)} = \lim_{x \to -7}{\frac{1}{x^3 - 7x^2}} = \text{Does not exist}$$
		$$\lim_{x \to 0}{f(x)} = \lim_{x \to 0}{\frac{1}{x^3 - 7x^2}} = - \infty$$
	\item (Section 2.4, Related Exercise 39):
		$$\lim_{\theta \to 0^+}{\csc{\theta}} = \infty$$
	\item (Section 2.4, Related Exercise 40):
		$$\lim_{x \to 0^-}{\csc{x}} = - \infty$$
\end{enumerate}
\blfootnote{A copy of my notes (in \LaTeX) are available on my \href{https://github.com/onlinechronically/MATH-211}{GitHub}}
\end{document}
