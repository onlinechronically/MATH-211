\documentclass{article}
\usepackage{graphicx}
\usepackage{amsthm}
\usepackage{amsmath}
\usepackage{amssymb}
\usepackage{geometry}
\usepackage{tikz}
\usepackage[hidelinks]{hyperref}
\usetikzlibrary{arrows}

\geometry{a4paper, total={170mm,257mm}, left=20mm, top=20mm}
\AtBeginEnvironment{align}{\setcounter{equation}{0}} 
\AtBeginEnvironment{eqnarray}{\setcounter{equation}{0}} 

\newcommand\blfootnote[1]{
    \begingroup
    \renewcommand\thefootnote{}\footnote{#1}
    \addtocounter{footnote}{-1}
    \endgroup
}

\title{Module 1 Notes (MATH-211)}
\author{Lillie Donato}
\date{10 June 2024}

\begin{document}

\maketitle

\section*{General Notes (and Definitions)}
\begin{itemize}
    \item Limit Definition(s):
        \begin{itemize}
            \item Simple: The value that the outputs of a function approach as inputs approach a certain value
            \item Preliminary: Suppose a function $f$ is defined for all $x$ near $a$ except possibly at $a$. If $f(x)$ is arbitrarily close to $L$ all $x$ sufficiently close (but not equal) to $a$, we write the following.
        \end{itemize}
        $$\lim_{x \to a}=L$$
    \item Secant Line: a line passing through two points $(t_0, s(t_0))$ and $(t_1, s(t_1))$. The slope is given by
    $$\frac{s(t_1)-s(t_0)}{t_1-t_0}$$
    \item Tangent Line: the line passing through $(t_0, s(t_0))$ with slope $$\lim_{t \to t_0}\frac{s(t)-s(t_0)}{t-t_0}$$
    \item One Sided limits:
        \begin{itemize}
            \item Right-hand (Definition): Suppose a function $f$ is defined for all $x$ near $a$ with $x > a$. If $f(x)$ is arbitrarily close to $L$ for all $x$ sufficiently close to $a$ with $x > a$ we write
            $$\lim_{x \to a^+}{f(x) = L}$$
            \item Left-hand (Definition): Suppose a function $f$ is defined for all $x$ near $a$ with $x < a$. If $f(x)$ is arbitrarily close to $L$ for all $x$ sufficiently close to $a$ with $x < a$ we write
            $$\lim_{x \to a^-}{f(x) = L}$$
            \item In order for their to be a double sided limit, we must have:
            $$\lim_{x \to a^-}{f(x)} = \lim_{x \to a^+}{f(x)}$$
            \item If the limits from sides are not equal, then a the double sided limit, "does not exist"
        \end{itemize}
    \item Limits can be simplified/solved in an easier way (as compared to numerically/graphically) using Limit Rules/Laws
    \item Limit Example Types:
        \begin{itemize}
            \item Tangent lines
            \item Velocity
        \end{itemize}
    \item Velocity
        \begin{itemize}
            \item Average Veolcity
            \begin{itemize}
                \item The average velocity over some interval $[t_0, t_1]$ is defined as
                $$v_{av} = \frac{s(t_1) - s(t_0)}{t_1 - t_0}$$
            \end{itemize}
            \item Instantaneous Veolcity
            \begin{itemize}
                \item The average velocity over some interval $[t_0, t_1]$ is defined as
                $$v_{inst} = \lim_{t \to a}{v_{av}} = = \frac{s(t) - s(a)}{t - a}$$
            \end{itemize}
        \end{itemize}
    \item Solving Techniques
        \begin{itemize}
    	    \item Factoring and canceling out
    	    \item Using conjugates
            \begin{itemize}
    	    	\item When direct substitution is not possible, you may rationalize the numerator
            \end{itemize}
        \end{itemize}
	\item Infinite Limits: In either case, the limit does not exist (not a real number) if it is infinite
	\begin{itemize}
		\item Suppose $f$ is defined for all $x$ near $a$. If $f(x)$ gorws arbitrarily large for all $x$ sufficiently close (but not equal) to $a$, we write
		$$\lim_{x \to a}{f(x)} = \infty$$
		\item If $f(x)$ is negative and gorws arbitrarily large in magnitude for all $x$ sufficiently close (but not equal) to $a$, we write
		$$\lim_{x \to a}{f(x)} = - \infty$$
		\item The line $x = a$ is a vertical asymptote for $f$ if any of the following hold
		$$\lim_{x \to a}{f(x)} = \pm \infty$$
		$$\lim_{x \to a^+}{f(x)} = \pm \infty$$
		$$\lim_{x \to a^-}{f(x)} = \pm \infty$$
		\item A vertical asymptote exists at $x = a$ if any one sided limit as $x \to a$ is $\infty$ or $- \infty$
		\item If you have a limit of a rational function, where $p(a) = L \neq 0$ and $q(a) = 0$, then the one sided limits for $\frac{p(x)}{q(x)}$ approach $\pm \infty$
		$$\lim_{x \to a}{\frac{p(x)}{q(x)}} = \frac{L}{0}$$
	\end{itemize}
	\item Limits as Infinity
	\begin{itemize}
		\item \textbf{Definition}: If $f(x)$ becoomes arbitrarily close to a finite number $L$ for all sufficiently large and positive $x$, the we write
		$$\lim_{x \to \infty}{f(x)} = L$$
		The definition for
		$$\lim_{x \to - \infty}{f(x)} = M$$
		is analogous.
		\item If $\lim\limits_{x \to \infty}{f(x)} = L$ we say that the function $f(x)$ has a horizontal asymptote at $y = L$
		\item If $\lim\limits_{x \to - \infty}{f(x)} = M$ we say that the function $f(x)$ has a horizontal asymptote at $y = M$
		\item \textbf{Principle}: If $n > 0$ is an integer then
		$$\lim_{x \to \pm \infty}{\frac{1}{x^n}} = 0$$
		\item Suppose $f(x) = \frac{p(x)}{q(x)}$ is a rational function where
		$$p(x) = a_mx^m + a_{m-1}x^{x-1} + ... + a_1x + a_0$$
		$$q(x) = b_nx^n + b_{n-1}x^{x-1} + ... + b_1x + b_0$$
		If the degree of $p(x)$ is less than the degree of $q(x)$ then
		$$\lim_{x \to \pm \infty}{f(x)} = 0$$
		If the degree of $p(x)$ equals the degree of $q(x)$ then
		$$\lim_{x \to \pm \infty}{f(x)} = \frac{a_m}{b_n}$$
		If the degree of $p(x)$ is greater than the degree of $q(x)$ then
		$$\lim_{x \to \pm \infty}{f(x)} = - \infty \text{ or } \infty$$
        If the graph of a function $f$ approaches a line (with finite and nonzero slope) as $x \to \pm \infty$, then that line is a slant asymptote/oblique asymptote of $f$
		\item End behaviour for transcendental functions
		$$\lim_{x \to \pm \infty}{\sin{x}} = \text{Does not exist}$$
		$$\lim_{x \to \infty}{e^x} = \infty$$
		$$\lim_{x \to \infty}{e^{-x}} = \lim_{x \to \infty}{\frac{1}{e^{x}}} = 0$$
		$$\lim_{x \to - \infty}{e^x} = 0$$
		$$\lim_{x \to - \infty}{e^{-x}} = \infty$$
		$$\lim_{x \to \infty}{\ln{x}} = \infty$$
		$$\lim_{x \to 0^+}{\ln{x}} = - \infty$$
		\item Continuity \\
		\textbf{Definition}: A function $f$ is continuous at $a$ if
		$$\lim_{x \to a}{f(x)} = f(a)$$
		A function $f$ is continuous at $a$ if
		\begin{enumerate}
			\item $f(a)$ is defined (Removable Discontinuity)
			\item $\lim\limits_{x \to a}{f(x)}$ exists (Jump Discontinuity)
			\item $\lim\limits{x \to a}{f(x)} = f(a)$ (Removable Discontinuity)
		\end{enumerate}
		A function $f$ has an \textbf{Infinite Discontinuity} at $a$ if the function has a Vertical Asymptote at $a$ \\
		Suppose $f$ is a function defined on an interval $I$. We say that $f$ is continuous on interval $I$ if $f$ is continuous at every point on the interior of $I$ and the following hold: \\
		\begin{enumerate}
			\item If $a$ is the the left-hand endpoint of $I$ and $a$ is contained in $I$ then
			$$\lim_{x \to a^+}{f(x)} = f(a) \text{ ($f$ is continuous from the right)}$$
			\item If $b$ is the the righ-hand endpoint of $I$ and $b$ is contained in $I$ then
			$$\lim_{x \to b^-}{f(x)} = f(b) \text{ ($f$ is continuous from the left)}$$
		\end{enumerate}
		\textbf{Theorem}: All of the following functions are continuous on the intervals where they are defined.
		\begin{enumerate}
			\item Polynomials (continuous everywhere)
			\item Rational Functions (continuous except where denominator is zero)
			\item Exponential functions
			\item Logarithmic functions
			\item Trigonometric functions
			\item Inverse trigonometric functions
		\end{enumerate}
		\textbf{Theorem}: If $f$ and $g$ are continuous at $a$, then the following functions are also continuous at $a$. Assume $c$ is a constant and $n > 0$ is an integer.
		\begin{enumerate}
			\item $f + g$
			\item $f - g$
			\item $cf$
			\item $fg$
			\item $\frac{f}{g}$ provided $g(a) \neq 0$
			\item $(f(x))^n$
		\end{enumerate}
		\textbf{Theorem}:
		\begin{enumerate}
			\item A polynomial function is continuous for all $x$
			\item A rational function (a function of the form $\frac{p}{q}$, where $p$ and $q$ are polynomials) is continuous for all $x$ for which $q(x) \neq 0$
		\end{enumerate}
		\textbf{Theorem}: If $g$ is continuous at $a$ and $f$ is continuous at $g(a)$ then the composite function $f \circ g$ is continuous at $a$. \\
		\textbf{Theorem}: Assume $n$ is a positive integer. If $n$ is odd then $(f(x))^{1/n}$ is continuous at all points at which $f$ is continuous. If $n$ is even then $(f(x))^{1/n}$ is continuous at all points $a$ at which $f$ is continuous and $f(a) > 0$ \\
		\textbf{Intermediate Value Theorem}: Suppose $f$ is continuous on the interval $[a, b]$ and $L$ is a number strictly between $f(a)$ and $f(b)$. Then there exists at least one number $c$ in $(a,b)$ satisfying $f(c) = L$.
	\end{itemize}
\end{itemize}
\section*{Limit Rules/Laws}
$$\text{Assume }\lim_{x \to a}{f(x)}\text{ and }\lim_{x \to a}{g(x)}\text{ exist.}$$
The following properties hold where $c$ is a real number, and $n > 0$ is an integer.
\begin{itemize}
\item \textbf{Sum Rule}
  $$\lim_{x \to a}{(f(x) + g(x))} = \lim_{x \to a}{f(x)} + \lim_{x \to a}{g(x)}$$
\item \textbf{Difference Rule}
  $$\lim_{x \to a}{(f(x) - g(x))} = \lim_{x \to a}{f(x)} - \lim_{x \to a}{g(x)}$$
\item \textbf{Constant Multiple Rule}
  $$\lim_{x \to a}{(cf(x))} = c \lim_{x \to a}{f(x)}$$
\item \textbf{Product Rule}
  $$\lim_{x \to a}{(f(x)g(x))} = (\lim_{x \to a}{f(x)})(\lim_{x \to a}{g(x)})$$
\item \textbf{Quotient Rule}
  $$\lim_{x \to a}{\frac{f(x)}{g(x)}} = \frac{\lim\limits_{x \to a}{f(x)}}{\lim\limits_{x \to a}{g(x)}}\text{, provided}\lim_{x \to a}{g(x) \neq 0}$$
\item \textbf{Power Rule}
$$\lim_{x \to a}{f(x)^n} = (\lim_{x \to a}{f(x)})^n$$
\item \textbf{Root Rule}
$$\lim_{x \to a}{\sqrt[n]{f(x)}} = \sqrt[n]{\lim_{x \to a}{f(x)}}\text{, provided} f(x) > 0 \text{, for } x \text{ near } a \text{, if } n \text{ is even}$$
\item \textbf{Polynomials} \\
    A \textbf{Polynomial} is defined as A function of the form $x_n x^n + a_{n-1} x^{n-1} + ... + a_1x + a_0$ where $n \geq 0$ is an integer
    If $p(x)$ is a polynomial then:
    $$\lim_{x \to a}{p(x)} = p(a)$$
    If $p(x)$ and $q(x)$ are polynomials and $q(a) \neq 0$ then (Direct Substitution):
    $$\lim_{x \to a}{\frac{p(x)}{q(x)}} = \frac{p(a)}{q(a)}$$
\item \textbf{The Squeeze Theorem} \\
    Assume for some functions $f$, $g$ and $h$ that satisfy $f(x) \leq g(x) \leq h(x)$ for $x$ near $a$ (except possibly at $x = a$). If
    $$\lim_{x \to a}{f(x)} = \lim_{x \to a}{h(x)} = L$$
    then
    $$\lim_{x \to a}{g(x)} = L$$
    As $x \to a, h(x) \to L$. Therefore, $g(x) \to L$.
    As $x$ approaches $a$, if $f$ and $h$ approach the same value, so does $g$.
\end{itemize}
\section*{Examples}
\begin{enumerate}
    \item (Describing Limits) As $x$ approaches $3$, $x^2$ approaches $9$
    $$\lim_{x \to 3}{x^2 = 9}$$
    \item (Common Use) Values that are undefined can still have limits, given a graph $G$ where $f(3) = \text{undefined}$ ($f(3)$ is a hole), the following limit is valid:
    $$\lim_{x \to 3}{f(x) = 4}$$
    \item Calculating Limits Numerically:
	$$f(x) = \frac{x^3-1}{x-1}$$
	\begin{tabular}{| c | c | c | c |}
        \hline
	    0.9 & 0.99 & 0.999 & 0.9999 \\
        \hline
	    2.71 & 2.9701 & 2.997001 & 2.99970001 \\
        \hline
    \end{tabular}
	  

	\begin{tabular}{| c | c | c | c |}
        \hline
	    1.1 & 1.01 & 1.001 & 1.0001 \\
        \hline
	    3.31 & 3.0301 & 3.003001 & 3.00030001 \\
        \hline
    \end{tabular} \\
	As $x$ approaches $1$, $f(x)$ approaches $3$: $\lim\limits_{x \to 1}{\frac{x^3-1}{x-1}} = 3$

    \item Calculating One-sided limits:
	$$g(x) = \frac{x^3 - 4x}{8|x-2|}$$
	  
	\begin{tabular}{| c | c | c | c |}
        \hline
	    1.9 & 1.09 & 1.009 & 1.0009 \\
        \hline
	    -0.92625 & -0.9925125 & -0.999250125 & -0.9999250013 \\
        \hline
    \end{tabular} \\
	  
	\begin{tabular}{| c | c | c | c |}
        \hline
	    2.1 & 2.01 & 2.001 & 2.0001 \\
        \hline
	    1.07625 & 1.0075125 & 1.000750125 & 1.000075001 \\
        \hline
    \end{tabular} \\
	  
	$$\lim_{x \to 2}{g(x)} = \text{Does not exist}$$
	$$\lim_{x \to 2^-}{g(x)} = -1$$
	$$\lim_{x \to 2^+}{g(x)} = 1$$

    \item Calculating piecewise function limits
	    $$f(x) = \begin{cases}3 - x \text{ if } x < 2 \\ x - 1 \text{ if } x > 2 \end{cases}$$
	    $$a = 2$$
	  
	\begin{tabular}{| c | c | c | c |}
        \hline
	    1.92 & 1.99 & 1.999 & 1.9999 \\
        \hline
	    1.1 & 1.01 & 1.001 & 1.0001 \\
        \hline
    \end{tabular} \\
	  
	\begin{tabular}{| c | c | c | c |}
        \hline
	    2.1 & 2.01 & 2.001 & 2.0001 \\
        \hline
	    1.1 & 1.01 & 1.001 & 1.0001 \\
        \hline
    \end{tabular} \\
	  
	\textbf{Explanation}: Since $f(2)$ is not defined within the piece wise function, a graph representing this function would have a whole where $x = a$ and have two lines with inverse slopes
	      
	$$f(a) = \text{undefined}$$
	$$\lim_{x \to a}{f(x)} = 1$$
	$$\lim_{x \to a^-}{f(x)} = 1$$
	$$\lim_{x \to a^+}{f(x)} = 1$$

    \item Limit Rules/Laws:
    \begin{enumerate}
        \item Definitions:
    		$$\lim\limits_{x \to 3}{f(x)} = 2$$
    		$$\lim\limits_{x \to 3}{g(x)} = -1$$
    		$$\lim\limits_{x \to 3}{h(x)} = 6$$
        \item Problems:
        \begin{enumerate}
    		\item Sum, Constant Multiple
    			\begin{eqnarray}
    			\lim_{x \to 3}{(f(x) + 2g(x))} &=& \lim_{x \to 3}{f(x)} + \lim_{x \to 3}{2g(x)} \\
    			&=& \lim_{x \to 3}{f(x)} + 2(\lim_{x \to 3}{g(x)}) \\
    			&=& 2 + 2(-1) \\
    			&=& 0
    			\end{eqnarray}
    		\item Quotient
    			\begin{eqnarray}
    			\lim_{x \to 3}{\frac{h(x)}{g(x)}} &=& \frac{\lim\limits_{x \to 3}{h(x)}}{\lim\limits_{x \to 3}{g(x)}} \\
    			&=& \frac{6}{-1} \\
    			&=& -6
    			\end{eqnarray}
    		\item Quotient, Root, Difference
    			\begin{eqnarray}
    			\lim_{x \to 3}{\frac{h(x)}{\sqrt{f(x) - g(x)}}} &=& \frac{\lim\limits_{x \to 3}{h(x)}}{\lim\limits_{x \to 3}{\sqrt{f(x) - g(x)}}}\\
    			&=& \frac{\lim\limits_{x \to 3}{h(x)}}{\sqrt{\lim\limits_{x \to 3}{(f(x) - g(x))}}} \\
    			&=& \frac{\lim\limits_{x \to 3}{h(x)}}{\sqrt{\lim\limits_{x \to 3}f(x) - \lim\limits_{x \to 3}g(x)}} \\
    			&=& \frac{6}{\sqrt{2 + 1}} \\
    			&=& \frac{6}{\sqrt{3}} \\
    			&=& 2 \sqrt{3}
    			\end{eqnarray}
        \end{enumerate}
    \end{enumerate}
    \item
    	\begin{eqnarray}
    	    \lim_{x \to 1}{\frac{3x^2 - 7x + 1}{x + 2}} &=& \frac{3(1)^2 - 7(1) + 1}{1 + 2} \\
    	    &=& \frac{3 - 7 + 1}{1+2} \\
    	    &=& \frac{-3}{3} \\
    	    &=& -1
    	\end{eqnarray}
    \item
    	\begin{eqnarray}
    	    \lim_{x \to 4}{\frac{\left ( \frac{1}{x} - \frac{1}{4} \right )}{x - 4}} &=& \lim_{x \to 4}{\frac{\left ( \frac{4}{4x} - \frac{x}{4x} \right )}{x - 4}} \\
    	    &=& \lim_{x \to 4}{\frac{\left ( \frac{4 - x}{4x} \right )}{x - 4}} \\
    	    &=& \lim_{x \to 4}{\frac{\left ( \frac{4 - x}{4x} \right )}{\left ( \frac{x - 4}{1} \right )}} \\
    	    &=& \lim_{x \to 4}{ \left (\frac{4 - x}{4x} \right ) \left (\frac{1}{x-4} \right )} \\
    	    &=& \lim_{x \to 4}{\frac{4 - x}{4x(x - 4)}} \\
    	    &=& \lim_{x \to 4}{\frac{-(-4 + x)}{4x(x - 4)}} \\
    	    &=& \lim_{x \to 4}{\frac{-(x - 4)}{4x(x - 4)}} \\
    	    &=& \lim_{x \to 4}{\frac{-1}{4x}} \\
    	    &=& \lim_{x \to 4}{\frac{-1}{4(4)}} \\
    	    &=& -\frac{1}{16}
    	\end{eqnarray}
    \item
    	\begin{eqnarray}
    	    \lim_{x \to 9}{\frac{x - 9}{\sqrt{x} - 3}} &=& \lim_{x \to 9}{\frac{x - 9}{\sqrt{x} - 3} \cdot \frac{\sqrt{x} + 3}{\sqrt{x} + 3}} \\
    	    &=& \lim_{x \to 9}{\frac{(x - 9)(\sqrt{x} + 3)}{(\sqrt{x} - 3)(\sqrt{x} + 3)}} \\
    	    &=& \lim_{x \to 9}{\frac{(x - 9)(\sqrt{x} + 3)}{x - 9}} \\
    	    &=& \lim_{x \to 9}{\sqrt{x} + 3} \\
    	    &=& \sqrt{9} + 3 \\
    	    &=& 3 + 3\\
    	    &=& 6
    	\end{eqnarray}
    \item
    	$$1 - \frac{x^2}{2} \leq \cos{x} \leq 1$$
    	\begin{eqnarray}
    	    \lim_{x \to 0}{\left ( 1 - \frac{x^2}{2} \right )} &=& 1 - \frac{0^2}{2} \\
    	    &=& 1 - 0 \\
    	    &=& 1 \\
    	    &=& \lim_{x \to 0}{1} \\
    	    \lim_{x \to 0}{\cos{x}} &=& 1 \hspace{2cm} \text{(By the Squeeze Theorem)}
    	\end{eqnarray}
    \item
    	\begin{eqnarray}
    	    \lim_{x \to 0}{\sin{x}} &=& 0 \hspace{1.5cm} \text{(By the Squeeze Theorem)} \\
    	    \lim_{x \to 0}{\cos{x}} &=& 1 \hspace{1.5cm} \text{(By the Squeeze Theorem)}
    	\end{eqnarray}
    
    	\begin{eqnarray}
    	    \lim_{x \to 0}{\frac{\sin{2x}}{\sin{x}}} &=& \lim_{x \to 0}{\frac{2 \sin{x} \cos{x}}{\sin{x}}} \\
    	    &=& \lim_{x \to 0}{2 \cos{x}} \\
    	    &=& 2 \lim_{x \to 0}{\cos{x}} \\
    	    &=& 2 \cdot 1 \\
    	    &=& 2
    	\end{eqnarray}
	\item Infinite Limits Numerically
		$$f(x) = \frac{x}{(x-2)^2}$$
		\begin{tabular}{| c | c | c | c |}
			\hline
			2.1 & 2.01 & 2.001 & 2.0001 \\
			\hline
			210 & 20100 & 2001000 & 200010000 \\
			\hline
		\end{tabular}

		\begin{tabular}{| c | c | c | c |}
			\hline
			1.9 & 1.99 & 1.999 & 1.9999 \\
			\hline
			190 & 19900 & 1999000 & 199990000 \\
			\hline
		\end{tabular}

		$$\lim_{x \to 2}{f(x)} = \infty$$
	\item Infinite Limits Graphically
		$$\lim_{x \to -2^-}{h(x)} = - \infty$$
		$$\lim_{x \to -2^+}{h(x)} = - \infty$$
		$$\lim_{x \to -2}{h(x)} = - \infty$$
		$$\lim_{x \to 3^-}{h(x)} = \infty$$
		$$\lim_{x \to 3^+}{h(x)} = - \infty$$
		$$\lim_{x \to 3}{h(x)} = \text{Does not exist}$$
	\item Infinite Limits Analytically
		\\ Hint: Look at the signs of the fractions
		$$\frac{x^2-5x+6}{x^4-4x^2} = \frac{(x - 3)(x - 2)}{x^2(x + 2)(x - 2)} = \frac{x-3}{x^2(x+2)}$$
		$$\lim_{x \to -2^+}{\frac{x^2 - 5x + 6}{x^4 - 4x^2}} = \lim_{x \to -2^+}{\frac{x-3}{x^2(x+2)}} = - \infty$$
		$$\lim_{x \to -2^-}{\frac{x^2 - 5x + 6}{x^4 - 4x^2}} = \lim_{x \to -2^-}{\frac{x-3}{x^2(x+2)}} = \infty$$
		$$\lim_{x \to -2}{\frac{x^2 - 5x + 6}{x^4 - 4x^2}} = \text{Does not exist}$$
	\item Infinite Limits Analytically with Square Root
		$$\lim_{x \to 1^+}{\frac{x + 3}{\sqrt{x^2 - 5x + 4}}} = \lim_{x \to 1^+}{\frac{x + 3}{\sqrt{(x-4)(x-1)}}} = \text{Does not exist}$$
		$$\lim_{x \to 1^-}{\frac{x + 3}{\sqrt{x^2 - 5x + 4}}} = \lim_{x \to 1^-}{\frac{x + 3}{\sqrt{(x-4)(x-1)}}} = \infty$$
		$$\lim_{x \to 1}{\frac{x + 3}{\sqrt{x^2 - 5x + 4}}} = \text{Does not exist}$$
	\item Infinite Limit with a Trigonometric Function
		$$\lim_{\theta \to 0^-}{\frac{\sin{\theta}}{\cos^2{\theta} - 1}} = \lim_{\theta \to 0^-}{\frac{\sin{\theta}}{- \sin^2{\theta}}} = \lim_{\theta \to 0^-}{\frac{1}{- \sin^2{\theta}}} = \infty$$
	\item Locating Veritical Asymptotes
		$$f(x) = \frac{x + 7}{x^4 - 49x^2} = \frac{x+7}{x^2(x^2-49)} = \frac{x+7}{x^2((x-7)(x+7))} = \frac{1}{x^2(x-7)}$$
		Denominator is $0$ at $x=0$, $x=-7$, $x=7$ \\
		$x=-7$ does not fit, as it is connected with $x+7$, but cancels out \\
		Vertical Asymptotes: $x = 0$, $x = 7$
	\item Limits at Infinity
		$$\lim_{x \to \infty}{5 + \frac{1}{x} + \frac{10}{x^2}} = 5 + 0 + 0 = 5$$
		$$\lim_{x \to \infty}{5} = 5$$
		$$\lim_{x \to \infty}{\frac{1}{x}} = 0$$
		$$\lim_{x \to \infty}{\frac{10}{x^2}} = 0$$
	\item End behaviour for rational functions (different degrees)
		\\ Hint: the degree of the numerator is less than the denominator
		$$\lim_{x \to \infty}{\frac{6x+1}{2x^2 - 5x + 2}} = \lim_{x \to \infty}{\frac{\frac{6}{x} + \frac{1}{x^2}}{2 - \frac{5}{x} + \frac{2}{x^2}}} = \lim_{x \to \infty}{\frac{0 + 0}{2 - 0 + 0}} = \frac{0}{2} = 0$$
	\item End behaviour for rational functions (equal degrees)
		\\ Hint: the degree of the numerator is the same as the denominator
		$$\lim_{x \to \infty}{\frac{6x^2 + 1}{2x^2 - 5x + 2}} = \lim_{x \to \infty}{\frac{6 + \frac{1}{x^2}}{2 - \frac{5}{x} + \frac{2}{x^2}}} = \lim_{x \to \infty}{\frac{6 + 0}{2 - 0 + 0}} = \frac{6}{2} = 3$$
	\item End behaviour for rational functions (different degrees)
		\\ Hint: the degrees of the numerator is greater than the degree of the denominator
		$$\lim_{x \to \infty}{\frac{6x^4 + 1}{2x^2 - 5x + 2}} = \lim_{x \to \infty}{\frac{6x^2 + \frac{1}{x^2}}{2 - \frac{5}{x} + \frac{2}{x^2}}} = \lim_{x \to \infty}{\frac{6x^2 + 0}{2 - 0 + 0}} = \frac{\infty}{2} = \infty$$
	\item End behaviours for rational functions
		\\ Hint: If there is a negative exponent like $2x^{-2}$, we can rewrite that as $\frac{2}{x^2}$
		\\ Hint: Keep in mind the direction at which $x$ is changing (increasing or decreasing)
		$$\lim_{x \to - \infty}{2x^{-8} + 4x^3} = \lim_{x \to - \infty}{\frac{2}{x^8} + 4x^3} = 0 - \infty = - \infty$$
		$$\lim_{x \to \infty}{\frac{14x^3 + 3x^2 - 2x}{21x^3 + x^2 + 2x + 1}} = \lim_{x \to \infty}{\frac{14 + \frac{3}{x} - \frac{2}{x^2}}{21 + \frac{1}{x} + \frac{2}{x^2} + \frac{1}{x^3}}} = \frac{14}{21} = \frac{2}{3}$$
		$$\lim_{x \to \infty}{\frac{9x^3 + x^2 - 5}{3x^4 + 4x^2}} = \lim_{x \to \infty}{\frac{\frac{9}{x} + \frac{1}{x^2} - \frac{5}{x^4}}{3 + \frac{4}{x^2}}} = \frac{0}{3} = 0$$
	\item Asymptotes for a rational function
		$$f(x) = \frac{3x^2 - 7}{x^2 + 5x}$$
		Horizontal Asymptote(s): $y = 3$
		\\ \textbf{Explanation}: The end behaviour for this function approaches $3$ (on both ends), so there is a single horizontal asymptote
	\item End behaviour for algebraic function
		\begin{eqnarray}
			\lim_{x \to - \infty}{\frac{\sqrt{16x^2 + x}}{x}} &=& \lim_{x \to - \infty}{\frac{\frac{\sqrt{16x^2 + x}}{x}}{\frac{x}{x}}} \\
			&=& \lim_{x \to - \infty}{\frac{\frac{1}{x}\sqrt{16x^2 + x}}{1}} \\
			&=& \lim_{x \to - \infty}{\frac{1}{-\sqrt{x^2}}\sqrt{16x^2 + x}} \\
			&=& \lim_{x \to - \infty}{- \sqrt{\frac{16x^2}{x^2} + \frac{x}{x^2}}} \\
			&=& \lim_{x \to - \infty}{- \sqrt{16 + \frac{1}{x}}} \\
			&=& \lim_{x \to - \infty}{- \sqrt{16}} \\
			&=& -4 \\
		\end{eqnarray}
	\item End behaviour for transcendental function
		$$\lim_{x \to \infty}{\frac{\sin{x}}{e^x + \ln{x}}} = \lim_{x \to \infty}{\frac{\sin{x}}{\infty + \infty}} = 0$$
		\textbf{Explanation}: Since $\sin{x}$ is bounded between $-1$ and $1$, and the denominator is a very large number, we know as $x$ increases, the function with approach zero
	\item Continuity graphically
		\\ $x = 1$ (Removable Discontinuity)
		\\ $x = 2$ (Jump Discontinuity)
		\\ $x = 3$ (Removable Discontinuity)
	\item Continuity analytically
		$$f(x) = \begin{aligned}
			\begin{cases}
				\frac{x^2 - 4x + 3}{x - 3} &\text{ if } x \neq 3 \\
				2 &\text{ if } x = 3
			\end{cases}
		\end{aligned}$$
		$$a = 3$$
		$$f(3) = 2$$
		$$\lim_{x \to 3}{f(x)} = \lim_{x \to 3}{\frac{x^2 - 4x + 3}{x - 3}} = \lim_{x \to 3}{\frac{(x - 3)(x - 1)}{x-3}} = \lim_{x \to 3}{(x - 1)} = 2$$
		The function $f$ is continuous at $a$
	\item Interval of continuity
		Hint: this is a composition of a polynomial and power function
		$$f(x) = (x^2 - 1)^{\frac{3}{2}}$$
		$f(x)$ is the composition $h \circ g(x)$ where $g(x) = x^2 - 1$, $h(x) = x^{\frac{3}{2}}$ \\
		$g(x)$ is continuous everywhere (because its a polynomial) \\
		$h(x)$ is conintuous on $[0, \infty)$ \\
		Because this is a compoisite function, $f(x) = h \circ g(x)$ is continuous at $a$ if $g(a) > 0$ \\
		The function $f(x)$ is continuous on $(- \infty, -1]$ and $[1, \infty)$
	\item Intermediate Value Theorem
		$$f(x) = x \ln{x} - 1$$
		Note: $f(x)$ is continuous on $(0, \infty)$ \\
		Interval: $(1, e)$ \\
		$$f(1) = 1 \ln{1} - 1 = 0 - 1 = -1 < 0$$
		$$f(e) = e \ln{e} - 1 = e - 1 > 0$$
		By the Intermediate Value Theorem, there is a $c \in (1, e)$ such that $f(c) = 0$ \\
		$c \ln{c} - 1 = 0$, meaning $c$ is a solution to $x \ln{x} - 1 = 0$
    \item (Section 2.1, Related Exercise 13):
    	\\Hint: use the secant line slope formula
    	$$s(t) = -16t^2 + 128t$$
        \begin{enumerate}
    	    \item $[1, 4]$
    	        $$\frac{256 - 112}{4 - 1} = \frac{144}{3} = 48$$
    	    \item $[1, 3]$
    	        $$\frac{240 - 112}{3 - 1} = \frac{128}{2} = 64$$
    	    \item $[1, 2]$
    	        $$\frac{192 - 112}{2 - 1} = \frac{80}{1} = 84$$
    	    \item $[1, 1 + h]$, where $h > 0$ is a real number
    	        $$\frac{112 + -16h^2 + 128h - 112}{1 + h - 1} = \frac{-16h^2 + 128h}{h} = -16h + 128 = 16(-h + 6)$$
        \end{enumerate}
    \item (Section 2.1, Related Exercise 15):
    	Hint: we use the slope formula for the secant line, and the relationship is referring to the interval
    	$$s(t) = -16^t + 100t$$
    	\begin{eqnarray}
    	\frac{s(t_1) - s(t_0)}{t_1 - t_0} &=& \frac{s(2) - s(0.5)}{2 - 0.5} \\
    	&=& \frac{136 - 46}{1.5} \\
    	&=& \frac{90}{1.5} \\
    	&=& 60
    	\end{eqnarray}
    	The slope of this secant line, through the lens of average velocity could be viewed as the average velocity over the interval $[0.5, 2]$
    \item (Section 2.1, Related Exercise 17):
        $$s(t) = -16t^2 + 128t$$
    
	    \begin{tabular}{| c | c | c | c | c |}
            \hline
            $[1,2]$&$[1,1.5]$&$[1,1.1]$&$[1,1.01]$&$[1,1.001]$ \\
            \hline
    	    80 & 88 & 94.4 & 95.84 & 95.984 \\
            \hline
        \end{tabular} \\

    	$$v_{inst} = \lim_{t \to 1}{s(t)} = 96$$
    \item (Section 2.1, Related Exercise 19):
	    $$s(t) = -16t^2 + 100t$$
    
	    \begin{tabular}{| c | c | c | c | c |}
            \hline
	        $[2,3]$&$[2.9,3]$&$[2.99,3]$&$[2.999,3]$&$[2.9999,3]$ \\
            \hline
	        20 & 5.6 & 4.16 & $4.016$ & $4.002$ \\
            \hline
        \end{tabular} \\
    
	    $$v_{inst} = \lim_{t \to 3}{s(t)} = 4$$
    \item (Section 2.2, Related Exercise 3):
        \begin{itemize}
		    \item $h(2) = 5$
		    \item $\lim\limits_{x \to 2}{h(x)} = 3$
		    \item $h(4) = \text{Does not exist}$
		    \item $\lim\limits_{x \to 4}{h(x)} = 1$
		    \item $\lim\limits_{x \to 5}{h(x)} = 2$
        \end{itemize}
    \item (Section 2.2, Related Exercise 4):
        \begin{itemize}
		    \item $g(0) = 0$
		    \item $\lim\limits_{x \to 0}{g(x)} = 1$
		    \item $g(1) = 2$
		    \item $\lim\limits_{x \to 1}{g(x)} = 2$
        \end{itemize}
    \item (Section 2.2, Related Exercise 7):
	    $$f(x) = \frac{x^2 - 4}{x - 2}$$
    
	    \begin{tabular}{| c | c | c | c |}
            \hline
	        1.9 & 1.99 & 1.999 & 1.9999 \\
            \hline
	        3.9 & 3.99 & 3.999 & 3.9999 \\
            \hline
        \end{tabular} \\
    
	    \begin{tabular}{| c | c | c | c |}
            \hline
	        2.1 & 2.01 & 2.001 & 2.0001 \\
            \hline
	        4.1 & 4.01 & 4.001 & 4.0001 \\
            \hline
        \end{tabular} \\
    
	    $$\lim_{x \to 2}{f(x)} = 4$$
    \item (Section 2.2, Related Exercise 8):
	    $$f(x) = \frac{x^3 - 1}{x - 1}$$
    
	    \begin{tabular}{| c | c | c | c |}
            \hline
	        0.9 & 0.99 & 0.999 & 0.9999 \\
            \hline
	        2.71 & 2.9701 & 3.997001 & 3.99970001 \\
            \hline
        \end{tabular} \\
    
	    \begin{tabular}{| c | c | c | c |}
            \hline
	        1.1 & 1.01 & 1.001 & 1.0001 \\
            \hline
	        3.31 & 3.0301 & 3.003001 & 3.00030001 \\
            \hline
        \end{tabular} \\
    
	    $$\lim_{x \to 1}{f(x)} = 3$$
    \item (Section 2.2, Related Exercise 27):
	    $$f(x) = \frac{x-2}{\ln|x-2|}$$
	    $$\lim_{x \to 2}{f(x)} = 2$$
    \item (Section 2.2, Related Exercise 28):
	    $$f(x) = \frac{e^{2x} - 2x - 1}{x^2}$$
	    $$\lim_{x \to 0}{f(x)} = 0$$
    \item (Section 2.2, Related Exercise 19):
	    $$\begin{aligned}f(x) = \begin{cases}
	    x^2 + 1 &\text{ if } x \leq -1 \\
	    3 &\text{ if } x > -1
	    \end{cases}\end{aligned}$$
		$$\lim_{x \to -1^-}{f(x)} = 2$$
		$$\lim_{x \to -1^+}{f(x)} = 3$$
		$$\lim_{x \to -1}{f(x)} = \text{Does not exist}$$
    \item (Section 2.2, Related Exercise 20):
	    $$\begin{aligned}f(x) = \begin{cases}
	    3 - x &\text{ if } x < 2 \\
	    x - 1 &\text{ if } x >2
	    \end{cases}\end{aligned}$$
	    $$\lim_{x \to 2^-}{f(x)} = 1$$
	    $$\lim_{x \to 2^+}{f(x)} = 1$$
	    $$\lim_{x \to 2}{f(x)} = 1$$

    \item (Section 2.3, Related Exercise 19):
	    $$\lim_{x \to 4}{3x-7} = 3(4)-7 = 12 - 7 = 5$$
    \item (Section 2.3, Related Exercise 22):
	    $$\lim_{x \to 6}{4} = 4$$
    \item (Section 2.3, Related Exercise 11): Quotient, Difference
	    \begin{eqnarray}
	    \lim_{x \to 1}{\frac{f(x)}{g(x) - h(x)}} &=& \frac{\lim\limits_{x \to 1}{f(x)}}{\lim\limits_{x \to 1}{g(x) - h(x)}} \\
	    &=& \frac{\lim\limits_{x \to 1}{f(x)}}{\lim\limits_{x \to 1}{g(x) - h(x)}} \\
	    &=& \frac{\lim\limits_{x \to 1}{f(x)}}{\lim\limits_{x \to 1}{g(x)} - \lim\limits_{x \to 1}h(x)} \\
	    &=& \frac{8}{3 - 2} \\
	    &=& \frac{8}{1} \\
	    &=& 8
	    \end{eqnarray}
    \item (Section 2.3, Related Exercise 12): Root, Sum, Product
	    \begin{eqnarray}
	    \lim_{x \to 1}{\sqrt[3]{f(x)g(x) + 3}} &=& \sqrt[3]{\lim_{x \to 1}{f(x)g(x) + 3}} \\
	    &=& \sqrt[3]{\lim_{x \to 1}{f(x)g(x) + 3}} \\
	    &=& \sqrt[3]{\lim_{x \to 1}{f(x)g(x)} + \lim_{x \to 1}{3}} \\
	    &=& \sqrt[3]{\lim_{x \to 1}{f(x)}\lim_{x \to 1}{g(x)} + \lim_{x \to 1}{3}} \\
	    &=& \sqrt[3]{8 \cdot 3 + 3} \\
	    &=& \sqrt[3]{24 + 3} \\
	    &=& \sqrt[3]{27} \\
	    &=& 3
	    \end{eqnarray}
    \item (Section 2.3, Related Exercise 25):
	    \begin{eqnarray}
	    \lim_{x \to 1}{\frac{5x^2+6x+1}{8x-4}} &=& \frac{5(1^2)+6(1)+1}{8(1)-4} \\
	    &=& \frac{5+6+1}{8-4} \\
	    &=& \frac{12}{4} \\
	    &=& 3
	    \end{eqnarray}
    \item (Section 2.3, Related Exercise 26):
	    \begin{eqnarray}
	    \lim_{t \to 3}{\sqrt[3]{t^2-10}} &=& \sqrt[3]{\lim_{t \to 3}{t^2-10}} \\
	    &=& \sqrt[3]{3^2 - 10} \\
	    &=& \sqrt[3]{9 - 10} \\
	    &=& \sqrt[3]{-1} \\
	    &=& -1
	    \end{eqnarray}
    \item (Section 2.3, Related Exercise 27):
	    \begin{eqnarray}
	    \lim_{p \to 2}{\frac{3p}{\sqrt{4p+1}-1}} &=& \frac{\lim\limits_{p \to 2}{3p}}{\lim\limits_{p \to 2}{\sqrt{4p+1}-1}} \\
	    &=& \frac{3(2)}{\sqrt{\lim\limits_{p \to 2}{4p+1}}-1} \\
	    &=& \frac{6}{\sqrt{4(2)+1}-1} \\
	    &=& \frac{6}{\sqrt{8+1}-1} \\
	    &=& \frac{6}{\sqrt{9}-1} \\
	    &=& \frac{6}{3-1} \\
	    &=& \frac{6}{2} \\
	    &=& 3
	    \end{eqnarray}
    \item (Section 2.3, Related Exercise 72):
	    $$g(x) = \begin{aligned}\begin{cases}
	    5x-15 &\text{ if } x < 4 \\
	    \sqrt{6x+1} &\text{ if } x \geq 4
	    \end{cases}\end{aligned}$$
	    $$\lim_{x \to 4^-}{g(x)} = 5$$
	    $$\lim_{x \to 4^+}{g(x)} = 5$$
	    $$\lim_{x \to 4}{g(x)} = 5$$
    \item (Section 2.3, Related Exercise 73):
        $$g(x) = \begin{aligned}
            \begin{cases}
	            x^2+1 &\text{ if } x < -1 \\
	            \sqrt{x+1} &\text{ if } x \geq -1
	        \end{cases}
        \end{aligned}$$
	    $$\lim_{x \to -1^-}{g(x)} = 2$$
	    $$\lim_{x \to -1^+}{g(x)} = 0$$
	    $$\lim_{x \to -1}{g(x)} = \text{Does not exist}$$
    \item (Section 2.3, Related Exercise 34):
	    \begin{eqnarray}
	    \lim_{x \to 3}{\frac{x^2 - 2x - 3}{x - 3}} &=& \lim_{x \to 3}{\frac{(x-3)(x+1)}{x - 3}} \\
	    &=& \lim_{x \to 3}{x+1} \\
	    &=& 3+1 \\
	    &=& 4
	    \end{eqnarray}
    \item (Section 2.3, Related Exercise 41):
	    \begin{eqnarray}
	    \lim_{x \to 9}{\frac{\sqrt{x} - 3}{x - 9}} &=& \lim_{x \to 9}{\frac{\sqrt{x} - 3}{x - 9}} \cdot \frac{\sqrt{x}+3}{\sqrt{x}+3} \\
	    &=& \lim_{x \to 9}{\frac{(\sqrt{x} - 3)(\sqrt{x} + 3)}{(x - 9)(\sqrt{x} + 3)}} \\
	    &=& \lim_{x \to 9}{\frac{x-9}{(x-9)(\sqrt{x}+3)}} \\
	    &=& \lim_{x \to 9}{\frac{1}{\sqrt{x}+3}} \\
	    &=& \frac{1}{\sqrt{9}+3} \\
	    &=& \frac{1}{3+3} \\
	    &=& \frac{1}{6}
	    \end{eqnarray}
    \item (Section 2.3, Related Exercise 69):
	    $$\lim_{x \to 1^+}{\frac{x-1}{\sqrt{x^2-1}}} = \text{Does not exist}$$
    \item (Section 2.3, Related Exercise 70):
	    \begin{eqnarray}
	    \lim_{x \to 1^+}{\frac{x-1}{\sqrt{x^2-1}}} &=& \lim_{x \to 1^+}{\frac{x-1}{\sqrt{x^2-1}} \cdot \frac{x+1}{x+1}} \\
	    &=& \lim_{x \to 1^+}{\frac{x^2-1}{\sqrt{x^2-1} (x+1)}} \\
	    &=& \lim_{x \to 1^+}{\frac{x^2-1}{(x^2-1)^{\frac{1}{2}} (x+1)}} \\
	    &=& \lim_{x \to 1^+}{\frac{(x^2-1)^{\frac{1}{2}}}{x+1}} \\
	    &=& \lim_{x \to 1^+}{\frac{\sqrt{x^2-1}}{x+1}} \\
	    &=& \frac{\sqrt{1-1}}{1+1} \\
	    &=& \frac{\sqrt{0}}{2} \\
	    &=& \frac{0}{2} \\
	    &=& 0
	    \end{eqnarray}
    \item (Section 2.3, Related Exercise 95):
		    $$\frac{2^x - 2^0}{x-0} = \frac{2^x - 1}{x}$$

            \begin{tabular}{| c | c | c | c | c | c |}
                \hline
		        -1 & -0.1 & -0.01 & -0.001 & -0.0001 & -0.00001 \\
                \hline
		        0.5 & 0.6696700846 & 0.6907504563 & 0.6929070095 & 0.6931231585 & 0.6931447783 \\
                \hline
            \end{tabular}

            $$\lim_{x \to 0^1}{\frac{2^x-1}{x}} = 0.693$$
    \item (Section 2.3, Related Exercise 96):
		$$\frac{3^x - 3^0}{x-0} = \frac{3^x - 1}{x}$$

            \begin{tabular}{| c | c | c | c |}
                \hline
		        -0.1 & -0.01 & -0.001 & -0.0001 \\
                \hline
		        1.040415402 & 1.092599583 & 1.098009035 & 1.098551943 \\
                \hline
            \end{tabular}

            \begin{tabular}{| c | c | c | c |}
                \hline
		        0.0001 & 0.001 & 0.01 & 0.1 \\
                \hline
		        1.098672638 & 1.099215984 & 1.104669194 & 1.161231740 \\
                \hline
            \end{tabular}
            
		    $$\lim_{x \to 0^1}{\frac{3^x-1}{x}} = 1.0986$$
    \item (Section 2.3, Related Exercise 81):
	    \\ $-|x| < 0 < |x|$ and $\sin{\frac{1}{x}} \leq 1$, so $|x| \sin{\frac{1}{x}} \leq |x|$ and $-|x| \sin{\frac{1}{x}} \geq -|x|$
		$$\lim_{x \to 0}{-|x|} = -|0| = 0$$
		$$\lim_{x \to 0}{|x|} = |0| = 0$$
		$$\lim_{x \to 0}{x \sin{\frac{1}{x}}} = 0$$
		By the Squeeze Theorem, since $\lim\limits_{x \to 0}{-|x|} = \lim\limits_{x \to 0}{|x|}$ and the functions are chronologically greater than the last
    \item (Section 2.3, Related Exercise 82):
		$$\lim_{x \to 0}{1 - \frac{x^2}{2}} = 1 -\frac{0}{2} = 1 - 0 = 1$$
		$$\lim_{x \to 0}{1} = 1$$
		$$\lim_{x \to 0}{\cos{x}} = 1$$
		By the Squeeze Theorem, since $\lim\limits_{x \to 0}{1 - \frac{x^2}{2}} = \lim\limits_{x \to 0}{1}$ and the functions are chronologically greater than the last
    \item (Section 2.3, Related Exercise 60):
	    \begin{eqnarray}
	    \lim_{x \to 0}{\frac{\sin{2x}}{\sin{x}}} &=& \lim_{x \to 0}{\frac{2\sin{x}\cos{x}}{\sin{x}}} \\
	    &=& \lim_{x \to 0}{2 \cos{x}} \\
	    &=& 2 \cos{0} \\
	    &=& 2 \cdot 1 \\
	    &=& 2
	    \end{eqnarray}
    \item (Section 2.3, Related Exercise 61):
	    \begin{eqnarray}
	    \lim_{x \to 0}{\frac{1 - \cos{x}}{\cos^2{x} - 3 \cos{x} + 2}} &=& \lim_{x \to 0}{\frac{1}{\cos^2{x} - 2 \cos{x} + 2}} \\
	    &=& \lim_{x \to 0}{\frac{1}{\cos{x} \cos{x} - 2 \cos{x} + 2}} \\
	    &=& \frac{1}{\cos{0} \cos{0} - 2 \cos{0} + 2} \\
	    &=& \frac{1}{1 \cdot 1 - 2(1) + 2} \\
	    &=& \frac{1}{1 - 2 + 2} \\
	    &=& \frac{1}{1} \\
	    &=& 1
	    \end{eqnarray}
	\item (Section 2.4, Related Exercise 6):
		$$f(x) = \frac{x}{(x^2 - 2x - 3)^2}$$
		$$\lim_{x \to -1}{f(x)} = - \infty$$
		$$\lim_{x \to 3}{f(x)} = \infty$$
	\item (Section 2.4, Related Exercise 7):
		$$\lim_{x \to 1^-}{f(x)} = \infty$$
		$$\lim_{x \to 1^+}{f(x)} = \infty$$
		$$\lim_{x \to 1}{f(x)} = \infty$$
		$$\lim_{x \to 2^-}{f(x)} = \infty$$
		$$\lim_{x \to 2^+}{f(x)} = - \infty$$
		$$\lim_{x \to 2}{f(x)} = \text{Does not exist}$$
	\item (Section 2.4, Related Exercise 8):
		$$\lim_{x \to 2^-}{g(x)} = \infty$$
		$$\lim_{x \to 2^+}{g(x)} = - \infty$$
		$$\lim_{x \to 2}{g(x)} = \text{Does not exist}$$
		$$\lim_{x \to 4^-}{g(x)} = - \infty$$
		$$\lim_{x \to 4^+}{g(x)} = - \infty$$
		$$\lim_{x \to 4}{g(x)} = - \infty$$
	\item (Section 2.4, Related Exercise 21):
		$$\lim_{x \to 2^+}{\frac{1}{x-2}} = \infty$$
		$$\lim_{x \to 2^-}{\frac{1}{x-2}} = - \infty$$
		$$\lim_{x \to 2}{\frac{1}{x-2}} = \text{Does not exist}$$
	\item (Section 2.4, Related Exercise 22):
		$$\lim_{x \to 3^+}{\frac{2}{(x - 3)^3}} = \infty$$
		$$\lim_{x \to 3^-}{\frac{2}{(x - 3)^3}} = - \infty$$
		$$\lim_{x \to 3}{\frac{2}{(x - 3)^3}} = \text{Does not exist}$$
	\item (Section 2.4, Related Exercise 28):
		$$\lim_{t \to -2^+}{\frac{t^3 - 5t^2 + 6t}{t^4 - 4t^2}} = \lim_{t \to -2^+}{\frac{t(t-2)(t-3)}{t^2(t^2 - 4)}} = \lim_{t \to -2^+}{\frac{t(t-2)(t-3)}{t^2(t-2)(t+2)}} = \lim_{t \to -2^+}{\frac{t(t-3)}{t^2(t+2)}} = \lim_{t \to -2^+}{\frac{t^2 - 3t}{t^3 + 2t^2}} = - \infty$$
		$$\lim_{t \to -2^-}{\frac{t^3 - 5t^2 + 6t}{t^4 - 4t^2}} = \lim_{t \to -2^-}{\frac{t(t-2)(t-3)}{t^2(t^2 - 4)}} = \lim_{t \to -2^-}{\frac{t(t-2)(t-3)}{t^2(t-2)(t+2)}} = \lim_{t \to -2^-}{\frac{t(t-3)}{t^2(t+2)}} = \lim_{t \to -2^-}{\frac{t^2 - 3t}{t^3 + 2t^2}} = - \infty$$
		$$\lim_{t \to -2}{\frac{t^3 - 5t^2 + 6t}{t^4 - 4t^2}} = \lim_{t \to -2}{\frac{t(t-2)(t-3)}{t^2(t^2 - 4)}} = \lim_{t \to -2}{\frac{t(t-2)(t-3)}{t^2(t-2)(t+2)}} = \lim_{t \to -2}{\frac{t(t-3)}{t^2(t+2)}} = \lim_{t \to -2}{\frac{t^2 - 3t}{t^3 + 2t^2}} = - \infty$$
		$$\lim_{t \to 2}{\frac{t^3 - 5t^2 + 6t}{t^4 - 4t^2}} = \lim_{t \to 2}{\frac{t(t-2)(t-3)}{t^2(t^2 - 4)}} = - \frac{1}{8}$$
	\item (Section 2.4, Related Exercise 31): Remember, if you are able to solve by direct substitution after canceling terms (where the denominator does not equal zero), that's your answer
		$$\lim_{x \to 0}{\frac{x - 3}{x^4 - 9x^2}} = \lim_{x \to 0}{\frac{x - 3}{x^2(x-3)(x+3)}} = \lim_{x \to 0}{\frac{1}{x^2(x+3)}} = \lim_{x \to 0}{\frac{1}{x^3+3x^2}} = \infty$$
		$$\lim_{x \to 3}{\frac{x - 3}{x^4 - 9x^2}} = \lim_{x \to 3}{\frac{x - 3}{x^2(x-3)(x+3)}} = \lim_{x \to 3}{\frac{1}{x^2(x+3)}} = \lim_{x \to -3}{\frac{1}{x^3+3x^2}} = \frac{1}{54}$$
		$$\lim_{x \to -3}{\frac{x - 3}{x^4 - 9x^2}} = \lim_{x \to -3}{\frac{x - 3}{x^2(x-3)(x+3)}} = \lim_{x \to -3}{\frac{1}{x^2(x+3)}} = \lim_{x \to -3}{\frac{1}{x^3+3x^2}} = \text{Does not exist}$$
	\item (Section 2.4, Related Exercise 45):
		$$f(x) = \frac{x-5}{x^2 - 25} = \frac{x-5}{(x-5)(x+5)} = \frac{1}{x+5}$$
		Vertical Asymptotes: $x=-5$
		$$\lim_{x \to 5}{f(x)} = \lim_{x \to 5}{\frac{1}{x+5}} = \frac{1}{5+5} = \frac{1}{10}$$
		$$\lim_{x \to -5^-}{f(x)} = \lim_{x \to -5^-}{\frac{1}{x+5}} = - \infty$$
		$$\lim_{x \to -5^+}{f(x)} = \lim_{x \to -5^+}{\frac{1}{x+5}} = \infty$$
	\item (Section 2.4, Related Exercise 46):
		$$f(x) = \frac{x+7}{x^4 - 49x^2} = \frac{x+7}{x^2(x^2 - 49)} = \frac{x+7}{x^2(x+7)(x-7)} = \frac{1}{x^2(x-7)} = \frac{1}{x^3 - 7x^2}$$
		Vertical Asymptotes: $x = 0$, $x = 7$, $x = -7$
		$$\lim_{x \to 7^-}{f(x)} = \lim_{x \to 7^-}{\frac{1}{x^3 - 6x^2}} = - \infty$$
		$$\lim_{x \to 7^+}{f(x)} = \lim_{x \to 7^+}{\frac{1}{x^3 - 6x^2}} = \infty$$
		$$\lim_{x \to -7}{f(x)} = \lim_{x \to -7}{\frac{1}{x^3 - 7x^2}} = \text{Does not exist}$$
		$$\lim_{x \to 0}{f(x)} = \lim_{x \to 0}{\frac{1}{x^3 - 7x^2}} = - \infty$$
	\item (Section 2.4, Related Exercise 39):
		$$\lim_{\theta \to 0^+}{\csc{\theta}} = \infty$$
	\item (Section 2.4, Related Exercise 40):
		$$\lim_{x \to 0^-}{\csc{x}} = - \infty$$
	\item (Section 2.5, Related Exercise 10)
		$$\lim_{x \to \infty}{5 + \frac{1}{x} + \frac{10}{x^2}} = 5 + 0 + 0 = 5$$
	\item (Section 2.5, Related Exercise 19)
		$$\lim_{x \to \infty}{\frac{\cos{x^5}}{\sqrt{x}}} = 0$$
	\item (Section 2.5, Related Exercise 21)
        $$\lim_{x \to \infty}{3x^12 - 9x^7} = \infty$$
	\item (Section 2.5, Related Exercise 23)
        $$\lim_{x \to - \infty}{-3x^16 + 2} = - \infty$$
	\item (Section 2.5, Related Exercise 38)
        $$f(x) = \frac{3x^2 - 7}{x^2 + 5x}$$
        Horizontal Asymptote: $y = 3$
	\item (Section 2.5, Related Exercise 41)
        $$f(x) = \frac{3x^3 - 7}{x^4 + 5x^2}$$
        Horizontal Asymptote: $y = 0$
	\item (Section 2.5, Related Exercise 43)
        $$f(x) = \frac{40x^5 + x^2}{16x^4 - 2x}$$
        Horizontal Asymptote: None
        \\ \textbf{Explanation}: Since the limit is infinity, there is no horizontal asymptote.
	\item (Section 2.5, Related Exercise 51)
        $$f(x) = \frac{x^2 - 3}{x+6}$$
		Slant Asymptote: $y = x - 6$ \\
        Vertical Asymptote: $x = -6$
	\item (Section 2.5, Related Exercise 52)
		$$f(x) = \frac{x^2 - 1}{x + 2}$$
		Slant Asymptote: $y = x - 2$ \\
        Vertical Asymptote: $x = -2$
	\item (Section 2.5, Related Exercise 46)
        $$f(x) = \frac{\sqrt{x^2 + 1}}{2x+1}$$
        \begin{eqnarray}
            \lim_{x \to \infty}{f(x)} &=& \lim_{x \to \infty}{\frac{\sqrt{x^2 + 1}}{2x+1}} \\
			&=& \lim_{x \to \infty}{\frac{\sqrt{\frac{x^2}{x^2} + \frac{1}{x^2}}}{\frac{2x}{x}+\frac{1}{x}}} \\
			&=& \lim_{x \to \infty}{\frac{\sqrt{1 + \frac{1}{x^2}}}{2+\frac{1}{x}}} \\
			&=& \frac{\sqrt{1 + 0}}{2+0} \\
			&=& \frac{\sqrt{1}}{2} \\
			&=& \frac{1}{2}
        \end{eqnarray}
		\begin{eqnarray}
			\lim_{x \to - \infty}{f(x)} &=& \lim_{x \to - \infty}{\frac{- \sqrt{x^2 + 1}}{2x+1}} \\
			&=& \lim_{x \to - \infty}{\frac{- \sqrt{\frac{x^2}{x^2} + \frac{1}{x^2}}}{\frac{2x}{x}+\frac{1}{x}}} \\
			&=& \frac{- \sqrt{1 + 0}}{2+0} \\
			&=& \frac{- 1}{2} \\
			&=& - \frac{1}{2}
		\end{eqnarray}
		Horizontal Asymptotes: $y = \frac{1}{2}$, $y = - \frac{1}{2}$
	\item (Section 2.5, Related Exercise 47)
		$$f(x) = \frac{4x^3 + 1}{2x^3 + \sqrt{16x^6 + 1}}$$
		\begin{eqnarray}
			\lim_{x \to \infty}{f(x)} &=& \lim_{x \to \infty}{\frac{4x^3 + 1}{2x^3 + \sqrt{16x^6 + 1}}} \\
			&=& \lim_{x \to \infty}{\frac{\frac{4x^3}{x^3} + \frac{1}{x^3}}{\frac{2x^3}{x^3} + \sqrt{\frac{16x^6}{x^6} + \frac{1}{x^6}}}} \\
			&=& \frac{4 + 0}{2 + \sqrt{16 + 0}} \\
			&=& \frac{4}{2 + \sqrt{16}} \\
			&=& \frac{4}{2 + 4} \\
			&=& \frac{4}{6} \\
			&=& \frac{2}{3}
		\end{eqnarray}
		\begin{eqnarray}
			\lim_{x \to - \infty}{f(x)} &=& \lim_{x \to - \infty}{\frac{4x^3 + 1}{2x^3 - \sqrt{16x^6 + 1}}} \\
			&=& \lim_{x \to - \infty}{\frac{\frac{4x^3}{x^3} + \frac{1}{x^3}}{\frac{2x^3}{x^3} - \sqrt{\frac{16x^6}{x^6} + \frac{1}{x^6}}}} \\
			&=& \frac{4 + 0}{2 - \sqrt{16 + 0}} \\
			&=& \frac{4}{2 - \sqrt{16}} \\
			&=& \frac{4}{2 - 4} \\
			&=& \frac{4}{-2} \\
			&=& -2
		\end{eqnarray}
		Horizontal Asymptotes: $y = \frac{2}{3}$, $y = -2$
	\item (Section 2.5, Related Exercise 57)
        $$f(x) = -3e^{-x}$$
        $$\lim_{x \to \infty}{-3e^{-x}} = 3(0) = 0$$
        $$\lim_{x \to - \infty}{-3e^{-x}} = 3(\infty) = \infty$$
        Horizontal Asymptotes: $y = 0$
	\item (Section 2.5, Related Exercise 59)
        $$f(x) = 1 - \ln{x}$$
        $$\lim_{x \to \infty}{(1 - \ln{x})} = 1 - \infty = - \infty$$
        $$\lim_{x \to 0^+}{(1 - \ln{x})} = 1 + \infty = \infty$$
	\item (Section 2.5, Related Exercise 62)
        $$f(x) = \frac{50}{e^{2x}}$$
        $$\lim_{x \to \infty}{\frac{50}{e^{2x}}} = \frac{50}{\infty} = 0$$
        $$\lim_{x \to - \infty}{\frac{50}{e^{2x}}} = \text{Does not exist}$$
	\item (Section 2.6, Related Exercise 5)
		\\ $x = 1$ (Removable Discontinuity)
		\\ $x = 2$ (Removable Discontinuity)
		\\ $x = 3$ (Jump Discontinuity)
	\item (Section 2.6, Related Exercise 6)
		\\ $x = 1$ (Removable Discontinuity)
		\\ $x = 2$ (Jump Discontinuity)
		\\ $x = 3$ (Removable Discontinuity)
	\item (Section 2.6, Related Exercise 17)
		$$f(x) = \frac{2x^2 + 3x + 1}{x^2 + 5x}$$
		$$a = -5$$
		$$f(a) = f(-5) = \frac{2(-5)^2 + 3(-5) + 1}{(-5)^2 + 5(-5)} = \frac{2(25) - 15 + 1}{25 - 25} = \frac{50 - 15 + 1}{0} = \frac{36}{0} = \text{undefined}$$
		$f$ is not continuous at $a$ as $f(a)$ is undefined.
	\item (Section 2.6, Related Exercise 22)
		$$f(x) = \begin{aligned}
			\begin{cases}
				\frac{x^2 - 4x + 3}{x - 3} &\text{ if } x \neq 3 \\
				2 &\text{ if } x = 3
			\end{cases}
		\end{aligned}$$
		$$a = 3$$
		$$f(a) = f(3) = 2$$
		$$\lim_{x \to a}{f(x)} = \lim_{x \to 3}{f(x)} = \lim_{x \to 3}{\frac{x^2 - 4x + 3}{x - 3}} = \lim_{x \to 3}{\frac{(x - 1)(x - 3)}{x - 3}} = \lim_{x \to 3}{(x - 1)} = 2$$
		$f$ is continuous at $a$
	\item (Section 2.6, Related Exercise 26)
		\\ $(- \infty, \infty)$
	\item (Section 2.6, Related Exercise 27)
		\\ $(- \infty, -3)$
		\\ $(-3, 3)$
		\\ $(3, \infty)$
		\\ \textbf{Explanation}: Because of the denominator, this function will be undefined when $x$ is $-3$ or $3$
	\item (Section 2.6, Related Exercise 31)
		$$\lim_{x \to 0}{(x^8 - 3x^6 - 1)^{40}} = (0 - 0 - 1)^{40} = -1^{40} = 1$$
	\item (Section 2.6, Related Exercise 32)
		$$\lim_{x \to 2}{\left (\frac{3}{2x^5 - 4x^2 - 50} \right )^4} = \left (\frac{3}{2(2)^5 - 4(2)^2 - 50} \right )^4 = \left (\frac{3}{64 - 16 - 50} \right )^4 = \left (\frac{3}{-2} \right )^4 = \left (- \frac{3}{2} \right )^4 = \frac{81}{16}$$
	\item (Section 2.6, Related Exercise 33)
		\begin{eqnarray}
			\lim_{x \to 4}{\sqrt{\frac{x^3 - 2x^2 - 8x}{x - 4}}} &=& \lim_{x \to 4}{\sqrt{\frac{x(x^2 - 2x - 8)}{x - 4}}} \\
			&=& \lim_{x \to 4}{\sqrt{\frac{x(x - 4)(x + 2)}{x - 4}}} \\
			&=& \lim_{x \to 4}{\sqrt{x(x + 2)}} \\
			&=& \lim_{x \to 4}{\sqrt{x^2 + 2x}} \\
			&=& \sqrt{4^2 + 2(4)} \\
			&=& \sqrt{16 + 8} \\
			&=& \sqrt{24} \\
			&=& 2\sqrt{6}
		\end{eqnarray}
	\item (Section 2.6, Related Exercise 34)
		$$\lim_{x \to 4}{\frac{t - 4}{\sqrt{t} - 2}} = \lim_{x \to 4}{\frac{t - 4}{\sqrt{t} - 2} \cdot \frac{\sqrt{t} + 2}{\sqrt{t} + 2}} = \lim_{x \to 4}{\frac{(t - 4)(\sqrt{t} + 2)}{t - 4}} = \lim_{x \to 4}{\sqrt{t} + 2} = \sqrt{4} + 2 = 2 + 2 = 4$$
	\item (Section 2.6, Related Exercise 39)
		$$f(x) = \begin{aligned}
			\begin{cases}
				2x &\text{ if } x < 1 \\
				x^2 + 3x &\text{ if } x \geq 1
			\end{cases}
		\end{aligned}$$
		$$a = 1$$
		$$f(a) = f(1) = 1^2 + 3(1) = 1+3 = 4$$
		$$lim_{x \to 1^-}{f(x)} = 2$$
		$$lim_{x \to 1^+}{f(x)} = 4$$
		$$lim_{x \to }$$
	\item (Section 2.6, Related Exercise 40)
	\item (Section 2.6, Related Exercise 44)
	\item (Section 2.6, Related Exercise 45)
	\item (Section 2.6, Related Exercise 62)
	\item (Section 2.6, Related Exercise 63)
	\item (Section 2.6, Related Exercise 67)
	\item (Section 2.6, Related Exercise 75)
\end{enumerate}
\blfootnote{A copy of my notes (in \LaTeX) are available on my \href{https://github.com/onlinechronically/MATH-211}{GitHub}}
\end{document}
