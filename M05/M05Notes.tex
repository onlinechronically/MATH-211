\documentclass{article}
\usepackage{graphicx}
\usepackage{amsthm}
\usepackage{amsmath}
\usepackage{amssymb}
\usepackage{geometry}
\usepackage{tikz}
\usepackage[hidelinks]{hyperref}
\usetikzlibrary{arrows}

\geometry{a4paper, total={170mm,257mm}, left=20mm, top=20mm}
\AtBeginEnvironment{align}{\setcounter{equation}{0}} 
\AtBeginEnvironment{eqnarray}{\setcounter{equation}{0}} 

\newcommand\blfootnote[1]{
    \begingroup
    \renewcommand\thefootnote{}\footnote{#1}
    \addtocounter{footnote}{-1}
    \endgroup
}

\title{Module 5 Notes (MATH-211)}
\author{Lillie Donato}
\date{08 July 2024}

\begin{document}

\maketitle

\section*{General Notes (and Definitions)}
\begin{itemize}
    \item Maxima and Minima \\
    \textbf{Absolute Maximum:} Assume a function $f$ is defined on a set $D$, and $x = c$ is a point in $D$. Then, $y = f(c)$ is an \textbf{absolute maximum value} of $f$ on $D$ if $f(c) \geq f(x)$ for every $x$ in $D$. Changing the set on which $f$ is defined \underline{may} change the absolute maximum value. \\
    \textbf{Absolute Minimum}: Assume a function $f$ is defined on a set $D$, and $x = c$ is a point in $D$. Then, $y = f(c)$ is an \textbf{absolute minimum value} of $f$ on $D$ if $f(c) \leq f(x)$ for every $x$ in $D$. Changing the set on which $f$ is defined \underline{may} change the absolute minimum value. \\
    \textbf{Extreme Value Theorem}: A function that is continuous on a closed interval is guarenteed to have both an absolute maximum value and an absolute minmum value. \\
    A discontinuous function, or a function defined on an interval that is not closed, may still have absolute extrema. \\
    \textbf{Local Maximum and Minimum Values}: Assume $x = c$ is an interior point (not an endpoint) of some interval $I$ in the domain of $f$. Then, $y = f(c)$ is a \textbf{local maximum value} of $f$ if $f(c) \geq f(x)$ for every $x$ in $I$, and $y = f(c)$ is a \textbf{local minimum value} of $f$ if $f(c) \leq f(x)$ for every $x$ in $I$. \\
    \textbf{Critical Points}: An interior point $x = c$ of the domain of $f$ is called a \textbf{critical point} of $f$ if either $f'(c) = 0$ or $f'(c)$ does not exist. \\
    \textbf{Local Extreme Value Theorem}: If a function $f$ has a local maximum or a local minimum at a point $x = c$, then either $f'(c) = 0$ or $f'(c)$ does not exist. \\
    If $f$ has a local extreme, it must occur at a critical point. \\
    Not every critical point is the location of a local extreme value. \\
    For a continuous function $f$ on a closed interval $[a, b]$, absolute extremes are guaranteed to exist, and they must occur either at the endpoints of interval or at critical points of $f$ within the interval.
    \item Mean Value Theorem \\
    \textbf{Rolle's Theorem}: Let $f$ be a continuous function on a closed interval $[a, b]$ that is differentiable on $(a, b)$, with $f(a) = f(b)$. Then, there is at least one point $x = c$ in $(a, b)$ where $f'(c) = 0$. \\
    \textbf{Mean Value Theorem}: If $f$ is a continuous function on a closed interval $[a, b]$ that is differentiable on $(a, b)$, then there is at least one point $x = c$ in $(a, b)$ where
    $$f'(c) = \frac{f(b) - f(a)}{b - a}$$
    \textbf{Zero Derivative Implies Constant Function}: If $f$ is differentiable on an open interval $I$, and $f'(x) = 0$ for all $x$ in $I$, then $f$ is a constant function on $I$. \\
    \textbf{Function with Equal Derivative Differ by a Constant}: If $f'(x) = g'(x)$ for all $x$ in an open interval $I$, then $f(x) = g(x) + C$ for some constant $C$.
    \item What Derivatives Tell Us \\
    \textbf{Increasing and Decreasing Functions}: Suppose a function $f$ is defined on an interval $I$. We say $f$ is \textbf{increasing} on $I$ if $f(x_2) > f(x_1)$ whenever $x_1$ and $x_2$ are in $I$ and $x_2 > x_1$, and we say $f$ is decreasing on $I$ if $f(x_2) < f(x_1)$ whenever $x_1$ and $x_2$ are in $I$ and $x_2 > x_1$. \\
    \textbf{Test for Interals of Increase and Decrease}: Suppose a function $f$ is defined on an interval $I$, and differentiable inside $I$. If $f'(x) > 0$ at all interior points of $I$, then $f$ is increasing on $I$; If $f'(x) < 0$ at all interior points of $I$, then $f$ is decreasing on $I$. \\
    \textbf{First Derivative Test}: Assume $f$ is continuous on an interval containing a critical point $c$, and that $f$ is differentiable on an interval containing $c$ (except possible at $c$ itself). Under these conditions:
    \begin{itemize}
        \item If $f'$ changes sign from positive to negative as $x$ increases through $c$, then $f$ has a local maximum at $c$.
        \item If $f'$ changes sign from negative to positive as $x$ increases through $c$, then $f$ has a local minimum at $c$.
        \item If $f'$ is positive on both sides of $c$, or negative on both sides of $c$, then $f$ has no local extreme value at $c$.
    \end{itemize}
\end{itemize}

\section*{Examples}
\begin{enumerate}
    \item Locate absolute maxima and minima from a graph \\
    Absolute Maximum: $f(c)$ and occurs at $x = c$ \\
    Absolute Minimum: None, as the $f(b)$ does not exist
    \item Locate local maxima and minima from a graph \\
    Absolute Min at $(a, f(a))$ \\
    Absolute Max at $(p, f(p))$ \\
    Local Max at $(p, f(p))$ \\
    Local Max at $(r, f(r))$ \\
    Local Min at $(q, f(q))$ \\
    Local Min at $(s, f(s))$
    \item Find critical points of a function
    $$f(t) = t^2 - 2\ln{\left(t^2 + 1\right)}$$
    $$f'(t) = \frac{2t\left(t+1\right)\left(t-1\right)}{t^2 + 1}$$
    Critical Point at $x = -1$ \\
    Critical Point at $x = 0$ \\
    Critical Point at $x = 1$
    \item Find absolute extremes of a continuous function on a closed interval
    $$f(x) = \frac{x}{\left(x^2 + 9\right)^5}$$
    $$f'(x) = \frac{-9x^2 + 9}{\left(x^2 + 9\right)^6}$$
    $$[-2, 2]$$
    $$f(-2) \approx -0.000005$$
    $$f(2) \approx 0.000005$$
    $$f(-1) = -0.00001$$
    $$f(1) = 0.00001$$
    Absolute Min at $(-1, f(-1))$ \\
    Absolute Max at $(1, f(1))$
    \item Application of finding absolute extreme values
    $$P(x) = 2x + \frac{128}{x}$$
    $$P'(x) = 2 + \frac{-128}{x^2}$$
    $$(0, \infty)$$
    $$f(8) = 18$$
    Absolute min at $(8, 32)$ or a perimeter of 32 units
    \item Verifying Rolle's Theorem
    $$f(x) = x^3 - 2x^2 - 8x$$
    $$f'(x) = 3x^2 - 4x - 8$$
    $$[-2, 4]$$
    $$f(-2) = 0 = f(4)$$
    $$x = \frac{2 + 2\sqrt{7}}{3} \approx 2.43$$
    $$x = \frac{2 - 2\sqrt{7}}{3} \approx -1.097$$
    $$x = \frac{2 \pm 2\sqrt{7}}{3}$$
    \item Verifying the Mean Value Theorem
    $$f(x) = x^3 - 2x^2$$
    $$f'(x) = 3x^2 - 4x$$
    $$[0, 1]$$
    $$f'(c) = -1$$
    $$(3x - 1)(x - 1) = 0$$
    $$x = \frac{1}{3}$$
    $$f\left(\frac{1}{3}\right) = -1$$
    \item Application of the Mean Value Theorem
    $$\frac{30}{27} = \frac{30}{0.45} \approx 66.667$$
    $$66.667 > 60$$
    \item Find the intervals of increase and decrease of a function
    $$f(x) = \frac{x^3}{3} - \frac{5x^2}{2} + 4x$$
    $$f'(x) = \left(x - 4\right)\left(x - 1\right)$$
    Critical Points: $x = 1$ and $x = 4$ \\
    For some $x \in (-\infty, 1)$, $f(x) > 0$ \\
    For some $x \in (1, 4)$, $f(x) < 0$ \\
    For some $x \in (4, \infty)$, $f(x) > 0$ \\
    $f$ is increasing at the following intervals: $(-\infty, 1)$ and $(4, \infty)$ \\
    $f$ is decreasing at the following intervals: $(1, 4)$
    \item Use the First Derivative Test to find local extrema
    $$f(x) = -x^3 + 9x$$
    $$f'(x) = -3x^2 + 9$$
    There are critical points at $x = \pm \sqrt{3}$ \\
    There is a local minimum at $x = -\sqrt{3}$ and $f(-\sqrt{3}) \approx -10.39230485$ \\
    There is a local maximum at $x = \sqrt{3}$ and $f(\sqrt{3}) \approx 10.39230485$ \\
    There is an absolute minimum at $x = -\sqrt{3}$ and $f(-\sqrt{3}) \approx -10.39230485$ \\
    There is an absolute maximum at $x = -4$ and $f(-4) = 28$
\end{enumerate}

\section*{Related Exercises}
\begin{enumerate}
    \item (Section 4.1, Exercise 11) \\
        Absolute Min at $x = c_2$ \\
        Absolute Max at $x = b$
    \item (Section 4.1, Exercise 14) \\
        Absolute Min at $x = c$ \\
        Absolute Max at $x = b$
    \item (Section 4.1, Exercise 15) \\
        Absolute Max at $x = b$ \\
        Absolute Min at $x = a$ \\
        Local Max at $x = p$ \\
        Local Max at $x = r$ \\
        Local Min at $x = q$ \\
        Local Min at $x = s$
    \item (Section 4.1, Exercise 18)
        Absolute Max at $x = p$ \\
        Absolute Min at $x = u$ \\
        Local Max at $x = p$ \\
        Local Max at $x = r$ \\
        Local Max at $x = t$ \\
        Local Min at $x = q$ \\
        Local Min at $x = s$ \\
        Local Min at $x = u$ \\
    \item (Section 4.1, Exercise 35)
        $$f(x) = \frac{1}{x} + \ln{x}$$
        $$f'(x) = \frac{x-1}{x^2}$$
        Critical Points at $x = 1$
    \item (Section 4.1, Exercise 36)
        $$f(t) = t^2 - 2\ln{\left(t^2 + 1\right)}$$
        $$f'(t) = \frac{2t\left(t+1\right)\left(t-1\right)}{t^2 + 1}$$
        Critical Points at $t = -1$, $t = 0$ and $t = 1$
    \item (Section 4.1, Exercise 46)
        $$f(x) = x^4 - 4x^3 + 4x^2$$
        $$f'(x) = 4x^3 - 12x^2 + 8x$$
        $$[-1, 3]$$
        $$f(-1) = 9$$
        $$f(0) = 0$$
        $$f(1) = 1$$
        $$f(2) = 0$$
        $$f(3) = 9$$
        Absolute Max at $(-1, 9)$ and $(3, 9)$ \\
        Absolute Min at $(0, 0)$ and $(2, 0)$ \\
    \item (Section 4.1, Exercise 52)
        $$f(x) = 3x^{\frac{2}{3}}$$
        $$f'(x) = \frac{2}{x^{\frac{1}{3}}}$$
        $$[0, 27]$$
        $$f(0) = 0$$
        $$f(27) = 27$$
        Absolute Min at $(0, 0)$ \\
        Absolute Min at $(27, 27)$
    \item (Section 4.1, Exercise 73)
        $$s(t) = -16t^2 + 64t + 192$$
        $$s'(t) = -32t + 64$$
        $$0 \leq t \leq 6$$
        $$s(0) = 192$$
        $$s(2) = 256$$
        $$s(6) = 0$$
        The stone will reach its maximum height at $2$ seconds
    \item (Section 4.2, Exercise 11)
        $$f(x) = x\left(x - 1\right)^2$$
        $$f'(x) = \left(x - 1\right)^2 + 2x\left(x - 1\right)$$
        $$[0, 1]$$
        $$f(0) = 0$$
        $$f(1) = 0$$
        $$f'\left(\frac{1}{3}\right) = 0$$
    \item (Section 4.2, Exercise 16)
        $$f(x) = x^3 - 2x^2 - 8x$$
        $$f'(x) = 3x^2 - 4x - 8$$
        $$[-2, 4]$$
        $$f(-2) = 0$$
        $$f(4) = 0$$
        $$x \approx -1.097$$
        $$x \approx 2.431$$
    \item (Section 4.2, Exercise 19)
        $$f(6.1) = -10.3$$
        $$f(3.2) = 8.0$$
        \begin{eqnarray}
            \frac{-10.3 - 8.0}{6.1 - 3.2} &=& \frac{-18.3}{2.9} \\
                                          &\approx& -6.3
        \end{eqnarray}
        Because the average lapse rate is approximately $-6.3$, we are unable to conclude that it exceeds $7$.
    \item (Section 4.2, Exercise 42)
        \begin{enumerate}
            \item Formations of a weak layer are likely as the following temperature gradient is greater than $10$ degrees celsius.
                $$\frac{14}{1.1} \approx 12.72$$
            \item Formations of a weak layer are not likely as the following temperature gradient is less than 10 degrees celsius.
                $$\frac{11}{1.4} \approx 7.86$$
            \item A weak layer is more likely to form when there is less of a difference in the deepness of the snowpack, as there is a higher chance of a greater temperature gradient.
            \item A weak layer most likely will not form in isothermal snow because if the temperatures are the same, then we know the value of the temperature gradient would be $0$.
        \end{enumerate}
    \item (Section 4.2, Exercise 21)
        $$f(x) = 7 - x^2$$
        $$f'(x) = -2x$$
        $$[-1, 2]$$
        $$f(-1) = 6$$
        $$f(2) = 3$$
        $$f(c) = \frac{-3}{3} = -1$$
        $$c = \frac{1}{2}$$
    \item (Section 4.2, Exercise 22)
        $$f(x) = x^3 - 2x^2$$
        $$f'(x) = 3x^2 - 4x$$
        $$[0, 1]$$
        $$f(0) = 0$$
        $$f(1) = -1$$
        $$f(c) = \frac{-1}{1} = -1$$
        $$c = \frac{1}{3}$$
\end{enumerate}

\blfootnote{A copy of my notes (in \LaTeX) are available on my \href{https://github.com/onlinechronically/MATH-211}{GitHub}}
\end{document}
