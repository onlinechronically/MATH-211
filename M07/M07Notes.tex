\documentclass{article}
\usepackage{graphicx}
\usepackage{amsthm}
\usepackage{amsmath}
\usepackage{amssymb}
\usepackage{geometry}
\usepackage{tikz}
\usepackage[hidelinks]{hyperref}
\usetikzlibrary{arrows}

\geometry{a4paper, total={170mm,257mm}, left=20mm, top=20mm}
\AtBeginEnvironment{align}{\setcounter{equation}{0}} 
\AtBeginEnvironment{eqnarray}{\setcounter{equation}{0}} 

\newcommand\blfootnote[1]{
    \begingroup
    \renewcommand\thefootnote{}\footnote{#1}
    \addtocounter{footnote}{-1}
    \endgroup
}

\title{Module 7 Notes (MATH-211)}
\author{Lillie Donato}
\date{22 July 2024}

\begin{document}

\maketitle

\section*{General Notes (and Definitions)}
\begin{itemize}
    \item Working with Integrals \\
        A function $f(x)$ is \textbf{even} if $f(-x) = f(x)$. \\
        A function $f(x)$ is \textbf{odd} if $f(-x) = -f(x)$. \\
        Let $a \in \mathbb{R}$ such that $a > 0$ and let $f$ be an integrable function on the interval $[-a,a]$.
        $$\text{If } f \text{ is even, } \int_{-a}^a{f(x)\,dx} = 2\int_0^a{f(x)\,dx}$$
        $$\text{If } f \text{ is odd, } \int_{-a}^a{f(x)\,dx} = 0$$
        The average value of an integrable function $f$ on the interval $[a,b]$ is
        $$\overline{f} = \frac{1}{b - a}\int_a^b{f(x)\,dx}$$
        Let $f$ be continuous on the interval $[a,b]$. There exists a point $c$ in $(a,b)$ such that (Mean Value Theorem)
        $$f(c) = \overline{f} = \frac{1}{b - a}\int_a^b{f(t)\,dx}$$
    \item Substitution Rule \\
        Let $u = g(x)$, where $g$ is differentiable on an interval, and let $f$ be continuous on the corresponding range of $g$. On that interval,
        $$\int{f(g(x))g'(x)\,dx} = \int{f(u)\,du}$$
        \begin{enumerate}
            \item Given an indefinite integral involving a commposite function $f(g(x))$, identify an inner function $u = g(x)$ such that a constant multiple of $g'(x)$ appears in the integrand.
            \item Substitute $u = g(x)$ and $du = g'(x)\,dx$ in the integral.
            \item Evaluate the new indefinite integral with respect to $u$.
            \item Write the result in terms of $x$ using $u = g(x)$.
        \end{enumerate}
        Let $u = g(x)$, where $g'$ is continuous on $[a,b]$, and let $f$ be continuous on the range of $g$. Then
        $$\int_a^b{f(g(x))g'(x)\,dx} = \int_{g(a)}^{g(b)}{f(u)\,du}$$
\end{itemize}

\section*{General formulas for indefinite integrals}
\begin{eqnarray}
    \int{\cos{ax}\,dx} &=& \frac{1}{a}\sin{ax} + C \\
    \int{\sin{ax}\,dx} &=& -\frac{1}{a}\cos{ax} + C \\
    \int{\sec^2{ax}\,dx} &=& \frac{1}{a}\tan{ax} + C \\
    \int{\csc^2{ax}\,dx} &=& -\frac{1}{a}\cot{ax} + C \\
    \int{\sec{ax}\tan{ax}\,dx} &=& \frac{1}{a}\sec{ax} + C \\
    \int{\csc{ax}\cot{ax}\,dx} &=& -\frac{1}{a}\csc{ax} + C \\
    \int{e^{ax}\,dx} &=& \frac{1}{a}e^{ax} + C \\
    \int{b^x\,dx} &=& \frac{1}{\ln{b}}b^x + C, b > 0, b \neq 1 \\
    \int{\frac{dx}{a^2 + x^2}} &=& \frac{1}{a}\tan^{-1}{\frac{x}{a}} + C \\
    \int{\frac{dx}{\sqrt{a^2 - x^2}}} &=& \sin^{-1}{\frac{x}{a}} + C, a > 0 \\
    \int{\frac{dx}{x\sqrt{x^2 - a^2}}} &=& \frac{1}{a}\sec^{-1}{\left|\frac{x}{a}\right|} + C, a > 0
\end{eqnarray}

\section*{Examples}
\begin{enumerate}
    \item Use symmetry to evaluate integrals
        $$\int_{-200}^{200}{2x^5\,dx} = 0$$
        \begin{eqnarray}
            \int_{-2}^{2}{\left(x^2 + x^3\right)\,dx} &=& \int_{-2}^{2}{x^2\,dx} + \int_{-2}^{2}{x^3\,dx} \\
                                                      &=& 2\int_{0}^{2}{x^2\,dx} + 0 \\
                                                      &=& 2\frac{x^3}{3} \\
                                                      &=& \frac{16}{3}
        \end{eqnarray}
    \item A derivative calculation
        $$s(t) = -16t^2 + 64t$$
        $$t = 4$$
        $$[0,4]$$
        \begin{eqnarray}
            v(t) &=& s'(t) \\
            \overline{v} &=& \frac{1}{4}\int_0^4{v(t)\,dx} \\
                         &=& \frac{1}{4}\int_0^4{s'(t)\,dx} \\
                         &=& \frac{1}{4}s(t) \\
                         &=& \frac{1}{4}\left(s(4) - s(0)\right) \\
                         &=& 0
        \end{eqnarray}
    \item Applying MVT for integrals
        $$f(x) = e^x$$
        $$[0,2]$$
        \begin{eqnarray}
            \overline{f} &=& \frac{1}{2}\left(\int_0^2{e^x\,dx}\right) \\
                         &=& \frac{e^x}{2} \\
                         &=& \frac{e^2}{2} - \frac{e^0}{2} \\
                         &=& \frac{e^2 - 1}{2} \\
            e^x &=& \frac{e^2 - 1}{2} \\
            \ln{e^x} &=& \ln{\frac{e^2 - 1}{2}}
        \end{eqnarray}
    \item Perfect substitutions in indefinite integrals
        \begin{eqnarray}
            u &=& 4x^3 - 8 \\
            du &=& 12x^2\,dx \\
            \int{12x^2\left(4x^3 - 8\right)^5\,dx} &=& \int{12x^2u^5\,dx} \\
                                                   &=& \frac{u^6}{6} + C \\
                                                   &=& \frac{\left(4x^3 - 8\right)^6}{6} + C
        \end{eqnarray}
        \begin{eqnarray}
            u &=& \sin{t} \\
            du &=& \cos{t}\,dt \\
            \int{\left(\cos{t}\right)e^{\sin{t}}\,dt} &=& \int{e^u\,du} \\
                                                      &=& e^u + C \\
                                                      &=& e^{\sin{t}} + C
        \end{eqnarray}
    \item Introducting constants when integrating by substitution
        \begin{eqnarray}
            u &=& 6x + 4 \\
            du &=& 6\,dx \\
            dx &=& \frac{du}{6} \\
            \int{\left(6x + 4\right)^9\,dx} &=& \int{\frac{1}{6} \cdot u^9\,du} \\
                                            &=& \frac{1}{6}\int{u^8\,du} \\
                                            &=& \frac{1}{6} \cdot \frac{u^9}{9} + C \\
                                            &=& \frac{\left(6x + 4\right)^9}{54} + C
        \end{eqnarray}
        \begin{eqnarray}
            u &=& \cot{x} \\
            du &=& -\csc^2{x}\,dx \\
            \int{\cot^2{x}\csc^2{x}\,dx} &=& \int{-u^2\,du} \\
                                         &=& -\frac{u^3}{3} + C \\
                                         &=& -\frac{\csc^3{x}}{3} + C
        \end{eqnarray}
    \item Variations on the substitution method
        \begin{eqnarray}
            u &=& x - 1 \\
            du &=& dx \\
            x &=& u + 1 \\
            \int{x\sqrt{x - 1}\,dx} &=& \int{\left(u + 1\right)\sqrt{u}\,du} \\
                                    &=& \int{u\sqrt{u} + \sqrt{u}\,du} \\
                                    &=& \int{u^{\frac{3}{2}} + u^{\frac{1}{2}}\,du} \\
                                    &=& \frac{2}{5}u^{\frac{5}{2}} + \frac{2}{3}u^{\frac{3}{2}} + C \\
                                    &=& \frac{2}{5}\left(x - 1\right)^{\frac{5}{2}} + \frac{2}{3}\left(x - 1\right)^{\frac{3}{2}} + C
        \end{eqnarray}
    \item Use known formulas to evaluate indefinite integrals
        \begin{eqnarray}
            \int{2e^{-4x}\,dx} &=& 2\int{e^{-4x}\,dx} \\
                               &=& \frac{2}{-4}e^{-4x} + C \\
                               &=& -\frac{1}{2}e^{-4x} + C \\
        \end{eqnarray}
        \begin{eqnarray}
            \int{\frac{dx}{\sqrt{36 - x^2}}} &=& \int{\frac{dx}{\sqrt{6^2 - x^2}}} \\
                                             &=& \sin^{-1}{\frac{x}{6}} + C
        \end{eqnarray}
    \item Evaluating definite integrals using substitution
        \begin{eqnarray}
            u &=& 2^x + 4 \\
            du &=& 2^x\ln{2}\,dx \\
            \frac{1}{\ln{2}}\,du &=& 2^x\,dx \\
            \int_1^3{\frac{2^x}{2^x + 4}\,dx} &=& \int_1^3{\frac{1}{u\ln{2}}\,du} \\
            \int_{g(1)}^{g(3)}{\frac{1}{u\ln{2}}\,du} &=& \int_{6}^{12}{\frac{1}{u\ln{2}}\,du} \\
                                                      &=& \frac{1}{\ln{2}}\int_{6}^{12}{\frac{du}{u}} \\
                                                      &=& \frac{1}{\ln{2}}\cdot\left(\ln{12} - \ln{6}\right) \\
                                                      &=& \frac{\ln{2}}{\ln{2}} \\
                                                      &=& 1
        \end{eqnarray}
        \begin{eqnarray}
            u &=& \ln{p} \\
            du &=& \frac{1}{p}\,dx \\
            \int_1^{e^2}{\frac{\ln{p}}{p}} &=& \int_0^2{u\,du} \\
                                           &=& \frac{2^2}{2} - \frac{0^2}{2} \\
                                           &=& \frac{4}{2} \\
                                           &=& 2
        \end{eqnarray}
    \item Integrals involving $\cos^2{x}$ and $\sin^2{x}$
        \begin{eqnarray}
            u &=& 2x \\
            du &=& 2\,dx \\
            dx &=& \frac{1}{2}du \\
            \sin^2{x} &=& \frac{1 - \cos{2x}}{2} \\
            \int_0^{\pi}{\sin^2{x}\,dx} &=& \int_0^{\pi}{\frac{1 - \cos{2x}}{2}\,dx} \\
                                        &=& \frac{1}{2}\int_0^{\pi}{1 - \cos{2x}\,dx} \\
                                        &=& \frac{1}{2}\left(\int_0^{\pi}{1\,dx} - \int_0^{\pi}{\cos{2x}\,dx}\right) \\
                                        &=& \frac{1}{2}\left(\left(\pi - 0\right) - \frac{1}{2}\int_0^{2\pi}{\cos{u}\,du}\right) \\
                                        &=& \frac{1}{2}\left(\pi - \frac{1}{2}\left(\sin{2\pi} - \sin{0}\right)\right) \\
                                        &=& \frac{1}{2}\left(\pi - 0\right) \\
                                        &=& \frac{\pi}{2}
        \end{eqnarray}
\end{enumerate}

\section*{Related Exercises}
\begin{enumerate}
    \item (Section 5.4, Exercise 15)
        \begin{eqnarray}
            \int_{-2}^2{\left(x^2 + x^3\right)\,dx} &=& \int_{-2}^2{x^2\,dx} + \int_{-2}^2{x^3\,dx} \\
                                                    &=& 2\int_{0}^2{x^2\,dx} + 0 \\
                                                    &=& 2\frac{x^3}{3} \\
                                                    &=& 2\frac{2^3}{3} - 2\frac{0^3}{3} \\
                                                    &=& 2\frac{8}{3} \\
                                                    &=& \frac{16}{3}
        \end{eqnarray}
    \item (Section 5.4, Exercise 16)
        $$\int_{-\pi}^{\pi}{t^2\sin{t}\,dx} = 0$$
    \item (Section 5.4, Exercise 26)
        $$f(x) = x^2 + 1$$
        $$[-2, 2]$$
        \begin{eqnarray}
            \overline{f} &=& \frac{1}{2 - (-2)}\int_{-2}^2{x^2 + 1\,dx} \\
                         &=& \frac{1}{4}\left(\int_{-2}^2{x^2\,dx} + 1\int_{-2}^2{x^0\,dx}\right) \\
                         &=& \frac{1}{4}\left(\frac{x^3}{3} + x\right) \\
                         &=& \frac{1}{4}\left(\int_{-2}^2{x^2\,dx} + \int_{-2}^2{1\,dx}\right) \\
                         &=& \frac{1}{4}\left(\frac{2^3}{3} - \frac{(-2)^3}{3} + 2 - (-2)\right) \\
                         &=& \frac{1}{4}\left(\frac{8}{3} - \frac{-8}{3} + 4\right) \\
                         &=& \frac{1}{4}\left(\frac{16}{3} + 4\right) \\
                         &=& \frac{1}{4}\left(\frac{28}{3}\right) \\
                         &=& \frac{7}{3}
        \end{eqnarray}
    \item (Section 5.4, Exercise 34)
        $$f(x) = x^3 - 5x^2 + 30$$
        $$[0,4]$$
        \begin{eqnarray}
            \overline{f} &=& \frac{1}{4}\left(\int_0^4{\left(x^3 - 5x^2 + 30\right)\,dx}\right) \\
                         &=& \frac{1}{4}\left(\int_0^4{x^3} - 5\int_0^4{x^2} + 30\int_0^4{x^0}\right) \\
                         &=& \frac{1}{4}\left(\frac{x^4}{4} - 5\frac{x^3}{3} + 30x\right) \\
                         &=& \frac{1}{4}\left(\left(\frac{4^4}{4} - \frac{0^4}{4}\right) - \left(5\frac{4^3}{3} - 5\frac{0^3}{3}\right) + \left(30(4) - 30(0)\right)\right) \\
                         &=& \frac{1}{4}\left(64 - \frac{320}{3} + 120\right) \\
                         &=& \frac{1}{4}\left(\frac{232}{3}\right) \\
                         &=& \frac{58}{3}
        \end{eqnarray}
    \item (Section 5.4, Exercise 41)
        $$f(x) = 1 - \frac{x^2}{a^2}$$
        $$[0, a]$$
        \begin{eqnarray}
            \overline{f} &=& \frac{1}{a}\left(\int_0^a{1 - \frac{x^2}{a^2}\,dx}\right) \\
                         &=& \frac{1}{a}\left(\int_0^a{1\,dx} - \int_0^a{\frac{x^2}{a^2}\,dx}\right) \\
                         &=& \frac{1}{a}\left(x - \frac{1}{a^2}\int_0^a{x^2\,dx}\right) \\
                         &=& \frac{1}{a}\left(x - \frac{1}{a^2}\frac{x^3}{3}\right) \\
                         &=& \frac{1}{a}\left(x - \frac{x^3}{3a^2}\right) \\
                         &=& \frac{1}{a}\left(\left(a - 0\right) - \frac{1}{a^2}\left(\frac{a^3}{3} - \frac{0^3}{3}\right)\right) \\
                         &=& \frac{1}{a}\left(a - \frac{a^3}{3a^2}\right) \\
                         &=& \frac{1}{a}\left(a - \frac{a}{3}\right) \\
                         &=& \frac{1}{a}\left(\frac{2a}{3}\right) \\
                         &=& \frac{2}{3} \\
            1 - \frac{c^2}{a^2} &=& \frac{2}{3} \\
            \frac{c^2}{a^2} &=& \frac{1}{3} \\
            c^2 &=& \frac{a^2}{3} \\
            c &=& \sqrt{\frac{a^2}{3}} \\
              &=& \frac{a}{\sqrt{3}}
        \end{eqnarray}
    \item (Section 5.4, Exercise 42)
        $$f(x) = \frac{\pi}{4}\sin{x}$$
        $$[0, \pi]$$
        \begin{eqnarray}
            \overline{f} &=& \frac{1}{\pi}\int_0^{\pi}{\frac{\pi}{4}\sin{x}} \\
                         &=& \frac{1}{\pi}\frac{\pi}{4}\int_0^{\pi}{\sin{x}} \\
                         &=& \frac{1}{\pi}\frac{\pi}{4}\left(-\cos{x}\right) \\
                         &=& \frac{1}{\pi}\frac{\pi}{4}\left(-\cos{\pi} + \cos{0}\right) \\
                         &=& \frac{1}{\pi}\frac{\pi}{4}\left(1 + 1\right) \\
                         &=& \frac{1}{\pi}\frac{\pi}{2} \\
                         &=& \frac{1}{2} \\
            \frac{\pi}{4}\sin{x} &=& \frac{1}{2} \\
            \sin{x} &=& \frac{2}{\pi} \\
            \sin^{-1}{\sin{x}} &=& \sin^{-1}{\frac{2}{\pi}} \\
            x &=& \sin^{-1}{\frac{2}{\pi}}
        \end{eqnarray}
\end{enumerate}

\blfootnote{A copy of my notes (in \LaTeX) are available on my \href{https://github.com/onlinechronically/MATH-211}{GitHub}}
\end{document}
