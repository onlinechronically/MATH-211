\documentclass{article}
\usepackage{graphicx}
\usepackage{amsthm}
\usepackage{amsmath}
\usepackage{amssymb}
\usepackage{geometry}
\usepackage{tikz}
\usepackage[hidelinks]{hyperref}
\usetikzlibrary{arrows}

\geometry{a4paper, total={170mm,257mm}, left=20mm, top=20mm}
\AtBeginEnvironment{align}{\setcounter{equation}{0}} 
\AtBeginEnvironment{eqnarray}{\setcounter{equation}{0}} 

\newcommand\blfootnote[1]{
    \begingroup
    \renewcommand\thefootnote{}\footnote{#1}
    \addtocounter{footnote}{-1}
    \endgroup
}

\title{Module 7 Notes (MATH-211)}
\author{Lillie Donato}
\date{22 July 2024}

\begin{document}

\maketitle

\section*{General Notes (and Definitions)}
\begin{itemize}
    \item Working with Integrals \\
        A function $f(x)$ is \textbf{even} if $f(-x) = f(x)$. \\
        A function $f(x)$ is \textbf{odd} if $f(-x) = -f(x)$. \\
        Let $a \in \mathbb{R}$ such that $a > 0$ and let $f$ be an integrable function on the interval $[-a,a]$.
        $$\text{If } f \text{ is even, } \int_{-a}^a{f(x)\,dx} = 2\int_0^a{f(x)\,dx}$$
        $$\text{If } f \text{ is odd, } \int_{-a}^a{f(x)\,dx} = 0$$
        The average value of an integrable function $f$ on the interval $[a,b]$ is
        $$\overline{f} = \frac{1}{b - a}\int_a^b{f(x)\,dx}$$
        Let $f$ be continuous on the interval $[a,b]$. There exists a point $c$ in $(a,b)$ such that (Mean Value Theorem)
        $$f(c) = \overline{f} = \frac{1}{b - a}\int_a^b{f(t)\,dx}$$
    \item Substitution Rule \\
        Let $u = g(x)$, where $g$ is differentiable on an interval, and let $f$ be continuous on the corresponding range of $g$. On that interval,
        $$\int{f(g(x))g'(x)\,dx} = \int{f(u)\,du}$$
        \begin{enumerate}
            \item Given an indefinite integral involving a commposite function $f(g(x))$, identify an inner function $u = g(x)$ such that a constant multiple of $g'(x)$ appears in the integrand.
            \item Substitute $u = g(x)$ and $du = g'(x)\,dx$ in the integral.
            \item Evaluate the new indefinite integral with respect to $u$.
            \item Write the result in terms of $x$ using $u = g(x)$.
        \end{enumerate}
        Let $u = g(x)$, where $g'$ is continuous on $[a,b]$, and let $f$ be continuous on the range of $g$. Then
        $$\int_a^b{f(g(x))g'(x)\,dx} = \int_{g(a)}^{g(b)}{f(u)\,du}$$
    \item Velocity and Net Change \\
        \textbf{Position, Velocity, Displacement, and Distance}:
        \begin{enumerate}
            \item The \textbf{position} of an object moving along a line at time $t$, denoted $s(t)$, is the location of the object relative to the origin.
            \item The \textbf{velocity} of an object at time $t$ is $v(t) = s'(t)$.
            \item The \textbf{displacement} of the object between $t = a$ and $t = b > a$ is
                $$s(b) - s(a) = \int_a^b{v(t)\,dt}$$
            \item The \textbf{distance traveled} by the object between $t = a$ and $t = b > a$ is
                $$\int_a^b{\left|v(t)\right|\,dt}$$
                where $|v(t)|$ is the \textbf{speed} of the object at time $t$.
        \end{enumerate}
        \textbf{Theorem: Position from Velocity} \\
        Given the velocity $v(t)$ of an object moving along a line and its initial position $s(0)$, the position function of the object for future times $t \geq 0$ is
        $$s(t) = s(0) + \int_0^t{v(x)\,dx}$$
        \textbf{Theorem: Velocity from Acceleration} \\
        Given the acceleration $a(t)$ of an object moving along a line and its initial velocity $v(0)$, the velocity of the object for future times $t \geq 0$ is
        $$v(t) = v(0) + \int_0^t{a(x)\,dx}$$
        \textbf{Theorem: Net Change and Future Value} \\
        Suppose a quantity $Q$ changes over time at a known rate $Q'$. Then the \textbf{net change} in $Q$ between $t = a$ and $t = b > a$ is
        $$Q(b) - Q(a) = \int_a^b{Q'(t)\,dt}$$
        Given the initial value $Q(0)$, the \textbf{future value} of $Q$ at time $t \geq 0$ is
        $$Q(t) = Q(0) + \int_0^t{Q'(x)\,dx}$$
    \item Area Between Curves \\
        \textbf{Area of a Region Between Two Curves}: \\
        Suppose that $f$ and $g$ are continuous functions with $f(x) \geq g(x)$ on the interval $[a,b]$. The area of the region bounded by the graphs of $f$ and $g$ on $[a,b]$ is
        $$A = \int_a^b{\left(f(x) - g(x)\right)\,dx}$$
        \textbf{Area of a Region Between Two Curves with Respect to $y$}: \\
        Suppose that $f$ and $g$ are continuous functions with $f(y) \geq g(y)$ on the interval $[c,d]$. The area of the region bounded by the graphs $x = f(y)$ and $x = g(y)$ on $[c,d]$ is
        $$A = \int_c^d{\left(f(y) - g(y)\right)\,dy}$$
\end{itemize}

\section*{General formulas for indefinite integrals}
\begin{eqnarray}
    \int{\cos{ax}\,dx} &=& \frac{1}{a}\sin{ax} + C \\
    \int{\sin{ax}\,dx} &=& -\frac{1}{a}\cos{ax} + C \\
    \int{\sec^2{ax}\,dx} &=& \frac{1}{a}\tan{ax} + C \\
    \int{\csc^2{ax}\,dx} &=& -\frac{1}{a}\cot{ax} + C \\
    \int{\sec{ax}\tan{ax}\,dx} &=& \frac{1}{a}\sec{ax} + C \\
    \int{\csc{ax}\cot{ax}\,dx} &=& -\frac{1}{a}\csc{ax} + C \\
    \int{e^{ax}\,dx} &=& \frac{1}{a}e^{ax} + C \\
    \int{b^x\,dx} &=& \frac{1}{\ln{b}}b^x + C, b > 0, b \neq 1 \\
    \int{\frac{dx}{a^2 + x^2}} &=& \frac{1}{a}\tan^{-1}{\frac{x}{a}} + C \\
    \int{\frac{dx}{\sqrt{a^2 - x^2}}} &=& \sin^{-1}{\frac{x}{a}} + C, a > 0 \\
    \int{\frac{dx}{x\sqrt{x^2 - a^2}}} &=& \frac{1}{a}\sec^{-1}{\left|\frac{x}{a}\right|} + C, a > 0
\end{eqnarray}

\section*{Examples}
\begin{enumerate}
    \item Use symmetry to evaluate integrals
        $$\int_{-200}^{200}{2x^5\,dx} = 0$$
        \begin{eqnarray}
            \int_{-2}^{2}{\left(x^2 + x^3\right)\,dx} &=& \int_{-2}^{2}{x^2\,dx} + \int_{-2}^{2}{x^3\,dx} \\
                                                      &=& 2\int_{0}^{2}{x^2\,dx} + 0 \\
                                                      &=& 2\frac{x^3}{3} \\
                                                      &=& \frac{16}{3}
        \end{eqnarray}
    \item A derivative calculation
        $$s(t) = -16t^2 + 64t$$
        $$t = 4$$
        $$[0,4]$$
        \begin{eqnarray}
            v(t) &=& s'(t) \\
            \overline{v} &=& \frac{1}{4}\int_0^4{v(t)\,dx} \\
                         &=& \frac{1}{4}\int_0^4{s'(t)\,dx} \\
                         &=& \frac{1}{4}s(t) \\
                         &=& \frac{1}{4}\left(s(4) - s(0)\right) \\
                         &=& 0
        \end{eqnarray}
    \item Applying MVT for integrals
        $$f(x) = e^x$$
        $$[0,2]$$
        \begin{eqnarray}
            \overline{f} &=& \frac{1}{2}\left(\int_0^2{e^x\,dx}\right) \\
                         &=& \frac{e^x}{2} \\
                         &=& \frac{e^2}{2} - \frac{e^0}{2} \\
                         &=& \frac{e^2 - 1}{2} \\
            e^x &=& \frac{e^2 - 1}{2} \\
            \ln{e^x} &=& \ln{\frac{e^2 - 1}{2}}
        \end{eqnarray}
    \item Perfect substitutions in indefinite integrals
        \begin{eqnarray}
            u &=& 4x^3 - 8 \\
            du &=& 12x^2\,dx \\
            \int{12x^2\left(4x^3 - 8\right)^5\,dx} &=& \int{12x^2u^5\,dx} \\
                                                   &=& \frac{u^6}{6} + C \\
                                                   &=& \frac{\left(4x^3 - 8\right)^6}{6} + C
        \end{eqnarray}
        \begin{eqnarray}
            u &=& \sin{t} \\
            du &=& \cos{t}\,dt \\
            \int{\left(\cos{t}\right)e^{\sin{t}}\,dt} &=& \int{e^u\,du} \\
                                                      &=& e^u + C \\
                                                      &=& e^{\sin{t}} + C
        \end{eqnarray}
    \item Introducting constants when integrating by substitution
        \begin{eqnarray}
            u &=& 6x + 4 \\
            du &=& 6\,dx \\
            dx &=& \frac{du}{6} \\
            \int{\left(6x + 4\right)^9\,dx} &=& \int{\frac{1}{6} \cdot u^9\,du} \\
                                            &=& \frac{1}{6}\int{u^8\,du} \\
                                            &=& \frac{1}{6} \cdot \frac{u^9}{9} + C \\
                                            &=& \frac{\left(6x + 4\right)^9}{54} + C
        \end{eqnarray}
        \begin{eqnarray}
            u &=& \cot{x} \\
            du &=& -\csc^2{x}\,dx \\
            \int{\cot^2{x}\csc^2{x}\,dx} &=& \int{-u^2\,du} \\
                                         &=& -\frac{u^3}{3} + C \\
                                         &=& -\frac{\csc^3{x}}{3} + C
        \end{eqnarray}
    \item Variations on the substitution method
        \begin{eqnarray}
            u &=& x - 1 \\
            du &=& dx \\
            x &=& u + 1 \\
            \int{x\sqrt{x - 1}\,dx} &=& \int{\left(u + 1\right)\sqrt{u}\,du} \\
                                    &=& \int{u\sqrt{u} + \sqrt{u}\,du} \\
                                    &=& \int{u^{\frac{3}{2}} + u^{\frac{1}{2}}\,du} \\
                                    &=& \frac{2}{5}u^{\frac{5}{2}} + \frac{2}{3}u^{\frac{3}{2}} + C \\
                                    &=& \frac{2}{5}\left(x - 1\right)^{\frac{5}{2}} + \frac{2}{3}\left(x - 1\right)^{\frac{3}{2}} + C
        \end{eqnarray}
    \item Use known formulas to evaluate indefinite integrals
        \begin{eqnarray}
            \int{2e^{-4x}\,dx} &=& 2\int{e^{-4x}\,dx} \\
                               &=& \frac{2}{-4}e^{-4x} + C \\
                               &=& -\frac{1}{2}e^{-4x} + C \\
        \end{eqnarray}
        \begin{eqnarray}
            \int{\frac{dx}{\sqrt{36 - x^2}}} &=& \int{\frac{dx}{\sqrt{6^2 - x^2}}} \\
                                             &=& \sin^{-1}{\frac{x}{6}} + C
        \end{eqnarray}
    \item Evaluating definite integrals using substitution
        \begin{eqnarray}
            u &=& 2^x + 4 \\
            du &=& 2^x\ln{2}\,dx \\
            \frac{1}{\ln{2}}\,du &=& 2^x\,dx \\
            \int_1^3{\frac{2^x}{2^x + 4}\,dx} &=& \int_1^3{\frac{1}{u\ln{2}}\,du} \\
            \int_{g(1)}^{g(3)}{\frac{1}{u\ln{2}}\,du} &=& \int_{6}^{12}{\frac{1}{u\ln{2}}\,du} \\
                                                      &=& \frac{1}{\ln{2}}\int_{6}^{12}{\frac{du}{u}} \\
                                                      &=& \frac{1}{\ln{2}}\cdot\left(\ln{12} - \ln{6}\right) \\
                                                      &=& \frac{\ln{2}}{\ln{2}} \\
                                                      &=& 1
        \end{eqnarray}
        \begin{eqnarray}
            u &=& \ln{p} \\
            du &=& \frac{1}{p}\,dx \\
            \int_1^{e^2}{\frac{\ln{p}}{p}} &=& \int_0^2{u\,du} \\
                                           &=& \frac{2^2}{2} - \frac{0^2}{2} \\
                                           &=& \frac{4}{2} \\
                                           &=& 2
        \end{eqnarray}
    \item Integrals involving $\cos^2{x}$ and $\sin^2{x}$
        \begin{eqnarray}
            u &=& 2x \\
            du &=& 2\,dx \\
            dx &=& \frac{1}{2}du \\
            \sin^2{x} &=& \frac{1 - \cos{2x}}{2} \\
            \int_0^{\pi}{\sin^2{x}\,dx} &=& \int_0^{\pi}{\frac{1 - \cos{2x}}{2}\,dx} \\
                                        &=& \frac{1}{2}\int_0^{\pi}{1 - \cos{2x}\,dx} \\
                                        &=& \frac{1}{2}\left(\int_0^{\pi}{1\,dx} - \int_0^{\pi}{\cos{2x}\,dx}\right) \\
                                        &=& \frac{1}{2}\left(\left(\pi - 0\right) - \frac{1}{2}\int_0^{2\pi}{\cos{u}\,du}\right) \\
                                        &=& \frac{1}{2}\left(\pi - \frac{1}{2}\left(\sin{2\pi} - \sin{0}\right)\right) \\
                                        &=& \frac{1}{2}\left(\pi - 0\right) \\
                                        &=& \frac{\pi}{2}
        \end{eqnarray}
    \item Displacement and distance from velocity
        $$v(t) = 4t^3 - 24t^2 + 20t$$
        \begin{enumerate}
            \item
                \begin{eqnarray}
                    v(t) &=& 0 \\
                    4t^3 - 24t^2 + 20t &=& 0 \\
                    4t(t^2 - 6t + 5) &=& 0 \\
                    4t(t - 1)(t - 5) &=& 5 \\
                    t &=& 0 \\
                    t &=& 1 \\
                    t &=& 5 \\
                    0 < t < 1 &=& \text{Positive} \\
                    1 < t < 5 &=& \text{Negative} \\
                    t > 5 &=& \text{Positive}
                \end{eqnarray}
            \item
                \begin{eqnarray}
                    \int_0^5{4t^3 - 24t^2 + 20t\,dt} &=& 4\int_0^5{t^3\,dt} - 24\int_0^5{t^2\,dt} + 20\int_0^5{t\,dt} \\
                                                     &=& 4\left(\frac{5^4}{4} - \frac{0^4}{4}\right) - 24\left(\frac{5^3}{3} - \frac{0^3}{3}\right) + 20\left(\frac{5^2}{2} - \frac{0^2}{2}\right) \\
                                                     &=& 4\left(\frac{625}{4}\right) - 24\left(\frac{125}{3}\right) + 20\left(\frac{25}{2}\right) \\
                                                     &=& -125
                \end{eqnarray}
            \item
                \begin{eqnarray}
                    \int_0^5{|4t^3 - 24t^2 + 20t|\,dt} &=& \int_0^1{4t^3 - 24t^2 + 20t\,dt} + \int_1^5{-4t^3 + 24t^2 - 20t\,dt} \\
                                                     &=& 3 + 128 \\
                                                     &=& 131
                \end{eqnarray}
        \end{enumerate}
    \item Position and velocity from acceleration
        $$a(t) = \frac{20}{\left(t + 2\right)^2}$$
        $$v(0) = 20$$
        $$s(0) = 10$$
        \begin{eqnarray}
            v(t) &=& v(0) + \int_0^t{a(t)\,d5} \\
                 &=& 20 + \int_0^t{\frac{20}{\left(t + 2\right)^2}\,d5} \\
                 &=& 20 - \frac{20}{t + 2} + 10 \\
                 &=& 30 - \frac{20}{t + 2} \\
            s(t) &=& s(0) + \int_0^t{v(t)\,dt} \\
                 &=& 10 + \int_0^t{\left(30 - \frac{20}{t + 2}\right)\,dt} \\
                 &=& 10 + 30t - 20\ln{|t + 2|} + 20\ln{2}
        \end{eqnarray}
    \item Acceleration application
        $$a(t) = -15$$
        $$v(0) = 60$$
        $$s(0) = 0$$
        \begin{eqnarray}
            v(t) &=& 60 + \int_0^t{-15\,dt} \\
                 &=& 60 + -15\int_0^t{t^0\,dt} \\
                 &=& -15t + 60 \\
            s(t) &=& 0 + \int_0^t{-15t + 60\,dt} \\
                 &=& -15\int_0^t{t\,dt} + 60t \\
                 &=& -\frac{15}{2}t^2 + 60t \\
            v(t) &=& 0 \\
            60 - 15t &=& 0 \\
            15t &=& 60 \\
            t &=& \frac{60}{15} \\
              &=& 4 \\
            s(4) - s(0) &=& 60(4) - \frac{15}{2}(4)^2 - 0 \\
                        &=& 120
        \end{eqnarray}
    \item Application of net change
        $$V'(t) = 70(1 + \sin{2\pi t})$$
        $$[0,t]$$
        $$V(0) = 0$$
        \begin{eqnarray}
            V(t) &=& V(0) + \int_0^t{V'(x)\,dx} \\
                 &=& 0 + \int_0^t{70(1 + \sin{(2\pi x)}\,dx} \\
                 &=& 70\left(t - \frac{\cos{2\pi t}}{2\pi} + \frac{1}{2\pi}\right) \\
            V(60) &=& 70\left(60 - \frac{\cos{120\pi}}{2\pi} + \frac{1}{2\pi}\right) \\
                  &=& 70 \cdot 60 \\
                  &=& 4200
        \end{eqnarray}
    \item Area between curves (one integral)
        $$y = x$$
        $$y = x^2 - 2$$
        \begin{eqnarray}
            x &=& x^2 - 2 \\
            0 &=& x^2 - x - 2 \\
              &=& \left(x-2\right)\left(x+1\right) \\
            x &=& -1 \\
            x &=& 2 \\
            \int_{-1}^2{x - \left(x^2 - 2\right)\,dx} &=& \int_{-1}^2{-x^2 + x + 2\,dx} \\
                                                      &=& \left(\frac{-2^3}{3} + \frac{2^2}{2} + 2(2)\right) - \left(\frac{-(-1)^3}{3} + \frac{(-1)^2}{2} + 2(-1)\right) \\
                                                      &=& \left(\frac{-8}{3} + 6\right) - \left(\frac{1}{3} + \frac{1}{2} - 2\right) \\
                                                      &=& \frac{-9}{3} + 8  - \frac{1}{2} \\
                                                      &=& 5 - \frac{1}{2} \\
                                                      &=& 4.5
        \end{eqnarray}
    \item Area between curves (multiple integrals)
        $$y = x$$
        $$y = x^3$$
        \begin{eqnarray}
            x^3 &=& x \\
            0 &=& x^3 - x \\
              &=& x\left(x^2 - 1\right) \\
            x &=& -1 \\
            x &=& 0 \\
            x &=& 1 \\
            \int_{-1}^0{x^3 - x\,dx} + \int_0^1{x - x^3\,dx} &=& - \left(\frac{(-1)^4}{4} - \frac{(-1)^2}{2}\right) + \left(\frac{1^2}{2} - \frac{(-1)^4}{4}\right) \\
                                                             &=& - \left(\frac{1}{4} - \frac{1}{2}\right) + \left(\frac{1}{2} - \frac{1}{4}\right) \\
                                                             &=& - \left(-\frac{1}{4}\right) + \frac{1}{4} \\
                                                             &=& \frac{1}{4} + \frac{1}{4} \\
                                                             &=& \frac{1}{2}
        \end{eqnarray}
    \item Area between curves (multiple integrals 2)
        $$y = 4x - x^2$$
        $$y = 4x - 4$$
        $$[0,2]$$
        \begin{eqnarray}
            \int_0^1{4x - x^2\,dx} + \int_1^2{4x - x^2 - 4x + 4\,dx} &=& \left(2 + \frac{1}{3}\right) + \left(\left(-\frac{2^3}{3} + 4(2)\right) - \left(-\frac{1^3}{3} + 4(1)\right)\right) \\
                                                                     &=& \left(2 + \frac{1}{3}\right) + \left(\left(-\frac{8}{3} + 8\right) - \left(-\frac{1}{3} + 4\right)\right) \\
                                                                     &=& \frac{7}{3} + \left(\frac{16}{3} - \frac{11}{3}\right) \\
                                                                     &=& \frac{7}{3} + \left(\frac{16}{3} - \frac{11}{3}\right) \\
                                                                     &=& 4
        \end{eqnarray}
    \item Area between curves (integrating $dy$)
        $$x = \sqrt{y}$$
        $$x \frac{y}{4}$$
        \begin{eqnarray}
            \int_0^4{\left(\sqrt{y} - \frac{y}{4}\right)\,dy} &=& \left(\frac{2}{3}4^{\frac{3}{2}} - \frac{4^2}{8}\right) \\
                                                              &=& \left(\frac{2}{3}\cdot 8 - \frac{16}{8}\right) \\
                                                              &=& \left(\frac{16}{3} - 2\right) \\
                                                              &=& \frac{10}{3}
        \end{eqnarray}
    \item Area between curves (choosing a method)
        $$y = \sqrt{\frac{x}{2} + 1}$$
        $$y = \sqrt{1 - x}$$
        \begin{eqnarray}
            x &=& 2y^2 - 2 \\
            x &=& 1 - y^2 \\
            2y^2 - 2 &=& 1 - y^2 \\
            3y^2 - 3 &=& 0 \\
            3\left(y^2 - 1\right) &=& 0 \\
            y &=& -1 \\
            y &=& 1 \\
            \int_0^1{\left(1 - y^2 - 2y^2 + 2\right)\,dy} &=& \int_0^1{\left(3 - 3y^2\right)\,dy} \\
                                                             &=& \left(3(1) - 1^3\right) \\
                                                             &=& 3 - 1 \\
                                                             &=& 2
        \end{eqnarray}
    \item Using geometry to find area between curves
        $$x = 2y$$
        $$x = y + 1$$
        $$(2,1)$$
        \begin{eqnarray}
            A_1 &=& 1 \\
            A_2 &=& \frac{1}{2} \\
            A &=& 1 - \frac{1}{2} \\
              &=& \frac{1}{2}
        \end{eqnarray}
\end{enumerate}

\section*{Related Exercises}
\begin{enumerate}
    \item (Section 5.4, Exercise 15)
        \begin{eqnarray}
            \int_{-2}^2{\left(x^2 + x^3\right)\,dx} &=& \int_{-2}^2{x^2\,dx} + \int_{-2}^2{x^3\,dx} \\
                                                    &=& 2\int_{0}^2{x^2\,dx} + 0 \\
                                                    &=& 2\frac{x^3}{3} \\
                                                    &=& 2\frac{2^3}{3} - 2\frac{0^3}{3} \\
                                                    &=& 2\frac{8}{3} \\
                                                    &=& \frac{16}{3}
        \end{eqnarray}
    \item (Section 5.4, Exercise 16)
        $$\int_{-\pi}^{\pi}{t^2\sin{t}\,dx} = 0$$
    \item (Section 5.4, Exercise 26)
        $$f(x) = x^2 + 1$$
        $$[-2, 2]$$
        \begin{eqnarray}
            \overline{f} &=& \frac{1}{2 - (-2)}\int_{-2}^2{x^2 + 1\,dx} \\
                         &=& \frac{1}{4}\left(\int_{-2}^2{x^2\,dx} + 1\int_{-2}^2{x^0\,dx}\right) \\
                         &=& \frac{1}{4}\left(\frac{x^3}{3} + x\right) \\
                         &=& \frac{1}{4}\left(\int_{-2}^2{x^2\,dx} + \int_{-2}^2{1\,dx}\right) \\
                         &=& \frac{1}{4}\left(\frac{2^3}{3} - \frac{(-2)^3}{3} + 2 - (-2)\right) \\
                         &=& \frac{1}{4}\left(\frac{8}{3} - \frac{-8}{3} + 4\right) \\
                         &=& \frac{1}{4}\left(\frac{16}{3} + 4\right) \\
                         &=& \frac{1}{4}\left(\frac{28}{3}\right) \\
                         &=& \frac{7}{3}
        \end{eqnarray}
    \item (Section 5.4, Exercise 34)
        $$f(x) = x^3 - 5x^2 + 30$$
        $$[0,4]$$
        \begin{eqnarray}
            \overline{f} &=& \frac{1}{4}\left(\int_0^4{\left(x^3 - 5x^2 + 30\right)\,dx}\right) \\
                         &=& \frac{1}{4}\left(\int_0^4{x^3} - 5\int_0^4{x^2} + 30\int_0^4{x^0}\right) \\
                         &=& \frac{1}{4}\left(\frac{x^4}{4} - 5\frac{x^3}{3} + 30x\right) \\
                         &=& \frac{1}{4}\left(\left(\frac{4^4}{4} - \frac{0^4}{4}\right) - \left(5\frac{4^3}{3} - 5\frac{0^3}{3}\right) + \left(30(4) - 30(0)\right)\right) \\
                         &=& \frac{1}{4}\left(64 - \frac{320}{3} + 120\right) \\
                         &=& \frac{1}{4}\left(\frac{232}{3}\right) \\
                         &=& \frac{58}{3}
        \end{eqnarray}
    \item (Section 5.4, Exercise 41)
        $$f(x) = 1 - \frac{x^2}{a^2}$$
        $$[0, a]$$
        \begin{eqnarray}
            \overline{f} &=& \frac{1}{a}\left(\int_0^a{1 - \frac{x^2}{a^2}\,dx}\right) \\
                         &=& \frac{1}{a}\left(\int_0^a{1\,dx} - \int_0^a{\frac{x^2}{a^2}\,dx}\right) \\
                         &=& \frac{1}{a}\left(x - \frac{1}{a^2}\int_0^a{x^2\,dx}\right) \\
                         &=& \frac{1}{a}\left(x - \frac{1}{a^2}\frac{x^3}{3}\right) \\
                         &=& \frac{1}{a}\left(x - \frac{x^3}{3a^2}\right) \\
                         &=& \frac{1}{a}\left(\left(a - 0\right) - \frac{1}{a^2}\left(\frac{a^3}{3} - \frac{0^3}{3}\right)\right) \\
                         &=& \frac{1}{a}\left(a - \frac{a^3}{3a^2}\right) \\
                         &=& \frac{1}{a}\left(a - \frac{a}{3}\right) \\
                         &=& \frac{1}{a}\left(\frac{2a}{3}\right) \\
                         &=& \frac{2}{3} \\
            1 - \frac{c^2}{a^2} &=& \frac{2}{3} \\
            \frac{c^2}{a^2} &=& \frac{1}{3} \\
            c^2 &=& \frac{a^2}{3} \\
            c &=& \sqrt{\frac{a^2}{3}} \\
              &=& \frac{a}{\sqrt{3}}
        \end{eqnarray}
    \item (Section 5.4, Exercise 42)
        $$f(x) = \frac{\pi}{4}\sin{x}$$
        $$[0, \pi]$$
        \begin{eqnarray}
            \overline{f} &=& \frac{1}{\pi}\int_0^{\pi}{\frac{\pi}{4}\sin{x}} \\
                         &=& \frac{1}{\pi}\frac{\pi}{4}\int_0^{\pi}{\sin{x}} \\
                         &=& \frac{1}{\pi}\frac{\pi}{4}\left(-\cos{x}\right) \\
                         &=& \frac{1}{\pi}\frac{\pi}{4}\left(-\cos{\pi} + \cos{0}\right) \\
                         &=& \frac{1}{\pi}\frac{\pi}{4}\left(1 + 1\right) \\
                         &=& \frac{1}{\pi}\frac{\pi}{2} \\
                         &=& \frac{1}{2} \\
            \frac{\pi}{4}\sin{x} &=& \frac{1}{2} \\
            \sin{x} &=& \frac{2}{\pi} \\
            \sin^{-1}{\sin{x}} &=& \sin^{-1}{\frac{2}{\pi}} \\
            x &=& \sin^{-1}{\frac{2}{\pi}}
        \end{eqnarray}
    \item (Section 5.5, Exercise 17)
        \begin{eqnarray}
            u &=& x^2 - 1 \\
            du &=& 2x\,dx \\
            \int{2x\left(x^2 - 1\right)^{99}\,dx} &=& \int{u^{99}\,du} \\
                                                  &=& \frac{u^{100}}{100} + C \\
                                                  &=& \frac{\left(x^2 - 1\right)^{100}}{100} + C
        \end{eqnarray}
    \item (Section 5.5, Exercise 20)
        \begin{eqnarray}
            u &=& \sqrt{x} + 1 \\
            du &=& \frac{1}{2\sqrt{x}}\,dx \\
            \int{\frac{\left(\sqrt{x} + 1\right)^4}{2\sqrt{x}}\,dx} &=& \int{u^4\,du} \\
                                                                    &=& \frac{u^5}{5} + C \\
                                                                    &=& \frac{\left(\sqrt{x} + 1\right)^5}{5} + C
        \end{eqnarray}
    \item (Section 5.5, Exercise 21)
        \begin{eqnarray}
            u &=& x^2 + x \\
            du &=& 2x + 1\,dx \\
            \int{\left(x^2 + x\right)^{10}\left(2x + 1\right)\,dx} &=& \int{u^{10}\,du} \\
                                                                   &=& \frac{u^{11}}{11} + C \\
                                                                   &=& \frac{\left(x^2 + x\right)^{11}}{11} + C
        \end{eqnarray}
    \item (Section 5.5, Exercise 23)
        \begin{eqnarray}
            u &=& x^4 + 16 \\
            du &=& 4x^3\,dx \\
            x^3\,dx &=& \frac{1}{4}du \\
            \int{x^3\left(x^4 + 16\right)^6\,dx} &=& \int{\frac{1}{4}u^6\,du} \\
                                                 &=& \frac{1}{4}\int{u^6\,du} \\
                                                 &=& \frac{1}{4}\cdot\frac{u^7}{7} + C \\
                                                 &=& \frac{\left(x^4 + 16\right)^7}{28} + C
        \end{eqnarray}
    \item (Section 5.5, Exercise 24)
        \begin{eqnarray}
            u &=& \sin{\theta} \\
            du &=& \cos{\theta}\,d\theta \\
            \int{\sin^{10}{\theta}\cos{\theta}\,d\theta} &=& \int{u^{10}\,du} \\
                                                         &=& \frac{u^{11}}{11} + C \\
                                                         &=& \frac{\sin^{11}{\theta}}{11} + C
        \end{eqnarray}
    \item (Section 5.5, Exercise 78)
        \begin{eqnarray}
            u &=& x - 2 \\
            x &=& u + 2 \\
            du &=& \,dx \\
            \int{\frac{x}{x - 2}\,dx} &=& \int{\frac{u + 2}{u}\,du} \\
                                      &=& \int{\frac{u}{u}\,du} + \int{\frac{2}{u}\,du} \\
                                      &=& u + 2\int{\frac{1}{u}\,du} + C \\
                                      &=& u + 2\ln{|u|} + C \\
                                      &=& x - 2 + 2\ln{|u|} + C
        \end{eqnarray}
    \item (Section 5.5, Exercise 79)
        \begin{eqnarray}
            u &=& \sqrt{x - 4} \\
            u^2 &=& x - 4 \\
            x &=& u^2 + 4 \\
            dx &=& 2u\,du \\
            \int{\frac{x}{\sqrt{x - 4}}\,dx} &=& \int{2u\frac{u^2 + 4}{u}\,du} \\
                                             &=& \int{\frac{2u^3 + 8u}{u}\,du} \\
                                             &=& 2\left(\int{u^2\,du} + \int{4\,du}\right) \\
                                             &=& 2\left(\frac{\sqrt{x - 4}^3}{3} + 4\sqrt{x - 4}\right) + C \\
                                             &=& \frac{2}{3}\sqrt{x - 4} + 8\sqrt{x - 4} + C
        \end{eqnarray}
    \item (Section 5.5, Exercise 15)
        \begin{eqnarray}
            \int{e^{10x}\,dx} &=& \frac{1}{10}e^{10x} + C \\
            \int{\sec{5x}\tan{5x}\,dx} &=& \frac{1}{5}\sec{5x} + C \\
            \int{\sin{7x}\,dx} &=& -\frac{1}{7}\cos{7x} + C \\
            \int{\cos{\frac{x}{7}}\,dx} &=& 7\sin{\frac{x}{7}} + C \\
            \int{\frac{dx}{81 + 9x^2}} &=& \int{\frac{dx}{9^2 + 9x^2}} \\
                                       &=& \int{\frac{dx}{9(9 + x^2)}} \\
                                       &=& \frac{1}{9}\int{\frac{dx}{3^2 + x^2}} \\
                                       &=& \frac{1}{27}\tan^{-1}{\frac{x}{3}} + C \\
            \int{\frac{dx}{\sqrt{36 - x^2}}} &=& \int{\frac{dx}{\sqrt{6^2 - x^2}}} \\
                                             &=& \sin^{-1}{\frac{x}{6}} + C
        \end{eqnarray}
    \item (Section 5.5, Exercise 16)
        \begin{eqnarray}
            \int_0^1{10^x\,dx} &=& \frac{1}{\ln{10}}10 - \frac{1}{\ln{10}} \\
                               &=& \frac{9}{\ln{10}} \\
            \int_0^{\frac{\pi}{40}}{\cos{20x}\,dx} &=& \cos{\frac{20\pi}{40}} - \cos{20(0)} \\
                                                   &=& 0 - 1 \\
                                                   &=& -1 \\
            \int_{3\sqrt{2}}^6{\frac{\,dx}{x\sqrt{x^2 - 9}}} &=& \int_{3\sqrt{2}}^6{\frac{\,dx}{x\sqrt{x^2 - 3^2}}} \\
                                                             &=& \frac{1}{3}\sec^{-1}{\left|\frac{6}{3}\right|} - \frac{1}{3}\sec^{-1}{\left|\frac{3\sqrt{2}}{3}\right|} \\
                                                             &=& \frac{1}{3}\sec^{-1}{\frac{\pi}{3}} - \frac{1}{3}\sec^{-1}{\frac{\pi}{4}} \\
            \int_0^{\frac{\pi}{16}}{\sec^2{4x}\,dx} &=& \frac{1}{4}\tan{\frac{4\pi}{16}} - \frac{1}{4}\tan{0} \\
                                                    &=& \frac{1}{4} - 0 \\
                                                    &=& \frac{1}{4}
        \end{eqnarray}
    \item (Section 5.5, Exercise 49)
        \begin{eqnarray}
            u &=& 2^x + 4 \\
            du &=& 2^x\ln{2}\,dx \\
            2^x\,dx &=& \frac{1}{\ln{2}}\,du \\
            \int_1^3{\frac{2^x}{2^x + 4}\,dx} &=& \frac{1}{\ln{2}}\int_6^{12}{\frac{1}{u}\,du} \\
                                              &=& \frac{1}{\ln{2}}\left(\ln{12} - \ln{6}\right) \\
                                              &=& \frac{1}{\ln{2}}\cdot\ln{2} \\
                                              &=& 1
        \end{eqnarray}
    \item (Section 5.5, Exercise 51)
        \begin{eqnarray}
            u &=& \sin{\theta} \\
            du &=& \cos{\theta}\,d\theta \\
            \int_0^{\frac{\pi}{2}}{\sin^2{\theta}\cos{\theta}\,d\theta} &=& \int_0^{\frac{\pi}{2}}{u^2\,du} \\
                                                                        &=& \int_0^1{u^2\,du} \\
                                                                        &=& \frac{1^3}{3} - \frac{0^3}{3} \\
                                                                        &=& \frac{1}{3}
        \end{eqnarray}
    \item (Section 5.5, Exercise 64)
        \begin{eqnarray}
            u &=& 3 + 2e^x \\
            du &=& 2e^x\,dx \\
            e^x\,dx &=& \frac{1}{2}\,du \\
            \int_0^{\ln{4}}{\frac{e^x}{3 + 2e^x}\,dx} &=& \frac{1}{2}\int_{u = 5}^{u = 11}{\frac{1}{u}\,du} \\
                                                      &=& \frac{\ln{11} - \ln{5}}{2}
        \end{eqnarray}
    \item (Section 5.5, Exercise 87)
        \begin{eqnarray}
            u &=& 2x \\
            du &=& 2\,dx \\
            dx &=& \frac{1}{2}\,du \\
            \cos^2{x} &=& \frac{1 + \cos{2x}}{2} \\
            \int_{-\pi}^{\pi}{\cos^2{x}\,dx} &=& \int_{-\pi}^{\pi}{\frac{1}{2} + \frac{\cos{2x}}{2}\,dx} \\
                                             &=& \frac{1}{2}\int_{-\pi}^{\pi}{1 + \cos{2x}\,dx} \\
                                             &=& \frac{1}{2}\left(\int_{-\pi}^{\pi}{1\,dx} + \frac{1}{2}\int_{-\pi}^{\pi}{\cos{u}\,du}\right) \\
                                             &=& \frac{1}{2}\left(\left(\pi - \left(-\pi\right)\right) + \frac{1}{2}\int_{u = -2\pi}^{u = 2\pi}{\cos{u}\,du}\right) \\
                                             &=& \frac{1}{2}\left(2\pi + \frac{1}{2}\left(\sin{2\pi} - \sin{\left(-2\pi\right)}\right)\right) \\
                                             &=& \frac{1}{2}\left(2\pi + \frac{1}{2}\left(0\right)\right) \\
                                             &=& \frac{2\pi}{2} \\
                                             &=& \pi
        \end{eqnarray}
    \item (Section 5.5, Exercise 91)
        \begin{eqnarray}
            u &=& 4\theta \\
            du &=& 4\,d\theta \\
            d\theta &=& \frac{1}{4}\,du \\
            \sin^2{x} &=& \frac{1 - \cos{2x}}{2} \\
            \sin^2{2\theta} &=& \frac{1 - \cos{4\theta}}{2} \\
            \int_{-\pi}^{\pi}{\sin^2{2\theta}\,d\theta} &=& \int_{-\pi}^{\pi}{\frac{1 - \cos{4\theta}}{2}\,d\theta} \\
                                                        &=& \int_{-\pi}^{\pi}{\frac{1}{2} - \frac{\cos{u}}{2}\,d\theta} \\
                                                        &=& \frac{1}{2}\int_{-\pi}^{\pi}{1 - \cos{u}\,d\theta} \\
                                                        &=& \frac{1}{2}\left(\int_{-\pi}^{\pi}{1\,d\theta} - \int_{-\pi}^{\pi}{\cos{u}\,d\theta}\right) \\
                                                        &=& \frac{1}{2}\left(2\pi - \frac{1}{4}\int_{u = -4\pi}^{u = 4\pi}{\cos{u}\,du}\right) \\
                                                        &=& \frac{1}{2}\left(2\pi - \frac{1}{4}\left(\sin{4\pi} - \sin{\left(-4\pi\right)}\right)\right) \\
                                                        &=& \frac{2\pi}{2} \\
                                                        &=& \pi
        \end{eqnarray}
    \item (Section 6.1, Exercise 7)
        \begin{enumerate}
            \item $0 < t < 1$ and $3 < t < 5$
            \item $-4$
            \item $26$
            \item 6
        \end{enumerate}
    \item (Section 6.1, Exercise 8)
        \begin{enumerate}
            \item $0 < t < 2$ and $4 < t < 6$
            \item $4$
            \item $44$
            \item $-10$
        \end{enumerate}
    \item (Section 6.1, Exercise 17)
        $$v(t) = \sin{t}$$
        $$s(0) = 1$$
        \begin{eqnarray}
            s(t) &=& 1 + \int_0^t{\sin{t}\,dt} \\
                 &=& 1 - \cos{t} + \cos{0} \\
                 &=& 2 - \cos{t}
        \end{eqnarray}
    \item (Section 6.1, Exercise 20)
        $$v(t) = 3\sin{\pi t}$$
        $$s(0) = 1$$
        \begin{eqnarray}
            u &=& \pi t \\
            du &=& \pi\,dt \\
            s(t) &=& 1 + \int_0^t{3\sin{\pi t}\,dt} \\
                 &=& 1 + \frac{3}{\pi}\int_0^t{\sin{u}\,du} \\
                 &=& 1 + \frac{3}{\pi}\left(-\cos{\pi t} + \cos{0}\right) \\
                 &=& 1 + \frac{3}{\pi} - \frac{3}{\pi}\cos{\pi t}
        \end{eqnarray}
    \item (Section 6.1, Exercise 27)
        \begin{eqnarray}
            \int_0^{20}{3t\,dt} + \int_{20}^{30}{60} &=& \left(\frac{3(20)^2}{2} - \frac{3(0)^2}{2}\right) + \left(60(30) - 60(20)\right) \\
                                                     &=& \left(\frac{1200}{2}\right) + \left(1800 - 1200\right) \\
                                                     &=& 600 + 600 \\
                                                     &=& 1200 \\
            \int_0^{20}{3t\,dt} + \int_{20}^{45}{60} + \int_{45}^{60}{240 - 4t\,dt} &=& 600 + 2700 - 1200 + 14400 - 7200 - 10800 - 4050 \\
                                                                                    &=& 2100 + \left(\left(14400 - 7200\right) - \left(10800 - 4050\right)\right) \\
                                                                                    &=& 2100 + \left(7200 - 6750\right) \\
                                                                                    &=& 2550 \\
            s(t) &=& \int_0^{20}{3t\,dt} + \int_{20}^{45}{60} + \int_{45}^{t}{240 - 4t\,dt} \\
                 &=& 2100 + \left(240t - 2t^2 - 6750\right) \\
                 &=& 240t - 2t^2 -4650 \\
            s(75) &=& 2100
        \end{eqnarray}
    \item (Section 6.1, Exercise 28)
        \begin{eqnarray}
            \int_0^{10}{9.8t\,dt} + \int_{10}^{30}{10} &=& \left(4.9(10)^2 - 4.9(0^2)\right) + \left(10(30) - 10(10)\right) \\
                                                       &=& \left(4.9(100)\right) + \left(300 - 100\right) \\
                                                       &=& 490 + 200 \\
                                                       &=& 690
        \end{eqnarray}
    \item (Section 6.1, Exercise 30)
        $$a(t) = -32$$
        $$v(0) = 50$$
        $$s(0) = 0$$
        \begin{eqnarray}
            v(t) &=& v(0) + \int_0^t{-32} \\
                 &=& 50 - 32t \\
            s(t) &=& s(0) + \int_0^t{50 - 32t} \\
                 &=& 50t - 16t^2
        \end{eqnarray}
    \item (Section 6.1, Exercise 31)
        $$a(t) = -9.8$$
        $$v(0) = 20$$
        $$s(0) = 0$$
        \begin{eqnarray}
            v(t) &=& v(0) + \int_0^t{-9.8} \\
                 &=& 20 - 9.8t \\
            s(t) &=& s(0) + \int_0^t{20 - 9.8t} \\
                 &=& 20t - 4.9t^2
        \end{eqnarray}
    \item (Section 6.1, Exercise 43)
        $$N'(t) = 100e^{-0.25t}$$
        \begin{eqnarray}
            u &=& -0.25t \\
            du &=& -0.25\,dt \\
            N(t) &=& N(0) + \int_0^t{10e^{-0.25t}\,dt} \\
                 &=& 1900 + 100\int_0^t{\frac{1}{-0.25}e^{u}\,du} \\
                 &=& 1900 - 400e^{-0.25t} \\
            N(20) &=& 1900 - 400e^{-5} \\
                  &\approx& 1897.305 \\
            N(40) &=& 1900 - 400e^{-10} \\
                  &\approx& 1899.98
        \end{eqnarray}
    \item (Section 6.1, Exercise 44)
        $$r(t) = 0.0025e^{0.25t} - 0.1485e^{-0.15t}$$
        \begin{eqnarray}
            t_0 &=& \frac{\ln{0.0025} - \ln{0.1485}}{-0.4} \\
            \int_0^{t_0}{0.0025e^{0.25t} - 0.1485e^{-0.15t}} &\approx& 271
        \end{eqnarray}
    \item (Section 6.1, Exercise 55)
        $$C'(x) = 2000 - 0.5x$$
        \begin{eqnarray}
            \int_{100}^{150}{2000 - 0.5x\,dx} &=& \left(2000(150) - 0.25(150)^2\right) - \left(2000(100) - 0.25(100)^2\right) \\
                                              &=& \left(300000 - 5625\right) - \left(200000 - 2500\right) \\
                                              &=& 294375 - 197500 \\
                                              &=& 96875 \\
            \int_{500}^{550}{2000 - 0.5x\,dx} &=& \left(1024375\right) - \left(937500\right) \\
                                              &=& 86875
        \end{eqnarray}
    \item (Section 6.1, Exercise 56)
        $$C'(x) = 200 - 0.05x$$
        \begin{eqnarray}
            \int_{100}^{150}{200 - 0.05x\,dx} &=& \left(200(150) - 0.025(150)^2\right) - \left(200(100) - 0.025(100)^2\right) \\
                                              &=& \left(29437.5\right) - \left(19750\right) \\
                                              &=& 9687.5 \\
            \int_{500}^{500}{200 - 0.05x\,dx} &=& \left(200(550) - 0.025(550)^2\right) - \left(200(500) - 0.025(500)^2\right) \\
                                              &=& 102437.5 - 93750 \\
                                              &=& 8687.5
        \end{eqnarray}
    \item (Section 6.2, Exercise 9)
        $$y = x$$
        $$y = x^2 - 2$$
        \begin{eqnarray}
            x &=& x^2 - 2 \\
            0 &=& x^2 - x - 2 \\
              &=& \left(x + 1\right)\left(x - 2\right) \\
            x &=& -1, 2 \\
            \int_{-2}^1{\left(- x^2 + x + 2\right)\,dx} &=& \left(-\frac{2^3}{3} + \frac{2^2}{2} + 2(2)\right) - \left(-\frac{(-1)^3}{3} + \frac{(-1)^2}{2} + 2(-1)\right) \\
                                                        &=& \left(-\frac{8}{3} + \frac{4}{2} + 4\right) - \left(\frac{1}{3} + \frac{1}{2} - 2\right) \\
                                                        &=& -\frac{8}{3} + \frac{4}{2} + 4 - \frac{1}{3} - \frac{1}{2} + 2 \\
                                                        &=& -\frac{9}{3} + \frac{3}{2} + 6 \\
                                                        &=& -3 + \frac{3}{2} + 6 \\
                                                        &=& \frac{3}{2} + 3 \\
                                                        &=& 4.5
        \end{eqnarray}
    \item (Section 6.2, Exercise 10)
        $$y = -x^2 + 4x$$
        $$y = x^2 - 2x$$
        \begin{eqnarray}
            -x^2 + 4x &=& x^2 - 2x \\
            0 &=& 2x^2 - 6x \\
              &=& 2x\left(x - 6\right) \\
            x &=& 0, 6 \\
            \int_0^6{\left(-x^2 + 4x - x^2 + 2x\right)\,dx} &=& \left(\frac{-2(6)^3}{3} + 3(6)^2\right) \\
                                                            &=& \left(\frac{-2(216)}{3} + 3(36)\right) \\
                                                            &=& \left(\frac{-432}{3} + 108\right) \\
                                                            &=& -144 + 108 \\
                                                            &=& -36
        \end{eqnarray}
    \item (Section 6.2, Exercise 15)
        $$y = \sin{x}$$
        $$y = \cos{x}$$
        \begin{eqnarray}
            \sin{x} &=& \cos{x} \\
            0 &=& \sin{x} - \cos{x} \\
            x &=& \frac{\pi}{4} \\
            \int_0^{\frac{\pi}{4}}{\sin{x}\,dx} + \int_{\frac{\pi}{4}}^{\frac{\pi}{2}}{\cos{x}\,dx} &=& \left(-\cos{\frac{\pi}{4}} + \cos{0}\right) + \left(\sin{\frac{\pi}{2}} - \sin{\frac{\pi}{4}}\right) \\
                                                                                                    &=& \left(-\frac{1}{\sqrt{2}} + 1\right) + \left(1 - \frac{1}{\sqrt{2}}\right) \\
                                                                                                    &=& 2 - \sqrt{2}
        \end{eqnarray}
    \item (Section 6.2, Exercise 16)
        $$y = x^3$$
        $$y = x$$
        \begin{eqnarray}
            x^3 &=& x \\
            0 &=& x^3 - x \\
              &=& x\left(x^2 - 1\right) \\
            x &=& -1, 0, 1 \\
            \int_{-1}^0{\left(x^3 - x\right)\,dx} + \int_0^1{\left(x - x^3\right)\,dx} &=& -\left(\frac{(-1)^4}{4} - \frac{(-1)^2}{2}\right) + \left(\frac{1^2}{2} - \frac{1^4}{4}\right) \\
                                                                                       &=& -\left(\frac{1}{4} - \frac{1}{2}\right) + \left(\frac{1}{2} - \frac{1}{4}\right) \\
                                                                                       &=& \frac{1}{4} + \frac{1}{4} \\
                                                                                       &=& \frac{1}{2}
        \end{eqnarray}
    \item (Section 6.2, Exercise 19)
        $$x = y^2 - 3$$
        $$x = 2y$$
        \begin{eqnarray}
            2y &=& y^2 - 3 \\
            0 &=& y^2 - 2y - 3 \\
              &=& \left(y - 3\right)\left(y + 1\right) \\
            y &=& -1, 3 \\
            \int_{-1}^3{\left(2y - y^2 + 3\right)\,dy} &=& \left(3^2 - \frac{3^3}{3} + 3(3)\right) - \left((-1)^2 - \frac{(-1)^3}{3} + 3(-1)\right) \\
                                                       &=& \left(9 - \frac{27}{3} + 9\right) - \left(1 - \frac{-1}{3} - 3\right) \\
                                                       &=& \left(18 - 9\right) - \left(-2 + \frac{1}{3}\right) \\
                                                       &=& 9 + 2 - \frac{1}{3} \\
                                                       &=& 11 - \frac{1}{3} \\
                                                       &=& \frac{32}{3}
        \end{eqnarray}
    \item (Section 6.2, Exercise 20)
        $$x = \frac{y}{4}$$
        $$x = \sqrt{y}$$
        \begin{eqnarray}
            \int_0^4{\left(\sqrt{y} - \frac{y}{4}\right)} &=& \left(\frac{2}{3}4^{\frac{3}{2}} - \frac{4^2}{8}\right) \\
                                                          &=& \left(\frac{2}{3}\cdot8 - \frac{16}{8}\right) \\
                                                          &=& \frac{16}{3} - 2 \\
                                                          &=& \frac{10}{3}
        \end{eqnarray}
    \item (Section 6.2, Exercise 34)
        $$y = 2x^2$$
        $$y = 3 - x$$
        \begin{eqnarray}
            2x^2 &=& 3 - x \\
            0 &=& 2x^2 + x - 3 \\
              &=& \left(2x + 3\right)\left(x - 1\right) \\
            x &=& 0, 1 \\
            \int_0^1{3 - x - 2x^2\,dx} &=& \left(3 - \frac{1}{2} - 2\frac{1}{3}\right) \\
                                       &=& \left(3 - \frac{1}{2} - \frac{2}{3}\right) \\
                                       &=& \frac{11}{6} \\
            A_1 &=& \frac{11}{6} \\
            A &=& A_1 + A_2 \\
              &=& \frac{9}{2} \\
            A_2 &=& A - A_1 \\
                &=& \frac{9}{2} - \frac{11}{6} \\
                &=& \frac{8}{3}
        \end{eqnarray}
\end{enumerate}

\blfootnote{A copy of my notes (in \LaTeX) are available on my \href{https://github.com/onlinechronically/MATH-211}{GitHub}}
\end{document}
