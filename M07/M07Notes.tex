\documentclass{article}
\usepackage{graphicx}
\usepackage{amsthm}
\usepackage{amsmath}
\usepackage{amssymb}
\usepackage{geometry}
\usepackage{tikz}
\usepackage[hidelinks]{hyperref}
\usetikzlibrary{arrows}

\geometry{a4paper, total={170mm,257mm}, left=20mm, top=20mm}
\AtBeginEnvironment{align}{\setcounter{equation}{0}} 
\AtBeginEnvironment{eqnarray}{\setcounter{equation}{0}} 

\newcommand\blfootnote[1]{
    \begingroup
    \renewcommand\thefootnote{}\footnote{#1}
    \addtocounter{footnote}{-1}
    \endgroup
}

\title{Module 7 Notes (MATH-211)}
\author{Lillie Donato}
\date{22 July 2024}

\begin{document}

\maketitle

\section*{General Notes (and Definitions)}
\begin{itemize}
    \item Working with Integrals \\
        A function $f(x)$ is \textbf{even} if $f(-x) = f(x)$. \\
        A function $f(x)$ is \textbf{odd} if $f(-x) = -f(x)$. \\
        Let $a \in \mathbb{R}$ such that $a > 0$ and let $f$ be an integrable function on the interval $[-a,a]$.
        $$\text{If } f \text{ is even, } \int_{-a}^a{f(x)\,dx} = 2\int_0^a{f(x)\,dx}$$
        $$\text{If } f \text{ is odd, } \int_{-a}^a{f(x)\,dx} = 0$$
        The average value of an integrable function $f$ on the interval $[a,b]$ is
        $$\overline{f} = \frac{1}{b - a}\int_a^b{f(x)\,dx}$$
        Let $f$ be continuous on the interval $[a,b]$. There exists a point $c$ in $(a,b)$ such that (Mean Value Theorem)
        $$f(c) = \overline{f} = \frac{1}{b - a}\int_a^b{f(t)\,dx}$$
\end{itemize}

\section*{Examples}
\begin{enumerate}
    \item Use symmetry to evaluate integrals
        $$\int_{-200}^{200}{2x^5\,dx} = 0$$
        \begin{eqnarray}
            \int_{-2}^{2}{\left(x^2 + x^3\right)\,dx} &=& \int_{-2}^{2}{x^2\,dx} + \int_{-2}^{2}{x^3\,dx} \\
                                                      &=& 2\int_{0}^{2}{x^2\,dx} + 0 \\
                                                      &=& 2\frac{x^3}{3} \\
                                                      &=& \frac{16}{3}
        \end{eqnarray}
    \item A derivative calculation
        $$s(t) = -16t^2 + 64t$$
        $$t = 4$$
        $$[0,4]$$
        \begin{eqnarray}
            v(t) &=& s'(t) \\
            \overline{v} &=& \frac{1}{4}\int_0^4{v(t)\,dx} \\
                         &=& \frac{1}{4}\int_0^4{s'(t)\,dx} \\
                         &=& \frac{1}{4}s(t) \\
                         &=& \frac{1}{4}\left(s(4) - s(0)\right) \\
                         &=& 0
        \end{eqnarray}
    \item Applying MVT for integrals
        $$f(x) = e^x$$
        $$[0,2]$$
        \begin{eqnarray}
            \overline{f} &=& \frac{1}{2}\left(\int_0^2{e^x\,dx}\right) \\
                         &=& \frac{e^x}{2} \\
                         &=& \frac{e^2}{2} - \frac{e^0}{2} \\
                         &=& \frac{e^2 - 1}{2} \\
            e^x &=& \frac{e^2 - 1}{2} \\
            \ln{e^x} &=& \ln{\frac{e^2 - 1}{2}}
        \end{eqnarray}
\end{enumerate}

\section*{Related Exercises}
\begin{enumerate}
    \item (Section 5.4, Exercise 15)
        \begin{eqnarray}
            \int_{-2}^2{\left(x^2 + x^3\right)\,dx} &=& \int_{-2}^2{x^2\,dx} + \int_{-2}^2{x^3\,dx} \\
                                                    &=& 2\int_{0}^2{x^2\,dx} + 0 \\
                                                    &=& 2\frac{x^3}{3} \\
                                                    &=& 2\frac{2^3}{3} - 2\frac{0^3}{3} \\
                                                    &=& 2\frac{8}{3} \\
                                                    &=& \frac{16}{3}
        \end{eqnarray}
    \item (Section 5.4, Exercise 16)
        $$\int_{-\pi}^{\pi}{t^2\sin{t}\,dx} = 0$$
    \item (Section 5.4, Exercise 26)
        $$f(x) = x^2 + 1$$
        $$[-2, 2]$$
        \begin{eqnarray}
            \overline{f} &=& \frac{1}{2 - (-2)}\int_{-2}^2{x^2 + 1\,dx} \\
                         &=& \frac{1}{4}\left(\int_{-2}^2{x^2\,dx} + 1\int_{-2}^2{x^0\,dx}\right) \\
                         &=& \frac{1}{4}\left(\frac{x^3}{3} + x\right) \\
                         &=& \frac{1}{4}\left(\int_{-2}^2{x^2\,dx} + \int_{-2}^2{1\,dx}\right) \\
                         &=& \frac{1}{4}\left(\frac{2^3}{3} - \frac{(-2)^3}{3} + 2 - (-2)\right) \\
                         &=& \frac{1}{4}\left(\frac{8}{3} - \frac{-8}{3} + 4\right) \\
                         &=& \frac{1}{4}\left(\frac{16}{3} + 4\right) \\
                         &=& \frac{1}{4}\left(\frac{28}{3}\right) \\
                         &=& \frac{7}{3}
        \end{eqnarray}
    \item (Section 5.4, Exercise 34)
        $$f(x) = x^3 - 5x^2 + 30$$
        $$[0,4]$$
        \begin{eqnarray}
            \overline{f} &=& \frac{1}{4}\left(\int_0^4{\left(x^3 - 5x^2 + 30\right)\,dx}\right) \\
                         &=& \frac{1}{4}\left(\int_0^4{x^3} - 5\int_0^4{x^2} + 30\int_0^4{x^0}\right) \\
                         &=& \frac{1}{4}\left(\frac{x^4}{4} - 5\frac{x^3}{3} + 30x\right) \\
                         &=& \frac{1}{4}\left(\left(\frac{4^4}{4} - \frac{0^4}{4}\right) - \left(5\frac{4^3}{3} - 5\frac{0^3}{3}\right) + \left(30(4) - 30(0)\right)\right) \\
                         &=& \frac{1}{4}\left(64 - \frac{320}{3} + 120\right) \\
                         &=& \frac{1}{4}\left(\frac{232}{3}\right) \\
                         &=& \frac{58}{3}
        \end{eqnarray}
    \item (Section 5.4, Exercise 41)
        $$f(x) = 1 - \frac{x^2}{a^2}$$
        $$[0, a]$$
        \begin{eqnarray}
            \overline{f} &=& \frac{1}{a}\left(\int_0^a{1 - \frac{x^2}{a^2}\,dx}\right) \\
                         &=& \frac{1}{a}\left(\int_0^a{1\,dx} - \int_0^a{\frac{x^2}{a^2}\,dx}\right) \\
                         &=& \frac{1}{a}\left(x - \frac{1}{a^2}\int_0^a{x^2\,dx}\right) \\
                         &=& \frac{1}{a}\left(x - \frac{1}{a^2}\frac{x^3}{3}\right) \\
                         &=& \frac{1}{a}\left(x - \frac{x^3}{3a^2}\right) \\
                         &=& \frac{1}{a}\left(\left(a - 0\right) - \frac{1}{a^2}\left(\frac{a^3}{3} - \frac{0^3}{3}\right)\right) \\
                         &=& \frac{1}{a}\left(a - \frac{a^3}{3a^2}\right) \\
                         &=& \frac{1}{a}\left(a - \frac{a}{3}\right) \\
                         &=& \frac{1}{a}\left(\frac{2a}{3}\right) \\
                         &=& \frac{2}{3} \\
            1 - \frac{c^2}{a^2} &=& \frac{2}{3} \\
            \frac{c^2}{a^2} &=& \frac{1}{3} \\
            c^2 &=& \frac{a^2}{3} \\
            c &=& \sqrt{\frac{a^2}{3}} \\
              &=& \frac{a}{\sqrt{3}}
        \end{eqnarray}
    \item (Section 5.4, Exercise 42)
        $$f(x) = \frac{\pi}{4}\sin{x}$$
        $$[0, \pi]$$
        \begin{eqnarray}
            \overline{f} &=& \frac{1}{\pi}\int_0^{\pi}{\frac{\pi}{4}\sin{x}} \\
                         &=& \frac{1}{\pi}\frac{\pi}{4}\int_0^{\pi}{\sin{x}} \\
                         &=& \frac{1}{\pi}\frac{\pi}{4}\left(-\cos{x}\right) \\
                         &=& \frac{1}{\pi}\frac{\pi}{4}\left(-\cos{\pi} + \cos{0}\right) \\
                         &=& \frac{1}{\pi}\frac{\pi}{4}\left(1 + 1\right) \\
                         &=& \frac{1}{\pi}\frac{\pi}{2} \\
                         &=& \frac{1}{2} \\
            \frac{\pi}{4}\sin{x} &=& \frac{1}{2} \\
            \sin{x} &=& \frac{2}{\pi} \\
            \sin^{-1}{\sin{x}} &=& \sin^{-1}{\frac{2}{\pi}} \\
            x &=& \sin^{-1}{\frac{2}{\pi}}
        \end{eqnarray}
\end{enumerate}

\blfootnote{A copy of my notes (in \LaTeX) are available on my \href{https://github.com/onlinechronically/MATH-211}{GitHub}}
\end{document}
