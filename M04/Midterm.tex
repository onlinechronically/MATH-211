\documentclass{article}
\usepackage{graphicx}
\usepackage{amsthm}
\usepackage{amsmath}
\usepackage{amssymb}
\usepackage{geometry}
\usepackage{tikz}
\usepackage[hidelinks]{hyperref}
\usetikzlibrary{arrows}

\geometry{a4paper, total={170mm,257mm}, left=20mm, top=20mm}
\AtBeginEnvironment{align}{\setcounter{equation}{0}} 
\AtBeginEnvironment{eqnarray}{\setcounter{equation}{0}} 

\newcommand\blfootnote[1]{
    \begingroup
    \renewcommand\thefootnote{}\footnote{#1}
    \addtocounter{footnote}{-1}
    \endgroup
}

\title{Midterm Review (MATH-211)}
\author{Lillie Donato}
\date{3 July 2024}

\begin{document}

\maketitle

\section*{Limits (M01)}
\begin{itemize}
    \item Limit Definition(s):
        \begin{itemize}
            \item Simple: The value that the outputs of a function approach as inputs approach a certain value
            \item Preliminary: Suppose a function $f$ is defined for all $x$ near $a$ except possibly at $a$. If $f(x)$ is arbitrarily close to $L$ all $x$ sufficiently close (but not equal) to $a$, we write the following.
        \end{itemize}
        $$\lim_{x \to a}=L$$
    \item Secant Line: a line passing through two points $(t_0, s(t_0))$ and $(t_1, s(t_1))$. The slope is given by
    $$\frac{s(t_1)-s(t_0)}{t_1-t_0}$$
    \item Tangent Line: the line passing through $(t_0, s(t_0))$ with slope $$\lim_{t \to t_0}\frac{s(t)-s(t_0)}{t-t_0}$$
    \item One Sided limits:
        \begin{itemize}
            \item Right-hand (Definition): Suppose a function $f$ is defined for all $x$ near $a$ with $x > a$. If $f(x)$ is arbitrarily close to $L$ for all $x$ sufficiently close to $a$ with $x > a$ we write
            $$\lim_{x \to a^+}{f(x) = L}$$
            \item Left-hand (Definition): Suppose a function $f$ is defined for all $x$ near $a$ with $x < a$. If $f(x)$ is arbitrarily close to $L$ for all $x$ sufficiently close to $a$ with $x < a$ we write
            $$\lim_{x \to a^-}{f(x) = L}$$
            \item In order for their to be a double sided limit, we must have:
            $$\lim_{x \to a^-}{f(x)} = \lim_{x \to a^+}{f(x)}$$
            \item If the limits from sides are not equal, then a the double sided limit, "does not exist"
        \end{itemize}
    \item Velocity
        \begin{itemize}
            \item Average Veolcity
            \begin{itemize}
                \item The average velocity over some interval $[t_0, t_1]$ is defined as
                $$v_{av} = \frac{s(t_1) - s(t_0)}{t_1 - t_0}$$
            \end{itemize}
            \item Instantaneous Veolcity
            \begin{itemize}
                \item The average velocity over some interval $[t_0, t_1]$ is defined as
                $$v_{inst} = \lim_{t \to a}{v_{av}} = = \frac{s(t) - s(a)}{t - a}$$
            \end{itemize}
        \end{itemize}
    \item Solving Techniques
        \begin{itemize}
    	    \item Factoring and canceling out
    	    \item Using conjugates
            \begin{itemize}
    	    	\item When direct substitution is not possible, you may rationalize the numerator
            \end{itemize}
        \end{itemize}
	\item Infinite Limits: In either case, the limit does not exist (not a real number) if it is infinite
	\begin{itemize}
		\item Suppose $f$ is defined for all $x$ near $a$. If $f(x)$ gorws arbitrarily large for all $x$ sufficiently close (but not equal) to $a$, we write
		$$\lim_{x \to a}{f(x)} = \infty$$
		\item If $f(x)$ is negative and gorws arbitrarily large in magnitude for all $x$ sufficiently close (but not equal) to $a$, we write
		$$\lim_{x \to a}{f(x)} = - \infty$$
		\item The line $x = a$ is a vertical asymptote for $f$ if any of the following hold
		$$\lim_{x \to a}{f(x)} = \pm \infty$$
		$$\lim_{x \to a^+}{f(x)} = \pm \infty$$
		$$\lim_{x \to a^-}{f(x)} = \pm \infty$$
		\item A vertical asymptote exists at $x = a$ if any one sided limit as $x \to a$ is $\infty$ or $- \infty$
		\item If you have a limit of a rational function, where $p(a) = L \neq 0$ and $q(a) = 0$, then the one sided limits for $\frac{p(x)}{q(x)}$ approach $\pm \infty$
		$$\lim_{x \to a}{\frac{p(x)}{q(x)}} = \frac{L}{0}$$
	\end{itemize}
	\item Limits as Infinity
	\begin{itemize}
		\item \textbf{Definition}: If $f(x)$ becoomes arbitrarily close to a finite number $L$ for all sufficiently large and positive $x$, the we write
		$$\lim_{x \to \infty}{f(x)} = L$$
		The definition for
		$$\lim_{x \to - \infty}{f(x)} = M$$
		is analogous.
		\item If $\lim\limits_{x \to \infty}{f(x)} = L$ we say that the function $f(x)$ has a horizontal asymptote at $y = L$
		\item If $\lim\limits_{x \to - \infty}{f(x)} = M$ we say that the function $f(x)$ has a horizontal asymptote at $y = M$
		\item \textbf{Principle}: If $n > 0$ is an integer then
		$$\lim_{x \to \pm \infty}{\frac{1}{x^n}} = 0$$
		\item Suppose $f(x) = \frac{p(x)}{q(x)}$ is a rational function where
		$$p(x) = a_mx^m + a_{m-1}x^{x-1} + ... + a_1x + a_0$$
		$$q(x) = b_nx^n + b_{n-1}x^{x-1} + ... + b_1x + b_0$$
		If the degree of $p(x)$ is less than the degree of $q(x)$ then
		$$\lim_{x \to \pm \infty}{f(x)} = 0$$
		If the degree of $p(x)$ equals the degree of $q(x)$ then
		$$\lim_{x \to \pm \infty}{f(x)} = \frac{a_m}{b_n}$$
		If the degree of $p(x)$ is greater than the degree of $q(x)$ then
		$$\lim_{x \to \pm \infty}{f(x)} = - \infty \text{ or } \infty$$
        If the graph of a function $f$ approaches a line (with finite and nonzero slope) as $x \to \pm \infty$, then that line is a slant asymptote/oblique asymptote of $f$
		\item End behaviour for transcendental functions
		$$\lim_{x \to \pm \infty}{\sin{x}} = \text{Does not exist}$$
		$$\lim_{x \to \infty}{e^x} = \infty$$
		$$\lim_{x \to \infty}{e^{-x}} = 0$$
		$$\lim_{x \to - \infty}{e^x} = 0$$
		$$\lim_{x \to - \infty}{e^{-x}} = \infty$$
		$$\lim_{x \to \infty}{\ln{x}} = \infty$$
		$$\lim_{x \to 0^+}{\ln{x}} = - \infty$$
	\end{itemize}
	\item Continuity \\
	\begin{itemize}
		\item \textbf{Definition}: A function $f$ is continuous at $a$ if
		$$\lim_{x \to a}{f(x)} = f(a)$$
		\item A function $f$ is continuous at $a$ if
		\begin{enumerate}
			\item $f(a)$ is defined (Removable Discontinuity)
			\item $\lim\limits_{x \to a}{f(x)}$ exists (Jump Discontinuity)
			\item $\lim\limits{x \to a}{f(x)} = f(a)$ (Removable Discontinuity)
		\end{enumerate}
		\item A function $f$ has an \textbf{Infinite Discontinuity} at $a$ if the function has a Vertical Asymptote at $a$ \\
		\item Suppose $f$ is a function defined on an interval $I$. We say that $f$ is continuous on interval $I$ if $f$ is continuous at every point on the interior of $I$ and the following hold: \\
		\begin{enumerate}
			\item If $a$ is the the left-hand endpoint of $I$ and $a$ is contained in $I$ then
			$$\lim_{x \to a^+}{f(x)} = f(a) \text{ ($f$ is continuous from the right)}$$
			\item If $b$ is the the righ-hand endpoint of $I$ and $b$ is contained in $I$ then
			$$\lim_{x \to b^-}{f(x)} = f(b) \text{ ($f$ is continuous from the left)}$$
		\end{enumerate}
		\item \textbf{Theorem}: All of the following functions are continuous on the intervals where they are defined.
		\begin{enumerate}
			\item Polynomials (continuous everywhere)
			\item Rational Functions (continuous except where denominator is zero)
			\item Exponential functions
			\item Logarithmic functions
			\item Trigonometric functions
			\item Inverse trigonometric functions
		\end{enumerate}
		\item \textbf{Theorem}: If $f$ and $g$ are continuous at $a$, then the following functions are also continuous at $a$. Assume $c$ is a constant and $n > 0$ is an integer.
		\begin{enumerate}
			\item $f + g$
			\item $f - g$
			\item $cf$
			\item $fg$
			\item $\frac{f}{g}$ provided $g(a) \neq 0$
			\item $(f(x))^n$
		\end{enumerate}
		\item \textbf{Theorem}:
		\begin{enumerate}
			\item A polynomial function is continuous for all $x$
			\item A rational function (a function of the form $\frac{p}{q}$, where $p$ and $q$ are polynomials) is continuous for all $x$ for which $q(x) \neq 0$
		\end{enumerate}
		\item \textbf{Theorem}: If $g$ is continuous at $a$ and $f$ is continuous at $g(a)$ then the composite function $f \circ g$ is continuous at $a$. \\
		\item \textbf{Theorem}: Assume $n$ is a positive integer. If $n$ is odd then $(f(x))^{1/n}$ is continuous at all points at which $f$ is continuous. If $n$ is even then $(f(x))^{1/n}$ is continuous at all points $a$ at which $f$ is continuous and $f(a) > 0$ \\
		\item \textbf{Intermediate Value Theorem}: Suppose $f$ is continuous on the interval $[a, b]$ and $L$ is a number strictly between $f(a)$ and $f(b)$. Then there exists at least one number $c$ in $(a,b)$ satisfying $f(c) = L$.
	\end{itemize}
\end{itemize}

\section*{Derivatives (M02)}
\begin{itemize}
    \item Derivatives
    \begin{itemize}
        \item A \textbf{derivative} is a new function made up of the slopes of the tangent lines as they change along a curve
        \item If a curve represents the trajectory of a moving object, the tangent line at a point indicates the direction of motion at that point
        \item As $x \to a$, the slope of the secant lines approaches the slope of the tangent line
        \item Alternative definition for Tangent Line(s): Consider the curve $y = f(x)$ and a secant line intersecting the curve at points $P(a, f(a))$ and $Q(a + h, f(a + h))$, with $m_{sec}$ and $m_{tan}$
        $$\text{Interval: } (a, a + h)$$
        $$m_{\sec} = \frac{f(a + h) - f(a)}{h}$$
        $$m_{\tan} = \lim_{h \to 0}{\frac{f(a + h) - f(a)}{h}}$$
        $$y - f(a) = m_{\tan}(x - a)$$
        \item \textbf{Definition}: The derivative of $f$ at $a$, denoted $f'(a)$, is given by either the two following limits, provided the limits exist and $a$ is in the domain of $f$
        \begin{eqnarray}
            f'(a) &=& \lim_{x \to a}{\frac{f(x) - f(a)}{x - a}} \\
            f'(a) &=& \lim_{h \to 0}{\frac{f(a + h) - f(a)}{h}}
        \end{eqnarray}
        If $f'(a)$ exists, we say that $f$ is \textbf{differentiable} at $a$
    \end{itemize}
\item Derivatives as Functions
    \begin{itemize}
        \item The slope of the tangent line of somefunction $f$ is a function called the derivative of $f$
            $$f'(x) = \lim_{h \to 0}{\frac{f(x + h) - f(x)}{h}}$$
        \item If $f'(x)$ exists, we say that $f$ is \textbf{differentiable} at x
        \item If $f$ is differentiable at every point in some open interval $I$, we say that $f$ is differentiable on $I$
        \item For some function $f$ we can denote the derivative of $f$ like such:
        \begin{eqnarray}
            f'(x) \\
            \frac{dy}{dx} \\
            \frac{df}{dx} \\
            \frac{d}{dx}(f(x)) \\
            D_x (f(x)) \\
            y'(x)
        \end{eqnarray}
        \item When evaluating some derivative $f$ at $a$, we can use the following:
        \begin{eqnarray}
            f'(a) \\
            y'(a) \\
            \frac{df}{dx}\Bigr|_{\substack{x=a}} \\
            \frac{dy}{dx}\Bigr|_{\substack{x=a}}
        \end{eqnarray}
        \item If $f$ is differentiable at $a$, then $f$ is continuous at $a$
        \item If $f$ is not continuous at $a$, then $f$ is not differentiable at $a$
    \end{itemize}
    \item Derivatives as Rate of Change
    \begin{itemize}
        \item Secant Lines give average velocities
        $$v_{av} = \frac{f(a + \Delta t) - f(a)}{\Delta t}$$
        \item Tangent Line gives instantaneous velocity
        $$v(a) = \lim_{\Delta t \to 0}{\frac{f(a + \Delta t) - f(a)}{\Delta t}} = f'(a)$$
        \item Velocity, speed, and acceleration
        \\ Suppose and object moves along a line with position $s = f(t)$
        $$\text{the \textbf{velocity} at time } t \text{ is } v = \frac{ds}{dt} = f'(t)$$
        $$\text{the \textbf{speed} at time } t \text{ is } |v| = |f'(t)|$$
        $$\text{the \textbf{acceleration} at time } t \text{ is } a = \frac{dv}{dt} = \frac{d^2s}{dt^2} = f''(t)$$
        \item Average and marginal cost
        $$\text{The \textbf{cost function} } C(x) \text{ gives the cost to produce the first } x \text{ items in a manufacturing process}$$
        $$\text{The \textbf{average cost} to produce } x \text{ items is } \overline{C}(x) = \frac{C(x)}{x}$$
        $$\text{The \textbf{marginal cost} } C'(x) \text{ is the approximate cost to produce one additional item after producing } x \text{ items}$$
        \item Elasticity
        $$E(p) = \frac{dD}{dp}\frac{p}{D} \text{ where } D = f(p)$$
    \end{itemize}
\end{itemize}

\section*{Rules of Differentiation}
\begin{itemize}
    \item Constant Rule
    $$\text{If } c \in \mathbb{R} \text{, then } \frac{d}{dx}\left(c\right) = 0$$
    \item Power Rule
    $$\text{If } n \in \mathbb{Z} \text{ and } n > 0 \text{, then } \frac{d}{dx}\left(x^n\right) = nx^{n - 1}$$
    \item Derivative of a Root
    $$\frac{d}{dx}\left(\sqrt{x}\right) = \frac{1}{2\sqrt{x}}$$
    \item Constant Multiple Rule
    $$\text{If } f \text{ is differentiable at } x \text{ and } c \text{ is a constant, then } \frac{d}{dx}\left(cf(x)\right) = cf'\left(x\right)$$
    \item Sum Rule
    $$\text{If } f \text{ and } g \text{ are differentiable at } x \text{, then } \frac{d}{dx}\left(f(x) + g(x)\right) = f'(x) + g'(x)$$
    \item Generalized Sum Rule
    $$\frac{d}{dx}\left(f_1(x) + f_2(x) + ... + f_x(x)\right) = f_1'(x) + f_2'(x) + ... + f_n'(x)$$
    \item Difference Rule
    $$\frac{d}{dx}\left(f(x) - g(x)\right) = f'(x) - g'(x)$$
    \item Euler's Number
        $$\text{The function } f(x) = e^x \text{ is differentiable for all } x \in \mathbb{R} \text{, and } \frac{d}{dx}\left(e^x\right) = e^x$$
    \item Higher-order Derivatives
    \\ Assuming $y = f(x)$ can be differentiated as often as necessary, the \textbf{second derivative} of $f$ is
    $$f''(x) = \frac{d}{dx}\left(f'(x)\right)$$
    For $n \in \mathbb{Z}$ where $n \geq 1$, the \textbf{nth derivative} of $f$ is
    $$f^{(n)}\left(x\right) = \frac{d}{dx}\left(f^{(n - 1)}\left(x\right)\right)$$
    \item Product Rule
    $$\text{If } f \text{ and } g \text{ are differentiable at } x \text{, then } \frac{d}{dx}\left(f(x)g(x)\right) = f'(x)g(x) + f(x)g'(x)$$
    \item Quotient Rule
        $$\text{If } f \text{ and } g \text{ are differentiable at } x \text{ and } g(x) \neq 0 \text{, then the derivative of } \frac{f}{g} \text{ at } x \text{ exists and}$$
        $$\frac{d}{dx}\left(\frac{f(x)}{g(x)}\right) = \frac{g(x)f'(x) - f(x)g'(x)}{(g(x))^2}$$
\end{itemize}

\section*{Trigonometric (and inverse) Derivatives}
\begin{eqnarray}
    \lim_{x \to 0}{\frac{\sin{x}}{x}} &=& 1 \\
    \lim_{x \to 0}{\frac{\cos{x} - 1}{x}} &=& 0 \\
    \frac{d}{dx}\left(\sin{x}\right) &=& \cos{x} \\
    \frac{d}{dx}\left(\cos{x}\right) &=& - \sin{x} \\
    \frac{d}{dx}\left(\tan{x}\right) &=& \sec^2{x} \\
    \frac{d}{dx}\left(\cot{x}\right) &=& - \csc^2{x} \\
    \frac{d}{dx}\left(\sec{x}\right) &=& \sec{x}\tan{x} \\
    \frac{d}{dx}\left(\csc{x}\right) &=& - \csc{x}\cot{x} \\
    \frac{d}{dx}\left(\sin^{-1}{x}\right) &=& \frac{1}{\sqrt{1 - x^2}} \text{, for } -1 < x < 1 \\
    \frac{d}{dx}\left(\cos^{-1}{x}\right) &=& -\frac{1}{\sqrt{1 - x^2}} \text{, for } -1 < x < 1 \\
    \frac{d}{dx}\left(\tan^{-1}{x}\right) &=& \frac{1}{1 + x^2} \text{, for } -\infty < x < \infty \\
    \frac{d}{dx}\left(\cot^{-1}{x}\right) &=& -\frac{1}{1 + x^2} \text{, for } -\infty < x < \infty \\
    \frac{d}{dx}\left(\sec^{-1}{x}\right) &=& \frac{1}{|x|\sqrt{x^2 - 1}} \text{, for } |x| > 1 \\
    \frac{d}{dx}\left(\csc^{-1}{x}\right) &=& -\frac{1}{|x|\sqrt{x^2 - 1}} \text{, for } |x| > 1 \\
\end{eqnarray}

\section*{More Derivatives (M03)}
\begin{itemize}
    \item The Chain Rule
        \\ Suppose $y = f(u)$ is differentiable at $u = g(x)$ and $u = g(x)$ is differentiable at $x$. The composite function $y = f(g(x))$ is differentiable at $x$, and its derivative can be expressed in two equivalent ways.
        \begin{eqnarray}
            \frac{dy}{dx} &=& \frac{dy}{dy} \cdot \frac{du}{dx} \\
            \frac{d}{dx}\left(f\left(g\left(x\right)\right)\right) &=& f'\left(g\left(x\right)\right) \cdot g'\left(x\right)
        \end{eqnarray}
        Application of the Chain Rule (Assume the differentiable function $y = f(g(x))$ is given):
        \begin{enumerate}
            \item Identify an outer function $f$ and an inner function $g$, and let $u = g(x)$.
            \item Replace $g(x)$ with $u$ to express $y$ in terms of $u$:
            $$y = f(g(x)) = f(u)$$
            \item Calculate the product
            $$\frac{dy}{du} \cdot \frac{du}{dx}$$
            \item Replace $u$ with $g(x)$ in $\frac{dy}{du}$ to obtain $\frac{dy}{dx}$
        \end{enumerate}
        $$\text{If } g \text{ is differentiable for all } x \text{ in its domain and } p \in \mathbb{R} \text{,}$$
        $$\frac{d}{dx}\left(\left(g\left(x\right)\right)^p\right) = p\left(g\left(x\right)\right)^{p - 1}g'\left(x\right)$$
    \item Implicit Differentiation
    \\ When we are unable to solve for $y$ explicitly, we treat $y$ as a function of $x \left(y = y(x)\right)$ and apply the Chain Rule:
    $$y' = \frac{dy}{dx}$$
    $$\frac{d}{dx}y^n = ny^{n - 1}\frac{dy}{dx}$$
    \item Derivatives of Logarithmic and Exponential Functions
    $$\frac{d}{dx}\left(\ln{x}\right) = \frac{1}{x} \text{, for } x > 0$$
    $$\frac{d}{dx}\left(\ln{|x|}\right) = \frac{1}{x} \text{, for } x \neq 0$$
    If $u$ is differentiable at $x$ and $u(x) \neq 0$, then
    $$\frac{d}{dx}\left(\ln{|u(x)|}\right) = \frac{u'(x)}{u(x)}$$
    If $b > 0$ and $b \neq 1$, then for all $x$,
    $$\frac{d}{dx}\left(b^x\right) = b^x\ln{b}$$
    General Power Rule:
    $$\text{For } p \in \mathbb{R} \text{ and for } x > 0 \text{, } \frac{d}{dx}\left(x^p\right) = px^{p - 1}$$
    Furthermore, if $u$ is a positive differentiable function on its domain, then
    $$\frac{d}{dx}\left(u\left(x\right)^p\right) = p\left(u\left(x\right)\right)^{p - 1} \cdot u'\left(x\right)$$
    Functions of the form $f(x) = \left(g(x)\right)^{h(x)}$, where both $g$ and $h$ are nonconstant functions, are neither exponential function nor power functions (they are sometimes called \textbf{tower functions}). To compute their derivatives, we use the identity $b^x = e^{x\ln{b}}$ to rewrite $f$ with base $e$:
    $$f(x) = \left(g(x)\right)^{h(x)} = e^{h(x)\ln{g(x)}}$$
    If $b > 0$ and $b \neq 1$, then
    $$\frac{d}{dx}\left(\log_b{x}\right) = \frac{1}{x\ln{b}} \text{, for } x > 0$$
    $$\frac{d}{dx}\left(\log_b{|x|}\right) = \frac{1}{x\ln{b}} \text{, for } x \neq 0$$
    Useful Properties of Logarithms
    \begin{eqnarray}
        \ln{xy} &=& \ln{x} + \ln{y} \\
        \ln{\left(\frac{x}{y}\right)} &=& \ln{x} - \ln{y} \\
        \ln{x^z} &=& z\ln{x}
    \end{eqnarray}
    Let $f$ be differentiable and have an inverse on an interval $I$. If $x_0$ is a point of $I$ at which $f'(x_0) \neq 0$, then $f^{-1}$ is differentiable at $y_0 = f(x_0)$ and
    $$\left(f^{-1}\right)'\left(y_0\right) = \frac{1}{f'\left(x_0\right)} \text{, where } y_0 = f\left(x_0\right)$$
    \item Related Rates
    \\ \textbf{Procedure}
    \begin{enumerate}
        \item Read the problem carefully, making a sketch to organize the given information. Identify the rates that are given and the rate that is to be determined.
        \item Write one or more equations that express the basic relationships among the variables.
        \item Introduce rates of change by differentiating the appropriate equation(s) with respect to time $t$.
        \item Substitute known values and solve for the desired quantity.
        \item Check that units are consistent and the answer is reasonable. (For example, does it have the correct sign?)
    \end{enumerate}
\end{itemize}

\blfootnote{A copy of my notes (in \LaTeX) are available on my \href{https://github.com/onlinechronically/MATH-211}{GitHub}}
\end{document}
