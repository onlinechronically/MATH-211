\documentclass{article}
\usepackage{graphicx}
\usepackage{amsthm}
\usepackage{amsmath}
\usepackage{amssymb}
\usepackage{geometry}
\usepackage{tikz}
\usepackage[hidelinks]{hyperref}
\usetikzlibrary{arrows}

\geometry{a4paper, total={170mm,257mm}, left=20mm, top=20mm}
\AtBeginEnvironment{align}{\setcounter{equation}{0}} 
\AtBeginEnvironment{eqnarray}{\setcounter{equation}{0}} 

\newcommand\blfootnote[1]{
    \begingroup
    \renewcommand\thefootnote{}\footnote{#1}
    \addtocounter{footnote}{-1}
    \endgroup
}

\title{Module 3 Notes (MATH-211)}
\author{Lillie Donato}
\date{24 June 2024}

\begin{document}

\maketitle

\section*{General Notes (and Definitions)}
\begin{itemize}
    \item The Chain Rule
        \\ Suppose $y = f(u)$ is differentiable at $u = g(x)$ and $u = g(x)$ is differentiable at $x$. The composite function $y = f(g(x))$ is differentiable at $x$, and its derivative can be expressed in two equivalent ways.
        \begin{eqnarray}
            \frac{dy}{dx} &=& \frac{dy}{dy} \cdot \frac{du}{dx} \\
            \frac{d}{dx}\left(f\left(g\left(x\right)\right)\right) &=& f'\left(g\left(x\right)\right) \cdot g'\left(x\right)
        \end{eqnarray}
        Application of the Chain Rule (Assume the differentiable function $y = f(g(x))$ is given):
        \begin{enumerate}
            \item Identify an outer function $f$ and an inner function $g$, and let $u = g(x)$.
            \item Replace $g(x)$ with $u$ to express $y$ in terms of $u$:
            $$y = f(g(x)) = f(u)$$
            \item Calculate the product
            $$\frac{dy}{du} \cdot \frac{du}{dx}$$
            \item Replace $u$ with $g(x)$ in $\frac{dy}{du}$ to obtain $\frac{dy}{dx}$
        \end{enumerate}
        $$\text{If } g \text{ is differentiable for all } x \text{ in its domain and } p \in \mathbb{R} \text{,}$$
        $$\frac{d}{dx}\left(\left(g\left(x\right)\right)^p\right) = p\left(g\left(x\right)\right)^{p - 1}g'\left(x\right)$$
    \item Implicit Differentiation
    \\ When we are unable to solve for $y$ explicitly, we treat $y$ as a function of $x \left(y = y(x)\right)$ and apply the Chain Rule:
    $$y' = \frac{dy}{dx}$$
    $$\frac{d}{dx}y^n = ny^{n - 1}\frac{dy}{dx}$$
    \item Derivatives of Logarithmic and Exponential Functions
    $$\frac{d}{dx}\left(\ln{x}\right) = \frac{1}{x} \text{, for } x > 0$$
    $$\frac{d}{dx}\left(\ln{|x|}\right) = \frac{1}{x} \text{, for } x \neq 0$$
    If $u$ is differentiable at $x$ and $u(x) \neq 0$, then
    $$\frac{d}{dx}\left(\ln{|u(x)|}\right) = \frac{u'(x)}{u(x)}$$
    If $b > 0$ and $b \neq 1$, then for all $x$,
    $$\frac{d}{dx}\left(b^x\right) = b^x\ln{b}$$
    General Power Rule:
    $$\text{For } p \in \mathbb{R} \text{ and for } x > 0 \text{, } \frac{d}{dx}\left(x^p\right) = px^{p - 1}$$
    Furthermore, if $u$ is a positive differentiable function on its domain, then
    $$\frac{d}{dx}\left(u\left(x\right)^p\right) = p\left(u\left(x\right)\right)^{p - 1} \cdot u'\left(x\right)$$
    Functions of the form $f(x) = \left(g(x)\right)^{h(x)}$, where both $g$ and $h$ are nonconstant functions, are neither exponential function nor power functions (they are sometimes called \textbf{tower functions}). To compute their derivatives, we use the identity $b^x = e^{x\ln{b}}$ to rewrite $f$ with base $e$:
    $$f(x) = \left(g(x)\right)^{h(x)} = e^{h(x)\ln{g(x)}}$$
    If $b > 0$ and $b \neq 1$, then
    $$\frac{d}{dx}\left(\log_b{x}\right) = \frac{1}{x\ln{b}} \text{, for } x > 0$$
    $$\frac{d}{dx}\left(\log_b{|x|}\right) = \frac{1}{x\ln{b}} \text{, for } x \neq 0$$
    Useful Properties of Logarithms
    \begin{eqnarray}
        \ln{xy} &=& \ln{x} + \ln{y} \\
        \ln{\left(\frac{x}{y}\right)} &=& \ln{x} - \ln{y} \\
        \ln{x^z} &=& z\ln{x}
    \end{eqnarray}
    \item Derivatives of Inverse Trigonometric Functions
    $$\frac{d}{dx}\left(\sec^{-1}{x}\right) = \begin{aligned}
		\begin{cases}
            \frac{1}{x\sqrt{x^2 - 1}} &\text{ if } x > 1 \\
            -\frac{1}{x\sqrt{x^2 - 1}} &\text{ if } x < 1
		\end{cases}
	\end{aligned}$$
    \begin{eqnarray}
        \frac{d}{dx}\left(\sin^{-1}{x}\right) &=& \frac{1}{\sqrt{1 - x^2}} \text{, for } -1 < x < 1 \\
        \frac{d}{dx}\left(\cos^{-1}{x}\right) &=& -\frac{1}{\sqrt{1 - x^2}} \text{, for } -1 < x < 1 \\
        \frac{d}{dx}\left(\tan^{-1}{x}\right) &=& \frac{1}{1 + x^2} \text{, for } -\infty < x < \infty \\
        \frac{d}{dx}\left(\cot^{-1}{x}\right) &=& -\frac{1}{1 + x^2} \text{, for } -\infty < x < \infty \\
        \frac{d}{dx}\left(\sec^{-1}{x}\right) &=& \frac{1}{|x|\sqrt{x^2 - 1}} \text{, for } |x| > 1 \\
        \frac{d}{dx}\left(\csc^{-1}{x}\right) &=& -\frac{1}{|x|\sqrt{x^2 - 1}} \text{, for } |x| > 1 \\
    \end{eqnarray}
    Let $f$ be differentiable and have an inverse on an interval $I$. If $x_0$ is a point of $I$ at which $f'(x_0) \neq 0$, then $f^{-1}$ is differentiable at $y_0 = f(x_0)$ and
    $$\left(f^{-1}\right)'\left(y_0\right) = \frac{1}{f'\left(x_0\right)} \text{, where } y_0 = f\left(x_0\right)$$
    \item Related Rates
    \\ \textbf{Procedure}
    \begin{enumerate}
        \item Read the problem carefully, making a sketch to organize the given information. Identify the rates that are given and the rate that is to be determined.
        \item Write one or more equations that express the basic relationships among the variables.
        \item Introduce rates of change by differentiating the appropriate equation(s) with respect to time $t$.
        \item Substitute known values and solve for the desired quantity.
        \item Check that units are consistent and the answer is reasonable. (For example, does it have the correct sign?)
    \end{enumerate}
\end{itemize}

\section*{Examples}
\begin{enumerate}
    \item The Chain Rule
    \begin{eqnarray}
        y &=& \left(5x^2+11x\right)^{\frac{4}{3}} \\
        u &=& 5x^2 + 11x \\
        f &=& u^{\frac{4}{3}} \\
        \frac{dy}{dx} &=& \frac{dy}{du} \cdot \frac{du}{dx} \\
                      &=& \frac{4}{3}u^{\frac{1}{3}} \cdot 10x + 11 \\
                      &=& \frac{4}{3}\left(5x^2 + 11x\right)^{\frac{1}{3}} \cdot 10x + 11 \\
    \end{eqnarray}
    \begin{eqnarray}
        y &=& e^{4x^2 + 1} \\
        u &=& 4x^2 + 1 \\
        y &=& e^u \\
        \frac{dy}{dx} &=& \frac{dy}{du} \cdot \frac{du}{dx} \\
                      &=& e^u \cdot 8x \\
                      &=& e^{4x^2 + 1} \cdot 8x \\
                      &=& 8xe^{4x^2 + 1}
    \end{eqnarray}
    \item The Chain Rule (with Tables)
    $$h(x) = f(g(x))$$
    \begin{eqnarray}
        y &=& f(g(x)) \\
        u &=& g(x) \\
        y &=& f(u) \\
        \frac{dy}{dx} &=& \frac{dy}{du} \cdot \frac{du}{dx} \\
                      &=& f(u) \cdot g'(x) \\
                      &=& f'(g(x)) \cdot g'(x) \\
                      \nonumber \\
        h'(1) &=& f'(g(1)) \cdot g'(1) \\
              &=& f'(4) \cdot 9 \\
              &=& 7 \cdot 9 \\
              &=& 63 \\
              \nonumber \\
        h'(2) &=& f'(g(2)) \cdot g'(2) \\
              &=& f'(1) \cdot 7 \\
              &=& -6 \cdot 7 \\
              &=& -42 \\
              \nonumber \\
        h'(3) &=& f'(g(3)) \cdot g'(3) \\
              &=& f'(5) \cdot 3 \\
              &=& 2 \cdot 3 \\
              &=& 6
    \end{eqnarray}
    $$k(x) = g(g(x))$$
    \begin{eqnarray}
        y &=& g(g(x)) \\
        u &=& g(x) \\
        y &=& g(u) \\
        \frac{dy}{dx} &=& \frac{dy}{du} \cdot \frac{du}{dx} \\
                      &=& g'(u) \cdot g'(x) \\
                      &=& g'(g(x)) \cdot g'(x) \\
                      \nonumber \\
        k'(3) &=& g'(g(3)) \cdot g'(3) \\
              &=& g'(5) \cdot 3 \\
              &=& -5 \cdot 3 \\
              &=& -15 \\
              \nonumber \\
        k'(1) &=& g'(g(1)) \cdot g'(1) \\
              &=& g'(4) \cdot 9 \\
              &=& -1 \cdot 9 \\
              &=& -9 \\
              \nonumber \\
        k'(5) &=& g'(g(5)) \cdot g'(5) \\
              &=& g'(3) \cdot -5 \\
              &=& 3 \cdot -5 \\
              &=& -15
    \end{eqnarray}
    \item The Chain Rule (All Forms)
    \begin{eqnarray}
        y &=& \sqrt[3]{2x^2 - x - 5} \\
        u &=& 2x^2 - x - 5 \\
        y &=& \sqrt[3]{u} \\
        y' &=& u^{-\frac{2}{3}} \cdot 4x - 1 \\
           &=& \frac{1}{3}\left(2x^2 - x - 5\right)^{-\frac{2}{3}} \cdot 4x - 1
    \end{eqnarray}
    \begin{eqnarray}
        y &=& \csc{(\tan{t})} \\
        u &=& \tan{t} \\
        y &=& \csc{u} \\
        y' &=& -\csc{u}\cot{u} \cdot \sec^2{t} \\
           &=& -\csc{(\tan{t})}\cot{(\tan{t})} \cdot \sec^2{t}
    \end{eqnarray}
    \item The Chain Rule (Nested)
    \begin{eqnarray}
        y &=& \tan{(\sin{e^x})} \\
        u_2 &=& e^x \\
        u_1 &=& \sin{u_2} \\
        y &=& \tan{u_1} \\
        y' &=& \sec^2{(\sin{e^x})} \cdot \cos{e^x} \cdot e^x
    \end{eqnarray}
    \item The Chain Rule (Combination of Rules)
    \begin{eqnarray}
        y &=& \left(\frac{e^x}{x+1}\right)^8 \\
        y' &=& 8\left(\frac{e^x}{x+1}\right)^7 \cdot \frac{xe^x}{(x+1)^2} \\
           &=& 8\frac{e^{7x}}{(x+1)^7} \cdot \frac{xe^x}{(x+1)^2} \\
           &=& \frac{8xe^{8x}}{(x+1)^9}
    \end{eqnarray}
    \item Implicit Differentiation
    \begin{eqnarray}
        x^4 + y^4 &=& 2 \\
        4x^3 + 4y^3\frac{dy}{dx} &=& 0 \\
        4y^3\frac{dy}{dx} &=& -4x^3 \\
        \frac{dy}{dx} &=& \frac{-4x^3}{4y^3} \\
                      &=& \frac{-x^3}{y^3}
    \end{eqnarray}
    $$\frac{dy}{dx}\Bigr|_{\substack{(1,-1)}} = \frac{-(1)^3}{(-1)^3} = \frac{-1}{-1} = 1$$
    \item Implicit Differentiation (Finding $y$)
    \begin{eqnarray}
        y &=& y = xe^y \\
        y' &=& e^y + xe^yy' \\
        y' - y'xe^y &=& e^y \\
        y'(1 - xe^y) &=& e^y \\
        y' &=& \frac{e^y}{1 - xe^y}
    \end{eqnarray}
    \item Implicit Differentiation (Tangent Line)
    \begin{eqnarray}
        \cos{(x - y)} + \sin{y} &=& \sqrt{2} \\
        \left(-\sin{(x - y)}\right)\left(1 - 1y'\right) + \cos{y}\left(y'\right) &=& 0 \\
        -\sin{(x - y)} + y'\sin{(x - y)} + y'\cos{y} &=& 0 \\
        y'\left(\sin{(x - y)} + \cos{y}\right) &=& \sin{(x - y)} \\
        y' &=& \frac{\sin{(x - y)}}{\sin{(x - y)} + \cos{y}} \\
        y'\Bigr|_{\substack{\left(\frac{\pi}{2},\frac{\pi}{4}\right)}} &=& \frac{\sin{(\frac{\pi}{2} - \frac{\pi}{2})}}{\sin{(\frac{\pi}{2} - \frac{\pi}{2})} + \cos{\frac{\pi}{4}}} \\
        &=& \frac{1}{2} \\
        y &=& \frac{1}{2}x
    \end{eqnarray}
    \item Implicit Differentiation (Higher Order)
    \begin{eqnarray}
        x^4 + y^4 &=& 64 \\
        4x^3 + 4y^3\frac{dy}{dx} &=& 0 \\
        4y^3\frac{dy}{dx} &=& -4x^3 \\
        \frac{dy}{dx} &=& \frac{-4x^3}{4y^3} \\
                      &=& \frac{-x^3}{y^3} \\
        \frac{d^2y}{dx^2} &=& \frac{-3x^2y^3 - \left(-x^33y^2\frac{dy}{dx}\right)}{\left(y^3\right)^2} \\
                          &=& \frac{-3x^2y^3 + 3x^3y^2\frac{dy}{dx}}{y^6} \\
                          &=& \frac{-3x^2y^3 + 3x^3y^2\frac{-x^3}{y^3}}{y^6} \\
                          &=& \frac{-3x^2y^3 + \frac{-3x^6y^2}{y^3}}{y^6} \\
                          &=& \frac{-3x^2y^3 - \frac{3x^6}{y}}{y^6} \\
                          &=& \frac{\frac{-3x^2y^4 - 3x^6}{y}}{y^6} \\
                          &=& \frac{-3x^2y^4 - 3x^6}{y^7}
    \end{eqnarray}
    \item Derivatives with $\ln{x}$
    \begin{eqnarray}
        y &=& \ln{2x^8} \\
        \frac{dy}{dx} &=& \frac{1}{2x^8} \cdot 16x^7 \\
                      &=& \frac{16x^7}{2x^8} \\
                      &=& \frac{8}{x}
    \end{eqnarray}
    \begin{eqnarray}
        y &=& x^2\left(1 - \ln{x^2}\right) \\
        \frac{dy}{dx} &=& 2x\left(1 - \ln{x^2}\right) + x^2\left(- \frac{2x}{x^2}\right) \\
                      &=& 2x - 2x\ln{x^2} - 2x \\
                      &=& - 2x\ln{x^2}
    \end{eqnarray}
    \item Derivatives with $b^x$
    \begin{eqnarray}
        y &=& 2^{2x} \\
        \frac{dy}{dx} &=& 2^{2x}\ln{2} \cdot 2 \\
                      &=& 2^{2x + 1}\ln{2}
    \end{eqnarray}
    \begin{eqnarray}
        f(x) &=& 7^{-x}\cos{x} \\
        \frac{dy}{dx} &=& -7^{-x}\ln{7}\cos{x} - 7^{-x}\sin{x} \\
                      &=& -7^{-x}\left(\ln{7}\cos{x} + \sin{x}\right)
    \end{eqnarray}
    \item Derivatives with the General Power Rule
    \begin{eqnarray}
        y &=& x^e \\
        \frac{dy}{dx} &=& ex^{e - 1}
    \end{eqnarray}
    \begin{eqnarray}
        f(x) &=& \left(x^3 + 3^x\right)^\pi \\
        \frac{dy}{dx} &=& \pi\left(x^3 + 3^x\right)^{\pi - 1} \cdot \left(3x^2 + 3^x\ln{3}\right)
    \end{eqnarray}
    \item Derivatives with Tower Functions
    \begin{eqnarray}
        g(x) &=& x^{\ln{x}} \\
             &=& e^{\ln{x}\ln{x}} \\
             &=& e^{\left(\ln{x}\right)^2} \\
        g'(x) &=& e^{\left(\ln{x}\right)^2} \cdot \frac{2\ln{x}}{x} \\
        g'(e) &=& e^{\left(\ln{e}\right)^2} \cdot \frac{2\ln{e}}{e} \\
              &=& \frac{2e}{e} \\
              &=& 2 \\
        y &=& 2x - 2e + e \\
          &=& 2x - e
    \end{eqnarray}
    \item Derivatives of Logarithmic Functions
    \begin{eqnarray}
        y &=& \log_7{5x} \\
        \frac{dy}{dx} &=& \frac{5}{5x\ln{7}} \\
                      &=& \frac{1}{x\ln{7}}
    \end{eqnarray}
    \begin{eqnarray}
        y &=& \log{\left(\log{x}\right)} \\
        \frac{dy}{dx} &=& \frac{1}{\log{x}\ln{10}} \cdot \frac{1}{x\ln{10}} \\
                      &=& \frac{1}{x\log{x}\ln{\left(10\right)}^2}
    \end{eqnarray}
    \item Logarithmic Differentiation
    \begin{eqnarray}
        f(x) &=& \left(\cos{x}\right)^{\sec{x}} \\
        \ln{f(x)} &=& \ln{\left(\left(\cos{x}\right)^{\sec{x}}\right)} \\
                  &=& \sec{x} \cdot \ln{\left(\cos{x}\right)} \\
        \frac{f'(x)}{f(x)} &=& \sec{x} \cdot \tan{x} \cdot \ln{\left(\cos{x}\right)} + \sec{x} \cdot \frac{-\sin{x}}{\cos{x}} \\
                      &=& \sec{x} \cdot \tan{x} \cdot \ln{\left(\cos{x}\right)} + \sec{x} \cdot \left(-\tan{x}\right) \\
                      &=& \tan{x}\sec{x}\left(\ln{\left(\cos{x}\right)} - 1\right) \\
        f'(x) &=& f(x)\tan{x}\sec{x}\left(\ln{\left(\cos{x}\right)} - 1\right) \\
              &=& \left(\cos{x}\right)^{\sec{x}}\tan{x}\sec{x}\left(\ln{\left(\cos{x}\right)} - 1\right) \\
    \end{eqnarray}
    \item Derivatives with $\sin^{-1}{x}$
    \begin{eqnarray}
        \frac{d}{dx} \left(\sin^{-1}{\left(\ln{x}\right)}\right) &=& \frac{1}{\sqrt{1 - \left(\ln{x}\right)^2}} \cdot \frac{1}{x} \\
                                                                 &=& \frac{1}{x\sqrt{1 - \left(\ln{x}\right)^2}}
    \end{eqnarray}
    \begin{eqnarray}
        \frac{d}{dx}\left(\sin^{-1}{\left(e^{-2x}\right)}\right) &=& \frac{1}{\sqrt{1 - e^{-4x}}} \cdot e^{-2x} \cdot -2 \\
                                                                 &=& \frac{-2e^{-2x}}{\sqrt{1 - e^{-4x}}} \\
    \end{eqnarray}
    \item Finding the Tangent Line of Inverse Trigonometric Functions
    \begin{eqnarray}
        f(x) &=& \cos^{-1}{x^2} \\
        \left(\frac{1}{\sqrt{2}}, \frac{\pi}{3}\right) \\
        f'(x) &=& -\frac{2x}{\sqrt{1 - x^4}} \\
        f'\left(\frac{1}{\sqrt{2}}\right) &=& -\frac{\frac{2}{\sqrt{2}}}{\sqrt{1 - \frac{1}{\sqrt{2}}^4}} \\
                                          &=& -\frac{\sqrt{2}}{\sqrt{1 - \frac{1}{4}}} \\
                                          &=& -\frac{\sqrt{2}}{\sqrt{\frac{3}{4}}} \\
                                          &=& -\frac{\sqrt{2}}{\frac{\sqrt{3}}{2}} \\
                                          &=& -\frac{2\sqrt{2}}{\sqrt{3}} \\
                                          &=& \frac{-4}{\sqrt{6}} \\
        y &=& \frac{-4}{\sqrt{6}}x + \frac{2}{\sqrt{3}} + \frac{\pi}{3}
    \end{eqnarray}
    \item Application of Derivatives of Inverse Trigonometric Functions
    \begin{eqnarray}
        \tan{\theta} &=& \frac{400}{x} \\
        \theta &=& \tan^{-1}{\frac{400}{x}} \\
        \frac{d\theta}{dx} &=& \frac{-400}{x^2\left(1 + \left(\frac{400}{x}\right)^2\right)} \\
        \frac{d\theta}{dx}\Bigr|_{\substack{x = 500}} &=& \frac{-400}{500^2\left(1 + \left(\frac{400}{500}\right)^2\right)} \\
                                                      &=& \frac{-400}{500^2\left(1 + \left(\frac{4}{5}\right)^2\right)} \\
                                                      &=& \frac{-400}{500^2 \cdot \frac{41}{25}} \\
                                                      &=& -0.000976
    \end{eqnarray}
    \item Derivatives of Inverse Functions
        $$f(x) = x^3 + 3$$
        If $(2, -1)$ is on the graph of $f^{-1}(x)$, then $(-1, 2)$ is on the graph of $f(x)$.
        $$\left(f^{-1}\right)\left(2\right) = \frac{1}{f'\left(-1\right)} = \frac{1}{3\left(-1\right)^2} = \frac{1}{3}$$
    \item Related Rates (Geometric Ideas)
    \\ Hint: we want the decreasing length
    \\ Hint (2): use the pythagorean theorem
    \begin{eqnarray}
        A &=& x^2 \\
        \frac{dA}{dt} &=& 2x \cdot \frac{dx}{dt} \\
        f'(5) &=& 2(5) \cdot -1 \\
              &=& -10 \text{ m/s}^2 \\
        L^2 &=& x^2 + x^2 \\
            &=& 2x^2 \\
        L &=& \sqrt{2x^2} \\
          &=& \sqrt{2}x \\
        \frac{dL}{dt} &=& \sqrt{2} \cdot -1 \\
                      &=& -\sqrt{2} \text{ m/s}
    \end{eqnarray}
    \item Related Rates (Distance Formula)
    \\ Hint: use the distance formula
    \begin{eqnarray}
        A(h) &=& 20h \\
        B(h) &=& 15h \\
        A(0.5) &=& 20(0.5) \\
               &=& 10 \\
        B(0.5) &=& 15(0.5) \\
               &=& 7.5 \\
        L^2 &=& x^2 + y^2 \\
        2L\frac{dL}{dt} &=& 2x\frac{dx}{dt} + 2y\frac{dy}{dt} \\
        \frac{dL}{dt} &=& \frac{2x\frac{dx}{dt} + 2y\frac{dy}{dt}}{2L} \\
                      &=& \frac{x\frac{dx}{dt} + y\frac{dy}{dt}}{L} \\
                      &=& \frac{10(20) + 7.5(15)}{\sqrt{10^2 + 7.5^2}} \\
                      &=& \frac{200 + 112.5}{\sqrt{100 + 56.25}} \\
                      &=& \frac{312.5}{\sqrt{156.25}} \\
                      &=& \frac{312.5}{\sqrt{156.25}} \\
                      &=& \frac{312.5}{12.5} \\
                      &=& 25
    \end{eqnarray}
    \item Related Rates (Cylinders \& Cones)
    \begin{eqnarray}
        V &=& \pi r^2h \\
        \frac{dh}{dt} = -0.25 \\
        \frac{dv}{dt} &=& \pi r^2 \cdot \frac{dh}{dt} \\
                      &=& \pi 2^2 \cdot -0.25 \\
                      &=& -\pi
    \end{eqnarray}
    \item Related Rates (Trigonometry)
    \begin{eqnarray}
        \frac{dy}{dt} &=& 20 \\
        \tan{\theta} &=& \frac{y}{300} \\
        \theta &=& \tan^{-1}{\frac{y}{300}} \\
        \frac{d\theta}{dt} &=& \frac{1}{1 + \left(\frac{y}{300}\right)^2} \cdot \frac{1}{300} \cdot \frac{dy}{dt} \\
        \frac{d\theta}{dt}\Bigr|_{\substack{y = 400}} &=& \frac{20}{300\left(1 + \frac{16}{9}\right)} \\
                                                      &=& 0.024
    \end{eqnarray}
\end{enumerate}

\section*{Related Exercises}
\begin{enumerate}
    \item (Section 3.7 Exercise 15)
        \begin{eqnarray}
            y &=& \left(3x+7\right)^{10} \\
            u &=& 3x+7 \\
            f(u) &=& u^{10} \\
            y' &=& 10u^9 \cdot 3 \\
               &=& 10\left(3x+7\right)^9 \cdot 3 \\
               &=& 30\left(3x+7\right)^9
        \end{eqnarray}
    \item (Section 3.7 Exercise 17)
        \begin{eqnarray}
            y &=& \sin^5{x} \\
            u &=& \sin{x} \\
            f(u) &=& u^5 \\
            y' &=& 5u^4 \cdot \cos{x} \\
               &=& 5\sin^4{x} \cos{x}
        \end{eqnarray}
    \item (Section 3.7 Exercise 28)
        \begin{eqnarray}
            y &=& \left(x^2+2x+7\right)^8 \\
            u &=& x^2 + 2x + 7 \\
            f(u) &=& u^8 \\
            y' &=& 8u^7 \cdot (2x + 2) \\
               &=& 8\left(x^2+2x+7\right)^7 \cdot (2x + 2) \\
               &=& \left(16x + 16\right)\left(x^2+2x+7\right)^7
        \end{eqnarray}
    \item (Section 3.7 Exercise 29)
        \begin{eqnarray}
            y &=& \sqrt{10x + 1} \\
            u &=& 10x + 1 \\
            f(u) &=& \sqrt{u} \\
            y' &=& \frac{1}{2\sqrt{u}} \cdot 10 \\
               &=& \frac{1}{2\sqrt{10x + 1}} \cdot 10 \\
               &=& \frac{10}{2\sqrt{10x + 1}} \\
               &=& \frac{5}{\sqrt{10x + 1}}
        \end{eqnarray}
    \item (Section 3.7 Exercise 41)
        \begin{eqnarray}
            y &=& \sqrt[4]{\frac{2x}{4x-3}} \\
            u &=& \frac{2x}{4x-3} \\
            f(u) &=& \sqrt[4]{u} \\
            y' &=& \frac{1}{4}u^{-\frac{3}{4}} \cdot -\frac{6}{\left(4x-3\right)^2} \\
               &=& \frac{1}{4}\left(\frac{2x}{4x-3}\right)^{-\frac{3}{4}} \cdot -\frac{6}{\left(4x-3\right)^2} \\
               &=& -\frac{6}{4\left(4x-3\right)^2}\left(\frac{2x}{4x-3}\right)^{-\frac{3}{4}}
        \end{eqnarray}
    \item (Section 3.7 Exercise 23)
        \begin{eqnarray}
            y &=& \tan{5x^2} \\
            u &=& 5x^2 \\
            f(u) &=& \tan{u} \\
            y' &=& \sec^2{u} \cdot 10x \\
               &=& \sec^2{5x^2} \cdot 10x \\
               &=& 10x\sec^2{5x^2} \\
        \end{eqnarray}
    \item (Section 3.7 Exercise 24)
        \begin{eqnarray}
            y &=& \sin{\frac{x}{4}} \\
            u &=& \frac{x}{4} \\
            f(u) &=& \sin{u} \\
            y' &=& \cos{u} \cdot \frac{4}{16} \\
               &=& \cos{\frac{x}{4}} \cdot \frac{1}{4} \\
               &=& \frac{1}{4}\cos{\frac{x}{4}}
        \end{eqnarray}
    \item (Section 3.7 Exercise 45)
        \begin{eqnarray}
            y &=& (2x^6 - 3x^3 + 3)^{25} \\
            u &=& 2x^6 - 3x^3 + 3 \\
            f(u) &=& u^{25} \\
            y' &=& 25\left(u\right)^{24} \cdot 12x^5 - 9x^2 \\
               &=& 25\left(2x^6 - 3x^3 + 3\right)^{24} \cdot 12x^5 - 9x^2 \\
               &=& 25\left(12x^5 - 9x^2\right)\left(2x^6 - 3x^3 + 3\right)^{24}
        \end{eqnarray}
    \item (Section 3.7 Exercise 46)
        \begin{eqnarray}
            y &=& (\cos{x} + 2\sin{x})^8 \\
            u &=& \cos{x} + 2\sin{x} \\
            f(u) &=& u^8 \\
            y' &=& 8u^7 \cdot (-\sin{x} + 2\cos{x}) \\
               &=& 8\left(\cos{x} + 2\sin{x}\right)^7 \cdot (-\sin{x} + 2\cos{x}) \\
               &=& 8(-\sin{x} + 2\cos{x})\left(\cos{x} + 2\sin{x}\right)^7 \\
        \end{eqnarray}
    \item (Section 3.7 Exercise 53)
        \begin{eqnarray}
            y &=& \sin{(\sin{(e^x)})} \\
            y' &=& \cos{(\sin{e^x})}\cos{e^x}e^x
        \end{eqnarray}
    \item (Section 3.7 Exercise 54)
        \begin{eqnarray}
            y &=& \sin^2{e^{3x + 1}} \\
            y' &=& 6\sin{e^{3x+1}}\cos{e^{3x+1}}
        \end{eqnarray}
    \item (Section 3.7 Exercise 68)
        \begin{eqnarray}
            y &=& \left(\frac{3x}{4x+2}\right)^5 \\
            y' &=& 5u^4 \cdot \frac{12x+6 - 12x}{\left(4x+2\right)^2} \\
               &=& 5\left(\frac{3x}{4x+2}\right)^4 \cdot \frac{6}{\left(4x+2\right)^2} \\
               &=& \frac{30}{\left(4x+2\right)^2}\left(\frac{3x}{4x+2}\right)^4
        \end{eqnarray}
    \item (Section 3.7 Exercise 69)
        \begin{eqnarray}
            y &=& \left(\left(x+2\right)\left(x^2+1\right)\right)^4 \\
            y' &=& 4u^3 \cdot x^2 + 1 + 2x^2 + 4x \\
               &=& 4\left(3x^2 + 4x + 1\right)\left(\left(x+2\right)\left(x^2+1\right)\right)^3 \\
               &=& 4\left(3x + 1\right)\left(x + 1\right)\left(\left(x+2\right)\left(x^2+1\right)\right)^3
        \end{eqnarray}
    \item (Section 3.8 Exercise 13)
        \begin{eqnarray}
            x^4 + y^4 &=& 2 \\
            (1, -1) \\
            4x^3 + 4y^3\frac{dy}{dx} &=& 0 \\
            4y^3\frac{dy}{dx} &=& -4x^3 \\
            \frac{dy}{dx} &=& \frac{-4x^3}{4y^3} \\
                          &=& \frac{-x^3}{y^3} \\
            \frac{dy}{dx}\Bigr|_{\substack{(1,-1)}} &=& \frac{-(1^3)}{(-1)^3} \\
                                                    &=& \frac{-1}{-1} \\
                                                    &=& 1
        \end{eqnarray}
    \item (Section 3.8 Exercise 15)
        \begin{eqnarray}
            y^2 &=& 4x \\
            (1, 2) \\
            2y\frac{dy}{dx} &=& 4 \\
            \frac{dy}{dx} &=& \frac{4}{2y} \\
                          &=& \frac{2}{y} \\
            \frac{dy}{dx}\Bigr|_{\substack{(1,2)}} &=& \frac{2}{2} \\
                                                   &=& 1
        \end{eqnarray}
    \item (Section 3.8 Exercise 31)
        \begin{eqnarray}
            \sin{xy} &=& x + y \\
            \cos{xy} \cdot \left(y + x \frac{dy}{dx}\right) &=& 1 + \frac{dy}{dx} \\
            y\cos{xy} + \frac{dy}{dx}x\cos{xy} &=& 1 + \frac{dy}{dx} \\
            \frac{dy}{dx}x\cos{xy} - \frac{dy}{dx} &=& 1 - y\cos{xy} \\
            \frac{dy}{dx}(x\cos{xy} - 1) &=& 1 - y\cos{xy} \\
            \frac{dy}{dx} &=& \frac{1 - y\cos{xy}}{x\cos{xy} - 1}
        \end{eqnarray}
    \item (Section 3.8 Exercise 33)
        \begin{eqnarray}
            \cos{y^2} + x &=& e^y \\
            -\frac{dy}{dx}2y\sin{y^2} + 1 &=& \frac{dy}{dx}e^y \\
            \frac{dy}{dx}2y\sin{y^2} + \frac{dy}{dx}e^y &=& 1 \\
            \frac{dy}{dx}\left(2y\sin{y^2} + e^y\right) &=& 1 \\
            \frac{dy}{dx} &=& \frac{1}{2y\sin{y^2} + e^y}
        \end{eqnarray}
    \item (Section 3.8 Exercise 47)
        \begin{eqnarray}
            x^2 + xy + y^2 &=& 7 \\
            (2, 1) \\
            2x + y + \frac{dx}{dy}x + \frac{dx}{dy}2y &=& 0 \\
            \frac{dx}{dy}\left(x + 2y\right) &=& - 2x - y \\
            \frac{dx}{dy} &=& \frac{- 2x - y}{x + 2y} \\
            \frac{dy}{dx}\Bigr|_{\substack{(2,1)}} &=& \frac{-2(2) - 1}{2 + 2(1)} \\
                                                   &=& \frac{-4 - 1}{2 + 2} \\
                                                   &=& \frac{-5}{4} \\
            y &=& \frac{-5}{4}x - 2\frac{-5}{4} + 1 \\
              &=& \frac{-5}{4}x + \frac{5}{2} + \frac{2}{2} \\
              &=& \frac{-5}{4}x + \frac{7}{2}
        \end{eqnarray}
    \item (Section 3.8 Exercise 48)
        \begin{eqnarray}
            x^4 - x^2y + y^4 &=& 1 \\
            (-1, 1) \\
            4x^3 - 2xy - \frac{dy}{dx}x^2 + \frac{dy}{dx}4y^3 &=& 0 \\
            \frac{dy}{dx}\left(-x^2 + 4y^3\right) &=& 2xy - 4x^3 \\
            \frac{dy}{dx} &=& \frac{2xy - 4x^3}{4y^3 - x^2} \\
            \frac{dy}{dx}\Bigr|_{\substack{(-1,1)}} &=& \frac{2(-1)(1) - 4(-1)^3}{4(1)^3 - (-1)^2} \\
                                                    &=& \frac{-2 + 4}{4 + 1} \\
                                                    &=& \frac{2}{5} \\
            y &=& \frac{2}{5}x + \frac{2}{5} + 1 \\
              &=& \frac{2}{5}x + \frac{7}{5}
        \end{eqnarray}
    \item (Section 3.8 Exercise 25)
        \begin{eqnarray}
            x\sqrt[3]{y} + y &=& 10 \\
            (1, 8) \\
            \sqrt[3]{y} + \frac{dy}{dx}\left(\frac{x}{3y^\frac{2}{3}}\right) + \frac{dy}{dx} &=& 0 \\
            \frac{dy}{dx}\left(\frac{x}{3y^\frac{2}{3}} + 1\right) &=& - \sqrt[3]{y} \\
            \frac{dy}{dx} &=& \frac{- 3y^\frac{2}{3}\sqrt[3]{y}}{x + 3y^{\frac{2}{3}}} \\
                          &=& \frac{- 3y}{3y^{\frac{2}{3}} + x} \\
            \frac{dy}{dx}\Bigr|_{\substack{(1,8)}} &=& \frac{- 3(8)}{3(8)^{\frac{2}{3}} + 1} \\
                                                   &=& \frac{-24}{3(4) + 1} \\
                                                   &=& \frac{-24}{13}
        \end{eqnarray}
    \item (Section 3.8 Exercise 26)
        \begin{eqnarray}
            \left(x + y\right)^{\frac{2}{3}} &=& y \\
            (4, 4) \\
            \frac{2}{3}\left(x + y\right)^{-\frac{1}{3}} \cdot \left(1 + \frac{dy}{dx}\right) &=& \frac{dy}{dx} \\
            \frac{2}{3}\left(x + y\right)^{-\frac{1}{3}} + \frac{dy}{dx}\frac{2}{3}\left(x + y\right)^{-\frac{1}{3}} &=& \frac{dy}{dx} \\
            \frac{2}{3}\left(x + y\right)^{-\frac{1}{3}} &=& \frac{dy}{dx} - \frac{dy}{dx}\frac{2}{3}\left(x + y\right)^{-\frac{1}{3}} \\
            \frac{dy}{dx}\left(1 - \frac{2}{3}\left(x + y\right)^{-\frac{1}{3}}\right) &=& \frac{2}{3}\left(x + y\right)^{-\frac{1}{3}} \\
            \frac{dy}{dx} &=& \frac{\frac{2}{3}\left(x + y\right)^{-\frac{1}{3}}}{1 - \frac{2}{3}\left(x + y\right)^{-\frac{1}{3}}} \\
            \frac{dy}{dx}\Bigr|_{\substack{(4,4)}} &=& \frac{\frac{2}{3}\left(4 + 4\right)^{-\frac{1}{3}}}{1 - \frac{2}{3}\left(4 + 4\right)^{-\frac{1}{3}}} \\
                                                   &=& \frac{\frac{2}{3}\frac{1}{2}}{1 - \frac{2}{3}\frac{1}{2}} \\
                                                   &=& \frac{\frac{1}{2}}{- \frac{1}{2}} \\
                                                   &=& -1
        \end{eqnarray}
    \item (Section 3.8 Exercise 51)
        \begin{eqnarray}
            x + y^2 &=& 1 \\
            1 + \frac{dy}{dx}2y &=& 0 \\
            \frac{dy}{dx}2y &=& -1 \\
            \frac{dy}{dx} &=& \frac{-1}{2y} \\
            \frac{d^2y}{dx^2} &=& \frac{-1}{2y} \\
                              &=& \frac{-1}{2}\frac{1}{y} \\
                              &=& \frac{-1}{2}\frac{dy}{dx}\frac{-1}{y^2} \\
                              &=& \frac{-1}{2}\frac{-1}{2y}\frac{-1}{y^2} \\
                              &=& \frac{-1}{4y^3}
        \end{eqnarray}
    \item (Section 3.8 Exercise 52)
        \begin{eqnarray}
            2x^2 + y^2 &=& 4 \\
            4x + 2y\frac{dy}{dx} &=& 0 \\
            2y\frac{dy}{dx} &=& -4x \\
            \frac{dy}{dx} &=& \frac{-4x}{2y} \\
                          &=& \frac{-2x}{y} \\
            \frac{d^2y}{dx^2} &=& \frac{-2x}{y} \\
                              &=& -2x\frac{1}{y} \\
                              &=& -2x\frac{dy}{dx}\frac{-1}{y^2} \\
                              &=& -2x\frac{-2x}{y}\frac{-1}{y^2} \\
                              &=& \frac{-4x^2}{y^3}
        \end{eqnarray}
    \item (Section 3.9 Exercise 15)
        \begin{eqnarray}
            y &=& \ln{7x} \\
            y' &=& \frac{1}{7x} \cdot 7 \\
               &=& \frac{7}{7x} \\
               &=& \frac{1}{x}
        \end{eqnarray}
    \item (Section 3.9 Exercise 16)
        \begin{eqnarray}
            y &=& x^2\ln{x} \\
            y' &=& 2x\ln{x} + \frac{x^2}{x} \\
               &=& 2x\ln{x} + x
        \end{eqnarray}
    \item (Section 3.9 Exercise 19)
        \begin{eqnarray}
            y &=& \ln{|\sin{x}|} \\
            y' &=& \frac{\cos{x}}{\sin{x}} \\
               &=& \cot{x}
        \end{eqnarray}
    \item (Section 3.9 Exercise 37)
        \begin{eqnarray}
            y &=& 8^x \\
            y' &=& 8^x\ln{8}
        \end{eqnarray}
    \item (Section 3.9 Exercise 39)
        \begin{eqnarray}
            y &=& 5 \cdot 4^x \\
            y' &=& 5 \cdot 4^x\ln{4}
        \end{eqnarray}
    \item (Section 3.9 Exercise 10)
        \begin{eqnarray}
            \frac{d}{dx}\left(x^e + e^x\right) &=& ex^{e-1} + e^x
        \end{eqnarray}
    \item (Section 3.9 Exercise 33)
        \begin{eqnarray}
            y &=& x^e \\
            y' &=& ex^{e-1}
        \end{eqnarray}
    \item (Section 3.9 Exercise 35)
        \begin{eqnarray}
            y &=& \left(2^x + 1\right)^\pi \\
            y' &=& \pi(2^x + 1)^{\pi - 1} \cdot 2^x\ln{2}
        \end{eqnarray}
    \item (Section 3.9 Exercise 49)
        \begin{eqnarray}
            f(x) &=& x^{\cos{x}} \\
                 &=& e^{\cos{x}\ln{x}} \\
            a &=& \frac{\pi}{2} \\
            f'(x) &=& e^{\cos{x}\ln{x}} \cdot \left(-\sin{x}\ln{x} + \frac{\cos{x}}{x}\right) \\
                  &=& x^{\cos{x}}\left(-\sin{x}\ln{x} + \frac{\cos{x}}{x}\right) \\
            f'\left(\frac{\pi}{2}\right) &=& \left(\frac{\pi}{2}\right)^0 \left(-1\cdot\ln{\frac{\pi}{2}} + \frac{0}{\pi}\right) \\
                                         &=& -\ln{\frac{\pi}{2}}
        \end{eqnarray}
    \item (Section 3.9 Exercise 59)
        \begin{eqnarray}
            f(x) &=& x^{\sin{x}} \\
                 &=& e^{\sin{x}\ln{x}} \\
            f'(x) &=& e^{\sin{x}\ln{x}} \cdot \left(\cos{x}\ln{x} + \frac{\sin{x}}{x}\right) \\
            f'(1) &=& e^0 \cdot \left(\cos{1}\ln{1} + \frac{\sin{1}}{1}\right) \\
                  &=& \cos{1} \cdot 0 + \sin{1} \\
                  &=& \sin{1} \\
            y &=& x\sin{1} - \sin{1} + 1
        \end{eqnarray}
    \item (Section 3.9 Exercise 63)
        \begin{eqnarray}
            y &=& 4\log_3{\left(x^2 - 1\right)} \\
            y' &=& 4 \cdot \frac{1}{\left(x^2 - 1\right)\ln{3}} \cdot 2x \\
               &=& \frac{8x}{\left(x^2 - 1\right)\ln{3}} \\
        \end{eqnarray}
    \item (Section 3.9 Exercise 64)
        \begin{eqnarray}
            y &=& \log_{10}{x} \\
            y' &=& \frac{1}{x\ln{10}}
        \end{eqnarray}
    \item (Section 3.9 Exercise 77)
        \begin{eqnarray}
            f(x) &=& \frac{\left(x + 1\right)^{10}}{\left(2x - 4\right)^8} \\
            \ln{f(x)} &=& \ln{\frac{\left(x + 1\right)^{10}}{\left(2x - 4\right)^8}} \\
                      &=& \ln{\left(x + 1\right)^{10}} - \ln{\left(2x - 4\right)^8} \\
                      &=& 10\ln{x + 1} - 8\ln{2x - 4} \\
            \frac{d}{dx}\left(\ln{f\left(x\right)}\right) &=& \frac{f'(x)}{f(x)} \\
            \frac{f'(x)}{f(x)} &=& 10\frac{1}{x + 1} - 8\frac{1}{2x - 4}2 \\
                               &=& \frac{10}{x + 1} - \frac{16}{2x - 4} \\
                               &=& \frac{10}{x + 1} - \frac{8}{x - 2} \\
            f'(x) &=& f(x)\left(\frac{10}{x + 1} - \frac{8}{x - 2}\right) \\
                  &=& \frac{\left(x + 1\right)^{10}}{\left(2x - 4\right)^8}\left(\frac{10}{x + 1} - \frac{8}{x - 2}\right) \\
        \end{eqnarray}
    \item (Section 3.9 Exercise 80)
        \begin{eqnarray}
            f(x) &=& \frac{\tan^{10}x}{\left(5x + 3\right)^6} \\
            \ln{f(x)} &=& \ln{\frac{\tan^{10}x}{\left(5x + 3\right)^6}} \\
                      &=& \ln{\tan^{10}{x}} - \ln{\left(5x + 3\right)^6} \\
                      &=& 10\ln{\tan{x}} - 6\ln{5x + 3} \\
            \frac{d}{dx}\left(\ln{f\left(x\right)}\right) &=& \frac{f'(x)}{f(x)} \\
            \frac{f'(x)}{f(x)} &=& 10\frac{1}{\tan{x}}\sec^2{x} - 6\frac{1}{5x + 3}5 \\
                               &=& \frac{10\sec^2{x}}{\tan{x}} - \frac{30}{5x + 3} \\
                               &=& \frac{10\sec{x}}{\sin{x}} - \frac{30}{5x + 3} \\
            f'(x) &=& f(x)\left(\frac{10\sec{x}}{\sin{x}} - \frac{30}{5x + 3}\right) \\
                  &=& \frac{\tan^{10}x}{\left(5x + 3\right)^6}\left(\frac{10\sec{x}}{\sin{x}} - \frac{30}{5x + 3}\right)
        \end{eqnarray}
    \item (Section 3.10 Exercise 13)
        \begin{eqnarray}
            f(x) &=& \sin^{-1}{2x} \\
            f'(x) &=& \frac{2}{\sqrt{1 - \left(2x\right)^2}} \\
                  &=& \frac{2}{\sqrt{1 - 4x^2}}
        \end{eqnarray}
    \item (Section 3.10 Exercise 15)
        \begin{eqnarray}
            f(w) &=& \cos{\left(\sin^{-1}{2w}\right)} \\
            f'(w) &=& -\sin{\left(\sin^{-1}{2w}\right)} \cdot \frac{1}{\sqrt{1 - \left(2w\right)^2}} \cdot 2 \\
                  &=& - \frac{2\sin{\left(\sin^{-1}{2w}\right)}}{\sqrt{1 - 4w^2}} \\
                  &=& - \frac{4w}{\sqrt{1 - 4w^2}}
        \end{eqnarray}
    \item (Section 3.10 Exercise 27)
        \begin{eqnarray}
            f(w) &=& w^2 - \tan^{-1}{w^2} \\
            f'(w) &=& 2w - \frac{2w}{1 + w^4} \\
                  &=& \frac{2w^5}{1 + w^4}
        \end{eqnarray}
    \item (Section 3.10 Exercise 41)
        \begin{eqnarray}
            f(x) &=& \tan^{-1}{2x} \\
            f'(x) &=& \frac{2}{1 + 4x^2} \\
            f'\left(\frac{1}{2}\right) &=& 1 \\
            y &=& x - \frac{1}{2} + \frac{\pi}{4}
        \end{eqnarray}
    \item (Section 3.10 Exercise 45)
        \begin{eqnarray}
            \tan{\theta} &=& \frac{150}{x} \\
            \theta &=& \tan^{-1}{\frac{150}{x}} \\
            \frac{d\theta}{dx} &=& -\frac{150}{x^2\left(1 + \left(\frac{150}{x}\right)^2\right)} \\
            \frac{d\theta}{dx}\Bigr|_{\substack{x = 500}} &=& -0.00055
        \end{eqnarray}
    \item (Section 3.10 Exercise 7)
        \begin{enumerate}
            \item $$\left(f^{-1}\right)'\left(4\right) = \frac{1}{f'(0)} = \frac{1}{2}$$
            \item $$\left(f^{-1}\right)'\left(6\right) = \frac{1}{f'(1)} = \frac{2}{3}$$
            \item $$\left(f^{-1}\right)'\left(1\right) = \text{Undeterminable}$$
            \item $$f'\left(1\right) = \frac{3}{2}$$
        \end{enumerate}
    \item (Section 3.10 Exercise 8)
        \begin{enumerate}
            \item $$f'\left(f\left(0\right)\right) = 2$$
            \item $$\left(f^{-1}\right)'\left(0\right) = \frac{1}{f'(-4)} = \frac{1}{5}$$
            \item $$\left(f^{-1}\right)'\left(1\right) = \frac{1}{f'(-2)} = \frac{1}{4}$$
            \item $$\left(f^{-1}\right)'\left(f\left(4\right)\right) = \frac{1}{f'(4)} = 1$$
        \end{enumerate}
\end{enumerate}

\blfootnote{A copy of my notes (in \LaTeX) are available on my \href{https://github.com/onlinechronically/MATH-211}{GitHub}}
\end{document}
